\begin{tframe}{Funzionalità implementate}
%\vspace{0.3cm}
\begin{minipage}{0.45\textwidth}
\begin{block}{1 - Applicazioni}
{\fontsize{7pt}{7.2}\selectfont
\begin{itemize} 
\item Aggiunta
\item Configurazione per l'accesso in SSO
\item Modifica
\item Rimozione
\item Categorizzazione
\end{itemize}
}
\end{block}

\begin{block}{2 - Gruppi}
{\fontsize{7pt}{7.2}\selectfont
Possibilità di raggruppare le applicazioni secondo caratteristiche comuni\newline(es: localizzazione).
}
\end{block}
\end{minipage}
\hfill
\begin{minipage}{0.45\textwidth}
\begin{block}{3 - Cataloghi aziendali}
{\fontsize{7pt}{7.2}\selectfont
Possibilità di creare cataloghi privati per le aziende registrate al servizio.
}
\end{block}
\begin{block}{4 - Logging e statistiche}
{\fontsize{7pt}{7.2}\selectfont
\begin{itemize} 
\item Salvataggio di log per qualsiasi operazione
\item Visualizzazione log
\item Visualizzazione statistiche
\end{itemize}
}
\end{block}

\begin{block}{5 - Autenticazione federata}
{\fontsize{7pt}{7.2}\selectfont
SSO secondo la modalità federata.
}
\end{block}

\end{minipage}
%\begin{center}
%\highlight{Catalogue Manager assicura che gli utenti accedano alle applicazioni web con la minor perdita di tempo possibile.}
%\end{center}
\end{tframe}