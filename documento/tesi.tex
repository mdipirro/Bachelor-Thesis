% I seguenti commenti speciali impostano:
% 1. 
% 2. PDFLaTeX come motore di composizione;
% 3. tesi.tex come documento principale;
% 4. il controllo ortografico italiano per l'editor.

% !TEX encoding = UTF-8
% !TEX TS-program = pdflatex
% !TEX root = tesi.tex
% !TEX spellcheck = it-IT

\documentclass[10pt,                    % corpo del font principale
               a4paper,                 % carta A4
               oneside,                 % impagina per solo fronte
               openright,               % inizio capitoli a destra
               english,                 
               italian           
               ]{book}    

\usepackage[utf8]{inputenc}             % codifica di input; anche [latin1] va bene
                                        % NOTA BENE! va accordata con le preferenze dell'editor

%**************************************************************
% Importazione package
%************************************************************** 

%\usepackage{amsmath,amssymb,amsthm}    % matematica

\usepackage[english, italian]{babel}    % per scrivere in italiano e in inglese;
                                        % l'ultima lingua (l'italiano) risulta predefinita

\usepackage{bookmark}                   % segnalibri

\usepackage{caption}                    % didascalie
\usepackage{subcaption}
\expandafter\def\csname ver@subfig.sty\endcsname{}

\usepackage{chngpage,calc}              % centra il frontespizio

\usepackage{csquotes}                   % gestisce automaticamente i caratteri (")

\usepackage{emptypage}                  % pagine vuote senza testatina e piede di pagina

\usepackage{epigraph}					% per epigrafi

\usepackage{eurosym}                    % simbolo dell'euro

\usepackage[T1]{fontenc}                % codifica dei font:
                                        % NOTA BENE! richiede una distribuzione *completa* di LaTeX

%\usepackage{indentfirst}               % rientra il primo paragrafo di ogni sezione

\usepackage{graphicx}                   % immagini

\usepackage{hyperref}                   % collegamenti ipertestuali



\usepackage[binding=5mm]{layaureo}      % margini ottimizzati per l'A4; rilegatura di 5 mm

\usepackage{listings}                   % codici

\usepackage{microtype}                  % microtipografia

\usepackage{mparhack,fixltx2e,relsize}  % finezze tipografiche

\usepackage{nameref}                    % visualizza nome dei riferimenti                                      

\usepackage[font=small]{quoting}        % citazioni

\usepackage{subfig}                     % sottofigure, sottotabelle

\usepackage[italian]{varioref}          % riferimenti completi della pagina

\usepackage[dvipsnames,table]{xcolor}         % colori

\usepackage{booktabs}                   % tabelle                                       
\usepackage{tabularx}                   % tabelle di larghezza prefissata                                    
\usepackage{longtable}                  % tabelle su più pagine                                        
\usepackage{ltxtable}                   % tabelle su più pagine e adattabili in larghezza

\usepackage[toc, acronym]{glossaries}   % glossario
                                        % per includerlo nel documento bisogna:
                                        % 1. compilare una prima volta tesi.tex;
                                        % 2. eseguire: makeindex -s tesi.ist -t tesi.glg -o tesi.gls tesi.glo
                                        % 3. eseguire: makeindex -s tesi.ist -t tesi.alg -o tesi.acr tesi.acn
                                        % 4. compilare due volte tesi.tex.

\usepackage[
	backend=biber,
	citestyle=numeric-comp,
	hyperref,
	backref,
	sorting=none
]{biblatex}
                                        % eccellente pacchetto per la bibliografia; 
                                        % produce uno stile di citazione autore-anno; 
                                        % lo stile "numeric-comp" produce riferimenti numerici
                                        % per includerlo nel documento bisogna:
                                        % 1. compilare una prima volta tesi.tex;
                                        % 2. eseguire: biber tesi
                                        % 3. compilare ancora tesi.tex.


%**************************************************************
% file contenente le impostazioni della tesi
%**************************************************************

%**************************************************************
% Frontespizio
%**************************************************************
\newcommand{\myName}{Matteo Di Pirro}                           % autore
\newcommand{\myTitle}{Gestione del catalogo applicativo per un sistema di Identity and Access Management cloud-based}                    				% titolo
\newcommand{\myDegree}{Tesi di laurea triennale}                % tipo di tesi
\newcommand{\myUni}{Università degli Studi di Padova}           % università
\newcommand{\myFaculty}{Corso di Laurea in Informatica}         % facoltà
\newcommand{\myDepartment}{Dipartimento di Matematica}          % dipartimento
\newcommand{\myProf}{Mauro Conti}                               % relatore
\newcommand{\myLocation}{Padova}                                % dove
\newcommand{\myAA}{2015-2016}                                   % anno accademico
\newcommand{\myTime}{Settembre 2016}                            % quando


%**************************************************************
% Impostazioni di impaginazione
% see: http://wwwcdf.pd.infn.it/AppuntiLinux/a2547.htm
%**************************************************************

\setlength{\parindent}{14pt}   % larghezza rientro della prima riga
\setlength{\parskip}{0pt}   % distanza tra i paragrafi


%**************************************************************
% Impostazioni di biblatex
%**************************************************************
\bibliography{bibliografia} % database di biblatex 


\defbibheading{bibliography}
{
    \phantomsection 
    \addcontentsline{toc}{chapter}{\bibname}
    \chapter*{\bibname\markboth{\bibname}{\bibname}}
}

\setlength\bibitemsep{1.5\itemsep} % spazio tra entry

\DeclareBibliographyCategory{opere}
\DeclareBibliographyCategory{web}

\addtocategory{web}{site:gartner}
\addtocategory{web}{site:scrum}
\addtocategory{web}{site:agilemanifesto}
\addtocategory{web}{site:scrumroles}
\addtocategory{web}{site:spidgov}
\addtocategory{web}{site:spidatt}
\addtocategory{web}{site:nodejava}
\addtocategory{web}{site:paypalNode}
\addtocategory{web}{site:mongodb}
\addtocategory{web}{site:expressjs}
\addtocategory{web}{site:git}
\addtocategory{web}{site:bitbucket}
\addtocategory{web}{site:guelphsso}
\addtocategory{web}{site:jwtintro}
\addtocategory{web}{site:signjwt}
\addtocategory{web}{site:SPvsIdPinitiated}
\addtocategory{web}{site:expressMiddleware}

\addtocategory{opere}{reg:spid}
\addtocategory{opere}{gilchrist05}
\addtocategory{opere}{doglio05}
\addtocategory{opere}{grt:G00269748}
\addtocategory{opere}{grt:G00292924}
\addtocategory{opere}{grt:G00279633}
\addtocategory{opere}{rountree12}
\addtocategory{opere}{std:saml}
\addtocategory{opere}{grt:G00261583}
\addtocategory{opere}{grt:G00296572}
\addtocategory{opere}{swebok}
\addtocategory{opere}{rfc:7519}
\addtocategory{opere}{designPattern}
\addtocategory{opere}{javascriptPatterns}
\addtocategory{opere}{abacNIST}
\addtocategory{opere}{grt:G00260705}

\defbibheading{opere}{\section*{Riferimenti bibliografici}}
\defbibheading{web}{\section*{Siti Web}}


%**************************************************************
% Impostazioni di caption
%**************************************************************
\captionsetup{
    tableposition=top,
    figureposition=bottom,
    font=normal,
    format=hang,
    labelfont=bf
}

%**************************************************************
% Impostazioni di glossaries
%**************************************************************

%**************************************************************
% Acronimi
%**************************************************************
\renewcommand{\acronymname}{Acronimi e abbreviazioni}

\newacronym[description=\glslink{IAM}{\textbf{I}dentity and \textbf{A}ccess \textbf{M}anagement}]
    {iam}{IAM}{\textbf{I}dentity and \textbf{A}ccess \textbf{M}anagement}
    
\newacronym[description=\glslink{IT}{\textbf{I}nformation \textbf{T}echnology}]
    {it}{IT}{\textbf{I}nformation \textbf{T}echnology}

\newacronym[description=\glslink{DLM}{\textbf{D}ata \textbf{L}ifetime \textbf{M}anagement}]
	{dlm}{DLM}{\textbf{D}ata \textbf{L}ifetime \textbf{M}anagement}

\newacronym[description=\glslink{ITIL}{\textbf{I}nformation \textbf{T}echnology \textbf{I}nfrastructure \textbf{L}ibrary}]
	{itil}{ITIL}{\textbf{I}nformation \textbf{T}echnology \textbf{I}nfrastructure \textbf{L}ibrary}
	
\newacronym[description=\glslink{DBA}{\textbf{D}ata\textbf{B}ase \textbf{A}dministrator}]
	{dba}{DBA}{\textbf{D}ata\textbf{B}ase \textbf{A}dministrator}
	
\newacronym[description=\glslink{SSO}{\textbf{S}ingle \textbf{S}ign \textbf{O}n}]
	{sso}{SSO}{\textbf{S}ingle \textbf{S}ign \textbf{O}n}
	
\newacronym[description=\glslink{SPID}{\textbf{S}istema \textbf{P}ubblico \textbf{I}dentità \textbf{D}igitale}]
	{spid}{SPID}{\textbf{S}istema \textbf{P}ubblico \textbf{I}dentità \textbf{D}igitale}
	
\newacronym[description=\glslink{AgID}{\textbf{Ag}enzia per l'\textbf{I}talia \textbf{D}igitale}]
	{agid}{AgID}{\textbf{A}genzia per l'\textbf{I}talia \textbf{D}igitale}
	
\newacronym[description=\glslink{IoT}{\textbf{I}nternet \textbf{o}f \textbf{T}hings}]
	{iot}{IoT}{\textbf{I}nternet \textbf{o}f \textbf{T}hings}
	
\newacronym[description=\glslink{IdM}{\textbf{Id}entity \textbf{M}anagement}]
	{idm}{IdM}{\textbf{Id}entity \textbf{M}anagement}
	
\newacronym[description=\glslink{AM}{\textbf{A}ccess \textbf{M}anagement}]
	{am}{AM}{\textbf{A}ccess \textbf{M}anagement}
	
\newacronym[description=\glslink{IO}{\textbf{I}nput/\textbf{O}utput}]
	{io}{I/O}{\textbf{I}nput/\textbf{O}utput}
	
\newacronym[description=\glslink{DBMS}{\textbf{D}ata\textbf{B}ase \textbf{M}anagement \textbf{S}ystem}]
	{dbms}{DBMS}{\textbf{D}ata\textbf{B}ase \textbf{M}anagement \textbf{S}ystem}
	
\newacronym[description=\glslink{NoSQL}{\textbf{N}ot \textbf{o}nly \textbf{SQL}}]
	{nosql}{NoSQL}{\textbf{N}ot \textbf{o}nly \textbf{SQL}}
	
\newacronym[description=\glslink{REST}{\textbf{RE}presentational \textbf{S}tate \textbf{T}ransfer}]
	{rest}{REST}{\textbf{RE}presentational \textbf{S}tate \textbf{T}ransfer}

\newacronym[description=\glslink{URI}{\textbf{U}niform \textbf{R}esource \textbf{I}dentifier}]
	{uri}{URI}{\textbf{U}niform \textbf{R}esource \textbf{I}dentifier}
	
\newacronym[description=\glslink{HTTP}{\textbf{H}yper\textbf{T}ext \textbf{T}ransfer \textbf{P}rotocol}]
	{http}{HTTP}{\textbf{H}yper\textbf{T}ext \textbf{T}ransfer \textbf{P}rotocol}

\newacronym[description=\glslink{IDoT}{\textbf{ID}entity \textbf{o}f \textbf{T}hings}]
	{idot}{IDoT}{\textbf{ID}entity \textbf{o}f \textbf{T}hings}
	
\newacronym[description=\glslink{API}{\textbf{A}pplication \textbf{P}rogramming \textbf{I}nterface}]
	{api}{API}{\textbf{A}pplication \textbf{P}rogramming \textbf{I}nterface}
	
\newacronym[description=\glslink{IDaaS}{\textbf{ID}entity \textbf{a}s \textbf{a} \textbf{S}ervice}]
	{idaas}{IDaaS}{\textbf{ID}entity \textbf{a}s \textbf{a} \textbf{S}ervice}
	
\newacronym[description=\glslink{SAML}{\textbf{S}ecurity \textbf{A}ssertion \textbf{M}arkup \textbf{L}angage}]
	{saml}{SAML}{\textbf{S}ecurity \textbf{A}ssertion \textbf{M}arkup \textbf{L}angage}
	
\newacronym[description=\glslink{AD}{\textbf{A}ctive \textbf{D}irectory}]
	{ad}{AD}{\textbf{A}ctive \textbf{D}irectory}
	
\newacronym[description=\glslink{AJAX}{\textbf{A}synchronous \textbf{J}avaScript \textbf{a}nd \textbf{X}ML}]
	{ajax}{AJAX}{\textbf{A}synchronous \textbf{J}avaScript \textbf{a}nd \textbf{X}ML}
	
\newacronym[description=\glslink{URL}{\textbf{U}niform \textbf{R}esource \textbf{L}ocator}]
	{url}{URL}{\textbf{U}niform \textbf{R}esource \textbf{L}ocator}
	
\newacronym[description=\glslink{MAC}{\textbf{M}andatory \textbf{A}ccess \textbf{C}ontrol}]
	{mac}{MAC}{\textbf{M}andatory \textbf{A}ccess \textbf{C}ontrol}
	
\newacronym[description=\glslink{DAC}{\textbf{D}iscretionary \textbf{A}ccess \textbf{C}ontrol}]
	{dac}{DAC}{\textbf{D}iscretionary \textbf{A}ccess \textbf{C}ontrol}
	
\newacronym[description=\glslink{RBAC}{\textbf{R}ole-\textbf{B}ased \textbf{A}ccess \textbf{C}ontrol}]
	{rbac}{RBAC}{\textbf{R}ole-\textbf{B}ased \textbf{A}ccess \textbf{C}ontrol}
	
\newacronym[description=\glslink{IdP}{\textbf{Id}entity \textbf{P}rovider}]
	{idp}{IdP}{\textbf{Id}entity \textbf{P}rovider}
	
\newacronym[description=\glslink{SP}{\textbf{S}ervice \textbf{P}rovider}]
	{sp}{SP}{\textbf{S}ervice \textbf{P}rovider}

\newacronym[description=\glslink{XML}{e\textbf{X}tensible \textbf{M}arkup \textbf{L}anguage}]
	{xml}{XML}{e\textbf{X}tensible \textbf{M}arkup \textbf{L}anguage}
	
\newacronym[description=\glslink{SaaS}{\textbf{S}oftware \textbf{a}s \textbf{a} \textbf{S}ervice}]
	{saas}{SaaS}{\textbf{S}oftware \textbf{a}s \textbf{a} \textbf{S}ervice}
	
\newacronym[description=\glslink{TCO}{\textbf{T}otal \textbf{C}ost of \textbf{O}wnership}]
	{tco}{TCO}{\textbf{T}otal \textbf{C}ost of \textbf{O}wnership}
	
\newacronym[description=\glslink{IGA}{\textbf{I}dentity \textbf{G}overnance and \textbf{A}dministration}]
	{iga}{IGA}{\textbf{I}dentity \textbf{G}overnance and \textbf{A}dministration}
	
\newacronym[description=\glslink{UML}{\textbf{U}nified \textbf{M}odeling \textbf{L}anguage}]
	{uml}{UML}{\textbf{U}nified \textbf{M}odeling \textbf{L}anguage}
	
\newacronym[description=\glslink{JWT}{\textbf{J}SON \textbf{W}eb \textbf{T}oken}]
	{jwt}{JWT}{\textbf{J}SON \textbf{W}eb \textbf{T}oken}	

\newacronym[description=\glslink{JSON}{\textbf{J}ava\textbf{S}cript \textbf{O}bject \textbf{N}otation}]
	{json}{JSON}{\textbf{J}ava\textbf{S}cript \textbf{O}bject \textbf{N}otation}
	
\newacronym[description=\glslink{HMAC}{keyed-\textbf{H}ash \textbf{M}essage \textbf{A}uthentication \textbf{C}ode}]
	{hmac}{HMAC}{keyed-\textbf{H}ash \textbf{M}essage \textbf{A}uthentication \textbf{C}ode}	
	
\newacronym[description=\glslink{SHA}{\textbf{S}ecure \textbf{H}ash \textbf{A}lgorithm}]
	{sha}{SHA}{\textbf{S}ecure \textbf{H}ash \textbf{A}lgorithm}	
	
\newacronym[description=\glslink{IANA}{\textbf{I}nternet \textbf{A}ssigned \textbf{N}umbers \textbf{A}uthority}]
	{iana}{IANA}{\textbf{I}nternet \textbf{A}ssigned \textbf{N}umbers \textbf{A}uthority}
	
\newacronym[description=\glslink{CORS}{\textbf{C}ross-\textbf{O}rigin \textbf{R}esource \textbf{S}haring}]
	{cors}{CORS}{\textbf{C}ross-\textbf{O}rigin \textbf{R}esource \textbf{S}haring}
	
\newacronym[description=\glslink{XSS}{Cross-site scripting}]
	{xss}{XSS}{Cross-site scripting}
	
\newacronym[description=\glslink{DRY}{\textbf{D}on't \textbf{R}epeat \textbf{Y}ourself}]
	{dry}{DRY}{\textbf{D}on't \textbf{R}epeat \textbf{Y}ourself}
	
\newacronym[description={\textbf{C}reate, \textbf{R}ead, \textbf{U}pdate, \textbf{D}elete}]
	{crud}{CRUD}{\textbf{C}reate, \textbf{R}ead, \textbf{U}pdate, \textbf{D}elete}
	
\newacronym[description=\glslink{CSV}{\textbf{C}omma \textbf{S}eparated \textbf{V}alues}]
	{csv}{CSV}{\textbf{C}omma \textbf{S}eparated \textbf{V}alues}
	
\newacronym[description={\textbf{C}ustomer \textbf{R}elationship \textbf{M}anagement}]
	{crm}{CRM}{\textbf{C}ustomer \textbf{R}elationship \textbf{M}anagement}

\newacronym[description={\textbf{E}nterprise \textbf{R}esource \textbf{P}lanning}]
	{erp}{ERP}{\textbf{E}nterprise \textbf{R}esource \textbf{P}lanning}
	
\newacronym[description={\textbf{A}ttribute \textbf{B}ased \textbf{A}ccess \textbf{C}ontrol}]
	{abac}{ABAC}{\textbf{A}ttribute \textbf{B}ased \textbf{A}ccess \textbf{C}ontrol}	

%**************************************************************
% Glossario
%**************************************************************
\renewcommand{\glossaryname}{Glossario}

\newglossaryentry{IAM}{
    name=\glslink{iam}{IAM},
    text=IAM,
    sort=iam,
    description={\acrlong{iam}\\Disciplina che consente ai giusti individui di accedere alle giuste risorse nel giusto momento per le giuste ragioni. L'IAM risponde alla necessità di garantire un adeguato accesso alle risorse in ambiti tecnologici sempre più eterogenei}
}

\newglossaryentry{IT}{
    name=\glslink{it}{IT},
    text=IT,
    sort=it,
    description={\acrlong{it}\\Settore caratterizzato dall'impiego di computer
    e tecnologie di telecomunicazione per immagazzinare, prelevare, trasmettere e
    manipolare dati in contesti aziendali}
}

\newglossaryentry{DLM}{
	name=\glslink{dlm}{DLM},
	text=DLM,
	sort=dlm,
	description={\acrlong{dlm}\\Processo di gestione dell'informazione di business durante tutto il suo ciclo di vita, a partire dalla creazione e memorizzazione iniziale fino alla sua obsolescenza e conseguente eliminazione. \\
	I prodotti a supporto del DLM consentono di automatizzare le attività coinvolte,
	organizzando i dati in livelli diversi e semplificando la migrazione da un livello ad un altro secondo i criteri definiti}
}

\newglossaryentry{bestpractice}{
	name={Best practice},
	text={Best practice},
	description={\mbox{}\\Esperienza, procedura o azione più significativa, o comunque che ha permesso di ottenere i migliori risultati, relativamente a svariati contesti e obiettivi preposti},
	plural={best practices}
}

\newglossaryentry{ITIL}{
	name=\glslink{itil}{ITIL},
	text=ITIL,
	sort=itil,
	description={\acrlong{itil}\\Insieme di linee guida ispirate dalla pratica (best practice) nella gestione di servizi IT. Consistono in una serie di pubblicazioni che forniscono indicazione sull'erogazione di servizi IT di	qualità e sui processi e mezzi necessari a supportarli}
}

\newglossaryentry{DBA}{
	name=\glslink{dba}{DBA},
	sort=dba,
	description={\acrlong{dba}\\
	Professionista che, all'interno di un'azienda o di un ente, si occupa di installare, configurare e gestire sistemi di archiviazione dei dati, più o meno complessi, consultabili e spesso aggiornabili per via telematica.\\
	Configura gli accessi al database, realizza il monitoraggio dei sistemi di archiviazione, si occupa della manutenzione del server, della sicurezza degli accessi interni ed esterni alla banca dati e definisce, al contempo, le politiche aziendali di impiego e utilizzo delle risorse costituite dal database. \\Tra i principali compiti di questa figura professionale c'è quello di preservare la sicurezza e l'integrità dei dati contenuti nell'archivio}
}

\newglossaryentry{identita}{
	name=Identità,
	text=identità,
	sort=identità,
	description={\mbox{}\\Combinazione di attributi generici (come nome, cognome, indirizzo, ecc.) e	specifici (rilevanti a livello aziendale) che consentono di identificare in modo univoco un utente}
}

\newglossaryentry{SSO}{
	name=\glslink{sso}{SSO},
	text=sso,
	sort=sso,
	description={\acrlong{sso}\\
	Il Single Sign On consente di autenticarsi una volta sola e di avere accesso, in modo del tutto automatico, a varie applicazioni di un sistema. Elimina il bisogno di autenticarsi separatamente a ciascuna applicazione e/o sistema}
}

\newglossaryentry{workflow}{
	name=Workflow,
	text=workflow,
	sort=workflow,
	description={\mbox{}\\Applicazione che automatizza le procedure e i processi aziendali di lavoro cooperativo}
}

\newglossaryentry{provisioning}{
	name=Provisioning,
	text=provisioning,
	sort=provisioning,
	description={\mbox{}\\Insieme di attività attraverso le quali viene garantito all'utente l'autorizzazione presso un sistema o un’applicazione. Il processo include l’assegnazione di diritti e privilegi all’utente, in modo tale da garantire la sicurezza del sistema}
}

\newglossaryentry{autenticazione}{
	name=Autenticazione,
	text=autenticazione,
	sort=autenticazione,
	description={\mbox{}\\Processo attraverso il quale l’utente fornisce le credenziali necessarie per ottenere l’accesso ad un sistema o ad una particolare risorsa; una volta che l’utente ha effettuato l’autenticazione, viene creata una sessione, riferita in tutte	le interazioni fra l’utente ed il sistema, finché l’utente effettua log out o la sessione viene terminata per altre ragioni (ad esempio per un timeout)}
}

\newglossaryentry{autorizzazione}{
	name=Autorizzazione,
	text=autorizzazione,
	sort=autorizzazione,
	description={\mbox{}\\Processo che garantisce che gli utenti correttamente autenticati possano accedere solo alle giuste risorse. Il collegamento tra utente e risorse viene stabilito in base alle politiche di accesso alla risorsa, precedentemente decise dal proprietario della risorsa stessa}
}

\newglossaryentry{efficienza}{
	name=Efficienza,
	text=efficienza,
	sort=efficienza,
	description={\mbox{}\\Capacità di evitare sprechi di energia, risorse, tempo e denaro durante lo svolgimento di una specifica attività. In senso più matematico, è la misura di quanto l'input è ben impiegato per produrre un output}
}

\newglossaryentry{affidabilita}{
	name=Affidabilità,
	text=affidabilità,
	sort=affidabilità,
	description={\mbox{}\\Certezza di corretto funzionamento che un impianto, un apparecchio, un dispositivo può dare in base alle sue caratteristiche tecniche e di fabbricazione}
}

\newglossaryentry{scalabilita}{
	name=Scalabilità,
	text=scalabilità,
	sort=scalabilità,
	description={\mbox{}\\Capacità di aumentare le risorse per ottenere un incremento (idealmente) lineare nella capacità del servizio. La caratteristica principale di un'applicazione scalabile è costituita dal fatto che un carico aggiuntivo richiede solamente risorse aggiuntive anziché un'estesa modifica dell'applicazione stessa}
}

\newglossaryentry{framework}{
	name=Framework,
	text=framework,
	sort=framework,
	description={\mbox{}\\Modalità strutturata, pianificata e permanente, che supporta una prassi, una metodologia, un progetto, un sistema di gestione; nello sviluppo software, inoltre, indica più specificamente una logica di supporto su cui un software può essere	progettato e realizzato}
}

\newglossaryentry{agile}{
	name=Agile,
	text=agile,
	sort=agile,
	description={\mbox{}\\Serie di principi per lo sviluppo di software secondo i quali i requisiti e le soluzioni si evolvono attraverso la collaborazione tra individui. Si basa su quattro principi fondamentali, definiti nel Manifesto for Agile Software Development:
	\begin{itemize}
	\item gli individui e le interazioni sono più importanti di processi e strumenti;
	\item il software funzionante è più importante di una documentazione esaustiva;
	\item la collaborazione con il cliente è più importante della negoziazione dei contratti;
	\item la rapida risposta al cambiamento è più importante di seguire una pianificazione.
	\end{itemize}}
}

\newglossaryentry{stakeholder}{
	name=Stakeholder,
	text=stakeholder,
	sort=stakeholder,
	plural=stakeholders,
	description={\mbox{}\\Soggetto (o un gruppo di soggetti) influente nei confronti di un'iniziativa economica, che sia un'azienda o un progetto}
}

\newglossaryentry{SPID}{
	name=\glslink{spid}{SPID},
	text=spid,
	sort=spid,
	description={\acrlong{spid}\\
	Sistema unico di login per l'accesso ai servizi online della pubblica amministrazione e dei privati aderenti
	}
}

\newglossaryentry{AgID}{
	name=\glslink{agid}{AgID},
	text=agid,
	sort=agid,
	description={\acrlong{agid}\\
	Agenzia pubblica italiana che si occupa di perseguire il massimo livello di innovazione tecnologica nell'organizzazione e nello sviluppo della pubblica amministrazione}
}

\newglossaryentry{IoT}{
	name=\glslink{iot}{IoT},
	text=iot,
	sort=iot,
	description={\acrlong{iot}\\
	Neologismo riferito all'estensione di Internet al mondo degli oggetti e dei luoghi concreti. Gli oggetti si rendono riconoscibili e acquisiscono intelligenza grazie al fatto di poter comunicare dati su se stessi e accedere ad informazioni aggregate da parte di altri}
}

\newglossaryentry{IdM}{
	name=\glslink{idm}{IdM},
	text=Identity Management,
	sort=idm,
	description={\acrlong{idm}\\
	Sistema integrato di tecnologie, criteri e procedure in grado di consentire alle organizzazioni di facilitare, e al tempo stesso controllare, gli accessi degli utenti ad applicazioni e dati critici
	}
}

\newglossaryentry{AM}{
	name=\glslink{am}{AM},
	text=am,
	sort=am,
	description={\acrlong{am}\\
	Sistema integrato di tecnologie in grado di consentire alle organizzazioni di definire politiche di accesso alle risorse}
}

\newglossaryentry{rollback}{
	name=Rollback,
	text=rollback,
	sort=rollback,
	description={\mbox{}\\
	Operazione che permette di riportare il database a una versione o stato precedente}
}

\newglossaryentry{IO}{
	name=\glslink{io}{I/O},
	text=I/O,
	sort=io,
	description={\acrlong{io}\\
	Interfacce messe a disposizione dal sistema operativo, o più in generale da qualunque sistema di basso livello, ai programmi per effettuare uno scambio di dati o segnali con altri programmi, col computer o con lo stesso sistema}
}

\newglossaryentry{thread}{
	name=Thread,
	text=thread,
	sort=thread,
	description={\mbox{}\\
	Suddivisione di un processo in due o più filoni o sottoprocessi, che vengono eseguiti concorrentemente da un sistema di elaborazione monoprocessore (multithreading) o multiprocessore (multicore)}
}


\newglossaryentry{filesystem}{
	name={File system},
	text={file system},
	sort={file system},
	description={\mbox{}\\
	Meccanismo con il quale i file sono posizionati e organizzati o su un dispositivo di archiviazione o su una memoria di massa, come un disco rigido}
}

\newglossaryentry{DBMS}{
	name=\glslink{dbms}{DBMS},
	text=DBMS,
	sort=dbms,
	description={\acrlong{dbms}\\
	Sistema software progettato per consentire la creazione, la manipolazione e l'interrogazione efficiente di database. È ospitato su architettura hardware dedicata (server) oppure su semplice computer}
}

\newglossaryentry{NoSQL}{
	name=\glslink{nosql}{NoSQL},
	text=NoSQL,
	sort=nosql,
	description={\acrlong{nosql}\\
	Movimento che promuove sistemi software dove la persistenza dei dati è caratterizzata dal fatto di non utilizzare il modello relazionale, di solito usato dai database tradizionali. L'espressione NoSQL fa riferimento al linguaggio SQL, che è il più comune linguaggio di interrogazione dei dati nei database relazionali, qui preso a simbolo dell'intero paradigma relazionale. Questi archivi di dati il più delle volte non richiedono uno schema fisso (schemaless), evitano spesso le operazioni di giunzione (join) e puntano a scalare in modo orizzontale
	}
}

\newglossaryentry{REST}{
	name=\glslink{rest}{REST},
	text=REST,
	sort=rest,
	description={\acrlong{rest}\\
	Tipo di architettura software per i sistemi di ipertesto distribuiti come il World Wide Web. Si basa su tre principi:
	\begin{itemize}
	\item lo stato dell'applicazione e le funzionalità sono divisi in risorse web;
	\item ogni risorsa è unica e indirizzabile usando sintassi universale per uso nei link ipertestuali;
	\item tutte le risorse sono condivise come interfaccia uniforme per il trasferimento di stato tra client e risorse, questo consiste in:
		\begin{itemize}
		\item un insieme vincolato di operazioni ben definite;
		\item un insieme vincolato di contenuti, opzionalmente supportato da codice a richiesta;
		\item un protocollo:
			\begin{itemize}
			\item client-server;
			\item privo di stato (stateless)
			\item memorizzabile in cache (cacheable)
			\item a livelli
			\end{itemize}
		\end{itemize}
	\end{itemize}
	}
}

\newglossaryentry{URI}{
	name=\glslink{uri}{URI},
	text=URI,
	sort=uri,
	description={\acrlong{uri}\\
	Stringa che identifica univocamente una risorsa generica che può essere un indirizzo Web, un documento, un'immagine, un file, un servizio, un indirizzo di posta elettronica, eccetera}
}

\newglossaryentry{HTTP}{
	name=\glslink{http}{HTTP},
	text=HTTP,
	sort=http,
	description={\acrlong{http}\\
	Protocollo a livello applicativo usato come principale sistema per la trasmissione d'informazioni sul web, ovvero in un'architettura tipica client-server. Le specifiche del protocollo sono gestite dal World Wide Web Consortium (W3C)}
}

\newglossaryentry{resilienza}{
	name=Resilienza,
	text=resilienza,
	sort=resilienza,
	description={\mbox{}\\Capacità di un sistema di adattarsi alle condizioni d'uso e di resistere all'usura in modo da garantire la disponibilità dei servizi erogati}
}

\newglossaryentry{IDoT}{
	name=\glslink{idot}{IDoT},
	text=IDot,
	sort=idot,
	description={\acrlong{idot}\\
	Assegnazione di identificatori univoci e attributi specifici ad oggetti per consentire loro di comunicare e interagire con le altre entità attraverso Internet}
}

\newglossaryentry{cloud}{
	name=Cloud,
	text=cloud,
	sort=cloud,
	description={\mbox{}\\Paradigma di erogazione di risorse informatiche, come l'archiviazione, l'elaborazione o la trasmissione di dati, caratterizzato dalla disponibilità on demand attraverso Internet a partire da un insieme di risorse preesistenti e configurabili. Le risorse non vengono pienamente configurate e messe in opera dal fornitore apposta per l'utente, ma gli sono assegnate, rapidamente e convenientemente, grazie a procedure automatizzate, a partire da un insieme di risorse condivise con altri utenti lasciando all'utente parte dell'onere della configurazione}
}

\newglossaryentry{branching}{
	name=Branching,
	text=branching,
	sort=branching,
	description={\mbox{}\\Duplicazione di un oggetto soggetto a controllo di versionamento, in modo che le modifiche su tale oggetto possano procedere in parallelo su due (o più) ramificazioni diverse}
}

\newglossaryentry{merging}{
	name=Merging,
	text=merging,
	sort=merging,
	description={\mbox{}\\Fusione delle modifiche effettuate su un oggetto su due ramificazioni (branch) distinti. il risultato è un unico oggetto che contiene l'unione delle modifiche}
}

\newglossaryentry{repository}{
	name=Repository,
	text=repository,
	sort=repository,
	description={\mbox{}\\Ambiente di un sistema informativo, in cui vengono gestiti i metadati, attraverso tabelle relazionali},
	plural=repositories
}

\newglossaryentry{API}{
	name=\glslink{api}{API},
	text=API,
	sort=api,
	description={\acrlong{api}\\
	Insieme di procedure disponibili al programmatore, di solito raggruppate a formare un set di strumenti specifici per il completamento di un determinato compito all'interno di un certo programma}
}

\newglossaryentry{IDaaS}{
	name=\glslink{idaas}{IDaaS},
	text=idaas,
	sort=idaas,
	description={\acrlong{idaas}\\
	Identity and Access Management fornito as a Service, ovvero in modo cloud based}
}

\newglossaryentry{SAML}{
	name=\glslink{saml}{SAML},
	text=SAML,
	sort=saml,
	description={\acrlong{saml}\\
	Standard informatico per lo scambio di dati di autenticazione e autorizzazione (dette asserzioni) tra domini di sicurezza distinti, tipicamente un identity provider (entità che fornisce informazioni di identità) e un service provider (entità che fornisce servizi)}
}

\newglossaryentry{AD}{
	name=\glslink{ad}{AD},
	text=AD,
	sort=ad,
	description={\acrlong{ad}\\
	Servizio di directory sviluppato da Microsoft che consente di autenticare e autorizzare utenti e computer in reti di dominio Windows, assegnando policy di sicurezza e aggiornando i software}
}

\newglossaryentry{serviziodirectory}{
	name={Servizio di directory},
	text={servizio di directory},
	sort={servizio di directory},
	description={\mbox{}\\Programma (o insiemi di programmi) che provvede ad organizzare e memorizzare informazioni e a gestire risorse condivise all’interno di reti di computer, fornendo anche un controllo degli accessi sul loro utilizzo},
	plural={servizi di directory}
}

\newglossaryentry{AJAX}{
	name=\glslink{ajax}{AJAX},
	text=AJAX,
	sort=ajax,
	description={\acrlong{ajax}\\
	Tecnica di sviluppo software per la realizzazione di applicazioni web interattive. Lo sviluppo di applicazioni web con AJAX si basa su uno scambio di dati in background fra web browser e server, che consente l'aggiornamento dinamico di una pagina web senza esplicito ricaricamento da parte dell'utente. AJAX è asincrono nel senso che i dati extra sono richiesti al server e caricati in background senza interferire con il comportamento della pagina esistente. Normalmente le funzioni richiamate sono scritte con il linguaggio JavaScript. Tuttavia, e a dispetto del nome, l'uso di JavaScript e di XML non è obbligatorio, come non è detto che le richieste di caricamento debbano essere necessariamente asincrone}
}

\newglossaryentry{URL}{
	name=\glslink{url}{URL},
	text=URL,
	sort=url,
	description={\acrlong{url}\\
	Sequenza di caratteri che identifica univocamente l'indirizzo di una risorsa in Internet, tipicamente presente su un host server, come ad esempio un documento, un'immagine, un video, rendendola accessibile ad un client che ne faccia richiesta attraverso l'utilizzo di un web browser}
}

\newglossaryentry{onpremises}{
	name=On-premises,
	text=on-premises,
	sort=on-premises,
	description={\mbox{}\\Installazione ed esecuzione del software direttamente su macchina locale, sia essa aziendale che privata. L'approccio on-premises per la distribuzione/utilizzo del software è stato ritenuto la norma fino al 2005, data oltre la quale si è progressivamente ampliato l'utilizzo di software che esegue su computer remoti}
}

\newglossaryentry{phishing}{
	name=Phishing,
	text=phishing,
	sort=phishing,
	description={\mbox{}\\Tipo di truffa effettuata su Internet attraverso la quale un malintenzionato cerca di ingannare la vittima convincendola a fornire informazioni personali, dati finanziari o codici di accesso, fingendosi un ente affidabile in una comunicazione digitale}
}

\newglossaryentry{MAC}{
	name=\glslink{mac}{MAC},
	text=MAC,
	sort=mac,
	description={\acrlong{mac}\\
	Tipo di controllo d'accesso attraverso il quale il sistema operativo vincola la capacità di un soggetto di eseguire diverse operazioni su un oggetto o un obiettivo}
}

\newglossaryentry{DAC}{
	name=\glslink{dac}{DAC},
	text=DAC,
	sort=dac,
	description={\acrlong{dac}\\
	Meccanismo di controllo degli accessi alle risorse messe a disposizione da un sistema informatico definito dalla Trusted Computer System Evaluation Criteria, nel quale i soggetti possiedono l'ownership degli oggetti da loro creati e possono concedere o revocare a loro discrezione alcuni privilegi ad altri soggetti. In un sistema informatico nel quale è realizzata una politica di controllo degli accessi di tipo DAC, l'autorizzazione degli accessi alle risorse del sistema è basata sull'identità degli utenti e/o sul gruppo utenti di appartenenza
	}
}

\newglossaryentry{RBAC}{
	name=\glslink{rbac}{RBAC},
	text=RBAC,
	sort=rbac,
	description={\acrlong{rbac}\\
	Approccio a sistemi ad accesso ristretto per utenti autorizzati. Si basa su tre regole fondamentali:
	\begin{itemize}
	\item assegnazione dei ruoli;
	\item autorizzazione dei ruoli;
	\item autorizzazione alla transazione.
	\end{itemize}
	}
}

\newglossaryentry{IdP}{
	name=\glslink{idp}{IdP},
	text=Identity Provider,
	sort=idp,
	description={\acrlong{idp}\\
	Entità responsabile di:
	\begin{itemize}
	\item fornire un'identità agli utenti che interagiscono con un sistema;
	\item assicurare che l'utente sia chi dice di essere;
	\item fornire ulteriori informazioni sull'utente.
	\end{itemize}
	}
}

\newglossaryentry{SP}{
	name=\glslink{sp}{SP},
	text=Service Provider,
	sort=sp,
	description={\acrlong{sp}\\
	Organizzazione che fornisce agli utenti servizi di vario tipo}
}

\newglossaryentry{XML}{
	name=\glslink{xml}{XML},
	text=XML,
	sort=xml,
	description={\acrlong{xml}\\
	Metalinguaggio per la definizione di linguaggi di markup, ovvero un linguaggio basato su un meccanismo sintattico che consente di definire e controllare il significato degli elementi contenuti in un documento o in un testo}
}

\newglossaryentry{SaaS}{
	name=\glslink{saas}{SaaS},
	text=saas,
	sort=saas,
	description={\acrlong{saas}\\
	Modello di distribuzione del software applicativo dove un produttore di software sviluppa, opera (direttamente o tramite terze parti) e gestisce un'applicazione web che mette a disposizione dei propri clienti via Internet. I clienti non pagano per il possesso del software bensì per l'utilizzo dello stesso}
}

\newglossaryentry{TCO}{
	name=\glslink{tco}{TCO},
	text=TCO,
	sort=tco,
	description={\acrlong{tco}\\
	Approccio sviluppato da Gartner nel 1987, utilizzato per calcolare tutti i costi del ciclo di vita di un'apparecchiatura informatica IT, per l'acquisto, l'installazione, la gestione, la manutenzione e il suo smaltimento}
}

\newglossaryentry{IGA}{
	name=\glslink{iga}{IGA},
	text=IGA,
	sort=iga,
	description={\acrlong{iga}\\
	I prodotti IGA consentono ai responsabili di amministrare e gestire i ruoli, le licenze e i privilegi degli utenti nell'azienda estesa e anche nel cloud. Queste soluzioni consentono alle organizzazioni di evitare le violazioni relative alla suddivisioni degli incarichi, supportare le politiche di business ed eliminare gli accessi inappropriati. Controllando e verificando l'attività degli utenti e rafforzando il controllo degli accessi, le organizzazioni possono ottenere una governance più efficace, prevenire le minacce alle informazioni riservate e la frode di identità}
}

\newglossaryentry{UML}{
	name=\glslink{uml}{UML},
	text=UML,
	sort=uml,
	description={\acrlong{uml}\\
	Linguaggio di modellazione basato sul paradigma orientato agli oggetti. Svolge un'importantissima funzione di "lingua franca" nella comunità della progettazione e programmazione a oggetti. Gran parte della letteratura di settore usa UML per descrivere soluzioni analitiche e progettuali in modo sintetico e comprensibile a un vasto pubblico}
}

\newglossaryentry{bug}{
	name=Bug,
	text=bug,
	sort=bug,
	description={\mbox{}\\Errore nella scrittura del codice sorgente di un programma software},
	plural={bugs}
}

\newglossaryentry{failure}{
	name=Failure,
	text=failure,
	sort=failure,
	description={\mbox{}\\Effetto indesiderato visibile durante l'esecuzione del prodotto},
	plural={failures}
}

\newglossaryentry{fault}{
	name=Fault,
	text=fault,
	sort=fault,
	description={\mbox{}\\Causa del malfunzionamento del prodotto. Di solito è un errore nella scrittura del codice sorgente ed è chiamato comunemente bug},
	plural={faults}
}

\newglossaryentry{softdeletion}{
	name={Soft deletion},
	text={soft deletion},
	sort={soft deletion},
	description={\mbox{}\\Cancellazione unicamente ''logica'' di informazioni da un database. Solitamente viene effettuata impostando a true uno specifico flag, ad esempio (is\_deleted)}
}

\newglossaryentry{JWT}{
	name=\glslink{jwt}{JWT},
	text=JWT,
	sort=jwt,
	description={\acrlong{jwt}\\
	Stringa che rappresenta un insieme di affermazioni come un oggetto JSON, in modo da poterle firmare o criptare digitalmente}
}

\newglossaryentry{JSON}{
	name=\glslink{json}{JSON},
	text=JSON,
	sort=json,
	description={\acrlong{json}\\
	Semplice formato completamente indipendente dal linguaggio di programmazione per lo scambio di dati. JSON è basato su due strutture:
	\begin{itemize}
	\item un insieme di coppie nome/valore;
	\item un elenco ordinato di valori
	\end{itemize}
	}
}

\newglossaryentry{HMAC}{
	name=\glslink{hmac}{HMAC},
	text=HMAC,
	sort=hmac,
	description={\acrlong{hmac}\\
	Modalità per l'autenticazione di messaggi basata su una funzione di hash, utilizzata in diverse applicazioni legate alla sicurezza informatica. Tramite HMAC è possibile garantire sia l'integrità che l'autenticità di un messaggio: utilizza una combinazione del messaggio originale e una chiave segreta per la generazione del codice. Una caratteristica peculiare di HMAC è il non essere legato a nessuna funzione di hash in particolare, per rendere possibile una sostituzione della funzione nel caso non fosse abbastanza sicura. Nonostante ciò le funzioni più utilizzate sono MD5 (attualmente considerata poco sicura) e SHA-2
	}
}

\newglossaryentry{SHA}{
	name=\glslink{sha}{SHA},
	text=SHA,
	sort=sha,
	description={\acrlong{sha}\\
	Famiglia di cinque diverse funzioni crittografiche di hash sviluppate a partire dal 1993 dalla National Security Agency (NSA) e pubblicate come standard federale dal governo degli USA. Il SHA produce un message digest, o "impronta del messaggio", di lunghezza fissa partendo da un messaggio di lunghezza variabile. La sicurezza di un algoritmo di hash risiede nel fatto che la funzione non sia reversibile (non sia cioè possibile risalire al messaggio originale conoscendo solo questo dato) e che non deve essere mai possibile creare intenzionalmente due messaggi diversi con lo stesso digest. Gli algoritmi della famiglia sono denominati SHA-1, SHA-224, SHA-256, SHA-384 e SHA-512: le ultime 4 varianti sono spesso indicate genericamente come SHA-2, per distinguerle dal primo. Il primo produce un digest del messaggio di soli 160 bit, mentre gli altri producono digest di lunghezza in bit pari al numero indicato nella loro sigla}
}

\newglossaryentry{rsa}{
	name=RSA,
	text=RSA,
	sort=rsa,
	description={\mbox{}\\Algoritmo di crittografia asimmetrica, inventato nel 1977 da Ronald \textbf{R}ivest, Adi \textbf{S}hamir e Leonard \textbf{A}dleman utilizzabile per cifrare o firmare informazioni}
}

\newglossaryentry{base64}{
	name=Base64,
	text=Base64,
	sort=base64,
	description={\mbox{}\\È un sistema di numerazione posizionale che usa 64 simboli che	viene usato principalmente come codifica di dati binari nelle email. L'algoritmo che effettua la conversione suddivide il file in gruppi da 6 bit, i quali possono quindi contenere valori da 0 a 63. Ogni possibile valore viene convertito in un carattere ASCII. L'algoritmo causa un aumento delle dimensioni dei dati del 33\%, poiché ogni gruppo di 3 byte viene convertito in 4 caratteri. Questo supponendo che per rappresentare un carattere si utilizzi un intero byte}
}

\newglossaryentry{IANA}{
	name=\glslink{iana}{IANA},
	text=IANA,
	sort=iana,
	description={\acrlong{iana}\\
	Organismo che ha responsabilità nell'assegnazione degli indirizzi IP}
}

\newglossaryentry{cookie}{
	name=Cookie,
	text=cookie,
	sort=cookie,
	description={\mbox{}\\Un cookie è simile ad un piccolo file, memorizzato nel computer da siti web durante la navigazione, utile a salvare le preferenze e a migliorare le prestazioni dei siti web. In questo modo si ottimizza l'esperienza di navigazione da parte dell'utente. Nel dettaglio, un cookie è una stringa di testo di piccola dimensione inviata da un web server ad un web client (di solito un browser) e poi rimandati indietro dal client al server (senza subire modifiche) ogni volta che il client accede alla stessa porzione dello stesso dominio web}
}

\newglossaryentry{CORS}{
	name=\glslink{cors}{CORS},
	text=CORS,
	sort=cors,
	description={\acrlong{cors}\\
	Meccanismo che permette alle risorse sul web di essere richieste da domini differenti da quello di appartenenza della risorsa stessa
	}
}

\newglossaryentry{XSS}{
	name=\glslink{xss}{XSS},
	text=XSS,
	sort=xss,
	description={\acrlong{xss}\\
	Vulnerabilità che affligge siti web dinamici che impiegano un insufficiente controllo dell'input nei form. Un XSS permette di inserire o eseguire codice lato client al fine di attuare un insieme variegato di attacchi. Nell'accezione odierna, la tecnica ricomprende l'utilizzo di qualsiasi linguaggio di scripting lato client tra i quali JavaScript. Secondo un rapporto di Symantec nel 2007 l'80\% di tutte le violazioni è dovuto ad attacchi XSS}
}

\newglossaryentry{DRY}{
	name=\glslink{dry}{DRY},
	text=DRY,
	sort=dry,
	description={\acrlong{dry}\\
	Principio di progettazione e sviluppo secondo cui andrebbe evitata ogni forma di ripetizione e ridondanza logica nell'implementazione di un sistema software. Il principio venne inizialmente enunciato da Andy Hunt e Dave Thomas nel loro libro The Pragmatic Programmer:
	\begin{center}''\textit{Every piece of knowledge must have a single, unambiguous, authoritative representation within a system.}''\end{center}
	Il DRY viene spesso citato in relazione alla duplicazione del codice, ovvero nell'accezione stretta secondo cui il software non dovrebbe contenere sequenze di istruzioni uguali fra loro. Si tratta però di un concetto più ampio, che si applica a ogni parte di un sistema software}
}

\newglossaryentry{CSV}{
	name=\glslink{csv}{CSV},
	text=CSV,
	sort=csv,
	description={\acrlong{csv}\\
	Formato di file basato su file di testo utilizzato per l'importazione ed esportazione (ad esempio da fogli elettronici o database) di una tabella di dati. Non esiste uno standard formale che lo definisca, ma solo alcune prassi più o meno consolidate}
}

\newglossaryentry{ABAC}{
	name=\glslink{abac}{ABAC},
	text=ABAC,
	sort=abac,
	description={\acrlong{abac}\\
	Meccanismo di controllo degli accessi nel quale le richieste sono concesse o negate sulla base di un insieme attributi dell'utente, della risorsa, sulle condizioni del sistema e su un insieme di politiche specificate in funzione di questi attributi}
} % database di termini
\makeglossaries

\renewcommand*{\glstextformat}[1]{\textcolor{black}{#1}} %toglie il colore blu dai link per il glossario
%\defglsentryfmt{\color{black}\glsgenentryfmt}

\newcommand{\glossaryItem}[1]{\textit{\gls{#1}\ped{\ped{G}}}} % termini da glossario
\newcommand{\glossaryItemPl}[1]{\textit{\glspl{#1}\ped{\ped{G}}}} % termini da glossario


%**************************************************************
% Impostazioni di graphicx
%**************************************************************
\graphicspath{{immagini/}} % cartella dove sono riposte le immagini


%**************************************************************
% Impostazioni di hyperref
%**************************************************************
\hypersetup{
    %hyperfootnotes=false,
    %pdfpagelabels,
    %draft,	% = elimina tutti i link (utile per stampe in bianco e nero)
    colorlinks=false,
    linktocpage=true,
    pdfstartpage=1,
    pdfstartview=FitV,
    % decommenta la riga seguente per avere link in nero (per esempio per la stampa in bianco e nero)
    %colorlinks=false, linktocpage=false, pdfborder={0 0 0}, pdfstartpage=1, pdfstartview=FitV,
    breaklinks=true,
    pdfpagemode=UseNone,
    pageanchor=true,
    pdfpagemode=UseOutlines,
    plainpages=false,
    bookmarksnumbered,
    bookmarksopen=true,
    bookmarksopenlevel=1,
    hypertexnames=true,
    pdfhighlight=/O,
    %nesting=true,
    %frenchlinks,
    urlcolor=webbrown,
    linkcolor=RoyalBlue,
    citecolor=RoyalBlue,
    %pagecolor=RoyalBlue,
    %urlcolor=Black, linkcolor=Black, citecolor=Black, %pagecolor=Black,
    pdftitle={\myTitle},
    pdfauthor={\textcopyright\ \myName, \myUni, \myFaculty},
    pdfsubject={},
    pdfkeywords={},
    pdfcreator={pdfLaTeX},
    pdfproducer={LaTeX}
}

%**************************************************************
% Impostazioni di itemize
%**************************************************************
\renewcommand{\labelitemi}{$\bullet$}
\renewcommand{\labelitemii}{$\circ$}
%\renewcommand{\labelitemiii}{$\diamond$}
%\renewcommand{\labelitemiv}{$\ast$}


%**************************************************************
% Impostazioni di listings
%**************************************************************
\lstset{
    language=[LaTeX]Tex,%C++,
    keywordstyle=\color{RoyalBlue}, %\bfseries,
    basicstyle=\small\ttfamily,
    %identifierstyle=\color{NavyBlue},
    commentstyle=\color{Green}\ttfamily,
    stringstyle=\rmfamily,
    numbers=none, %left,%
    numberstyle=\scriptsize, %\tiny
    stepnumber=5,
    numbersep=8pt,
    showstringspaces=false,
    breaklines=true,
    frameround=ftff,
    frame=single
} 

% congigurazione dei comandi per mostrare i nomi nel documento
\renewcommand{\lstlistingname}{Frammento di codice} % nome nel caption
\renewcommand{\lstlistlistingname}{Elenco dei frammenti di codice} %nome nell'indice

% configurazione listati JSON
\colorlet{punct}{red!60!black}
\definecolor{background}{HTML}{FFFFFF}
\definecolor{delim}{RGB}{20,105,176}
\colorlet{numb}{magenta!60!black}

\lstdefinelanguage{json}{
    basicstyle=\normalfont\ttfamily,
    numbers=left,
    numberstyle=\scriptsize,
    stepnumber=1,
    numbersep=8pt,
    showstringspaces=false,
    breaklines=true,
    frame=lines,
    comment=[l]{//},
    backgroundcolor=\color{background},
    keywords=[2]{typ, alg, id, email, iat, exp, sub, name, admin, status, message, error\_code, APP\_CREATED, PRIVATE\_APP\_CREATED, APP\_REMOVED, PRIVATE\_APP\_REMOVED, ACCESS\_ALLOWED, ACCESS\_DENIED, \_id, code, user, created\_at, infos, key, value, element\_id, date, application, group, domain, browser, errored},
    keywordstyle=[2]{\color{WildStrawberry}},
    keywords=[3]{Array, String, Number, Date, ObjectId},
    keywordstyle=[3]{\color{Black}},
    literate=
     *{:}{{{\color{punct}{:}}}}{1}
      {,}{{{\color{punct}{,}}}}{1}
      {\{}{{{\color{delim}{\{}}}}{1}
      {\}}{{{\color{delim}{\}}}}}{1}
      {[}{{{\color{delim}{[}}}}{1}
      {]}{{{\color{delim}{]}}}}{1},
}

% configurazione listati JavaScript
\definecolor{lightgray}{rgb}{.9,.9,.9}
\definecolor{darkgray}{rgb}{.4,.4,.4}
\definecolor{purple}{rgb}{0.65, 0.12, 0.82}

\lstdefinelanguage{JavaScript}{
  keywords={typeof, new, true, false, catch, function, return, null, catch, switch, var, if, in, while, do, else, case, break},
  keywordstyle=\color{blue}\bfseries,
  ndkeywords={class, export, boolean, throw, implements, import, this},
  ndkeywordstyle=\color{darkgray}\bfseries,
  identifierstyle=\color{purple},
  sensitive=false,
  comment=[l]{//},
  morecomment=[s]{/*}{*/},
 % commentstyle=\color{lighgray}\ttfamily,
  stringstyle=\color{red}\ttfamily,
  morestring=[b]',
  morestring=[b]"
}

% configurazione listato bearer
\lstdefinelanguage{bearer}{
basicstyle=\small\ttfamily,
columns=fullflexible,
keywords=[2]{Bearer, token},
keywordstyle=[2]{\color{WildStrawberry}},
literate=
     *{<}{{{\color{WildStrawberry}{<}}}}{1}
      {>}{{{\color{WildStrawberry}{>}}}}{1}
}

% configurazione listato signature JWT
\lstdefinelanguage{signature}{
basicstyle=\small\ttfamily,
columns=fullflexible,
keywords=[2]{HMACSHA256},
keywords=[3]{base64UrlEncode},
keywordstyle=[2]{\color{Bittersweet}},
keywordstyle=[3]{\color{Aquamarine}},
literate=
     *{+}{{{\color{WildStrawberry}{+}}}}{1}
}


%**************************************************************
% Impostazioni di xcolor
%**************************************************************
\definecolor{webgreen}{rgb}{0,.5,0}
\definecolor{webbrown}{rgb}{.6,0,0}


%**************************************************************
% Altro
%**************************************************************

\newcommand{\omissis}{[\dots\negthinspace]} % produce [...]

% eccezioni all'algoritmo di sillabazione
\hyphenation
{
    ma-cro-istru-zio-ne
}

% nomi per le traduzioni italiane
\newcommand{\sectionname}{Sezione}
\addto\captionsitalian{\renewcommand{\figurename}{Figura}
                       \renewcommand{\tablename}{Tabella}}


\newcommand{\intro}[1]{\emph{\textsf{#1}}}

%**************************************************************
% Environment per ``rischi''
%**************************************************************
\newcounter{riskcounter}                % define a counter
\setcounter{riskcounter}{0}             % set the counter to some initial value

%%%% Parameters
% #1: Title
\newenvironment{risk}[1]{
    \refstepcounter{riskcounter}        % increment counter
    \par \noindent                      % start new paragraph
    \textbf{\arabic{riskcounter}. #1}   % display the title before the 
                                        % content of the environment is displayed 
}{
    \par\medskip
}

\newcommand{\riskname}{Rischio}

\newcommand{\riskdescription}[1]{\textbf{\\Descrizione:} #1.}

\newcommand{\risksolution}[1]{\textbf{\\Soluzione:} #1.}

%**************************************************************
% Environment per ``use case''
%**************************************************************
\newcounter{usecasecounter}             % define a counter
\setcounter{usecasecounter}{0}          % set the counter to some initial value

%%%% Parameters
% #1: ID
% #2: Nome
\newenvironment{usecase}[2]{
    \renewcommand{\theusecasecounter}{\usecasename #1}  % this is where the display of 
                                                        % the counter is overwritten/modified
    \refstepcounter{usecasecounter}             % increment counter
    \vspace{10pt}
    \par \noindent                              % start new paragraph
    {\large \textbf{\usecasename #1: #2}}       % display the title before the 
                                                % content of the environment is displayed 
    \medskip
}{
    \medskip
}

\newcommand{\usecasename}{UC}

\newcommand{\usecaseactors}[1]{\textbf{\\Attori Principali:} #1. \vspace{4pt}}
\newcommand{\usecasepre}[1]{\textbf{\\Precondizioni:} #1. \vspace{4pt}}
\newcommand{\usecasedesc}[1]{\textbf{\\Descrizione:} #1. \vspace{4pt}}
\newcommand{\usecasepost}[1]{\textbf{\\Postcondizioni:} #1. \vspace{4pt}}
\newcommand{\usecasealt}[1]{\textbf{\\Scenario Alternativo:} #1. \vspace{4pt}}

%**************************************************************
% Environment per ``namespace description''
%**************************************************************
\newenvironment{namespacedesc}{
    \vspace{10pt}
    \par \noindent                              % start new paragraph
    \begin{description} 
}{
    \end{description}
    \medskip
}

\newcommand{\classdesc}[2]{\item[\textbf{#1:}] #2}                     % file con le impostazioni personali

\begin{document}
%**************************************************************
% Materiale iniziale
%**************************************************************
\frontmatter
% !TEX encoding = UTF-8
% !TEX TS-program = pdflatex
% !TEX root = ../tesi.tex
% !TEX spellcheck = it-IT

%**************************************************************
% Frontespizio 
%**************************************************************
\begin{titlepage}

\begin{center}

\begin{LARGE}
\textbf{\myUni}\\
\end{LARGE}

\vspace{10pt}

\begin{Large}
\textsc{\myDepartment}\\
\end{Large}

\vspace{10pt}

\begin{large}
\textsc{\myFaculty}\\
\end{large}

\vspace{30pt}
\begin{figure}[htbp]
\begin{center}
\includegraphics[height=6cm]{logo-unipd}
\end{center}
\end{figure}
\vspace{30pt} 

\begin{LARGE}
\begin{center}
\textbf{\myTitle}\\
\end{center}
\end{LARGE}

\vspace{10pt} 

\begin{large}
\textsl{\myDegree}\\
\end{large}

\vspace{20pt} 

\begin{large}
\begin{flushleft}
\textit{Relatore}\\ 
\vspace{5pt} 
Prof. \myProf
\end{flushleft}

\vspace{0pt} 

\begin{flushright}
\textit{Laureando}\\ 
\vspace{5pt} 
\myName
\end{flushright}
\end{large}

\vspace{40pt}

\line(1, 0){338} \\
\begin{normalsize}
\textsc{Anno Accademico \myAA}
\end{normalsize}

\end{center}
\end{titlepage} 
% !TEX encoding = UTF-8
% !TEX TS-program = pdflatex
% !TEX root = ../tesi.tex
% !TEX spellcheck = it-IT

%**************************************************************
% Colophon
%**************************************************************
\clearpage
\phantomsection
\thispagestyle{empty}

\hfill

\vfill

\begin{center}
\noindent\myName: 

\textit{\myTitle,}


\myDegree,
\textcopyright\ \myTime
\end{center}
% !TEX encoding = UTF-8
% !TEX TS-program = pdflatex
% !TEX root = ../tesi.tex
% !TEX spellcheck = it-IT

%**************************************************************
% Dedica
%**************************************************************
\cleardoublepage
\phantomsection
\thispagestyle{empty}
\pdfbookmark{Dedica}{Dedica}

\topskip0pt
\vspace*{\fill}
\begin{center}
\textit{Dedicato alla mia famiglia}
\end{center}
\vspace*{\fill}

% !TEX encoding = UTF-8
% !TEX TS-program = pdflatex
% !TEX root = ../tesi.tex
% !TEX spellcheck = it-IT

%**************************************************************
% Sommario
%**************************************************************
\cleardoublepage
\phantomsection
\pdfbookmark{Abstract}{Abstract}
\begingroup
\let\clearpage\relax
\let\cleardoublepage\relax
\let\cleardoublepage\relax

\chapter*{Abstract}

Il presente documento riassume il lavoro svolto durante il periodo di stage, della durata di 320 ore, presso l'azienda iVoxIT S.r.l. di Padova. \\
Lo scopo principale del prodotto sviluppato è quello di semplificare ed automatizzare la gestione del catalogo applicativo di Monokee, un sistema di Identity as a Service (IDaaS). Nel corso del documento verranno anche esposte le basi teoriche del prodotto, come la gestione delle identità e il Single Sign-On (SSO). 

In una prima fase mi sono concentrato sull'apprendimento degli strumenti e delle tecnologie da utilizzare e preso confidenza con le tematiche della gestione delle identità, dell'autenticazione e dell'autorizzazione. Dopo aver identificato le principali funzionalità richieste e compreso il funzionamento di Monokee e dell'Identity and Access Management (IAM) ho iniziato la progettazione dell'applicazione. La definizione di un'architettura che fosse estendibile, manutenibile e integrabile con quella esistente ha richiesto molti sforzi, ma il risultato è stato all'altezza della aspettative dell'azienda e ha soddisfatto tutte i requisiti individuati.
%\vfill
%
%\selectlanguage{english}
%\pdfbookmark{Abstract}{Abstract}
%\chapter*{Abstract}
%
%\selectlanguage{italian}

\endgroup			

\vfill


% !TEX encoding = UTF-8
% !TEX TS-program = pdflatex
% !TEX root = ../tesi.tex
% !TEX spellcheck = it-IT

%**************************************************************
% Ringraziamenti
%**************************************************************
\cleardoublepage
\phantomsection
\pdfbookmark{Ringraziamenti}{ringraziamenti}

\begin{small}
\begin{flushright}{
	\slshape    
	``The modern world needs people with a complex identity who are intellectually autonomous and prepared to cope with uncertainty; who are able to tolerate ambiguity and not be driven by fear into a rigid, single-solution approach to problems, who are rational, foresightful and who look for facts; who can draw inferences and can control their behavior in the light of foreseen consequences, who are altruistic and enjoy doing for others, and who understand social forces and trends.''} \\ 
	\medskip
    --- Robert Havighurst
\end{flushright}
\end{small}

\begingroup
\let\clearpage\relax
\let\cleardoublepage\relax
\let\cleardoublepage\relax

\chapter*{Ringraziamenti}

\noindent \textit{Innanzitutto voglio ringraziare i miei genitori, Daniela e Paolo. Grazie per essermi stati vicini in questi lunghi anni di studio e per avermi lasciato intraprendere la mia strada, anche nelle occasioni in cui questa non ha seguito i vostri pensieri e desideri. Grazie per i vostri consigli, per le vostre critiche, per i vostri rimproveri, per aver condiviso con me sia i momenti di gioia che quelli di difficoltà e per avermi insegnato a rialzarmi quando le cose non vanno come vorrei. Questo primo, grande, traguardo è dedicato a voi.}\\

\noindent \textit{Grazie ad Elena, la mia fidanzata, per avermi sopportato tutti questi anni e per aver condiviso con me gioie, dolori, stress e felicità. Grazie per avermi motivato ed incitato tanto nei momenti felici quanto in quelli più difficili. Grazie per aver sempre creduto in me e nelle mie capacità e per avermi sempre dato il tuo totale ed incondizionato supporto. Se oggi sono come sono, è anche merito tuo.}\\

\noindent \textit{Grazie a Roberto, il mio tutor aziendale, per avermi permesso di partecipare a Monokee, un prodotto che ora sento un po' anche mio. Grazie ad Enrico, Maria Giovanna, Sara, Silvia e Valentina, per avermi accolto nel loro gruppo e per avermi fatto lavorare in un ambiente fantastico. Mi avete insegnato che c'è solo un modo per raggiungere grandi traguardi: è farlo insieme. Senza di voi sarebbe stato sicuramente tutto più difficile. Lavorare nel vostro team è stato un onore, prima che un piacere. Grazie a Fabio, Federica, Silvia, Simone e a tutto lo staff di Athesys S.r.l. per avermi permesso di vivere questa bellissima esperienza e per avermi coinvolto dal primo fino all'ultimo giorno, e oltre.}\\

\noindent \textit{Grazie a tutti i miei amici e a tutte le persone che mi sono state vicine in questi anni.}\\

\noindent \textit{Grazie, infine, al Prof. Mauro Conti, relatore della mia tesi, per le sue indicazioni e l'aiuto che mi ha fornito durante lo stage e la stesura del documento.}\\

\noindent\textit{\myLocation, \myTime}
\hfill \myName

\endgroup


% !TEX encoding = UTF-8
% !TEX TS-program = pdflatex
% !TEX root = ../tesi.tex
% !TEX spellcheck = it-IT

%**************************************************************
% Indici
%**************************************************************
\cleardoublepage
\pdfbookmark{\contentsname}{tableofcontents}
\setcounter{tocdepth}{3} 		% profondità dell'indice (così prende anche subsubsection)
\addtocounter{secnumdepth}{1} 	% enumera subsubsection
\tableofcontents
%\markboth{\contentsname}{\contentsname} 
\clearpage

\begingroup 
    \let\clearpage\relax
    \let\cleardoublepage\relax
    \let\cleardoublepage\relax
    %*******************************************************
    % Elenco delle figure
    %*******************************************************    
    \phantomsection
    \pdfbookmark{\listfigurename}{lof}
    \listoffigures

    \vspace*{8ex}

    %*******************************************************
    % Elenco delle tabelle
    %*******************************************************
    \phantomsection
    \pdfbookmark{\listtablename}{lot}
    \listoftables
    
    %*******************************************************
        % Elenco dei frammenti di codice
        %*******************************************************
        \phantomsection
        \pdfbookmark{\lstlistlistingname}{lol}
        \lstlistoflistings
        
    \vspace*{8ex}
\endgroup

\clearpage


%**************************************************************
% Materiale principale
%**************************************************************
\mainmatter
\nocite{*}
\chapter{Introduzione}
\textit{In questo primo capitolo verrà esposta l'organizzazione dell'elaborato e verranno spiegate le convenzioni tipografiche adottate.}

\section{Organizzazione dell'elaborato}
\hyperref[azienda]{Il secondo capitolo} esporrà il contesto aziendale nel quale si è svolta l'attività di stage. \\ \\
\hyperref[identita]{Il terzo capitolo} illustrerà la realtà attuale dei sistemi di \glossaryItem{iam}, le possibili evoluzioni future e gli ambiti di utilizzo. \\ \\
\hyperref[progetto]{Il quarto capitolo} esporrà il prodotto per il quale si è svolta l'attività di stage, Monokee. Verrà inoltre esposto il ruolo del progetto svolto. \\ \\
\hyperref[tecnologie]{Il quinto capitolo} esporrà le tecnologie, gli strumenti e i linguaggi utilizzati nel corso dell'attività di stage. \\ \\
\hyperref[adr]{Il sesto capitolo} presenterà le attività di analisi dei requisiti e l'individuazione delle principali funzionalità da implementare. \\ \\
\hyperref[progettazione]{Il settimo capitolo} presenterà ad alto livello la progettazione effettuata e le principali scelte progettuali implementate. \\ \\
\hyperref[implementazione]{L'ottavo capitolo} illustrerà i risultati dell'attività di implementazione di quanto progettato. \\ \\
\hyperref[vev]{Il nono capitolo} esporrà le attività di verifica e validazione. \\ \\
\hyperref[conclusioni]{Il decimo capitolo} riassumerà alcune considerazioni finali, relativamente al prodotto implementato, all'attività di stage nel suo complesso ed all'intero percorso di studi dello studente.

\section{Convenzioni adottate}
Nella stesura del presente documento sono state adottate le seguenti convenzioni tipografiche:
\begin{itemize}
\item ogni occorrenza di termini tecnici, ambigui o di acronimi verrà marcata in corsivo e con una G a pedice. Questo indica che quel termine è presente nel glossario in fondo al documento;
\item la prima occorrenza di un acronimo viene riportata con la dicitura estesa, in corsivo, e le lettere che compongono l'acronimo vengono riportate in grassetto. Ogni occorrenza successiva presenterà solo la forma ridotta;
\item le parole chiave presenti in ciascun paragrafo saranno marcate in grassetto, in modo da consentirne una facile individuazione;
\item ogni termine corrispondente a nomi di file, componenti dell'architettura o codice verrà marcato utilizzando un \textit{font} non proporzionale (a larghezza fissa);
\item i termini in lingua inglese non presenti nel glossario, e non marcati in grassetto, sono evidenziati semplicemente in corsivo, senza la G a pedice.
\end{itemize}
Inoltre, per ciascun diagramma \glossaryItem{uml} è utilizzato lo standard 2.0. 
\begin{tframe}{L'azienda}
\begin{itemize}
\item Azienda: iVoxIT S.R.L.
\item Ambito: Identity and Access Management
\item Prodotto: Monokee
\end{itemize}

\mbox{} \\

\begin{figure}[h]
\centering
\includegraphics[scale=0.25]{iVoxIT}
\end{figure}

\end{tframe}
\chapter{L'ambito: l'identità} \label{identita}

\section{Identity and Access Management - IAM}
\glsreset{iam}
La capacità di fornire il giusto accesso ai giusti utenti alle giuste risorse nei tempi e nelle modalità giuste è la chiave per aumentare la produttività e per mitigare rischi di qualsiasi tipo. Per questo la gestione dell'\glossaryItem{identita} è una parte fondamentale nella sicurezza del mondo \glossaryItem{it}.

Storicamente l'approccio a questa disciplina è stato molto semplice: tutto si basava su \textbf{qualcosa che l'utente conosceva}, tipicamente una \textit{password}. Per applicazioni a rischio più alto sono stati poi introdotti ulteriori controlli, come l'\glossaryItem{autenticazione} a più fattori (\textbf{qualcosa che l'utente possiede}, come \textit{smartphone}, carte speciali, chiavette, eccetera) o controlli biometrici (\textbf{qualcosa che l'utente è}, come scansione della retina o rilevamento delle impronte digitali). 

Questi sistemi sono ottimi se si deve gestire solamente l'\glossaryItem{identita} di una persona, ma l'avvento del \textit{mobile} ha cambiato tutto. Inoltre, in seguito alla nascita dell'\glossaryItem{iot}, anche qualsiasi dispositivo può connettersi ad Internet: non più solo utenti o \textit{smartphone}, ma anche sensori, elettrodomestici, e altro. 

In un contesto come questo è necessario essere in grado di gestire agevolmente le \glossaryItem{identita} e gli accessi alle risorse. Ecco perché nascono sistemi di \glossaryItem{iam}. Nelle successive sezioni saranno approfondite le due componenti fondamentali di un sistema di \glossaryItem{iam}: la gestione dell'\glossaryItem{identita} (\glossaryItem{idm}) e la gestione degli accessi (\glossaryItem{am}).

\subsection{Identity Management - IdM}
La funzionalità principale dell'\glossaryItem{IdM} è fornire \glossaryItem{identita} creando \textit{account} ed eventualmente disabilitando o cancellando quelli creati quando non sono più necessari (Figura~\ref{fig:identitylifecicle}). Tipicamente questo è un compito complicato, perché le \glossaryItem{identita} richieste sono molto eterogenee: impiegati, clienti, fornitori, \textit{partner}, eccetera. È pertanto necessario un fornitore di \glossaryItem{identita} unico per ogni tipologia di \textit{account} che automatizzi l'intero processo. Ogni tipologia deve, inoltre, avere degli attributi ben definiti (ad esempio il periodo di validità).

Solitamente ad ogni \textit{account} è associato un identificatore univoco (ID), in modo da associare un'\glossaryItem{identita} all'\textit{account} stesso e tracciare le operazioni effettuate a scopo di \textit{log} o \textit{report}. L'ID può essere rappresentato da un indirizzo email, oppure essere generato \textit{ad hoc}, ma è fondamentale che la sua unicità sia mantenuta: questo comporta la memorizzazione dell'ID stesso in caso di cancellazione, in modo da evitare la possibilità di inconsistenze.

\begin{figure}[hbpc]
\begin{center}
\includegraphics[scale=0.2]{identityLifecycle}
\caption[Ciclo di vita dell'identità]{Ciclo di vita dell'identità\protect\footnotemark}
\label{fig:identitylifecicle}
\end{center}
\end{figure}

Scegliere tra disabilitazione e cancellazione di un \textit{account} è fondamentale: la cancellazione è, ovviamente, più rischiosa, soprattutto se fatta in modo errato, ma consente il riutilizzo dell'\glossaryItem{identita} cancellata. La disabilitazione, invece, consente il \glossaryItem{rollback}, ma comporta un maggior numero di dati memorizzati. Generalmente gli \textit{account} vengono disabilitati quando l'utente associato non è presente (ad esempio per periodi di malattia) e riabilitati al rientro, oppure cancellati dopo un certo numero di giorni di inattività (\textbf{use it or lose it}). 

Le politiche di creazione, disabilitazione e cancellazione, comunque, variano da un'organizzazione all'altra e, in generale, si integrano con i sistemi di gestione del personale.
\footnotetext{Fonte: \url{https://www.linkedin.com/pulse/identity-management-internet-things-george-moraetes}}

\subsection{Access Management - AM}
La semplice associazione \glossaryItem{identita}/\textit{account} è poco utile se non ci sono autorizzazioni o politiche di accesso associate a quell'\textit{account}. L'\glossaryItem{AM} si occupa di creare e far rispettare queste politiche: applica regole diverse in base al ruolo e controlla le autorizzazioni degli \textit{account} generati e gestiti dai sistemi di \glossaryItem{idm}. 

Le politiche di \glossaryItem{autenticazione} e \glossaryItem{autorizzazione} possono essere rafforzate da un sistema di controllo degli accessi, usato per imporre dei permessi sui sistemi e sulle risorse. Principalmente ne esistono di quattro tipi:
\begin{itemize}
\item \glossaryItem{mac};
\item \glossaryItem{dac};
\item \glossaryItem{rbac};
\item \glossaryItem{abac}.
\end{itemize}
Di seguito verranno descritti vantaggi e svantaggi di ciascuno.

\paragraph{Mandatory Access Control} \mbox{} \\
\glossaryItem{mac} è fondato su un modello gerarchico basato sui livelli di sicurezza. Ogni utente, o oggetto, è associato ad un livello di sicurezza. Gli utenti possono accedere alle risorse che corrispondono al loro livello, o che hanno un livello inferiore nella gerarchia.

Gli accessi sono controllati solamente dagli amministratori, che impostano ogni permesso degli utenti. I sistemi \glossaryItem{mac} sono considerati molto sicuri proprio grazie a questa amministrazione centralizzata e sono generalmente utilizzati in ambiti governativi.

Il principale pregio è però anche un difetto, perché gli amministratori devono obbligatoriamente assegnare ogni permesso senza poter delegare niente: per sistemi di grandi dimensioni questo può portare ad un ''sovraccarico'' dello staff amministrativo. Proprio per questo motivo i sistemi \glossaryItem{mac} sono poco usati in ambito web.

\paragraph{Discretionary Access Control} \mbox{} \\
\glossaryItem{dac} si basa su un modello governato da una lista che indica chi può accedere a cosa e cosa può fare con quella risorsa. Nella lista vanno elencati tutti gli utenti e i permessi, per ogni risorsa. Sistemi di questo tipo utilizzano un'architettura più distribuita e sono generalmente più facili da gestire dei \glossaryItem{mac}, perché gli amministratori non hanno l'onere dell'impostazione dei permessi. Tuttavia sono anche meno sicuri, perché i permessi possono essere impostati in modo errato.

La maggior parte dei sistemi operativi usano modelli \glossaryItem{dac}.

\paragraph{Role-Based Access Control} \mbox{} \\
\glossaryItem{rbac} si basa su un modello di ruoli e responsabilità. Agli utenti non è permesso l'accesso al sistema: un insieme di ruoli è assegnato ad un utente, e un insieme di livelli di accesso è assegnato a ciascun ruolo. I ruoli sono gestiti dagli amministratori, che determinano quanti e quali devono essere e, successivamente, assegnano questi ruoli a funzioni e compiti. 

In questo modello è necessario conoscere in modo più specifico il sistema, perché ad ogni ruolo devono essere assegnate delle politiche di accesso e delle funzioni. È possibile delegare il ruolo di amministratore semplicemente creando un nuovo ruolo e assegnandolo a delle persone.

\glossaryItem{rbac} è difficile da implementare; per grandi organizzazioni, inoltre, è complicato identificare con precisione i ruoli. Nonostante questo, però, è molto utilizzato nell'ambito web, perché consente di automatizzare la creazione e la gestione degli utenti. 

\paragraph{Attribute Based Access Control} \mbox{} \\
\glossaryItem{abac} è un meccanismo di controllo degli accessi alle risorse che fornisce maggiore flessibilità rispetto a \glossaryItem{rbac}. L'\glossaryItem{autorizzazione} è basata su una serie di attributi associati con utenti, risorse, transazioni e attività; le decisioni sono rappresentate da regole condizionali sui valori degli attributi. In questo modo le politiche di sicurezza sono completamente indipendenti da utenti e oggetti.

Gli attributi utilizzabili in un sistema di questo tipo sono moltissimi, e variano da quelli più ''stabili'' (come ruoli, etichette di sicurezza, eccetera) a quelli più ''dinamici'' (come posizione geografica, tipologia di \glossaryItem{autenticazione} effettuata, ambiente operativo dal quale viene effettuato l'accesso, eccetera). Considerato che le decisioni vengono prese sulla base delle condizioni nel momento dell'accesso, \glossaryItem{abac} è un meccanismo di controllo degli accessi \textbf{dipendente dal contesto}.

\subsection{IdM e AM: IAM}
\glossaryItem{idm} e \glossaryItem{am} sono componenti fondamentali per qualsiasi sistema di sicurezza e dipendono fortemente l'uno dall'altro: non si possono associare ruoli e politiche se non si dispone di \textit{account}, e un \textit{account} senza permessi e politiche di accesso non serve a niente. La loro unione e combinazione porta alla definizione di sistemi di \glossaryItem{IAM}. La gestione di accessi e \glossaryItem{identita} non è più un male non necessario, ma è una componente fondamentale della sicurezza di qualsiasi organizzazione: non si tratta più di avere a che fare solamente con persone confinate nel loro ufficio, ma di gestire interazioni tra impiegati, fornitori, soci, clienti e oggetti (Figura~\ref{fig:interazioniIAM}). 

\begin{figure}[h]
\begin{center}
\includegraphics[scale=0.4]{interazioniIAM}
\caption[Gestione di identità, privilegi e accessi]{Gestione di identità, privilegi e accessi\protect\footnotemark}
\label{fig:interazioniIAM}
\end{center}
\end{figure}
\footnotetext{Immagine tratta da \cite{grt:G00292924}}

L'\glossaryItem{iot}, in particolare, è fonte di un elevatissimo numero di opportunità per le aziende e di servizi per i consumatori: un esempio è la possibilità di connettere le auto ad Internet, per aprirle, chiuderle, monitorare i consumi, accendere o spegnere il motore, eccetera. D'altro canto ha portato ad un grande incremento dei rischi legati alla sicurezza: ogni dispositivo connesso ad Internet rischia di essere compromesso e di fornire pericolose informazioni circa il suo utilizzo. Il dispositivo può poi essere controllato in modo da agire sull'ambiente circostante: sarebbe possibile aumentare o diminuire la temperatura di una stanza, spegnere, disabilitare o manomettere delle funzionalità e molto altro. È necessario gestire anche l'\glossaryItem{identita} delle cose (\glossaryItem{idot}) per assegnare ruoli e permessi anche agli oggetti, oltre che alle persone. 

La marcata eterogeneità delle entità coinvolte rende i sistemi di \glossaryItem{iam} molto complessi: è necessario gestire e proteggere enormi moli di dati (\textbf{sicurezza}), derivanti da interazioni tra persone e oggetti (\textbf{privacy}) effettuate con modalità in continua espansione (\textbf{\glossaryItem{resilienza}} e \textbf{\glossaryItem{scalabilita}}). Oltre a questo, le \glossaryItem{identita} da gestire sono sempre di più: non ci sono più solo quelle fornite dai governi, ma anche quelle digitali create direttamente dagli utenti (ad esempio quelle sui \textit{social networks}).

Tra i compiti principali dell'\glossaryItem{iam}, dunque, ci sono:
\begin{itemize}
\setlength\itemsep{2pt}
\item mantenimento della consistenza dell'\glossaryItem{identita} di un utente tra le varie applicazioni;
\item supporto del \glossaryItem{sso};
\item controllo degli accesi alle risorse basato sui ruoli o sugli attributi (\glossaryItem{rbac} e \glossaryItem{abac});
\item gestione di diverse \glossaryItem{identita}, ruoli e permessi: la stessa persona può essere un amministratore di un sistema e un semplice utente in un altro;
\item fornitura di processi e procedure per gestire il ciclo di vita dell'\glossaryItem{identita} (di utenti e di oggetti);
\item gestione dell'\glossaryItem{iot};
\item fornitura di meccanismi di accesso sicuri e senza interruzioni ad applicazioni interne, esterne e \glossaryItem{cloud};
\item identificazione di \textit{account} orfani (ancora attivi sebbene non più associati ad un utente);
\item incremento della produttività degli utenti, grazie alla possibilità di accedere alle applicazioni in meno tempo e in maggiore sicurezza;
\item incremento della sicurezza;
\item gestione delle relazioni tra oggetti e persone;
\item memorizzazione delle operazioni svolte da ciascun utente (o oggetto) nel sistema e generazione di \textit{log} e \textit{report};
\item fornitura di servizi \textit{self service} per l'utente: cambio \textit{password}, gestione profilo, eccetera;
\item pianificazione e giustificazione dell'allocazione delle risorse.
\end{itemize}

\section{Single Sign-On - SSO}
\begin{tframe}{Single Sign On - SSO}
\begin{center}
\begin{minipage}{0.45\textwidth}
\begin{figure}[h]
\centering
\includegraphics[scale=0.4]{beforeSSO}
\end{figure}
{\fontsize{7pt}{7.2}\selectfont
\begin{adv}
\item Risparmio di tempo
\item Una sola coppia di credenziali
\item Riduzione dei costi
\end{adv}
}
\end{minipage}
\begin{minipage}{0.45\textwidth}
\begin{figure}[h]
\centering
\includegraphics[scale=0.4]{afterSSO}
\end{figure}
{\fontsize{7pt}{7.2}\selectfont
\begin{disadv}
\item One key to kingdom
\item Single point of failure
\end{disadv}
}
\end{minipage}
\end{center}
\end{tframe}

\section{On-premises o IDaaS?}
Un numero sempre crescente di compagnie sta ''migrando'' da soluzioni \glossaryItem{iam} \glossaryItem{onpremises} a soluzioni \glossaryItem{cloud} (\glossaryItem{idaas}). Questa migrazione è in gran parte dovuta alla possibilità di utilizzare il \glossaryItem{sso} tra applicazioni \glossaryItem{cloud} (\glossaryItem{saas}), ma le differenze tra le due modalità sono molte altre. Di seguito verrà presentato un confronto basato sui seguenti punti:
\begin{itemize}
\item \textbf{caratteristiche intrinseche}: in particolare:
	\begin{itemize}
	\item \textbf{piattaforma software e facilità di installazione}: il costo iniziale per configurare ed installare il prodotto;
	\item \textbf{manutenzione e aggiornamento}: risorse richieste per manutenere ed aggiornare il prodotto;
	\item \textbf{sicurezza e protezione dei dati}: difficoltà per rendere il sistema sicuro;
	\item \textbf{privacy}: difficoltà per mantenere i dati privati;
	\item \textbf{agilità}: capacità di adattarsi ai cambiamenti;
	\item \textbf{distribuzione dell'architettura}: vantaggi sul posizionamento dell'architettura (\glossaryItem{onpremises} o \glossaryItem{cloud}).
	\end{itemize}
\item \textbf{maturità funzionale}: in particolare:
	\begin{itemize}
	\item \textbf{modernità dell'architettura}: consistenza, accesso facile a tutte le funzionalità, eccetera;
	\item \textbf{modernità dell'interfaccia grafica}: facilità con la quale l'utente trova quello che cerca;
	\item \textbf{integrazione con \glossaryItem{saas}}: facilità di integrazione del \glossaryItem{sso} con applicazioni di tipo \glossaryItem{saas};
	\item \textbf{integrazione con applicazioni \glossaryItem{onpremises}}: facilità di integrazione del \glossaryItem{sso} con applicazioni di tipo \glossaryItem{onpremises};
	\item \textbf{personalizzazione}: facilità di modifica per adattarsi a necessità specifiche;
	\item \textbf{aderenza agli standard}: grado di supporto degli standard;
	\item \textbf{maturità generale delle funzionalità \glossaryItem{iam}};
	\item \textbf{flessibilità dei termini di licenza}: supporto per varie opzioni appropriate per differenti casi d'uso. 
	\end{itemize}
\end{itemize}
In Figura~\ref{fig:IDaaSvsOnPremises} viene mostrato un confronto qualitativo sulle caratteristiche appena elencate.

\begin{figure}[hbpc]
\begin{center}
\includegraphics[scale=0.45]{IDaaSvsOnPremises}
\caption[Confronto tra sistemi IDaaS e on-premises nell'ambito IAM]{Confronto tra sistemi IDaaS e on-premises nell'ambito IAM\protect\footnotemark}
\label{fig:IDaaSvsOnPremises}
\end{center}
\end{figure}
\footnotetext{Immagine tratta da \cite{grt:G00261583}}

\subsection{Caratteristiche intrinseche}
\paragraph{Piattaforma software e facilità di installazione} \mbox{} \\
Per definizione l'\glossaryItem{idaas} è quello in cui il servizio mette a disposizione la piattaforma e il \textit{software}, quindi non richiede nessuno sforzo da parte degli utilizzatori, che risparmiano moltissimo tempo e denaro pagando solo le funzionalità di cui hanno bisogno. Inoltre:
\begin{itemize}
\item se l'organizzazione non dispone di uno \textit{staff} adatto a mantenere un sistema \glossaryItem{onpremises} o non può procurarselo la scelta \glossaryItem{idaas} è la migliore;
\item se l'organizzazione non ha nessuna esperienza di \glossaryItem{saas} allora conviene l'alternativa \glossaryItem{onpremises};
\item se il tempo a disposizione è poco conviene \glossaryItem{idaas}, in quanto non necessita della configurazione iniziale;
\item se il costo iniziale per la creazione dell'infrastruttura è elevato conviene, ancora una volta, \glossaryItem{idaas}.
\end{itemize}

\paragraph{Manutenzione e aggiornamento} \mbox{} \\
In questo ambito \glossaryItem{idaas} vince di misura. Infatti, non essendoci la necessità di installazione, non c'è neanche quella di manutenzione o aggiornamento. Al contrario, un prodotto \glossaryItem{onpremises} necessita di essere seguito e aggiornato.

\paragraph{Sicurezza e protezione dei dati} \mbox{} \\
Nell'ambito della sicurezza la decisione è più difficile, e in generale dipende dalla situazione. Si potrebbe pensare che sia più sicuro mantenere i dati \glossaryItem{onpremises} in quanto l'organizzazione ne ha il completo controllo. Questo è vero, ma molto spesso non si dispone di esperti di sicurezza che possano, a tempo pieno, dedicarsi alla protezione dei dati. Sebbene le organizzazioni di grandi dimensioni siano in grado di far fronte a questo problema, quelle di dimensioni medio/piccole non possono, e questo rende, per loro, più conveniente la scelta \glossaryItem{idaas}. 

\paragraph{Privacy} \mbox{} \\
Il tema della \textit{privacy} è molto controverso. Se i dati sono interamente mantenuti dall'organizzazione si ha l'assoluto controllo e si è sicuri di dove risiedono. D'altro canto, se il sistema è \glossaryItem{cloud} bisogna chiedersi dove sono memorizzati, come vengono trasferiti e in che giurisdizione ricadono. Se i dati sono molto sensibili è più sicuro usare sistemi \glossaryItem{onpremises}. Al contrario, se l'organizzazione ha particolari requisiti di riservatezza o protezione (come un'agenzia governativa) può essere utile cercare un sistema \glossaryItem{idaas} che possieda questi requisiti: alcuni recenti sistemi \glossaryItem{cloud}, ad esempio, soddisfano le necessità delle agenzie federali statunitensi. Infine \glossaryItem{idaas} consente di memorizzare i dati sotto una specifica giurisdizione: è sufficiente cercare un produttore che supporta quella voluta.

\paragraph{Agilità} \mbox{} \\
L'agilità è il maggior punto di vantaggio dei sistemi \glossaryItem{idaas}. Sistemi di questo tipo possono scalare con facilità. Ogni cambiamento viene fatto in gran velocità e i produttori non devono richiedere ai clienti di aggiornare per ottenere il supporto di altre funzionalità (il ciclo di rilascio si misura in settimane nel primo caso e mesi nell'altro). Sebbene molte aree siano già mature, in altre dei cicli così veloci sono un grande vantaggio. Per questi motivi, \glossaryItem{idaas} è la scelta giusta se si necessita l'integrazione con applicazioni in rapida evoluzione o un rapido rilascio di nuove funzionalità. 

\paragraph{Distribuzione dell'architettura} \mbox{} \\
Dipendentemente dalle situazioni può essere più vantaggioso avere un sistema eseguito su \glossaryItem{cloud} o no. Ad esempio, se l'organizzazione non dispone di una connessione veloce, l'\glossaryItem{idaas} può comportare latenze significative; d'altra parte, se l'integrazione con \glossaryItem{saas} è una priorità l'\glossaryItem{onpremises} deve essere scartato. Oggigiorno, comunque, molte organizzazioni fanno uso di \glossaryItemPl{serviziodirectory} \glossaryItem{onpremises}, quindi, se si è su \glossaryItem{cloud}, sono necessari dei connettori per accedere a questi sistemi.

Per scegliere bisogna chiedersi chi sono gli utenti e da dove accedono. Se sono persone esterne all'organizzazione e devono accedere ad applicazioni su \glossaryItem{cloud} la scelta \glossaryItem{idaas} è assolutamente la più conveniente.

\subsection{Maturità funzionale}
\paragraph{Modernità dell'architettura} \mbox{} \\
La maggior parte dei nuovi sistemi \glossaryItem{iam} è di tipo \glossaryItem{idaas}, quindi è naturale che questi presentino delle architetture più vicine alle \glossaryItemPl{bestpractice} correnti. Anche i sistemi \glossaryItem{onpremises} sono stati riprogettati per adattarsi, ma il cambiamento dell'architettura richiede pesanti cambiamenti e aggiornamenti.

\paragraph{Modernità dell'interfaccia} \mbox{} \\
La maggior parte delle innovazioni grafiche vengono introdotte prima in sistemi \glossaryItem{idaas}. Il mondo web è molto più dinamico, e un'interfaccia sempre aggiornata è un obbligo. Al contrario, i sistemi \glossaryItem{onpremises} continuano a rimanere ancorati alle loro scelte senza cambiare.

\paragraph{Integrazione con SaaS} \mbox{} \\
L'integrazione con \glossaryItem{saas} è molto più facile e veloce per sistemi \glossaryItem{idaas}, anche perché l'organizzazione non deve fare niente per integrare i nuovi \textit{software}. Le alternative \glossaryItem{onpremises}, invece, richiedono costi e tempo aggiuntivi per integrare nuove applicazioni.

\paragraph{Integrazione con applicazioni on-premises} \mbox{} \\
In generale l'integrazione con applicazioni \glossaryItem{onpremises} è migliore per le soluzioni \glossaryItem{onpremises}. Sebbene anche i sistemi \glossaryItem{idaas} stiano cercando di supportarle, se si necessita l'integrazione con un gran numero di applicazioni di questo tipo, allora la scelta migliore è ancora la prima.

\paragraph{Personalizzazione} \mbox{} \\
La personalizzazione è il principale vantaggio dei sistemi \glossaryItem{onpremises}. In teoria \glossaryItem{idaas} offre un approccio adatto a quasi tutte le organizzazioni grazie alla possibilità di configurare il prodotto in molti modi diversi. La realtà, però, è più confusa. 

La personalizzazione comporta un cambiamento nel codice del prodotto, perché richiede l'aggiunta di funzionalità specifiche e necessarie solamente ad un ristretto numero di interessati. Sebbene i prodotti \glossaryItem{onpremises} possano essere altamente personalizzati, la personalizzazione in sé dovrebbe essere evitata e nel futuro sarà sempre meno necessaria: più è matura un'area di interesse meno personalizzazioni sono richieste e più è possibile prevedere diverse configurazioni per incontrare i bisogni di tutti i clienti. D'altro canto, la maturità di aree dinamiche come il \glossaryItem{cloud} o il \textit{mobile} è ancora molto lontana. 

Se si ritiene che un prodotto configurabile non si adatti alle necessità dell'azienda, allora l'\glossaryItem{onpremises} è la scelta giusta.

\paragraph{Aderenza agli standard} \mbox{} \\
Vista la grande diffusione di applicazioni \glossaryItem{saas}, molti produttori di sistemi \glossaryItem{iam} tendono ad aggiungere supporto agli standard più velocemente nelle versioni \glossaryItem{idaas} rispetto a quelle \glossaryItem{onpremises}. Questa tendenza è destinata a crescere, in quanto permette di ricevere \textit{feedbacks} più velocemente e di reagire con maggiore rapidità alla nascita di nuovi standard.

\paragraph{Maturità generale delle funzionalità} \mbox{} \\
In media i sistemi \glossaryItem{onpremises} tendono ad essere più vecchi. Di conseguenza dispongono di tecnologie superate e di funzionalità più legate ad applicazioni \glossaryItem{onpremises}. \glossaryItem{idaas}, invece, offre un'ottima integrazione con \glossaryItem{saas} e un buon supporto di applicazioni \glossaryItem{onpremises}. Nel lungo periodo ci si aspetta che nel \glossaryItem{cloud} le innovazioni arrivino prima\footnote{Vedi \cite{grt:G00261583}}, quindi \glossaryItem{idaas} è la scelta migliore.

\paragraph{Flessibilità dei termini della licenza} \mbox{} \\
Storicamente, i sistemi \glossaryItem{onpremises} sono venduti con licenze pagate una sola volta, mentre quelli \glossaryItem{idaas} normalmente prevedono un pagamento ogni mese e si adattano meglio alle esigenze degli utenti perché consentono di pagare solo le funzionalità volute. 

Nonostante questo, anche le licenze dei \textit{software} \glossaryItem{onpremises} si stanno adattando e stanno diventando più flessibili, quindi la differenza tra le due alternative è poca e non c'è un vincitore netto.

\subsection{Cosa scegliere?}
La scelta tra \glossaryItem{cloud} e \glossaryItem{onpremises} dipende fortemente dalle necessità dell'organizzazione e dal \glossaryItem{tco}. Soluzioni \glossaryItem{onpremises} sono adatte se si hanno vincoli stretti sulla residenza e sulla protezione dei dati, se si vogliono implementazioni \glossaryItem{iam} mature e personalizzabili e un forte controllo sull'amministrazione delle \glossaryItem{identita} (\glossaryItem{iga}), attualmente carenti nei prodotti basati su \glossaryItem{cloud}. Al contrario, sistemi \glossaryItem{idaas} sono adatti se si cercano implementazioni \glossaryItem{iam} dinamiche e una maggiore integrazione con \glossaryItem{saas} (Figura~\ref{fig:IDaaSvsorPremises}).

\begin{figure}[h!]
\begin{center}
\includegraphics[scale=0.25]{IDaaSorOnPremises}
\caption[Scelta tra sistemi IDaaS e on-premises nell'ambito IAM]{Scelta tra sistemi IDaaS e on-premises nell'ambito IAM\protect\footnotemark}
\label{fig:IDaaSvsorPremises}
\end{center}
\end{figure}
\footnotetext{Immagine tratta da \cite{grt:G00296572}}
\newpage

\section{Un esempio: Sistema Pubblico di Identità Digitale}
Il \glossaryItem{spid} (logo in Figura~\ref{fig:spid}) permette a cittadini e imprese di accedere con un unico \textit{login} a tutti i servizi \textit{online} di pubbliche amministrazioni e imprese aderenti. \glossaryItem{spid} nasce con lo scopo di favorire la diffusione di servizi \textit{online} e di agevolarne l'utilizzo da parte di cittadini e imprese. 

Il modello proposto da \glossaryItem{spid} segue un approccio federato di aziende private e accreditate per la fornitura dei servizi di \glossaryItem{identita} digitale. Cittadini e imprese possono scegliere il loro fornitore di \glossaryItem{identita} preferito.


\begin{figure}[h]
\begin{center}
\includegraphics[scale=0.4]{spid}
\caption[Logo di SPID]{Logo di SPID\protect\footnotemark}
\label{fig:spid}
\end{center}
\end{figure}

\subsection{Attori e ruoli}
\footnotetext{Fonte: \url{http://www.agid.gov.it/agenda-digitale/infrastrutture-architetture/spid/percorso-attuazione}}
Lo \glossaryItem{spid} identifica differenti ruoli:
\begin{itemize}
\item \textbf{Fornitori di \glossaryItem{identita} digitale (o \glossaryItem{IdP})}: sono imprese accreditate dall'\glossaryItem{agid} con il compito di identificare l'utente in modo certo, di creare le \glossaryItem{identita} digitali, di assegnare le credenziali e di gestire gli attributi dell'utente. Devono garantire la correttezza dell'\glossaryItem{identita} digitale e la riservatezza delle informazioni.
\item \textbf{Fornitori di servizi (o \glossaryItem{SP})}: privati o pubbliche amministrazioni che erogano servizi \textit{online}.
\item \textbf{Utente}: titolare di un'\glossaryItem{identita} digitale \glossaryItem{spid} che utilizza i servizi erogati dai fornitori di servizi.
\item \textbf{Gestore di attributi qualificati (o Attribute Provider)}: ha il potere di attestare gli attributi qualificati su richiesta dei fornitori di servizi.
\item \textbf{Agenzia}: organismo di vigilanza che si occupa di gestire l'accreditamento e di monitorare i gestori dell'\glossaryItem{identita} digitale e i gestori di attributi qualificati.
\end{itemize}

\subsection{Come funziona}
Un utente si registra al servizio tramite un \glossaryItem{idp} che crea un’\glossaryItem{identita} digitale e gli assegna le credenziali per il riconoscimento. L'utente può utilizzare la sua \glossaryItem{identita} digitale per l'accesso ai servizi \textit{online} offerti dai \glossaryItem{sp}, che sono collegati a tutti gli \glossaryItem{idp}. Sarà compito dell’\glossaryItem{idp} verificare la correttezza dei dati di \textit{login} immessi dall'utente e fornire al \glossaryItem{sp} solo gli attributi dell'utente strettamente necessari alla fornitura del servizio.

\glossaryItem{spid} infatti introduce il concetto di informazioni necessarie e sufficienti per il servizio. I fornitori del servizio potranno richiedere solamente le informazioni minime necessarie all'erogazione del servizio stesso, come si legge in \cite{reg:spid}:

\begin{displayquote}
\textit{I fornitori di servizi, per verificare le \textit{policies} di sicurezza relativi all'accesso ai servizi da essi erogati potrebbero avere necessità di informazioni relative ad attributi riferibili ai soggetti richiedenti. Tali \textit{policies} dovranno essere concepite in modo da richiedere per la verifica il set minimo di attributi pertinenti e non eccedenti le necessità effettive del servizio offerto e mantenuti per il tempo strettamente necessario alla verifica stessa, come previsto dall'articolo 11 del decreto legislativo n. 196 del 2003.}
\end{displayquote}

\subsection{Livelli di sicurezza}
In Figura~\ref{fig:spidlivelli} sono mostrati i tre livelli di sicurezza di \glossaryItem{spid}.
\begin{figure}[h]
\begin{center}
\includegraphics[scale=0.5]{sicurezzaSPID}
\caption[Livelli di sicurezza SPID]{Livelli di sicurezza SPID\protect\footnotemark}
\label{fig:spidlivelli}
\end{center}
\end{figure}
\footnotetext{Fonte: \url{http://www.spid.gov.it/}}
\begin{itemize}
\item \textbf{Livello 1}: garantisce con un buon grado di affidabilità dell'\glossaryItem{identita} accertata nel corso dell'attività di \glossaryItem{autenticazione}. A tale livello è associato un rischio moderato e compatibile con l'impiego di un sistema \glossaryItem{autenticazione} a singolo fattore, ad es. la \textit{password}; questo livello può essere considerato applicabile nei casi in cui il danno causato da un utilizzo indebito dell’\glossaryItem{identita} digitale ha un basso impatto per le attività del cittadino/impresa/amministrazione. Per il livello 1 la credenziale sarà dunque una \textit{password} di almeno 8 caratteri, da rinnovarsi ogni 180 giorni, formulata secondo i consueti criteri di sicurezza.
\item \textbf{Livello 2}: garantisce con un alto grado di affidabilità dell'\glossaryItem{identita} accertata nel corso dell'attività di \glossaryItem{autenticazione}. A tale livello è associato un rischio ragguardevole e compatibile con l'impiego di un sistema di \glossaryItem{autenticazione} informatica a due fattori non necessariamente basato su certificati digitali; questo livello è adeguato per tutti i servizi per i quali un indebito utilizzo dell’\glossaryItem{identita} digitale può provocare un danno consistente. Per il livello 2, oltre alla \textit{password} sarà necessario inserire il codice proveniente da un dispositivo a chiave variabile (ad esempio una \textit{One Time Password}) che potrebbe essere anche un'applicazione sul cellulare.
\item \textbf{Livello 3}: garantisce con un altissimo grado di affidabilità dell'\glossaryItem{identita} accertata nel corso dell'attività di \glossaryItem{autenticazione}. A tale livello è associato un rischio altissimo e compatibile con l'impiego di un sistema di \glossaryItem{autenticazione} informatica a due fattori basato su certificati digitali e criteri di custodia delle chiavi private su altri dispositivi; questo è il livello di garanzia più elevato e da associare a quei servizi che possono subire un serio e grave danno per cause imputabili ad abusi di \glossaryItem{identita}. È anche interessante notare come la definizione di ''dispositivo'' includa sia sistemi di tipo \textit{hardware} sia di tipo \textit{software} (ad esempio sono tali i generatori di \textit{password} attraverso applicazioni per \textit{smartphone}).
\end{itemize}

\subsection{Come ottenere l'identità digitale}
Gli attori che possono assumere il ruolo di \glossaryItem{idp} sono molteplici (banche, operatori di telefonia mobile, \textit{certification authority}, fornitori di soluzioni \glossaryItem{it}) e giocano un ruolo fondamentale nel decretare il successo del sistema perché portano in “dote” allo \glossaryItem{spid} il proprio bacino di utenti potenziali. La condizione ottimale è che il numero di \glossaryItem{idp} sia sufficientemente elevato per raggiungere il maggior numero di utenti, e contemporaneamente limitato per minimizzare il numero di relazioni tra \glossaryItem{sp} e \glossaryItem{idp}. 

Il processo di richiesta  dell'\glossaryItem{identita} e di \glossaryItem{autenticazione} è fondamentale perché un'esperienza utente eccessivamente complicata potrebbe scoraggiare gli utenti che vorrebbero aderire al servizio.

\subsection{Vantaggi}
I \glossaryItem{sp} che aderiscono allo \glossaryItem{spid}, sia pubbliche amministrazioni sia imprese private, possono disporre di un parco utenti senza censirli, non hanno gli oneri derivanti dalla conservazione dei dati personali e non devono preoccuparsi di evitare attacchi volti al furto delle credenziali. Inoltre, i \glossaryItem{sp} possono avere profili con un’\glossaryItem{identita} certa, eliminando i cosiddetti ''falsi profili'', ed univoca, eliminando i duplicati.

Per gli utenti, \glossaryItem{spid} consente di semplificare la vita di cittadini e imprese nell'interazione con la pubblica amministrazione tramite servizi \textit{online} grazie ad un unico \textit{login}. Il sistema garantisce la massima sicurezza e \textit{privacy}. Il \glossaryItem{sp} non può conservare i dati dell'utente che riceve dall’\glossaryItem{idp} ed è assolutamente vietata la tracciatura delle attività di un individuo. 
\chapter{Il progetto}\label{progetto}
\section{Monokee}
\textbf{Monokee} è un sistema \glossaryItem{idaas} nato con lo scopo di realizzare una corretta gestione degli utenti e delle loro autorizzazioni all'interno di sistemi informativi eterogenei. Utilizzandolo, ciascun utente sarà in grado di gestire in modo centralizzato l'accesso in \glossaryItem{sso} ad applicazioni differenti. In parole povere, Monokee assicura che le persone giuste accedano alle risorse di propria competenza.

Ogni utente di Monokee ha a disposizione un \textbf{domain broker} (Figura~\ref{fig:domainBroker}) in cui sono elencati i diversi domini ad esso associati. Un dominio è uno spazio di un web server che identifica in maniera precisa il nome di un privato, un ente o un'organizzazione su Internet. Il dominio \textbf{personale} è obbligatorio: qui sono presenti le applicazioni personali; ci possono poi essere uno o più domini di tipo \textbf{company}, ovvero relativi ad aziende registrate al servizio. 

\begin{figure}[hbpc]
\begin{center}
\includegraphics[scale=0.25]{domainBroker}
\caption{Domain broker di Monokee}
\label{fig:domainBroker}
\end{center}
\end{figure}

Le utenze sono quindi di due tipi differenti:
\begin{itemize}
\item \textbf{consumer}: con il solo dominio personale (il rapporto utente:domini è 1:1);
\item \textbf{company}: oltre al dominio personale hanno anche la possibilità di avere domini aziendali (il rapporto utente:domini è 1:N). 
\end{itemize}
Attualmente i concetti di dominio e \textit{domain broker} sono propri di Monokee: nessun \textit{competitor} e nessun altro sistema di \glossaryItem{iam} supporta una gestione di questo tipo. In generale, infatti, il problema viene risolto imponendo all'utente di registrarsi più volte al sistema, creando un evidente controsenso per un sistema di gestione delle \glossaryItem{identita}.

Accedendo ad uno dei domini a sua disposizione, l'utente può visualizzare l'\textbf{application broker}, ovvero un elenco delle applicazioni associate a quel dominio (Figura~\ref{fig:applicationBroker}). Da qui è possibile, selezionando un'applicazione, effettuare l'accesso in \glossaryItem{sso}. 

\begin{figure}[hbpc]
\begin{center}
\includegraphics[scale=0.25]{applicationBroker}
\caption{Application broker di Monokee}
\label{fig:applicationBroker}
\end{center}
\end{figure}

Se i permessi associati all'\textit{account} lo consentono si possono inoltre gestire le applicazioni dell'\textit{application broker} rimuovendo quelle presenti o aggiungendone altre dal \textbf{catalogo} (illustrato in Figura~\ref{fig:catalogo}).

\begin{figure}[hbpc]
\begin{center}
\includegraphics[scale=0.25]{catalogo}
\caption{Catalogo applicativo di Monokee}
\label{fig:catalogo}
\end{center}
\end{figure}

Infine, se il dominio è di tipo \textit{company}, un utente amministratore può gestire gli utenti, le applicazioni ed eventuali gruppi appartenenti al dominio.

\subsection{I moduli}
L'architettura base di Monokee si compone di sette moduli che interagiscono tra di loro per il corretto funzionamento del servizio:
\begin{itemize}
\item \textbf{Front end}: responsabile delle interazioni con l'utente e scritto utilizzando Angular.js;
\item \textbf{Plug in}: ha una duplice funzione: da un lato permette di inserire nuove applicazioni non presenti del catalogo attraverso una fase di \textbf{learning} della struttura della pagina web, e dall'altro consente l'accesso in \glossaryItem{sso} alle applicazioni tramite il \textbf{form fulfillment};
\item \textbf{\glossaryItem{IdP}}: si occupa dell'accesso di tipo federato \glossaryItem{saml} elaborando le richieste del \glossaryItem{SP} (\textbf{SAMLRequest}) e producendo l'appropriata \textbf{SAMLResponse} consentendo o meno l'accesso all'applicazione. È scritto in Java;
\item \textbf{\glossaryItem{ad} Integration}: consente l'integrazione del sistema con uno o più \glossaryItemPl{serviziodirectory} di tipo \glossaryItem{ad} e si occupa di effettuare periodicamente delle \textit{query} volte a replicare l'intera struttura delle utenze nel \textit{database} del sistema inviando una copia di tutti i dati al modulo \textit{Core}; questo strumento garantisce a Monokee di operare sempre
sull'ultima versione aggiornata delle utenze;
\item \textbf{Mobile}: applicazione \textit{mobile} per \textit{smartphone} e \textit{tablet};
\item \textbf{\glossaryItem{SP}}: si occupa della generazione di \textit{SAMLRequest} per erogare dei servizi. Il \glossaryItem{sp} di Monokee è utilizzato per l'accesso all'applicazione di gestione del catalogo, obiettivo dell'attività di stage. Così come l'\glossaryItem{idp}, è scritto in Java;
\item \textbf{Core}: insieme di servizi \glossaryItem{rest} su protocollo \glossaryItem{http} che ricevono, ed elaborano, i dati provenienti dagli altri moduli, si occupano della loro gestione e memorizzazione nel \textit{database}, producono ed inviano le risposte in base alle situazioni e salvano i \textit{log} di quanto è stato fatto.
\end{itemize}
Un ulteriore modulo, obiettivo dell'attività di stage, verrà descritto nella successiva sezione.

\section{Monokee Catalogue Manager} \label{catmgr}
Lo scopo dell'attività di stage è quello di realizzare la parte di back end del gestore del catalogo applicativo dell'applicazione Monokee. 

La fonte principale delle applicazioni accedibili in \glossaryItem{sso} è il catalogo: attraverso di esso è possibile cercare e aggiungere applicazioni al proprio dominio. Il prodotto sviluppato permetterà la gestione, lato amministratore, del catalogo stesso, con le principali funzionalità di aggiunta, rimozione e configurazione.

\subsection{Funzionalità principali}
Attraverso il catalogo applicativo sarà possibile gestire le applicazioni utilizzabili in modo ''predefinito'' in Monokee. In particolare, le funzioni principali saranno:
\begin{itemize}
\item gestione di un'applicazione:
\begin{itemize}
	\item aggiunta e configurazione;
	\item rimozione;
	\item modifica della configurazione;
\end{itemize}
\item categorizzazione delle applicazioni;
\item visualizzazione e ricerca delle applicazioni;
\item gestione dei cataloghi specifici dei domini:
\begin{itemize}
	\item aggiunta di un catalogo ad un dominio \textit{company};
	\item rimozione di un catalogo da un dominio \textit{company};
	\item aggiunta di un'applicazione al catalogo;
	\item rimozione di un'applicazione dal catalogo;
\end{itemize}
\item gestione dei gruppi di applicazioni:
\begin{itemize}
	\item aggiunta di un gruppo;
	\item rimozione di un gruppo;
	\item modifica di un gruppo;
	\item aggiunta di un'applicazione al gruppo;
	\item rimozione di un'applicazione dal gruppo;
\end{itemize}
\item visualizzazione di statistiche sul numero di applicazioni, gruppi e utenti;
\item visualizzazione dei \textit{log} delle operazioni svolte dagli utenti e degli errori riscontrati durante l'esecuzione di queste operazioni.
\end{itemize}

Principalmente, quindi, le entità coinvolte sono tre:
\begin{enumerate}
	\item applicazione;
	\item gruppo;
	\item categoria.
\end{enumerate}

\paragraph{Applicazione} \mbox{} \\
Le applicazioni possono essere di due tipi:
\begin{itemize}
\item pubbliche;
\item private, ovvero specifiche di un dominio aziendale.
\end{itemize}
È presente, di base, un catalogo pubblico, contenente tutte le applicazioni pubbliche. Accanto a questo possono esistere anche dei cataloghi di dominio (un catalogo per ogni dominio aziendale esistente): questi raccolgono le applicazioni private. Così facendo è possibile inserire applicazioni specifiche di un'azienda (anche raggiungibili solamente attraverso la rete \textit{ethernet} aziendale) e permettere agli utenti di quell'azienda di usufruirne. 

Indipendentemente dal fatto che siano pubbliche o private, gli utenti di Monokee potranno accedere alle applicazioni in \glossaryItem{sso}. Attualmente, l'\glossaryItem{autenticazione} può avvenire in tre modi:
\begin{itemize}
\item \textbf{form-based}, attraverso due differenti modalità:
\begin{itemize}
	\item \textbf{form fulfillment}: una fase di \textit{learning} specifica per applicazione e \textit{browser} effettuata grazie a dei \textit{plug in} già sviluppati consente di ''istruire'' Monokee sulla struttura della pagina dell'applicazione web per poter eseguire il \glossaryItem{sso} ''riempiendo'' il \textit{form} di accesso;
	\item \textbf{richieste POST realizzate tramite \glossaryItem{ajax}}.
\end{itemize}
\item \textbf{federata}, con l'utilizzo dello standard \glossaryItem{saml} 2.0;
\item \textbf{accesso di terzo tipo}, ad applicazioni di terze parti che necessitano una richiesta POST \glossaryItem{ajax} completa di \textit{headers} \glossaryItem{http}.
\end{itemize}
Esiste, in realtà, anche un quarto tipo di applicazione supportato da Monokee, ma non è rilevante per la discussione in quanto le applicazioni di questo tipo non possono essere gestite da Catalogue Manager.

Le regole di accesso specifiche delle applicazioni dovranno quindi tenere conto della modalità di \glossaryItem{autenticazione}. Inoltre, l'accesso tramite \glossaryItem{sso} comporta il dover conoscere l'\glossaryItem{url} della pagina di \textit{login} dell'applicazione. Questa pagina può variare da \textit{browser} a \textit{browser}, come già citato, ma anche da Paese a Paese (\textbf{localizzazione}): per una stessa applicazione di base possono essere quindi memorizzate più pagine di \textit{login}, dipendentemente dal Paese ''bersaglio''. Un esempio è dato da Amazon (Figura~\ref{fig:amazon}):
\begin{figure}[h]
\centering
\includegraphics[scale=0.2]{amazonIT}\hfill
\includegraphics[scale=0.2]{amazonES}
\caption[Confronto tra Amazon Italia e Spagna]{Confronto fra le pagine di login di Amazon Italia e Amazon Spagna}
\label{fig:amazon}
\end{figure}\newpage
Come si può notare, nell'\glossaryItem{url} i due \textit{host} sono diversi (www.amazon.it in un caso, www.amazon.es nell'altro). Questo porta a dover memorizzare, per ogni applicazione localizzata, una lista delle diverse localizzazioni e dei corrispondenti \glossaryItem{url} per il \textit{login}. Dal punto di vista dell'utente, ogni localizzazione è considerata un'applicazione a sé stante. Nel caso appena citato, ad esempio, sarebbero presenti due diverse ''versioni'' di Amazon, Amazon Italia e Amazon Spagna. Ovviamente l'utente potrà inserire nel suo dominio entrambe le versioni, qualora voglia autenticarsi sia al sito italiano che a quello spagnolo.

Appena creata, l'applicazione non è \textbf{pubblicata}: questo significa che gli utenti di Monokee non possono vederla, in quanto i dati necessari al \glossaryItem{sso} non sono ancora stati impostati. Una volta pubblicata, l'applicazione è visibile nel catalogo. Può accadere, però, che sia necessario il cambiamento di alcuni dati in seguito ad una modifica dell'applicazione. Quest'ultima può quindi essere messa, momentaneamente, \textbf{in manutenzione} in modo da poter modificare i dati necessari e avvisare gli utenti che l'autenticazione potrebbe non funzionare correttamente.

\paragraph{Gruppo} \mbox{} \\
Applicazioni con caratteristiche simili (ad esempio nel caso della localizzazione sopra citata) possono essere raggruppate per consentire una migliore gerarchizzazione e per generare meno confusione nell'utente. I gruppi possono essere creati per applicazioni sia pubbliche che private, e aiutano gli amministratori del catalogo a mantenere ordine tra le applicazioni presenti. Non c'è limite al numero di applicazioni che possono essere inserite in un gruppo, ma un'applicazione può essere inserita in un solo gruppo per volta. 

Così come per le applicazioni, anche i gruppi possono essere pubblici o privati, ma in un gruppo privato non possono essere inserite applicazioni pubbliche e viceversa.

\paragraph{Categorie} \mbox{} \\
Ogni applicazione deve appartenere ad una o più categorie, in modo da rendere il più semplice possibile la ricerca nel catalogo. Il sistema di categorizzazione è più vicino ad un sistema di ''etichette'': la categorizzazione non è, infatti, mutuamente esclusiva e può accadere (e in generale accade) che ad un'applicazione siano associate più categorie. Questo è un esempio del cosiddetto ''\textbf{schema ambiguo per argomento}''. Schemi di questo genere sono più difficilmente manutenibili da parte degli sviluppatori, perché richiedono un costante lavoro di controllo e aggiornamento, ma risultano molto più utilizzabili da parte degli utenti, in quanto non è necessario sapere esattamente cosa si sta cercando. 

Considerato che Monokee può essere usato sia per uso personale che per uso aziendale si è resa necessaria una profonda ad approfondita categorizzazione delle applicazioni web. Tale categorizzazione tiene conto di entrambi gli utilizzi previsti di Monokee e questo ha portato all'ottenimento di un alto numero di ''etichette''. Tuttavia, un utente di Monokee rischierebbe di essere confuso e disorientato da un numero elevato di categorie: è stato pertanto fatto un ulteriore lavoro di raggruppamento, in modo da definire le seguenti 13 ''sovra categorie'':
\begin{itemize}
\item Arts;
\item Beauty \& Sport;
\item Career;
\item Computer \& Electronics;
\item Education;
\item Finance;
\item Food \& Health;
\item Hobby \& Travel;
\item Internet \& Communication;
\item News;
\item People \& Pets;
\item Productivity;
\item Vehicles.
\end{itemize}
Queste ultime vengono mostrate nella pagina di aggiunta di un'applicazione all'\textit{application broker}, come mostrato in Figura~\ref{fig:catalogo}. Le ''etichette'', invece, vengono mostrate all'utente del gestore del catalogo che, essendo un amministratore di Monokee, ha una maggiore conoscenza del mondo web e non rischia di essere sopraffatto dal gran numero di categorie. Di seguito vengono indicate le sovra categorie, complete delle categorie mostrate agli utenti del gestore del catalogo di Monokee:
\begin{itemize}
\item \textbf{Arts}:
	\begin{itemize}
	\item Animation \& Comics;
	\item Arts \& Entertainment;
	\item Movies;
	\item Music \& Audio;
	\item Photography.
	\end{itemize}
\item \textbf{Beauty \& Sport}:
	\begin{itemize}
	\item Beauty;
	\item Sports.
	\end{itemize}
\item \textbf{Career}:
	\begin{itemize}
	\item Jobs \& Employment.
	\end{itemize}
\item \textbf{Computer \& Electronics}:
	\begin{itemize}
	\item Computer Hardware;
	\item Computer Security;
	\item Electronics;
	\item Multimedia;
	\item Networking;
	\item Programming;
	\item Software.
	\end{itemize}
\item \textbf{Education}:
	\begin{itemize}
	\item Book Retailers;
	\item Dictionaries;
	\item Ebooks;
	\item Education;
	\item Science;
	\item Universities.
	\end{itemize}
\item \textbf{Finance}:
	\begin{itemize}
	\item Banking;
	\item Financial Investing;
	\item Financial Management;
	\item Insurance.
	\end{itemize}
\item \textbf{Food \& Health}:
	\begin{itemize}
	\item Food Delivery;
	\item Health;
	\item Nutrition;
	\item Restaurants.
	\end{itemize}
\item \textbf{Hobby \& Travel}:
	\begin{itemize}
	\item Games;
	\item Hotels;
	\item Maps;
	\item Online Games;
	\item Recreations \& Hobbies;
	\item TV \& Video;
	\item Video Games.
	\end{itemize}
\item \textbf{Internet \& Communication}:
	\begin{itemize}
	\item Chat;
	\item ECommerce;
	\item Email;
	\item File Sharing;
	\item Forum \& Blog;
	\item Search Engine;
	\item Social Networks;
	\item Web Hosting.
	\end{itemize}
\item \textbf{News}:
	\begin{itemize}
	\item News;
	\item Newspapers;
	\item Weather.
	\end{itemize}
\item \textbf{People \& Pets}:
	\begin{itemize}
	\item Pets;
	\item Relationship \& Dating.
	\end{itemize}
\item \textbf{Productivity}:
	\begin{itemize}
	\item Collaboration;
	\item \glossaryItem{crm};
	\item Data Analysis;
	\item \glossaryItem{erp};
	\item Human Resources;
	\item Supply Chain Management;
	\item Transportation.
	\end{itemize}
\item \textbf{Vehicles}:
	\begin{itemize}
	\item Aviation;
	\item Boating;
	\item Car Buying \& Rentals;
	\item Cars;
	\item Motorcycles;
	\item Railways.
	\end{itemize}
\end{itemize}

\section{Il futuro ed i competitors}
Quello delle soluzioni \glossaryItem{idaas} è un mercato recente ed in forte espansione, introdotto dagli analisti del gruppo Gartner come categoria chiave dell’\glossaryItem{it} solamente nel 2014. Confrontando i due \textit{report} (corredati dai celebri \textit{Magic Quadrant}, figure~\ref{fig:mq2015} e \ref{fig:mq2016}) prodotti da Gartner nel giugno 2015 e giugno 2016, infatti, si può notare come, nell'arco di un solo anno, molte aziende abbiano deciso di investire in tale ambito sviluppando e facendo crescere i propri prodotti. Particolarmente significativa è la presenza e la crescita di molte compagnie \textit{leader} nel settore \glossaryItem{it}, quali, ad esempio, Microsoft, Salesforce e IBM.

\begin{figure}[h]
\centering
\mbox{
	\begin{subfigure}[b]{\textwidth}
    \centering
    \includegraphics[scale=0.5,clip=false]{mq2015}
	\caption[Giugno 2015]{Giugno 2015\protect\footnotemark}
	\label{fig:mq2015}
    \end{subfigure}
}

\mbox{
	\begin{subfigure}[b]{\textwidth}
    \centering
    \includegraphics[scale=0.5,clip=false]{mq2016}
    \caption[Giugno 2016]{Giugno 2016\protect\footnotemark}
    \label{fig:mq2016}
    \end{subfigure}
}
\caption{Gartner Magic Quadrants}\label{fig:confrontomq}
\end{figure}
\footnotetext{Fonte: \url{http://get.onelogin.com/GartnerMQ-2015.html}}
\footnotetext{Immagine tratta da \cite{grt:G00279633}}

Nonostante ciò, e in particolare nonostante la crescita di Microsoft e Centrify, Gartner ha identificato, anche nel 2016, Okta come principale \textit{leader} del mercato \glossaryItem{idaas}. Okta, Inc. è stata fra le prime aziende a capire il potenziale dell'emergente ambito dell’\glossaryItem{idaas}. Ciò che ha determinato il suo successo e l'ha eletta \textit{leader} anche per il 2016 è l'ampia gamma di funzionalità offerte dal suo prodotto, la semplicità nell'utilizzo e l'impegno nel supporto e nell'integrazione del suo sistema con \glossaryItemPl{serviziodirectory} aziendali, come \glossaryItem{ad}.

Gartner stabilisce dei ''requisiti minimi'' per poter partecipare alla ricerca, e divide i partecipanti nelle quattro aree che formano il \textit{Magic Quadrant}:
\begin{itemize}
\item \textbf{leaders}: i molti clienti sono soddisfatti e le funzionalità offerte sono al passo con le richieste del mercato. I servizi offerti sono molti e il supporto è di alto livello;
\item \textbf{challengers}: sono sulla buona strada per diventare \textit{leaders}, ma la loro visione complessiva dell'\glossaryItem{idaas} è troppo limitata o focalizzata solo su un'area specifica. I clienti sono generalmente soddisfatti, ma devono fare richieste specifiche per nuove funzionalità;
\item \textbf{visionaries}: i loro prodotti incontrano i bisogni della maggior parte dei clienti, ma questi ultimi non sono sufficienti a renderli dei \textit{leaders}. Sono molto innovativi, e spesso presentano funzionalità che gli altri non hanno;
\item \textbf{niche players}: i loro prodotti sono adatti a specifici casi d'uso, come particolari settori industriali. Hanno di solito pochi clienti, e i prezzi sono troppo alti per permettere di essere competitivi sul mercato. Ad ogni modo, nel loro ambito possono essere molto conosciuti ed apprezzati.
\end{itemize}

\paragraph{Posizionamento di Monokee} \mbox{} \\
Nello sviluppo del suo prodotto, \textit{iVoxIT S.r.l} ha analizzato a fondo le funzionalità offerte dalle aziende \textit{competitor} ed ha dedicato particolare attenzione ad Okta quale riferimento del mercato. Lo studio di tali realtà ha portato all'individuazione di alcuni punti critici attorno ai quali l'azienda si è concentrata al fine di trovare delle soluzioni efficaci ed innovative che le permettano di distinguersi nell'ambito \glossaryItem{idaas}. Un esempio è la gestione dei domini già citata. In particolare, l'azienda punta ad affermarsi come realtà importante in ambito europeo, collocandosi in uno spazio che le permetta ampie possibilità di sviluppo e crescita (attualmente, fra le aziende indicate nel \textit{Magic Quadrant} di Gartner, solo iWelcome opera nativamente in Europa).
\chapter{Tecnologie, strumenti e linguaggi utilizzati}\label{tecnologie}
\section{Tecnologie}
Di seguito saranno presentate le tecnologie principali utilizzate durante l'attività di stage. Lo \textit{stack} tecnologico usato è denominato \textbf{MEAN} (Figura~\ref{fig:mean}):
\begin{itemize}
\item \textbf{M}ongoDB, \textit{database} non relazionale;
\item \textbf{E}xpress.js, \glossaryItem{framework} per applicazioni web eseguite su Node.js;
\item \textbf{A}ngular.js, \glossaryItem{framework} JavaScript eseguito su \textit{browser};
\item \textbf{N}ode.js, ambiente di esecuzione lato server per applicazioni ad eventi e relative comunicazioni.
\end{itemize}
\begin{figure}[hbpc]
\begin{center}
\includegraphics[scale=0.3]{mean}
\caption[Funzionamento dello stack MEAN]{Funzionamento dello stack MEAN\protect\footnotemark}
\label{fig:mean}
\end{center}
\end{figure}
\footnotetext{Fonte: \url{http://trinathswebapps.blogspot.it/2016/01/history-of-angularjs.html}}
In particolare, durante l'attività di stage sono state utilizzate le tecnologie lato server, quindi MongoDB, Express.js e Node.js, che saranno descritte in seguito.

\subsection{Node.js}

\paragraph{Asincronismo} \mbox{} \\
In Figura~\ref{fig:confrontoNodeJava} viene mostrato un confronto tra un server Java e uno Node.js. La differenza principale è che Node.js permette la realizzazione di \textbf{server asincroni non bloccanti}. Lo stile architetturale orientato agli eventi lo rende particolarmente adatto a catturare la reattività di un'applicazione web.

Quando un server viene invocato in modo sincrono, l'applicazione client che l'ha invocato deve attendere la risposta prima di poter continuare la sua esecuzione. Se la risposta è immediata va tutto bene, ma è ragionevole supporre che la maggior parte delle applicazioni eseguite da un server impieghino del tempo per essere eseguite, con conseguente rallentamento dell'intero sistema.

Se il server viene invocato in modo asincrono, però, il client non deve aspettare la risposta del server e può proseguire con l'esecuzione. 
\begin{figure}[hbpc]
    \begin{center}
    \begin{subfigure}[b]{0.6\textwidth}
        \includegraphics[width=\textwidth]{threading_node}
        \caption{Server Node.js}
        \label{fig:servernode}
    \end{subfigure}
    ~ %add desired spacing between images, e. g. ~, \quad, \qquad, \hfill etc. 
      %(or a blank line to force the subfigure onto a new line)
    \begin{subfigure}[b]{0.6\textwidth}
        \includegraphics[width=\textwidth]{threading_java}
        \caption{Server Java}
        \label{fig:serverjava}
    \end{subfigure}
    \caption[Confronto tra server Node.js e server Java]{Confronto tra server Node.js e server Java\protect\footnotemark}\label{fig:confrontoNodeJava}
    \end{center}
\end{figure}
\footnotetext{Fonte: \url{https://strongloop.com/strongblog/node-js-is-faster-than-java/}}

In Figura~\ref{fig:funzionamentoNode} viene mostrato il funzionamento di un server Node.js nel contesto di una richiesta di \glossaryItem{io}. Le funzionalità di \glossaryItem{io} sono molto importanti nell'architettura di Node.js. Un unico \glossaryItem{thread} (il \textbf{main \glossaryItem{thread}}) rimane in ascolto di connessioni. Quando si riceve una richiesta il \glossaryItem{thread} esegue l'evento richiesto, qualunque esso sia. Tuttavia, se l'operazione richiede del tempo, come una lettura da \glossaryItem{filesystem} o da \textit{database}, viene aperto un altro \glossaryItem{thread} (il \textbf{worker \glossaryItem{thread}}) che si occupa di eseguire quanto richiesto. Questo meccanismo di delega avviene attraverso l'uso di una funzione di \textit{callback}, ovvero una funzione invocata al termine dell'esecuzione di un'altra. Grazie a questo meccanismo di ''chiamata all'indietro'' il \glossaryItem{thread} delegato può notificare al \glossaryItem{thread} principale il completamento dell'operazione richiesta. Nel frattempo quest'ultimo è libero di continuare con l'esecuzione del programma principale. 
\begin{figure}[h]
\begin{center}
\includegraphics[scale=0.3]{funzionamentoNode}
\caption[Funzionamento di un server Node.js]{Funzionamento di un server Node.js\protect\footnotemark}
\label{fig:funzionamentoNode}
\end{center}
\end{figure}
\footnotetext{Immagine tratta da \cite{doglio05}}

Questo meccanismo asincrono migliora notevolmente le \textit{performance} e l'utilizzo delle risorse, ma presenta anche degli aspetti negativi. Il controllo del flusso di esecuzione risulta più complesso, e l'eccessivo uso di \textit{callback} può portare ad un rallentamento.

Comunque sia, per applicazioni web molto orientate all'\glossaryItem{io} l'uso di Node.js, rispetto ad un tradizionale server Java, consente un miglioramento complessivo delle \textit{performance} e ad una riduzione dei tempi di attesa.

\paragraph{Performance} \mbox{} \\
In Figura~\ref{fig:nodejavaperf} si riporta un confronto di prestazioni tra un server Node.js e uno Java effettuato da Paypal (\cite{site:paypalNode}) e basato sulla stessa applicazione scritta nei due modi diversi.

La prima differenza evidente riguarda il \textbf{numero di richieste al secondo}: Node.js riesce a gestirne il doppio rispetto a Java, e soprattutto lo fa dimezzando il \textbf{tempo di risposta medio} (questa è la seconda notevole differenza).
\begin{figure}[h]
\begin{center}
\includegraphics[scale=0.3]{node_java_perf}
\caption[Confronto di prestazioni tra Java e Node.js]{Confronto di prestazioni tra Java e Node.js\protect\footnotemark}
\label{fig:nodejavaperf}
\end{center}
\end{figure}
\footnotetext{Immagine tratta da \cite{site:paypalNode}}
\newpage
\subsection{Express.js} \label{express}
Per lo sviluppo del back end si è fatto uso di \textbf{Express.js} (logo in Figura~\ref{fig:express}), un \glossaryItem{framework} che costituisce lo standard \textit{de facto} per la creazione di servizi \glossaryItem{rest} in ambiente Node.js. Le sue caratteristiche principali sono il \textbf{routing} e le funzioni \textbf{middleware}.
\begin{figure}[hbpc]
\begin{center}
\includegraphics[scale=0.4]{express}
\caption[Logo di Express.js]{Logo di Express.js\protect\footnotemark}
\label{fig:express}
\end{center}
\end{figure}
\footnotetext{Fonte: \url{http://expressjs.com/}}

\paragraph{Routing} \mbox{} \\
Express.js fornisce un metodo semplice per gestire gli \textit{endpoint} a cui l'applicazione risponde. Per creare una \textit{route} è sufficiente utilizzare l'oggetto \texttt{Router}, specificando a quale metodo \glossaryItem{http} (\textbf{GET}, \textbf{POST}, \textbf{PUT} o \textbf{DELETE}) e \glossaryItem{uri} rispondere. In risposta all'invocazione è possibile inviare qualsiasi tipo di dato, dai più semplici ai più complessi, utilizzando i metodi che l'oggetto \texttt{Response} di Express.js fornisce.

\paragraph{Middleware} \mbox{} \\
Per \textit{middleware} si intendono delle strutture che permettono di astrarre funzionalità locali in funzionalità distribuite. In Express.js, le funzioni \textit{middleware} permettono di eseguire delle operazioni prima, durante o dopo l'esecuzione del codice associato all'\textit{endpoint} invocato, modificando gli oggetti associati alla richiesta, terminando l'invocazione o chiamando un altro \textit{middleware}.

Possono essere di vario tipo:
\begin{itemize}
\item a livello di \textbf{applicazione}: definiti per ogni possibile \textit{endpoint};
\item a livello di \textbf{router}: definiti quindi per gli \textit{endpoints} associati a quel \textit{router};
\item per \textbf{gestione degli errori};
\item di \textbf{terze parti}: possono essere importati per aggiungere delle funzionalità all'applicazione.
\end{itemize}

\subsection{MongoDB}
Il \glossaryItem{dbms} utilizzato è \textbf{MongoDB} (logo in Figura~\ref{fig:mongodb}). 

\begin{figure}[hbpc]
\begin{center}
\includegraphics[scale=0.2]{mongodb}
\caption[Logo di MongoDB]{Logo di MongoDB\protect\footnotemark}
\label{fig:mongodb}
\end{center}
\end{figure}
%\footnotetext{Fonte: \url{https://www.mongodb.com/}}

L'integrazione con Node.js è effettuata grazie al modulo \textbf{mongoose.js}, che semplifica le operazioni che coinvolgono il database aggiungendo metodi per validare e controllare i dati e per eseguire \textit{query}. 

Per facilitare la lettura dei dati, inoltre, è stato utilizzato \textbf{Robomongo}, che verrà descritto in \hyperref[sec:robomongo]{seguito}.

\paragraph{Confronto con un database relazionale} \mbox{} \\
MongoDB è un \textit{database} non relazionale (o \glossaryItem{nosql}) appartenente alla famiglia dei \textit{database} \textbf{orientati ai documenti}: è in grado quindi di memorizzare oggetti con struttura complessa e non fissata inizialmente. I \textbf{documenti} memorizzati vengono organizzati in \textbf{collezioni}, con la possibilità di definire indici di vario tipo per velocizzare le ricerche ed esprimere vincoli. L'utilizzo di tali indici è inoltre necessario per mantenere le relazioni tra collezioni di documenti diversi.

In Tabella~\ref{tab:terminiSQLNoSQL} è mostrato un confronto di terminologia tra i due tipi di \textit{database}.
\begin{table}[h]
\centering
\caption{Confronto di terminologia tra un database relazionale e MongoDB}
\label{tab:terminiSQLNoSQL}
\begin{tabular}{|l|l|}
\hline
\textbf{Database relazionale} & \textbf{MongoDB} \\ \hline
Database                      & Database         \\ \hline
Tabella                       & Collezione       \\ \hline
Riga                          & Documento        \\ \hline
Colonna                       & Campo	         \\ \hline
\end{tabular}
\end{table}

\newpage
\paragraph{Caratteristiche principali} \mbox{} \\
Le caratteristiche principali di MongoDB sono:
\begin{itemize}
\item \textbf{Prestazioni elevate}: MongoDB riesce a garantire alte prestazioni nella persistenza dei dati. Fornisce modelli che riducono le operazioni di \glossaryItem{io} nel \textit{database} e consente la creazioni di indici su qualsiasi tipo di dato.
\item \textbf{Alta disponibilità}: MongoDB è provvisto di mezzi di replica, chiamati \textit{replica sets}, ovvero gruppi di server che mantengono lo stesso insieme di dati.
\item \textbf{Scalabilità automatica}: MongoDB scala, automaticamente, in orizzontale, ovvero aumenta le prestazioni aggiungendo altre macchine.
\end{itemize}

\section{Strumenti}
Nel corso del progetto sono stati utilizzati numerosi strumenti di supporto per facilitare lo svolgimento delle diverse attività, dal controllo di versione all'analisi del codice scritto. La scelta si è basata su un attento studio delle funzionalità offerte, sul supporto presente \textit{online} e sulla facilità di utilizzo, in modo da assicurare un'alta qualità. Di seguito saranno descritte le principali funzionalità di ogni strumento.

\subsection{Git}
Git (logo in Figura~\ref{fig:git}) è un \textit{software} di controllo di versione distribuito creato da Linus Torvalds nel 2005. Il controllo di versione permette di tener traccia di tutte le versioni di un progetto, con la possibilità di ripristinare un file o un intero lavoro ad uno stato precedente, evitando quindi la perdita dei dati e le sovrascritture accidentali.

Le sue principali caratteristiche sono:
\begin{itemize}
\item supporto allo sviluppo non lineare (\glossaryItem{branching} e \glossaryItem{merging});
\item sviluppo distribuito: ad ogni sviluppatore viene fornita una copia locale dell'intera cronologia di sviluppo. Quasi tutte le operazioni fatte da Git sono in locale, quindi non è soggetto a latenze di rete.
\item gestione efficiente di grandi progetti: git è veloce e scalabile;
\item gestione della cronologia: il nome di un \textit{commit} dipende dall'intera cronologia di sviluppo che ha condotto a quel \textit{commit}.
\end{itemize}
\begin{figure}[h]
\begin{center}
\includegraphics[scale=0.6]{git}
\caption[Logo di Git]{Logo di Git\protect\footnotemark}
\label{fig:git}
\end{center}
\end{figure}
\footnotetext{Fonte: \url{https://git-scm.com/} (\cite{site:git})}

\subsection{Bitbucket}
Bitbucket (logo in Figura~\ref{fig:bitbucket}) è un sistema web di \textit{hosting}, dedicato a progetti che usano Mercurial o git per il controllo di versione. Permette di avere un \textit{account} gratuito e un numero illimitato di \glossaryItemPl{repository} private con la possibilità di aggiungere al progetto fino a cinque membri del \textit{team}. Bitbucket offre un sistema di discussione su codice sorgente con commenti in linea, visualizzazione per \textit{branch} o \textit{tag} per vedere i progressi del \textit{team} e richieste di \textit{pull}.
\begin{figure}[hbpc]
\begin{center}
\includegraphics[scale=0.3]{bitbucket}
\caption[Logo di Bitbucket]{Logo di Bitbucket\protect\footnotemark}
\label{fig:bitbucket}
\end{center}
\end{figure}
\footnotetext{Fonte: \url{https://bitbucket.org/} (\cite{site:bitbucket})}

\subsection{JSHint}
JSHint è uno strumento di analisi statica del codice usato per controllare se il codice JavaScript è conforme ad alcune regole di codifica. È presente sia una versione installabile come modulo di Node.js che una \textit{online}, oltre ad una grande varietà di \textit{plug ins} per i principali \textit{editor} di testo.

\subsection{Mocha.js} \label{mocha}
Mocha.js è un \glossaryItem{framework} di \textit{test} per JavaScript eseguito su ambiente Node.js. Tra le sue caratteristiche principali troviamo:
\begin{itemize}
\item supporto per i \textit{test} su \textit{browser};
\item \textit{testing} asincrono;
\item report sulla copertura dei \textit{test};
\item utilizzo di una qualsiasi libreria di asserzioni.
\end{itemize}
In particolare, come libreria di asserzioni è stato usato Express.js.
\begin{figure}[h]
\begin{center}
    \includegraphics[scale=0.08]{mocha}
    \caption{Logo di Mocha.js}
    \label{fig:mocha}
\end{center}
\end{figure}
\footnotetext{Fonte: \url{https://mochajs.org/}}

\subsection{DHC} \label{dhc}
DHC (logo in Figura~\ref{fig:dhc}) è uno strumento per rendere più semplice l'uso
e il \textit{testing} delle risorse \glossaryItem{HTTP}/\glossaryItem{REST} ed è disponibile come \textit{plug in} nel \textit{browser} Google Chrome. Oltre alla sua funzione principale: inviare o ricevere rispettivamente richieste o risposte \glossaryItem{HTTP}/\glossaryItem{REST}, esso permette di salvare in modo permanente una richiesta e le sue variabili in un \glossaryItem{repository} locale per un successivo riutilizzo.
\begin{figure}[hbpc]
\begin{center}
\includegraphics[scale=0.5]{dhc}
\caption[Logo di DHC]{Logo di DHC\protect\footnotemark}
\label{fig:dhc}
\end{center}
\end{figure}
\footnotetext{Fonte: \url{https://restlet.com/products/dhc/}}

\subsection{Robomongo} \label{sec:robomongo}
Robomongo è uno strumento di gestione di MongoDB che permette di utilizzare tutte le funzionalità della \textit{shell} di MongoDB fornendo però un'intuitiva interfaccia grafica. Tra i vantaggi vi sono, oltre alla maggiore leggibilità dei dati contenuti nel \textit{database}, la possibilità di avere più finestre con i risultati visibili contemporaneamente, di poter visualizzare nella stessa \textit{shell} i risultati di due \textit{query} differenti e un aiuto nella scrittura delle interrogazioni, dovuta alla funzione di auto-completamento.
\begin{figure}[hbpc]
\begin{center}
\includegraphics[scale=0.5]{robomongo}
\caption[Logo di Robomongo]{Logo di Robomongo\protect\footnotemark}
\label{fig:robomongo}
\end{center}
\end{figure}
\footnotetext{Fonte: \url{https://robomongo.org/}}

\subsection{APIDoc e JSDoc}
Per documentare il codice JavaScript e le \glossaryItem{api} scritte sono stati usati, rispettivamente, JSDoc e APIDoc, che consentono di documentare \textit{inline} il codice e di generare, a partire dalla documentazione, dei piccoli siti web che consentono una facile e veloce consultazione.

\section{Linguaggi}
L'unico linguaggio utilizzato è JavaScript: la scelta è stata imposta dal fatto che l'applicazione viene eseguita su un server Node.js. Di seguito verranno descritte le principali caratteristiche di questo linguaggio. 

\subsection{JavaScript}
JavaScript (logo in Figura~\ref{fig:javascript}) è un linguaggio di programmazione importante perché è il linguaggio più usato nei \textit{browser} web, ma al contempo è uno dei più ''disprezzati''. Presenta notevoli differenze rispetto a tutti gli altri e molte delle sue caratteristiche possono essere viste come un bene o un male. 
\begin{figure}[hbpc]
\begin{center}
\includegraphics[scale=0.2]{javascript}
\caption[Logo di JavaScript]{Logo di JavaScript\protect\footnotemark}
\label{fig:javascript}
\end{center}
\end{figure}
\footnotetext{Fonte: \url{https://code.support/category/code/javascript/}}


Una di queste è la tipizzazione debole. In un linguaggio fortemente tipizzato il compilatore è in grado di riconoscere gli errori di tipo e di segnalarli al programmatore, impedendo che questi si tramutino in errori di esecuzione difficili da trovare. D'altro canto, però, nel mondo delle applicazioni web una tipizzazione forte costringerebbe ad usare complicate gerarchie di classi che in JavaScript possono essere evitate. Il lato negativo è che il programmatore deve essere molto più attento e sapere molto meglio quello che sta facendo. Questo si traduce in un maggior numero di \textit{test}. 

Un punto forte è la facilità con cui possono essere creati gli oggetti: basta semplicemente elencare tutte le componenti (o proprietà) di un oggetto per crearlo. In JavaScript anche le funzioni sono oggetti, quindi i metodi si definiscono esattamente come gli attributi. Tuttavia, come sarà descritto in seguito, questa flessibilità porta alla mancanza di incapsulazione.

Un altro punto controverso è l'ereditarietà prototipale. In JavaScript gli oggetti possono ereditare liberamente proprietà da altri oggetti: questo meccanismo è molto potente, ma è difficile padroneggiarlo, soprattutto per chi è abituato all'ereditarietà ''classica'' dei linguaggi fortemente tipizzati. Il problema si riflette soprattutto nell'applicazione dei \textit{design pattern} per la progettazione ad oggetti: è impossibile applicarli così come sono, e imparare ad applicarli può risultare frustrante. 

L'enorme diffusione di JavaScript è dovuta principalmente al fiorire di numerose librerie nate allo scopo di semplificare la programmazione sul \textit{browser}, ma anche alla nascita di \glossaryItem{framework} lato server e nel mondo \textit{mobile} che lo supportano come linguaggio principale. Node.js, infatti, si basa su JavaScript. Di conseguenza ogni applicazione eseguita su un server Node.js deve essere scritta in JavaScript. 

\chapter{Analisi dei requisiti}\label{adr}
\section{Attori coinvolti}
L'attore principale dell'applicazione Catalogue Manager è l'Utente Autenticato. Una gerarchia di utenti, ognuno associato a permessi e funzionalità crescenti, è prevista per le successive versioni del gestore del catalogo applicativo di Monokee, ma non è stata inserita tra i requisiti del progetto per volontà del tutor aziendale. 

L'autenticazione è di tipo federato e fa uso di \glossaryItem{saml}: non è pertanto presente una fase di registrazione. L'applicazione ha il ruolo di \glossaryItem{sp} e utilizza Monokee come \glossaryItem{idp}. Un utente non autenticato (ma autenticato presso Monokee) dipendentemente dal suo ruolo può:
\begin{itemize}
\item aggiungere manualmente l'applicazione ad un dominio;
\item accedere all'applicazione.
\end{itemize}
Tutte le funzionalità di Catalogue Manager sono accessibili solamente dopo aver effettuato l'accesso e, come già detto, non esiste nessuna distinzione tra utenti: questo significa che un amministratore di dominio e un semplice utente di Monokee in Catalogue Manager hanno gli stessi poteri.

\section{Casi d'uso ad alto livello}
Considerando l'elevato numero di funzionalità previste in Catalogue Manager di seguito vengono riportati solo i casi d'uso di alto livello, in modo da poter dare una visione d'insieme più precisa del prodotto sviluppato.

Ad ogni caso d'uso è associato un identificatore univoco così formato:

\begin{center}
\textbf{UCTX.Y.Z}
\end{center}
dove:
\begin{itemize}
\item \textbf{T} corrisponde al tipo di caso d'uso:
	\begin{itemize}
	\item \textbf{U} per l'Utente Non Autenticato;
	\item \textbf{A} per l'Utente Autenticato;
	\end{itemize}
\item \textbf{X} corrisponde all'identificatore del caso d'uso ''padre'';
\item \textbf{Y} corrisponde all'identificatore del caso d'uso ''figlio'' di primo livello;
\item \textbf{Z} corrisponde all'identificatore del caso d'uso ''figlio'' di secondo livello;
\end{itemize}
Man mano che si scende nella gerarchia (da padre a figlio di primo e poi secondo livello) si aumenta il dettaglio del caso d'uso.

\subsection{Operazioni permesse ad un Utente Non Autenticato}
\begin{figure}[hbpc]
  \begin{center}
    \includegraphics[width=12cm]{UC/UtenteNonAutenticato}
  \caption[Operazioni Generali per l'Utente Non Autenticato]{Operazioni di alto livello permesse ad un Utente Non Autenticato}
  \end{center} 
\end{figure}

\begin{center}
  \bgroup
  \def\arraystretch{1.8}     
  \begin{longtable}{  p{3.5cm} | p{8cm} } 
    \hline
    \multicolumn{2}{ | c | }{ \cellcolor[gray]{0.9} \textbf{Operazioni generali per l'Utente Non Autenticato}} \\
    \textbf{Attori Primari} & Utente Non Autenticato \\ 
    \textbf{Scopo e Descrizione} & L'Utente Non Autenticato può aggiungere, tramite le funzionalità di Monokee, l'applicazione Catalogue Manager ad un \textit{application broker} di un dominio. Dopo averla aggiunta può decidere in qualsiasi momento di cambiarne gli attributi di configurazione attraverso la pagina di modifica di Monokee. Infine può accedere all'applicazione per gestire il catalogo applicativo. Si ricorda che, sebbene venga definito come Utente Non Autenticato, l'utente in questione deve essere autenticato ed autorizzato tramite Monokee. \\ 
    \textbf{Precondizioni}  & L'Utente Non Autenticato ha effettuato l'accesso a Monokee. \\
    \textbf{Postcondizioni} & Monokee ha preso in carico, ed eseguito, l'operazione voluta dall'Utente Non Autenticato.  \\ 
    \textbf{Flusso Principale} & 
    1. L'Utente Non Autenticato aggiunge l'applicazione Catalogue Manager ad un \textit{application broker} di un dominio di Monokee. (UCU1) \newline
    2. L'Utente Non Autenticato modifica gli attributi di Catalogue Manager. (UCU2) \newline
    3. L'Utente Non Autenticato accede a Catalogue Manager. (UCU3) \\
    \textbf{Estensioni} & L'Utente Non Autenticato visualizza un messaggio di errore perché non è autorizzato ad accedere a Catalogue Manager. (UCU4)
  \end{longtable}
  \egroup
\end{center}

\subsubsection{UCU1 - Aggiunta di Catalogue Manager ad un Application Broker}
\begin{center}
  \bgroup
  \def\arraystretch{1.8}     
  \begin{longtable}{  p{3.5cm} | p{8cm} } 
    \multicolumn{2}{ | c | }{ \cellcolor[gray]{0.9} \textbf{UCU1 - Aggiunta di Catalogue Manager ad un Application Broker}} \\
    \hline
    
    \textbf{Attori Primari} & Utente Non Autenticato \\ 
    \textbf{Scopo e Descrizione} & L'Utente Non Autenticato può aggiungere l'applicazione Catalogue Manager ad un \textit{application broker} di un dominio di Monokee. Dato che l'\glossaryItem{autenticazione} avviene tramite \glossaryItem{saml}, per aggiungere l'applicazione è necessario inserire tutti i parametri richiesti da questo standard. \\ 
    
    \textbf{Precondizioni}  & L'Utente Non Autenticato ha effettuato l'accesso a Monokee e si trova nella pagina dell'aggiunta di una nuova applicazione \glossaryItem{saml}. \\ 
    
    \textbf{Postcondizioni} & Monokee ha aggiunto all'\textit{application broker} del dominio selezionato dall'Utente Non Autenticato l'applicazione Catalogue Manager. \\ 
    \textbf{Flusso Principale} & 
    1. L'Utente Non Autenticato inserisce l'\glossaryItem{url} dell'\textit{assertion consumer service}. \newline
    2. L'Utente Non Autenticato inserisce l'\glossaryItem{uri} del \glossaryItem{sp}. \newline
    3. L'Utente Non Autenticato inserisce il certificato del \glossaryItem{sp}. \newline
    4. L'Utente Non Autenticato inserisce l'\glossaryItem{url} della pagina successiva al \textit{login}. \newline
    5. L'Utente Non Autenticato inserisce le regole dell'asserzione \glossaryItem{saml}. \newline
    6. L'Utente Non Autenticato inserisce l'algoritmo di firma. \newline
    7. L'Utente Non Autenticato inserisce l'\glossaryItem{uri} della pagina di \textit{log out}. \newline
    8. L'Utente Non Autenticato inserisce l'\glossaryItem{uri} della pagina di risposta al \textit{log out}. 
  \end{longtable}
  \egroup
\end{center}

\subsubsection{UCU2 - Modifica della configurazione specifica di un dominio di Catalogue Manager}
\begin{center}
  \bgroup
  \def\arraystretch{1.8}     
  \begin{longtable}{  p{3.5cm} | p{8cm} } 
    \multicolumn{2}{ | c | }{ \cellcolor[gray]{0.9} \textbf{UCU2 - Modifica della configurazione specifica di un dominio di Catalogue Manager}} \\
    \hline
    
    \textbf{Attori Primari} & Utente Non Autenticato \\ 
    \textbf{Scopo e Descrizione} & L'Utente Non Autenticato può modificare gli attributi di configurazione di Catalogue Manager per un dominio tramite la pagina di modifica di un'applicazione prevista da Monokee. \\ 
    
    \textbf{Precondizioni}  & L'Utente Non Autenticato ha effettuato l'accesso a Monokee e si trova nella pagina di modifica di un'applicazione esistente. \\ 
    
    \textbf{Postcondizioni} & Monokee ha modificato l'applicazione Catalogue Manager per il dominio selezionato dall'Utente Non Autenticato. \\ 
    \textbf{Flusso Principale} & 
    1. L'Utente Non Autenticato modifica l'\glossaryItem{url} dell'\textit{assertion consumer service}. \newline
    2. L'Utente Non Autenticato modifica l'\glossaryItem{uri} del \glossaryItem{sp}. \newline
    3. L'Utente Non Autenticato modifica il certificato del \glossaryItem{sp}. \newline
    4. L'Utente Non Autenticato modifica l'\glossaryItem{url} della pagina successiva al \textit{login}. \newline
    5. L'Utente Non Autenticato modifica le regole dell'asserzione \glossaryItem{saml}. \newline
    6. L'Utente Non Autenticato modifica l'algoritmo di firma. \newline
    7. L'Utente Non Autenticato modifica l'\glossaryItem{uri} della pagina di \textit{log out}. \newline
    8. L'Utente Non Autenticato modifica l'\glossaryItem{uri} della pagina di risposta al \textit{log out}. 
  \end{longtable}
  \egroup
\end{center}

\subsubsection{UCU3 - Accesso a Catalogue Manager}
\begin{center}
  \bgroup
  \def\arraystretch{1.8}     
  \begin{longtable}{  p{3.5cm} | p{8cm} } 
    \multicolumn{2}{ | c | }{ \cellcolor[gray]{0.9} \textbf{UCU3 - Accesso a Catalogue Manager}} \\
    \hline
    
    \textbf{Attori Primari} & Utente Non Autenticato \\ 
    \textbf{Scopo e Descrizione} & L'Utente Non Autenticato può accedere a Catalogue Manager tramite l'\textit{application broker} di un dominio di Monokee. \\ 
    
    \textbf{Precondizioni}  & L'Utente Non Autenticato ha effettuato l'accesso a Monokee e si trova nell'\textit{application broker} di un dominio. \\ 
    
    \textbf{Postcondizioni} & L'Utente Non Autenticato ha effettuato l'accesso a Catalogue Manager e si trova nella pagina principale della nuova applicazione. \\ 
    \textbf{Flusso Principale} & 
    1. L'Utente Non Autenticato richiede (tramite un \textit{click}) l'accesso a Catalogue Manager. \\
    \textbf{Estensioni} & L'Utente Non Autenticato visualizza un messaggio di errore perché l'\glossaryItem{idp} di Monokee non gli ha consentito l'accesso a Catalogue Manager.
  \end{longtable}
  \egroup
\end{center}

\subsubsection{UCU4 - Visualizzazione messaggio di errore per utente non autorizzato}
\begin{center}
  \bgroup
  \def\arraystretch{1.8}     
  \begin{longtable}{  p{3.5cm} | p{8cm} } 
    \multicolumn{2}{ | c | }{ \cellcolor[gray]{0.9} \textbf{UCU4 - Visualizzazione messaggio di errore per utente non autorizzato}} \\
    \hline
    
    \textbf{Attori Primari} & Utente Non Autenticato \\ 
    \textbf{Scopo e Descrizione} & L'Utente Non Autenticato visualizza un messaggio di errore in seguito all'accesso negato, da parte dell'\glossaryItem{idp} di Monokee, all'applicazione Catalogue Manager. \\ 
    
    \textbf{Precondizioni}  & L'Utente Non Autenticato ha effettuato l'accesso a Monokee e si trova nell'\textit{application broker} di un dominio. ha inoltre richiesto l'accesso all'applicazione Catalogue Manager, ma gli è stata negata dall'\glossaryItem{idp} di Monokee. \\ 
    
    \textbf{Postcondizioni} & L'Utente Non Autenticato ha visualizzato il messaggio di errore per autenticazione non riuscita. \\ 
    \textbf{Flusso Principale} & 
    1. L'Utente Non Autenticato visualizza il messaggio di errore conseguente all'accesso negato a Catalogue Manager.
  \end{longtable}
  \egroup
\end{center} % casi d'uso per utente non autenticato
\newpage
\subsection{Operazioni generali per l'Utente Autenticato}
\begin{figure}[hbpc]
  \begin{center}
    \includegraphics[width=12cm]{UC/UtenteAutenticato}
  \caption[Operazioni Generali per l'Utente Autenticato]{Operazioni di alto livello permesse ad un Utente Autenticato}
  \end{center} 
\end{figure}
\newpage
\begin{center}
  \bgroup
  \def\arraystretch{1.8}     
  \begin{longtable}{  p{3.5cm} | p{8cm} } 
    \hline
    \multicolumn{2}{ | c | }{ \cellcolor[gray]{0.9} \textbf{Operazioni generali per l'Utente Autenticato}} \\
    \textbf{Attori Primari} & Utente Autenticato \\ 
    \textbf{Scopo e Descrizione} & All'Utente Autenticato viene mostrata una \textit{dashboard} a partire dalla quale può eseguire varie operazioni. Queste operazioni permettono di gestire l'intero catalogo applicativo, i raggruppamenti di applicazioni e i cataloghi specifici per i domini aziendali. È possibile anche cercare le applicazioni sulla base di vari filtri, visualizzare statistiche e \textit{log}. \\ 
    \textbf{Precondizioni}  & L'Utente Autenticato ha effettuato l'accesso a Catalogue Manager a partire da Monokee attraverso \glossaryItem{saml}. \\
    \textbf{Postcondizioni} & Catalogue Manager ha preso in carico, ed eseguito, l'operazione voluta dall'Utente Autenticato.  \\ 
    \textbf{Flusso Principale} & 
    1. L'Utente Autenticato visualizza la lista di applicazioni presenti nel catalogo. (UCA1) \newline
    2. L'Utente Autenticato aggiunge un'applicazione al catalogo. (UCA2) \newline
    3. L'Utente Autenticato modifica i dati di un'applicazione del catalogo. (UCA3) \newline 
    4. L'Utente Autenticato rimuove un'applicazione. (UCA4) \newline
    5. L'Utente Autenticato gestisce i cataloghi specifici dei domini aziendali. (UCA5) \newline
    5. L'Utente Autenticato ricerca un'applicazione. (UCA6) \newline
    6. L'Utente Autenticato gestisce i raggruppamenti di applicazioni. (UCA10) \newline
    7. L'Utente Autenticato visualizza le statistiche sul numero di applicazioni e gruppi (UCA14) \newline
    8. L'Utente Autenticato visualizza i \textit{log} sulle operazioni effettuate da lui e dagli altri utenti dell'applicazione. (UCA15) \newline
    I \textit{log} riguardano sia le operazioni eseguite con successo (UCA16) sia quelle che hanno generato errori (UCA17).\\
    \textbf{Estensioni} & L'Utente Autenticato cerca di inserire un'applicazione pubblica che è già presente nel catalogo. (UCA13)
  \end{longtable}
  \egroup
\end{center}

\subsubsection{UCA1 - Visualizzazione Applicazioni}
\begin{center}
  \bgroup
  \def\arraystretch{1.8}     
  \begin{longtable}{  p{3.5cm} | p{8cm} } 
    \multicolumn{2}{ | c | }{ \cellcolor[gray]{0.9} \textbf{UCA1 - Visualizzazione Applicazioni}} \\
    \hline
    
    \textbf{Attori Primari} & Utente Autenticato \\ 
    \textbf{Scopo e Descrizione} & L'Utente Autenticato può visualizzare le applicazioni presenti nel catalogo di Monokee. \\ 
    
    \textbf{Precondizioni}  & Catalogue Manager presenta all'Utente Autenticato una pagina contenente l'elenco delle applicazioni presenti in Monokee. \\ 
    
    \textbf{Postcondizioni} & L'Utente Autenticato ha visualizzato le applicazioni presenti in Monokee. \\ 
    \textbf{Flusso Principale} & 
    1. L'Utente Autenticato visualizza i dettagli di un'applicazione presente in Monokee.   
  \end{longtable}
  \egroup
\end{center}

\subsubsection{UCA2 - Aggiunta Applicazione Pubblica}
\begin{center}
  \bgroup
  \def\arraystretch{1.8}     
  \begin{longtable}{  p{3.5cm} | p{8cm} } 
    \multicolumn{2}{ | c | }{ \cellcolor[gray]{0.9} \textbf{UCA2 - Aggiunta Applicazione Pubblica}} \\
    \hline
    
    \textbf{Attori Primari} & Utente Autenticato \\ 
    \textbf{Scopo e Descrizione} & L'Utente Autenticato può aggiungere una nuova applicazione al catalogo di Monokee. \\ 
    
    \textbf{Precondizioni}  & Catalogue Manager presenta all'Utente Autenticato una pagina contenente un \textit{form} per l'aggiunta di un'applicazione al catalogo di Monokee. \\ 
    
    \textbf{Postcondizioni} & L'Utente Autenticato ha aggiunto un'applicazione al catalogo di Monokee. \\ 
    \textbf{Flusso Principale} & 
    1. L'Utente Autenticato inserisce il nome dell'applicazione. \newline
    2. L'Utente Autenticato inserisce la descrizione dell'applicazione. \newline
    3. L'Utente Autenticato inserisce l'\glossaryItem{url} dell'applicazione. \newline
    4. L'Utente Autenticato inserisce l'immagine dell'applicazione. \newline
    5. L'Utente Autenticato seleziona le categorie di appartenenza dell'applicazione. \newline
    6. L'Utente Autenticato seleziona la tipologia di \glossaryItem{autenticazione}. \newline
    7. L'Utente Autenticato seleziona il nome del gruppo di applicazioni nel quale inserire l'applicazione. \\
    \textbf{Estensioni} & 
        1. L'Utente Autenticato visualizza un messaggio di errore come conseguenza al tentativo di inserimento dello stesso nome di un'altra applicazione pubblica di Monokee.
  \end{longtable}
  \egroup
\end{center}

\subsubsection{UCA3 - Modifica Applicazione Pubblica}
\begin{center}
  \bgroup
  \def\arraystretch{1.8}     
  \begin{longtable}{  p{3.5cm} | p{8cm} } 
    \multicolumn{2}{ | c | }{ \cellcolor[gray]{0.9} \textbf{UCA3 - Modifica Applicazione Pubblica}} \\
    \hline
    
    \textbf{Attori Primari} & Utente Autenticato \\ 
    \textbf{Scopo e Descrizione} & L'Utente Autenticato può modificare un'applicazione esistente nel catalogo di Monokee. \\ 
    
    \textbf{Precondizioni}  & Catalogue Manager presenta all'Utente Autenticato una pagina contenente un \textit{form} per la modifica di un'applicazione nel catalogo di Monokee. \\ 
    
    \textbf{Postcondizioni} & L'Utente Autenticato ha modificato un'applicazione nel catalogo Monokee. \\ 
    \textbf{Flusso Principale} & 
    1. L'Utente Autenticato modifica il nome dell'applicazione. \newline
    2. L'Utente Autenticato modifica la descrizione dell'applicazione. \newline
    3. L'Utente Autenticato modifica l'\glossaryItem{url} dell'applicazione. \newline
    4. L'Utente Autenticato modifica l'immagine dell'applicazione. \newline
    5. L'Utente Autenticato seleziona le categorie di appartenenza dell'applicazione. \newline
    6. L'Utente Autenticato modifica i dati necessari all'\glossaryItem{autenticazione}. \newline
    	Tali dati possono riguardare l'accesso \textit{form-based}, tramite \glossaryItem{saml} o di terzo tipo. \newline
    7. L'Utente Autenticato seleziona il nome del gruppo di applicazioni nel quale modificare l'applicazione. \newline
    8. L'Utente Autenticato decide se pubblicare l'applicazione. \newline
    9. L'Utente Autenticato decide se mettere l'applicazione in manutenzione.\\
    \textbf{Estensioni} & 
    1. L'Utente Autenticato visualizza un messaggio di errore come conseguenza al tentativo di inserimento dello stesso nome di un'altra applicazione pubblica di Monokee.
  \end{longtable}
  \egroup
\end{center}

\subsubsection{UCA4 - Rimozione Applicazione Pubblica}
\begin{center}
  \bgroup
  \def\arraystretch{1.8}     
  \begin{longtable}{  p{3.5cm} | p{8cm} } 
    \multicolumn{2}{ | c | }{ \cellcolor[gray]{0.9} \textbf{UC4 - Rimozione Applicazione Pubblica}} \\
    \hline
    \textbf{Attori Primari} & Utente Autenticato \\ 
    \textbf{Scopo e Descrizione} & L'Utente Autenticato può rimuovere un'applicazione pubblica dal catalogo di Monokee. \\ 
    
    \textbf{Precondizioni}  & L'applicazione selezionata è presente nel catalogo di Monokee e l'Utente Autenticato ha selezionato il comando di rimozione su di essa. \\ 
    
    \textbf{Postcondizioni} & L'applicazione selezionata è stata rimossa dal catalogo di Monokee. \\ 
    \textbf{Flusso Principale} &
    1. L'Utente Autenticato conferma l'operazione di rimozione. \\
    \textbf{Inclusioni} & Richiesta di conferma rimozione.
  \end{longtable}
  \egroup
\end{center}

\subsubsection{UCA5 - Gestione Cataloghi di Dominio}
\begin{center}
  \bgroup
  \def\arraystretch{1.8}     
  \begin{longtable}{  p{3.5cm} | p{8cm} } 
    \multicolumn{2}{ | c | }{ \cellcolor[gray]{0.9} \textbf{UCA5 - Gestione Cataloghi di Dominio}} \\
    \hline
    
    \textbf{Attori Primari} & Utente Autenticato \\ 
    \textbf{Scopo e Descrizione} & L'Utente Autenticato può gestire i cataloghi associati a domini aziendali. \\ 
    
    \textbf{Precondizioni}  & Catalogue Manager presenta all'Utente Autenticato la pagina per la gestione dei cataloghi associati a domini aziendali. \\ 
    
    \textbf{Postcondizioni} & Catalogue Manager ha preso in carico, ed eseguito, le operazioni richieste dall'Utente Autenticato. \\ 
    \textbf{Flusso Principale} & 
    1. L'Utente Autenticato può aggiungere un nuovo catalogo di dominio ad un dominio senza catalogo. \newline 
    2. L'Utente Autenticato può visualizzare i cataloghi di dominio esistenti. \newline 
    3. L'Utente Autenticato può aggiungere un'applicazione ad un catalogo di dominio esistente. \newline
    4. L'Utente Autenticato può rimuovere un'applicazione da un catalogo di dominio esistente. \newline
    5. L'Utente Autenticato può rimuovere un catalogo di dominio esistente. \newline 
    6. L'Utente Autenticato può cercare un catalogo di dominio tra quelli esistenti. \newline
    	La ricerca può avvenire in base al nome del dominio. \\
    \textbf{Estensioni} &
    1. L'Utente Autenticato visualizza un messaggio di errore dovuto all'esistenza di un altro catalogo per il dominio selezionato.
  \end{longtable}
  \egroup
\end{center}

\subsubsection{UCA6 - Ricerca Applicazione}
\begin{center}
  \bgroup
  \def\arraystretch{1.8}     
  \begin{longtable}{  p{3.5cm} | p{8cm} } 
    \multicolumn{2}{ | c | }{ \cellcolor[gray]{0.9} \textbf{UCA6 - Ricerca Applicazione}} \\
    \hline
    
    \textbf{Attori Primari} & Utente Autenticato \\ 
    \textbf{Scopo e Descrizione} & L'Utente Autenticato può cercare un'applicazione del catalogo di Monokee. La ricerca può avvenire per nome o per categoria. \\ 
    
    \textbf{Precondizioni}  & Catalogue Manager ha mostrato all'utente la pagina di ricerca. L'Utente Autenticato può effettuare ricerche di vario tipo. \\ 
    
    \textbf{Postcondizioni} & Catalogue Manager ha mostrato i risultati della ricerca. \\ 
    \textbf{Flusso Principale} & 
    1. L'Utente Autenticato seleziona le modalità di ricerca e inserisce le informazioni richieste.
  \end{longtable}
  \egroup
\end{center}

\subsubsection{UCA10 - Gestione Gruppi di Applicazioni}
\begin{center}
  \bgroup
  \def\arraystretch{1.8}     
  \begin{longtable}{  p{3.5cm} | p{8cm} } 
    \multicolumn{2}{ | c | }{ \cellcolor[gray]{0.9} \textbf{UCA10 - Gestione Gruppi di Applicazioni}} \\
    \hline
    \textbf{Attori Primari} & Utente Autenticato \\ 
    \textbf{Scopo e Descrizione} & L'Utente Autenticato può gestire i gruppi di applicazioni del catalogo di Monokee. \\ 
    
    \textbf{Precondizioni}  & Catalogue Manager mostra all'Utente Autenticato la pagina di gestione dei gruppi di applicazioni. \\ 
    
    \textbf{Postcondizioni} & Catalogue Manager ha preso in carico le richieste dell'Utente Autenticato e le ha eseguite. \\ 
    \textbf{Flusso Principale} &
    1. L'Utente Autenticato può aggiungere un gruppo di applicazioni. \newline
    2. L'Utente Autenticato può visualizzare i gruppi di applicazioni presenti. \newline
    3. L'Utente Autenticato può gestire un gruppo di applicazioni esistente. \newline
    	In particolare, L'Utente Autenticato può aggiungere o rimuovere un'applicazione dal gruppo selezionato. \newline
    4. L'Utente Autenticato può modificare gli attributi di un gruppo di applicazioni esistente. \newline
    5. L'Utente Autenticato può rimuovere un gruppo di applicazioni esistente. 
  \end{longtable}
  \egroup
\end{center}

\subsubsection{UCA13 - Visualizzazione di un messaggio di errore per applicazione pubblica già presente}
\begin{center}
  \bgroup
  \def\arraystretch{1.8}     
  \begin{longtable}{  p{3.5cm} | p{8cm} } 
    \multicolumn{2}{ | c | }{ \cellcolor[gray]{0.9} \textbf{UCA13 - Visualizzazione di un messaggio di errore per applicazione pubblica già presente}} \\
    \hline
    
    \textbf{Attori Primari} & Utente Autenticato \\ 
    \textbf{Scopo e Descrizione} & L'Utente Autenticato ha cercato di inserire un nome corrispondente ad un'applicazione pubblica già presente nel catalogo di Monokee. Catalogue Manager presenta all'Utente Autenticato un messaggio di errore. \\ 
    
    \textbf{Precondizioni}  & L'Utente Autenticato ha inserito (nella pagina di aggiunta o di modifica di un'applicazione) un nome già utilizzato per un'applicazione pubblica. \\ 
    
    \textbf{Postcondizioni} & L'Utente Autenticato ha visualizzato il messaggio di errore presentato da Catalogue Manager. \\
    \textbf{Flusso Principale} & 
    1. L'Utente Autenticato visualizza il messaggio di errore.
  \end{longtable}
  \egroup
\end{center}

\subsubsection{UCA14 - Visualizzazione Statistiche}
\begin{center}
  \bgroup
  \def\arraystretch{1.8}     
  \begin{longtable}{  p{3.5cm} | p{8cm} } 
    \multicolumn{2}{ | c | }{ \cellcolor[gray]{0.9} \textbf{UCA14 - Visualizzazione Statistiche}} \\
    \hline
    
    \textbf{Attori Primari} & Utente Autenticato \\ 
    \textbf{Scopo e Descrizione} & L'Utente Autenticato può visualizzare delle statistiche riguardanti l'applicazione Catalogue Manager. In particolare, le statistiche riguardano:
    	\begin{enumerate}
 	    \item numero di applicazioni aggiunte e rimosse in intervalli di tempo definiti a priori: ultime 24 ore, ultima settimana, ultimo mese e ultimo anno;
 	    \item numero di accessi in intervalli di tempo definiti a priori: ultime 24 ore e ultima settimana;
 	    \item numero di utenti attivi;
 	    \item numero di applicazioni e gruppi pubblici e privati;
 	    \item numero di applicazioni, pubbliche e private, appartenenti ad ogni categoria (intesa come ''sotto categoria'');
 	    \item numero di applicazioni, pubbliche e private, appartenenti ad una specifica ''sovra categoria''.
 	  	\end{enumerate} \\ 
    
    \textbf{Precondizioni}  & L'Utente Autenticato ha richiesto la visualizzazione delle statistiche dell'applicazione Catalogue Manager. \\ 
    
    \textbf{Postcondizioni} & L'Utente Autenticato ha visualizzato le statistiche dell'applicazione Catalogue Manager. \\
    \textbf{Flusso Principale} & 
    1. L'Utente Autenticato visualizza il numero di applicazioni aggiunte e rimosse in intervalli di tempo definiti a priori: ultime 24 ore, ultima settimana, ultimo mese e ultimo anno. \newline
    2. L'Utente Autenticato visualizza il numero di accessi in intervalli di tempo definiti a priori: ultime 24 ore e ultima settimana. \newline
    3. L'Utente Autenticato visualizza il numero di utenti attivi. \newline
    4. L'Utente Autenticato visualizza il numero di applicazioni e gruppi pubblici e privati. \newline
    5. L'Utente Autenticato visualizza il numero di applicazioni, pubbliche e private, appartenenti ad ogni categoria (intesa come ''sotto categoria''). \newline
    6. L'Utente Autenticato visualizza il numero di applicazioni, pubbliche e private, appartenenti ad una specifica ''sovra categoria''.
  \end{longtable}
  \egroup
\end{center}

\subsubsection{UCA15 - Visualizzazione Log}
\begin{center}
  \bgroup
  \def\arraystretch{1.8}     
  \begin{longtable}{  p{3.5cm} | p{8cm} } 
    \multicolumn{2}{ | c | }{ \cellcolor[gray]{0.9} \textbf{UCA15 - Visualizzazione Log}} \\
    \hline
    
    \textbf{Attori Primari} & Utente Autenticato \\ 
    \textbf{Scopo e Descrizione} & L'Utente Autenticato può visualizzare i \textit{log} delle operazioni eseguite nell'applicazione Catalogue Manager. In particolare i \textit{log} possono riguardare le operazioni eseguite con successo (UCA16) e quelle che hanno generato un errore (UCA17). I \textit{log} salvati possono essere filtrati per intervallo di date, per \textit{keywords} e per tipologia. \\
    
    \textbf{Precondizioni}  & L'Utente Autenticato ha richiesto la visualizzazione dei \textit{log} dell'applicazione Catalogue Manager. \\ 
    
    \textbf{Postcondizioni} & L'Utente Autenticato ha visualizzato i \textit{log} dell'applicazione Catalogue Manager. \\
    \textbf{Flusso Principale} & 
    1. L'Utente Autenticato visualizza i \textit{log} dell'applicazione. 
  \end{longtable}
  \egroup
\end{center}

\subsubsection{UCA16 - Visualizzazione Log sulle operazioni che hanno avuto successo}
\begin{center}
  \bgroup
  \def\arraystretch{1.8}     
  \begin{longtable}{  p{3.5cm} | p{8cm} } 
    \multicolumn{2}{ | c | }{ \cellcolor[gray]{0.9} \textbf{UCA16 - Visualizzazione Log sulle operazioni che hanno avuto successo}} \\
    \hline
    
    \textbf{Attori Primari} & Utente Autenticato \\ 
    \textbf{Scopo e Descrizione} & L'Utente Autenticato può visualizzare i \textit{log} delle operazioni eseguite con successo nell'applicazione Catalogue Manager. I \textit{log} salvati possono essere filtrati per intervallo di date, per \textit{keywords} e per tipologia. \\
    
    \textbf{Precondizioni}  & L'Utente Autenticato ha richiesto la visualizzazione dei \textit{log} dell'applicazione Catalogue Manager. Ha successivamente selezionato la visualizzazione dei \textit{log} delle operazioni eseguite con successo. \\ 
    
    \textbf{Postcondizioni} & L'Utente Autenticato ha visualizzato i \textit{log} delle operazioni eseguite con successo dell'applicazione Catalogue Manager. \\
    \textbf{Flusso Principale} & 
    1. L'Utente Autenticato visualizza i \textit{log} delle operazioni eseguite con successo.
  \end{longtable}
  \egroup
\end{center}

\subsubsection{UCA17 - Visualizzazione Log delle operazioni che hanno generato errori}
\begin{center}
  \bgroup
  \def\arraystretch{1.8}     
  \begin{longtable}{  p{3.5cm} | p{8cm} } 
    \multicolumn{2}{ | c | }{ \cellcolor[gray]{0.9} \textbf{UCA17 - Visualizzazione Log delle operazioni che hanno generato errori}} \\
    \hline
    
    \textbf{Attori Primari} & Utente Autenticato \\ 
    \textbf{Scopo e Descrizione} & L'Utente Autenticato può visualizzare i \textit{log} delle operazioni che hanno generato errori nell'applicazione Catalogue Manager. I \textit{log} salvati possono essere filtrati per intervallo di date, per \textit{keywords} e per tipologia. \\
    
    \textbf{Precondizioni}  & L'Utente Autenticato ha richiesto la visualizzazione dei \textit{log} dell'applicazione Catalogue Manager. Ha successivamente selezionato la visualizzazione dei \textit{log} delle operazioni che hanno generato errori. \\ 
    
    \textbf{Postcondizioni} & L'Utente Autenticato ha visualizzato i \textit{log} delle operazioni che hanno generato errori dell'applicazione Catalogue Manager. \\
    \textbf{Flusso Principale} & 
    1. L'Utente Autenticato visualizza i \textit{log} delle operazioni che hanno generato errore. 
  \end{longtable}
  \egroup
\end{center}

\subsubsection{UCA18 - Ricerca Domini Aziendali di Monokee}
\begin{center}
  \bgroup
  \def\arraystretch{1.8}     
  \begin{longtable}{  p{3.5cm} | p{8cm} } 
    \multicolumn{2}{ | c | }{ \cellcolor[gray]{0.9} \textbf{UCA18 - Ricerca Domini Aziendali di Monokee}} \\
    \hline
    \textbf{Attori Primari} & Utente Autenticato \\ 
    \textbf{Scopo e Descrizione} & L'Utente Autenticato può effettuare una ricerca tra i domini aziendali di Monokee. La ricerca avviene per nome. \\ 
    
    \textbf{Precondizioni}  & Catalogue Manager mostra all'Utente Autenticato la pagina di gestione dei cataloghi di dominio di Monokee. \\ 
    
    \textbf{Postcondizioni} & Catalogue Manager ha preso cercato tra i domini aziendali di Monokee e ha mostrato all'Utente Autenticato i risultati. \\ 
    \textbf{Flusso Principale} &
    1. L'Utente Autenticato cerca un dominio aziendale di Monokee. 
  \end{longtable}
  \egroup
\end{center}

\subsubsection{UCA19 - Visualizzazione di un messaggio di errore per catalogo già presente}
\begin{center}
  \bgroup
  \def\arraystretch{1.8}     
  \begin{longtable}{  p{3.5cm} | p{8cm} } 
    \multicolumn{2}{ | c | }{ \cellcolor[gray]{0.9} \textbf{UCA19 - Visualizzazione di un messaggio di errore per catalogo già presente}} \\
    \hline
    
    \textbf{Attori Primari} & Utente Autenticato \\ 
    \textbf{Scopo e Descrizione} & L'Utente Autenticato ha cercato di aggiungere un catalogo di dominio ad un dominio che ha già un catalogo. Catalogue Manager presenta all'Utente Autenticato un messaggio di errore. \\ 
    
    \textbf{Precondizioni}  & L'Utente Autenticato ha cercato di aggiungere un catalogo di dominio ad un dominio che ha già un catalogo. \\ 
    
    \textbf{Postcondizioni} & L'Utente Autenticato ha visualizzato il messaggio di errore presentato da Catalogue Manager. \\
    \textbf{Flusso Principale} & 
    1. L'Utente Autenticato visualizza il messaggio di errore.
  \end{longtable}
  \egroup
\end{center} % casi d'uso per utente autenticato


\section{Requisiti}
I requisiti funzionali e di vincolo individuati sono riportati nelle seguenti tabelle. Viene inoltre indicato se si tratta di un requisito fondamentale, desiderabile o facoltativo e una sua descrizione.

Ogni requisito è identificato da un codice, che segue il seguente formalismo:
\begin{center}
		\textbf{RXY Gerarchia}
\end{center}

Dove:
\begin{itemize}
 \item \textbf{X} corrisponde alla tipologia del requisito e può assumere i seguenti valori:
		\begin{itemize}
		 \item[] \textbf{1} = Funzionale;
		 \item[] \textbf{2} = Vincolo.
		\end{itemize}

 \item \textbf{Y} corrisponde alla priorità del requisito e può assumere i seguenti valori:
		\begin{itemize}
		 \item[] \textbf{O} = Obbligatorio;
		 \item[] \textbf{D} = Desiderabile;
		 \item[] \textbf{F} = Facoltativo o Opzionale.
		\end{itemize}

 \item \textbf{Gerarchia} identifica la relazione gerarchica che c'è tra i requisiti di uno stesso tipo. Vi è dunque una struttura gerarchica per ogni tipologia di requisito.
\end{itemize}

\subsection{Principali requisiti funzionali}
Nella Tabella~\ref{tab:reqfunzionali} vengono elencati i principali requisiti funzionali dell'applicazione Catalogue Manager.
\begin{center}
  \bgroup
  \def\arraystretch{1.8}
  \begin{longtable}{ | l | p{8.4cm} |}
    \hline
    \cellcolor[gray]{0.9} \textbf{Requisito} & \cellcolor[gray]{0.9} \textbf{Descrizione} \\ \hline
    R1O 1 & Un utente deve poter visualizzare la lista delle applicazioni del catalogo di Monokee. \\ \hline 
R1O 1.1 & Un utente deve poter visualizzare i dettagli di un'applicazione presente nel catalogo di Monokee. \\ \hline 
R1O 1.2 & Un utente deve poter visualizzare il nome dell'applicazione del catalogo di Monokee. \\ \hline 
R1O 1.3 & Un utente deve poter visualizzare la descrizione dell'applicazione del catalogo di Monokee. \\ \hline 
R1O 1.4 & Un utente deve poter visualizzare l'immagine di un'applicazione del catalogo di Monokee. \\ \hline 
R1O 1.5 & Un utente deve poter visualizzare la lista delle categorie associate ad un'applicazione del catalogo di Monokee. \\ \hline 
R1O 1.6 & Un utente deve poter visualizzare il tipo di \glossaryItem{autenticazione} di un'applicazione del catalogo di Monokee. \\ \hline 
R1O 1.7 & Un utente deve poter visualizzare ogni dettaglio di \glossaryItem{autenticazione} per un'applicazione del catalogo di Monokee. \\ \hline 
R1O 1.8 & Un utente deve poter visualizzare la visibilità (pubblica o privata) dell'applicazione. \\ \hline 
R1O 1.9 & Un utente deve poter visualizzare il nome del gruppo nel quale è inserita l'applicazione. \\ \hline 
R1O 1.10 & Un utente deve poter visualizzare se l'applicazione è in manutenzione o meno. \\ \hline 
R1O 1.10 & Un utente deve poter visualizzare se l'applicazione è stata pubblicata o meno. \\ \hline 
R1O 2 & Un utente deve poter inserire una nuova applicazione. \\ \hline 
R1O 2.1 & Un utente deve poter inserire il nome della nuova applicazione. \\ \hline 
R1O 2.2 & Un utente deve poter inserire la descrizione della nuova applicazione. \\ \hline 
R1O 2.3 & Un utente deve poter inserire l'\glossaryItem{url} della nuova applicazione. \\ \hline 
R1O 2.4 & Un utente deve poter inserire l'immagine della nuova applicazione. \\ \hline 
R1O 2.5 & Un utente deve poter selezionare le categorie di appartenenza della nuova applicazione. \\ \hline 
R1O 2.6 & Un utente deve poter selezionare il tipo di \glossaryItem{autenticazione} della nuova applicazione. \\ \hline 
R1O 2.7 & Un utente deve poter inserire inserire i dati di \glossaryItem{autenticazione} della nuova applicazione. \\ \hline 
R1O 2.8 & Un utente deve poter inserire il nome del gruppo nel quale sarà inserita la nuova applicazione. \\ \hline 
R1O 3 & Un utente deve poter modificare un'applicazione. \\ \hline 
R1O 3.1 & Un utente deve poter modificare il nome dell'applicazione. \\ \hline 
R1O 3.2 & Un utente deve poter modificare la descrizione dell'applicazione.\\ \hline 
R1O 3.3 & Un utente deve poter modificare l'\glossaryItem{url} dell'applicazione.\\ \hline 
R1O 3.4 & Un utente deve poter modificare l'immagine dell'applicazione. \\ \hline 
R1O 3.5 & Un utente deve poter selezionare le categorie di appartenenza dell'applicazione. \\ \hline 
R1O 3.6 & Un utente deve poter modificare modificare i dati di autenticazione dell'applicazione. \\ \hline 
R1O 3.7 & Un utente deve poter modificare il nome del gruppo nel quale sarà inserita l'applicazione. \\ \hline 
R1O 3.8 & Un utente deve poter mettere un'applicazione in manutenzione. \\ \hline 
R1O 3.9 & Un utente deve poter pubblicare un'applicazione. \\ \hline 
R1O 4 & Un utente deve poter rimuovere un'applicazione. \\ \hline 
R1O 5 & Un utente deve poter gestire i cataloghi di dominio di Monokee. \\ \hline 
R1O 5.1 & Un utente deve poter aggiungere un catalogo di dominio. \\ \hline 
R1O 5.2 & Un utente deve visualizzare un messaggio di errore se il dominio aziendale è già collegato ad un catalogo di dominio. \\ \hline 
R1O 5.3 & Un utente deve poter visualizzare i cataloghi di dominio esistenti. \\ \hline 
R1O 5.4 & Un utente deve poter gestire le applicazioni presenti in un catalogo di dominio. \\ \hline 
R1O 5.4.1 & Un utente deve poter visualizzare le applicazioni nel catalogo. \\ \hline 
R1O 5.4.2 & Un utente deve poter aggiungere un'applicazione al catalogo. \\ \hline 
R1O 5.4.3 & Un utente deve poter rimuovere un'applicazione dal catalogo. \\ \hline 
R1O 5.5 & L'utente deve poter rimuovere un catalogo di dominio esistente. \\ \hline 
R1O 5.6 & La rimozione di un catalogo di dominio deve comportare la rimozione di tute le applicazioni collegate a quel catalogo. \\ \hline 
R1O 6 & Un utente deve poter cercare un'applicazione nel catalogo. \\ \hline 
R1O 10 & Un utente deve poter gestire i gruppi di applicazioni. \\ \hline 
R1O 10.1 & Un utente deve poter aggiungere un gruppo. \\ \hline 
R1O 10.1.1 & Un utente deve poter inserire il nome del gruppo. \\ \hline 
R1O 10.1.2 & Un utente deve poter inserire la descrizione del gruppo. \\ \hline 
R1O 10.1.3 & Un utente deve poter inserire l'immagine del gruppo. \\ \hline 
R1O 10.2 & Un utente deve poter visualizzare i gruppi di applicazioni esistenti. \\ \hline 
R1O 10.3 & Un utente deve poter gestire un gruppo di applicazioni esistente. \\ \hline 
R1O 10.3.1 & Un utente deve poter aggiungere un'applicazione al gruppo.  \\ \hline 
R1O 10.3.2 & Un utente deve poter rimuovere un'applicazione dal gruppo. \\ \hline 
R1O 10.3.3 & Un utente deve visualizzare un messaggio di errore se l'applicazione selezionata è già presente nel gruppo. \\ \hline 
R1O 10.4 & Un utente deve poter modificare un gruppo. \\ \hline 
R1O 10.4.1 & Un utente deve poter inserire il nuovo nome del gruppo. \\ \hline 
R1O 10.4.2 & Un utente deve poter inserire la nuova descrizione del gruppo. \\ \hline 
R1O 10.4.3 & Un utente deve poter inserire la nuova immagine del gruppo. \\ \hline 
R1O 10.5 & Un utente deve poter rimuovere un gruppo.\\ \hline 
R1O 13 & Un utente deve visualizzare un messaggio di errore se l'applicazione che si sta cercando di inserire è già presente. \\ \hline
R1D 14 & Un utente deve poter visualizzare le statistiche dell'applicazione Catalogue Manager. \\ \hline 
R1D 14.1 & Un utente deve poter visualizzare il numero di applicazioni aggiunte e rimosse in intervalli di tempo definiti a priori: ultime 24 ore, ultima settimana, ultimo mese e ultimo anno. \\ \hline 
R1D 14.2 & Un utente deve poter visualizzare il numero di accessi in intervalli di tempo definiti a priori: ultime 24 ore e ultima settimana. \\ \hline 
R1D 14.3 & Un utente deve poter visualizzare il numero di utenti attivi. \\ \hline 
R1D 14.4 & Un utente deve poter visualizzare il numero di applicazioni e gruppi pubblici e privati. \\ \hline 
R1D 14.5 & Un utente deve poter visualizzare il numero di applicazioni, pubbliche e private, appartenenti ad ogni categoria (intesa come ''sotto categoria''). \\ \hline 
R1D 14.6 & Un utente deve poter visualizzare il numero di applicazioni, pubbliche e private, appartenenti ad una specifica ''sovra categoria''. \\ \hline 
R1D 15 & Un utente deve poter visualizzare i \textit{log} dell'applicazione Catalogue Manager. \\ \hline 
R1D 16 & Un utente deve poter visualizzare i \textit{log} delle operazioni eseguite con successo. \\ \hline 
R1D 17 & Un utente deve poter visualizzare i \textit{log} delle operazioni che hanno generato errori. \\ \hline 
R1O 18 & Un utente deve poter effettuare una ricerca tra i domini aziendali di Monokee. \\ \hline 
R1O 19 & Un utente deve visualizzare un messaggio di errore se il dominio ha già un catalogo associato. \\ \hline
    \caption[Principali requisiti funzionali]{Principali requisiti funzionali}
    \label{tab:reqfunzionali}
  \end{longtable}
  \egroup
\end{center} 

\subsection{Requisiti di vincolo}
Nella Tabella~\ref{tab:reqvincolo} vengono elencati i requisiti di vincolo dell'applicazione.
\begin{center}
  \bgroup
  \def\arraystretch{1.8}
  \begin{longtable}{ | l | p{8.4cm} |}
    \hline
    \cellcolor[gray]{0.9} \textbf{Requisito} & \cellcolor[gray]{0.9} \textbf{Descrizione} \\ \hline
    R2O 1 & L'applicazione deve utilizzare Node.js e JavaScript. \\ \hline
    R2O 2 & L'applicazione deve memorizzare i dati su MongoDB. \\ \hline
    R2O 3 & L'accesso all'applicazione deve avvenire con \glossaryItem{saml}. \\ \hline
    \caption[Requisiti di vincolo]{Requisiti di vincolo}
    \label{tab:reqvincolo} 
    \end{longtable}
  \egroup
\end{center} 
\chapter{Progettazione} \label{progettazione}

\section{Interfaccia REST-like}
Il back end si basa su uno stile \glossaryItem{rest}-\textit{like}, ovvero con le seguenti caratteristiche:
\begin{itemize}
\item stato dell'applicazione e funzionalità divisi in risorse web;
\item ogni risorsa è unica e indirizzabile attraverso un \glossaryItem{uri};
\item tutte le risorse sono condivise come interfaccia uniforme per il trasferimento di stato tra client e risorse. Questo trasferimento consiste in:
\begin{itemize}
\item un insieme vincolato di operazioni ben definite;
\item un insieme vincolato di contenuti, opzionalmente supportato da codice a richiesta;
\item un protocollo:
\begin{itemize}
\item client-server;
\item privo di stato;
\item memorizzabile in cache;
\item a livelli.
\end{itemize}
\end{itemize}
\end{itemize}

\glossaryItem{rest} utilizza il concetto di risorsa (aggregato di dati con un nome e una rappresentazione interna), sulla quale è possibile invocare operazioni \glossaryItem{crud} secondo la corrispondenza indicata in Tabella~\ref{tab:RESTCRUD}.
\begin{center}
  \bgroup
  
  \begin{longtable}{ | m{2cm} | m{2cm} | p{7cm} |}
    \hline
    \cellcolor[gray]{0.9}\textbf{Metodo HTTP} & \cellcolor[gray]{0.9}\textbf{Operazione CRUD} & \cellcolor[gray]{0.9}\textbf{Descrizione} \\ \hline
    GET & Read & Ricava e ritorna informazioni su una risorsa \\ \hline
    PUT & Update & Aggiorna una risorsa \\ \hline
    POST & Create & Crea una risorsa \\ \hline
    DELETE & Delete & Cancella una risorsa \\ \hline
    \caption[Corrispondenza tra CRUD e HTTP]{Corrispondenza tra CRUD e HTTP}
    \label{tab:RESTCRUD} 
    \end{longtable}
  \egroup
\end{center} 
Per la rappresentazione dei dati si è scelto di utilizzare \glossaryItem{json} perché si integra molto bene con le tecnologie utilizzate e con il linguaggio JavaScript. Questo non è vero per \glossaryItem{xml} o \glossaryItem{csv}, che richiederebbero librerie specifiche. Inoltre \glossaryItem{json} è molto meno verboso e molto più flessibile di \glossaryItem{xml}, e si adatta molto bene al dominio dell'applicazione.

Uno stile architetturale di questo tipo permette l'indipendenza completa tra back end e front end, permettendo così espansioni su altre piattaforme senza dover modificare il back end dell'applicazione.

\section{Architettura}
In Figura~\ref{fig:architetturaGenerale} è rappresentata l'architettura di Catalogue Manager. Il diagramma dei \textit{package} rappresenta i componenti ad un livello di dettaglio molto basso, ma sufficiente a capire le relazioni principali. Come si nota, Catalogue Manager utilizza i modelli di mongoose.js (\textit{package} \texttt{Monokee.models}) di Monokee. Questa scelta è stata imposta dall'architettura esistente: alcuni servizi di Monokee utilizzavano, e utilizzano, alcuni modelli riguardanti il catalogo. Uno spostamento completo avrebbe causato numerosi problemi e cambiamenti. È stato pertanto deciso di importare solo e soltanto i modelli necessari allo svolgimento delle operazioni di Catalogue Manager. Successivamente verranno descritti i modelli importati.

\begin{figure}[hbpc]
  \begin{center}
    \includegraphics[scale=0.4]{Classi/architettura}
  \caption[Architettura generale]{Architettura generale}
  \label{fig:architetturaGenerale}
  \end{center} 
\end{figure}

\subsection{Moduli esterni}
Catalogue Manager fa uso di numerosi moduli esterni: nel diagramma in Figura~\ref{fig:architetturaGenerale} sono stati riportati solamente quelli principali.

\paragraph{mongoose.js} \mbox{} \\
\textbf{mongoose.js} è uno strumento di modellazione ad oggetti per MongoDB progettato per lavorare in ambiente asincrono (e quindi ottimo per Node.js) che offre grande supporto per le interrogazioni al \textit{database}. La modellazione ad oggetti consente di progettare con precisione le collezioni attraverso la definizione degli attributi, dei loro tipi e delle loro relazioni: in questo modo è possibile definire uno schema sul quale basarsi. 

La definizione dei tipi, in particolare, consente di controllare la consistenza dei dati inseriti. Grazie alla definizione di \textit{middlewares}\footnote{I \textit{middlewares} sono delle funzioni alle quali è passato il controllo durante l'esecuzione di funzioni asincrone. Sono specifici a livello di schema.}, inoltre, è possibile eseguire delle operazioni prima, o dopo, le interrogazioni al \textit{database}, evitando la replicazione di codice e aderendo al principio \glossaryItem{dry}. Esistono due tipi di \textit{middleware}: \textbf{document}, che agiscono a livello dell'intero \textit{document} MongoDB e che possono essere definiti su operazioni come salvataggio, rimozione o validazione, e \textbf{query}, che agiscono durante le interrogazioni al \textit{database}, in particolare per il \texttt{find} (ricerca), l'\texttt{update} (aggiornamento) e il \texttt{count} (conteggio). La possibilità di eseguire codice personale prima dello svolgimento di queste operazioni consente, ad esempio, di effettuare controlli specifici sui dati e di modificarli se necessario. È uno strumento molto potente del quale si è fatto grande uso.

Un \textit{middleware} meno conosciuto, ma importantissimo per la gestione degli errori in Catalogue Manager, è quello che viene eseguito dopo la sollevazione di un errore (\texttt{post error}). Durante la sua esecuzione si dispone dell'errore sollevato, ed è possibile aggiungerci delle informazioni sfruttando il permissivo paradigma ad oggetti di JavaScript. Questi dati aggiuntivi consentono di salvare dei \textit{log} precisi ed accurati.
\subparagraph{Dipendenze}
Come si nota dal diagramma, mongoose.js è utilizzato dal \textit{package} \texttt{models} di Monokee.

\paragraph{async.js} \mbox{} \\
\textbf{async.js} è un modulo di utilità che fornisce potenti funzioni per lavorare in modo asincrono con JavaScript. È stato progettato per un uso con Node.js, ma è utilizzabile anche direttamente nel \textit{browser}. Tra le circa 70 funzioni presenti si trovano molte utili per operare con gli \textit{array} (come \texttt{map}, \texttt{reduce}, \texttt{filter}, eccetera) e altre che implementano \textit{pattern} comuni per un flusso di controllo asincrono. Tutte queste seguono le convenzioni di Node.js e prevedono una sola funzione di \texttt{callback} (con due parametri: l'errore sollevato, se presente, e i risultati dell'intera esecuzione o fino al verificarsi dell'errore) che va chiamata una sola volta.
\subparagraph{Dipendenze}
async.js è utilizzato da tutto Catalogue Manager, sia dal \textit{package} \texttt{routes} sia da \texttt{modules}.

\paragraph{body-parser} \mbox{} \\
\textbf{body-parser} effettua il \textit{parsing} del \textit{body} delle richieste alle \glossaryItem{api} e lo rende disponibile nella proprietà \texttt{body} dell'oggetto \texttt{req} (\texttt{Request})di Express.js. Mette a disposizione numerose funzioni che definiscono come deve essere effettuato il \textit{parsing}: Catalogue Manager utilizza la funzione \texttt{json} in modo da analizzare solamente \textit{body} di tipo \glossaryItem{json}.
\subparagraph{Dipendenze}
body-parser è utilizzato dallo \textit{script} JavaScript utilizzato per avviare l'applicazione (\texttt{app.js}).

\paragraph{Express.js} \mbox{} \\
Come abbondantemente descritto in \ref{express}, \textbf{Express.js} è utilizzato per definire gli \textit{endpoints} del back end di Catalogue Manager. 

\subparagraph{Dipendenze}
Ogni elemento del \textit{package} \texttt{routes} dipende da Express.js, oltre allo script principale per l'avvio dell'applicazione, che lo utilizza per definire gli \glossaryItem{uri} dei servizi esposti.

\paragraph{expressjwt e jsonwebtoken} \mbox{} \\
Questi due moduli permettono di utilizzare i \glossaryItem{jwt}. \textbf{jsonwebtoken} è un'implementazione che rispetta il documento RFC7519 (\cite{rfc:7519}) ed è utilizzato per generare e firmare \textit{token} \glossaryItem{jwt}. \textbf{expressjwt}, invece, è utilizzato per validare i \glossaryItem{jwt} e per inserire il contenuto del \textit{token} nella proprietà \textbf{user} dell'oggetto \texttt{req} di Express.js. Quest'ultimo modulo consente di autenticare richieste \glossaryItem{http} utilizzando \textit{token} \glossaryItem{jwt} in applicazioni Node.js. 

\subparagraph{Dipendenze}
jsonwebtoken è utilizzato da un unico servizio, \texttt{/acs}, ovvero quello che, dopo aver ricevuto la \textit{SAMLResponse} dall'\glossaryItem{idp}, genera il \textit{token} e lo invia al front end di Catalogue Manager.

expressjwt, invece, è utilizzato nello \textit{script} per l'avvio dell'applicazione e ''protegge'' i servizi per i quali è richiesta l'\glossaryItem{autenticazione}. 

\subsection{Models di Monokee}
Il \textit{package} \texttt{models} di Monokee contiene i modelli di mongoose.js utilizzati da Catalogue Manager, che verranno descritti in dettaglio successivamente (vedi \ref{modelli}). In Figura~\ref{fig:models} sono riportati quelli di maggior interesse.

\begin{figure}[hbpc]
  \begin{center}
    \includegraphics[scale=0.4]{Classi/models}
  \caption[Package models]{Package models}
  \label{fig:models}
  \end{center} 
\end{figure}

Come si può notare dai nomi, la quasi totalità di questi riguarda esclusivamente il catalogo. Il modello principale è \texttt{Catalogue}, che specifica lo schema delle applicazioni da catalogo, non importa se quello di Monokee o uno di dominio. Le applicazioni appartenenti ad uno stesso dominio sono raggruppate nei \textit{document} di \texttt{CatalogueDomain}: ad ogni \textit{document} corrisponde un catalogo di dominio.

\texttt{CatalogueForm}, \texttt{CatalogueSAML} e \texttt{CatalogueThirdType} definiscono le informazioni specifiche per i tre tipi di accesso diversi.

\texttt{CatalogueGroup} quelle dei gruppi di applicazioni del catalogo, mentre \texttt{CatalogueLog} specifica lo schema dei \textit{log}.

\texttt{Domain}, invece, è utilizzato solamente per mantenere la consistenza dei dati e per recuperare il giusto catalogo da \texttt{CatalogueDomain}: definisce le informazioni dei domini di Monokee.

\subsection{Modules di Catalogue Manager}
Il \textit{package} \texttt{modules} di Catalogue Manager (Figura~\ref{fig:modules}) contiene i moduli di utilità usati dalle \textit{routes}, che verranno descritti in dettaglio successivamente (vedi \ref{moduli}).  Questi moduli permettono di raggruppare le operazioni comuni, evitando la duplicazione di codice e rendendo, di conseguenza, il prodotto più manutenibile. Ogni modulo è fortemente coeso, e dipende in misura quasi completamente nulla dagli altri moduli. L'unica eccezione è rappresentata dalla dipendenza nei confronti dei due moduli di \textit{logging}.

\begin{figure}[hbpc]
  \begin{center}
    \includegraphics[scale=0.4]{Classi/modules}
  \caption[Package modules]{Package modules}
  \label{fig:modules}
  \end{center} 
\end{figure}

\texttt{Logger} e \texttt{DBLogger} si occupano di salvare i \textit{log} rispettivamente su \textit{file} e nel \textit{database} (attraverso il modello \texttt{CatalogueLog}).

La gerarchia di \texttt{ImageHandler} implementa il \textit{design pattern} \textbf{Template Method} per il salvataggio e la rimozione delle immagini dei gruppi (\texttt{GroupImageHandler}), delle applicazioni (\texttt{ApplicationImageHandler}) e delle istruzioni per la configurazione di applicazioni di tipo \glossaryItem{saml} (\texttt{SAMLInstructionsImageHandler}). 

\texttt{ClearDB} è utilizzato per eliminare i residui della \glossaryItem{softdeletion} dal \textit{database} di Monokee. Per consentire il \glossaryItem{rollback} dei dati, infatti, i \textit{document} non sono direttamente eliminati dal \textit{database}, ma viene impostato a \textit{true} un \textit{flag} (\texttt{removed}). Alla fine del processo di cancellazione, se non sono stati rilevati errori, \texttt{ClearDB} si occupa di eliminare tutto ciò che ha \texttt{removed} a \texttt{true}. 

\texttt{CheckRequiredFields} controlla semplicemente che tutti i campi obbligatori per il servizio che lo invoca siano presenti nella proprietà \texttt{body} dell'oggetto \texttt{req} di Express.js.

\texttt{DomainCatalogue} popola il catalogo di un dominio specifico. 

\texttt{ErrorHandler} è utilizzato per gestire gli errori riscontrati: oltre a salvare il \textit{log} invia risposte diverse al client in base a quanto è successo. È inoltre in grado di intercettare gli errori sollevati direttamente da mongoose.js (ad esempio, una validazione fallita) e di effettuare un \textit{parsing} per presentare al client un \glossaryItem{json} conforme alla definizione definita durante la progettazione.

\texttt{GroupApplicationsHandler} raggruppa le operazioni che coinvolgono gruppi di applicazioni, come l'aggiunta e la rimozione di applicazioni e il controllo sull'associazione gruppo/applicazione.

\texttt{RemoveApp} si occupa di rimuovere (tramite \glossaryItem{softdeletion}) un'applicazione dal \textit{database} di Monokee, ed è usato anche come \glossaryItem{rollback} se si verificano errori durante una creazione.

\texttt{ResetFirstSignIn}, infine, reimposta i \textit{flag} di \textit{first sign in} in seguito alla modifica delle informazioni sull'\glossaryItem{autenticazione} \textit{form-based}. 

\subsection{Routes di Catalogue Manager}
Il \textit{package} \texttt{routes} (Figura~\ref{fig:routes}) di Catalogue Manager contiene gli \textit{endpoints} esposti dal server. In generale, ogni \textit{route} corrisponde a (e soddisfa un) requisito funzionale di alto livello. I servizi \glossaryItem{rest} definiti sono circa 40, e un diagramma delle classi che li mostri tutti risulterebbe illeggibile e inutile. Una descrizione di ciascuna \textit{route} viene fornita in \ref{servizi}.

\begin{figure}[hbpc]
  \begin{center}
    \includegraphics[scale=0.4]{Classi/routes}
  \caption[Package routes]{Package routes}
  \label{fig:routes}
  \end{center} 
\end{figure}

\section{Modalità di autenticazione}
Come già visto, l'\glossaryItem{autenticazione} a Catalogue Manager deve avvenire tramite \glossaryItem{saml}. Tuttavia non esiste un modo univoco di effettuare questa \glossaryItem{autenticazione}, in quanto \glossaryItem{saml} prevede due diverse modalità di accesso, ognuna corrispondente ad uno specifico caso d'uso: in \ref{ssoSAML} verranno analizzate entrambe. 

Risolto il problema dell'accesso, è importante stabilire come verrà mantenuta la sessione con l'utente autenticato, ovvero se questa sarà memorizzata sul server o meno. La scelta è stata quella di lasciare al client l'onere di far sapere se l'\glossaryItem{autenticazione} è avvenuta o no, attraverso l'uso dei \glossaryItem{jwt}, descritti in \ref{descrJWT}.

\subsection{JSON Web Token} \label{descrJWT}
\paragraph{Descrizione} \mbox{} \\
\glossaryItem{jwt} (logo in Figura~\ref{fig:jwt}) è uno standard \textit{open} (\cite{rfc:7519}) che definisce un modo \textbf{compatto} e \textbf{self-contained} per trasmettere informazioni in modo sicuro sotto forma di oggetti \glossaryItem{json}. Le informazioni possono essere verificate dato che sono firmate: la firma può avvenire attraverso una stringa (il cosiddetto \textbf{secret}), con l'algoritmo \glossaryItem{hmac}, o usando una coppia di chiavi pubbliche e private, grazie a \glossaryItem{rsa}.

\begin{figure}[h]
  \begin{center}
    \includegraphics[scale=0.5]{jwt}
  \caption[Logo di JWT]{Logo di JWT\protect\footnotemark}
  \label{fig:jwt}
  \end{center} 
\end{figure}
\footnotetext{Immagine tratta da \cite{site:jwtintro}}

\glossaryItem{jwt} è \textbf{compatto} perché grazie alla sua dimensione ridotta può essere inviato attraverso un \glossaryItem{url}, come parametro di una richiesta POST o dentro un \textit{header} \glossaryItem{http}. Inoltre, occupando poco spazio, può essere trasmesso velocemente. 

\glossaryItem{jwt} è \textbf{self-contained} perché il \textbf{payload} contiene tutte le informazioni riguardo l'utente, evitando di dover interrogare il \textit{database} più di una volta.

I due scenari di utilizzo più comuni sono i seguenti:
\begin{itemize}
\item \textbf{\glossaryItem{autenticazione}}: è il caso d'uso più comune. Dopo la prima \glossaryItem{autenticazione} ogni successiva richiesta conterrà il \textit{token} e permetterà all'utente di accedere a tutto ciò che il suo ruolo gli consente (servizi, risorse, eccetera). \glossaryItem{jwt} è molto usato con il \glossaryItem{sso} grazie alla sua interoperabilità e alla facilità di utilizzo;
\item \textbf{scambio di informazioni}: \glossaryItem{jwt} è un ottimo modo per trasmettere informazioni in modo sicuro tra parti diverse, grazie alla possibilità di firmarli. Visto che la firma è calcolata sul contenuto, si può anche controllare che le informazioni non siano state alterate.
\end{itemize}

\paragraph{Struttura} \mbox{} \\
In Figura~\ref{fig:token} è mostrato un esempio di \textit{token} utilizzato in Catalogue Manager. Come si può notare è composto da tre parti separate da '.':
\begin{itemize}
\item \textbf{Header};
\item \textbf{Payload};
\item \textbf{Signature}.
\end{itemize}

\begin{figure}[hbpc]
  \begin{center}
    \includegraphics[scale=0.4]{token}
  \caption[Esempio di token di Catalogue Manager]{Esempio di token di Catalogue Manager}
  \label{fig:token}
  \end{center} 
\end{figure}

L'\textbf{header} tipicamente consiste di due parti: il tipo del \textit{token} (\texttt{typ}, che deve essere \textbf{JWT}) e l'algoritmo di firma utilizzato (\texttt{alg}), che può essere \glossaryItem{hmac} con \textit{\glossaryItem{sha}-256} o \glossaryItem{rsa}. Nel \lstlistingname~\ref{headerEsempio} è mostrato un esempio di \textit{header}.
\begin{lstlisting}[
		caption={Esempio di header JWT},
		label=headerEsempio,
		language=json,
		firstnumber=1
	]
{
  typ: "JWT",
  alg: "RS256" // RSA
}
\end{lstlisting}
Il tutto è poi codificato con \glossaryItem{base64} e forma la prima parte del \textit{token}.

La seconda parte è il \textbf{payload} e contiene le ''affermazioni'' (o \textbf{claims}), ovvero delle asserzioni di sicurezza riguardo un'entità (tipicamente l'utente) e dati aggiuntivi. Ci sono tre tipi di affermazioni:
\begin{itemize}
\item \textbf{riservate}: insieme di asserzioni predefinite, non obbligatorie, ma raccomandate per fornire un \textit{token} utile. In particolare sono:
	\begin{itemize}
	\item \texttt{iss}: indica l'\texttt{issuer} (emittente) del \textit{token}, e il suo utilizzo generalmente è fortemente dipendente dall'applicazione;
	\item \texttt{sub}: indice il \texttt{subject} (soggetto) del \textit{token} e deve essere unico globalmente o per singolo \texttt{issuer}. Come per il campo \texttt{iss}, anche il suo utilizzo è  fortemente dipendente dall'applicazione;
	\item \texttt{aud}: indica l'\texttt{audience} (pubblico) del \textit{token}. Chiunque voglia utilizzare il \textit{token} deve identificarsi in uno dei casi descritti in questo campo. Se questo non accade, il \glossaryItem{jwt} deve essere scartato. Solitamente è un \textit{array} di stringhe che identificano i diversi casi d'uso;
	\item \texttt{exp}: indica l'\texttt{expiration time} (tempo di scadenza) oltre il quale il \textit{token} non deve essere accettato;
	\item \texttt{nbf}: indica il \texttt{not before time} (non prima di) prima del quale il \textit{token} non deve essere accettato;
	\item \texttt{iat}: indica il momento in cui il \textit{token} è stato emesso (\texttt{issued at}) e può essere usato per determinare l'età del \glossaryItem{jwt};
	\item \texttt{jti}: fornisce un identificatore univoco (\glossaryItem{jwt} ID) per il \textit{token}. L'ID deve essere assegnato in modo tale da garantire l'unicità; se l'applicazione usa più di un \texttt{issuer}, le collisioni devono essere previste ed evitate. È una stringa \textit{case sensitive}, e in generale è utilizzata per evitare la replicazione dei \textit{token}.
	\end{itemize}
I nomi sono tutti di tre caratteri per enfatizzare la compattezza di \glossaryItem{jwt};
\item \textbf{pubbliche}: possono essere definite dagli utilizzatori dei \glossaryItem{jwt}, ma devono essere registrate presso il \textbf{\glossaryItem{JWT} \glossaryItem{iana} Registry} o definite come \glossaryItem{uri}, in modo da evitare collisioni;
\item \textbf{private}: personalizzate ed utilizzate per scambiare informazioni tra due parti in accordo sulle modalità di utilizzo.
\end{itemize}
Un esempio di \textit{payload} è riportato nel \lstlistingname~\ref{payloadEsempio}.
\begin{lstlisting}[
		caption={Esempio di payload JWT},
		label=payloadEsempio,
		language=json,
		firstnumber=1
	]
{
  sub: "1234567890",
  name: "John Doe",
  admin: true
}
\end{lstlisting}
Il tutto è poi codificato con \glossaryItem{base64} e forma la seconda parte del \textit{token}.

Per creare la terza parte tutto quello che serve è l'\textit{header} codificato, il \textit{payload} codificato, una chiave (il \textbf{secret}) e l'algoritmo specificato nell'\textit{header}. Tutto questo va firmato. Nel \lstlistingname~\ref{signatureEsempio} è mostrato un esempio utilizzando \glossaryItem{hmac} con \glossaryItem{sha}-256.
\begin{lstlisting}[
		caption={Esempio di signature JWT},
		label=signatureEsempio,
		language=signature,
		firstnumber=1
	]
HMACSHA256(
  base64UrlEncode(header) + "." +
  base64UrlEncode(payload),
  secret)
\end{lstlisting}
La firma è utilizzata per verificare che chi invia il \glossaryItem{jwt} sia chi dice di essere e per assicurare che il messaggio non venga modificato.

L'\textit{output} finale è un insieme di tre stringhe codificate con \glossaryItem{base64} separate da punti ('.') che può essere facilmente trasmessa con il protocollo \glossaryItem{http}. La stringa ottenuta è inoltre molto più compatta di altre alternative, come le asserzioni \glossaryItem{saml}. Un esempio è stato mostrato in Figura~\ref{fig:token}.

È importante notare come le informazioni siano solo codificate con \glossaryItem{base64} e non criptate: questo le rende visibili da chiunque intercetti il \textit{token}. Vanno quindi inseriti solamente dati pubblici e assolutamente non sensibili.

\paragraph{Utilizzo} \mbox{} \\
Nel caso d'uso dell'\glossaryItem{autenticazione}, quando un utente effettua il \textit{login} utilizzando le proprie credenziali viene generato, e ritornato, un \glossaryItem{jwt} che deve essere salvato localmente. Questa modalità differisce da quella ''classica'', che prevedeva di creare una sessione lato server e di ritornare un \glossaryItem{cookie}.

Ogniqualvolta l'utente vuole accedere ad un servizio, o risorsa, protetto, il \textit{token} deve essere inviato al server. L'invio avviene tipicamente nell'\textit{header} \textbf{Authorization} usando lo schema \texttt{Bearer}. Il contenuto dell'\textit{header} sarà dunque quello mostrato nel \lstlistingname~\ref{bearer}
\begin{lstlisting}[
		caption={Esempio di header HTTP per l'invio di un JWT},
		label=bearer,
		language=bearer,
		firstnumber=1
	]
Authorization: Bearer <token>
\end{lstlisting}
L'\glossaryItem{autenticazione} effettuata in questo modo è \textit{stateless} perché lo stato dell'utente non è mai salvato nella memoria del server. I servizi protetti del server controlleranno se il \glossaryItem{jwt} inviato è valido e se permette di accedere alla risorsa richiesta. Dato che i \textit{token} \glossaryItem{jwt} sono \textit{self-contained}, tutte le informazioni necessarie sono già presenti, e non è necessario interrogare, nuovamente, il \textit{database}. Il diagramma in Figura~\ref{fig:jwtdiagram} mostra l'intero processo.
\begin{figure}[h]
  \begin{center}
    \includegraphics[scale=0.2]{jwt-diagram}
  \caption[Autenticazione con JWT]{Autenticazione con JWT\protect\footnotemark}
  \label{fig:jwtdiagram}
  \end{center} 
\end{figure}
\footnotetext{Immagine tratta da \cite{site:jwtintro}}

\paragraph{Vantaggi} \mbox{} \\
\begin{itemize}
\item \textbf{Compattezza}: le dimensioni ridotte dei \glossaryItem{jwt} li rendono molto versatili.
\item \textbf{Completezza}: il \textit{payload} può contenere tutti i dati necessari a definire l'identità dell'utente, evitando numerose e ricorrenti interrogazioni al \textit{database}.
\item \textbf{Sicurezza}: i \textit{token} vengono firmati in modo da poter verificare che nessuno modifichi i dati che contengono.
\item \textbf{Supporto a \glossaryItem{cors}}: essendo il processo \textit{stateless} non importa quale dominio fornisce le \glossaryItem{api}. Le informazioni necessarie sono tutte nel \textit{token} e nessun \glossaryItem{cookie} viene memorizzato lato server, quindi l'\glossaryItem{autenticazione} è assolutamente indipendente dal dominio.
\end{itemize}

\paragraph{Svantaggi} \mbox{} \\
\begin{itemize}
\item \textbf{\glossaryItem{xss}}: nonostante risultino molto più sicuri dei \textit{cookie} per certi tipi di minacce, anche i \glossaryItem{jwt} sono vulnerabili ad attacchi di tipo \glossaryItem{xss}; la
grande differenza, però, sta nel fatto che mentre i \textit{cookie} hanno la possibilità di rendersi invisibili a codice JavaScript, i \textit{token} (salvati nello \textit{storage} del \textit{browser}) non possono impedire a codice malevolo di accedervi. Una buona strategia per mitigare i rischi di un attacco \glossaryItem{xss} è quello di impostare un valore basso per la scadenza del \textit{token}, prediligendo un rinnovo più frequente dello stesso e limitando
temporalmente la possibilità che un \textit{token} rubato possa essere utilizzato per accedere a risorse protette.
\end{itemize}

\paragraph{Motivazioni della scelta} \mbox{} \\
Si è scelto l'utilizzo dei \textit{token} \glossaryItem{jwt} per la natura di Catalogue Manager: essendo un sistema \glossaryItem{saas} ogni azione effettuata dall'utente, attraverso il front end, genera una richiesta ad un apposito servizio fornito dal back end. La corretta identificazione dell'utente, quindi, risulta fondamentale, non tanto, attualmente, per una questione di \glossaryItem{autorizzazione} (che comunque è prevista nelle versioni successive dell'applicazione), ma per una questione di \textit{logging} delle operazioni svolte. I \glossaryItem{jwt} grazie alla capacità di includere al loro interno informazioni sull'utente in modo sicuro, firmate in modo tale che qualsiasi tentativo di corruzione del \textit{token} venga rilevato durante la verifica di integrità, sono lo strumento ideale a tale scopo. Si adattano, inoltre, molto bene a diversi tipi di dispositivi, consentendo agli utenti di poter utilizzare l'applicativo anche da ambienti \textit{mobile}. La facilità con cui i \glossaryItem{jwt} supportano comunicazioni di tipo \glossaryItem{cors} garantisce infine scalabilità all'applicazione.

\subsection{SSO con SAML} \label{ssoSAML}
In Figura~\ref{fig:spidpinit} sono mostrate le due modalità di accesso in \glossaryItem{sso} con \glossaryItem{saml}: \glossaryItem{idp} e \glossaryItem{sp} Initiated. 

\begin{figure}[h]
\centering
\mbox{
	\begin{subfigure}[b]{\textwidth}
    \centering
    \includegraphics[scale=0.6,clip=false]{idp-init-sso-post}
    \caption{SSO IdP Initiated}
    \label{fig:idpinit}
    \end{subfigure}
}

\mbox{
	\begin{subfigure}[b]{\textwidth}
    \centering
    \includegraphics[scale=0.6,clip=false]{sp-init-sso-post-post}
    \caption{SSO SP Initiated}
    \label{fig:spinit}
    \end{subfigure}
}
\caption[Confronto tra SSO SP e IdP Initiated]{Confronto tra SSO SP e IdP Initiated\protect\footnotemark}\label{fig:spidpinit}
\end{figure}

La differenza principale tra le due modalità è rappresentata dall'azione iniziale dell'utente. Con \textbf{\glossaryItem{idp} Initiated} (Figura~\ref{fig:idpinit}) l'utente richiede immediatamente di effettuare il \textit{login} e, successivamente, di accedere alla risorsa. Al contrario, con \textbf{\glossaryItem{sp} Initiated} (Figura~\ref{fig:spinit}) l'utente cerca fin da subito di accedere alla risorsa. È compito del \glossaryItem{sp} verificare che l'utente abbia effettuato il \textit{login} e, in caso contrario, di reindirizzarlo alla pagina di \textit{login} dell'\glossaryItem{idp}.

Di seguito viene descritta in dettaglio la sequenza di azioni di entrambe le modalità.

\paragraph{IdP Initiated}
\begin{enumerate}
\item L'utente effettua il \textit{login} presso l'\glossaryItem{idp}.
\item L'utente richiede l'accesso ad una risorsa protetta del \glossaryItem{sp}.
\item (\textbf{Opzionale}) Alcuni attributi aggiuntivi possono essere aggiunti alla \textit{SAMLResponse}. Questi attributi vengono ricavati dal \textit{database} delle \glossaryItem{identita} e sono determinati sulla base dei requisiti dell'applicazione.
\item L'\glossaryItem{idp} ritorna un \textit{form} al \textit{browser} dell'utente con l'asserzione \glossaryItem{saml} ed, eventualmente, gli attributi aggiuntivi. Il \textit{browser} invia come POST il \textit{form} al \glossaryItem{sp}.
\item Dopo aver validato l'asserzione e la firma dell'\glossaryItem{idp}, il \glossaryItem{sp} stabilisce una sessione con l'utente e il \textit{browser} viene reindirizzato alla risorsa richiesta.
\end{enumerate}
\footnotetext{Immagini tratte da \cite{site:SPvsIdPinitiated}}
Come si può notare l'utente effettua inizialmente il \textit{login} e, successivamente, cerca di accedere alla risorsa. Il primo componente che entra in gioco è l'\glossaryItem{idp}.

\paragraph{SP Initiated}
\begin{enumerate}
\item L'utente cerca di accedere ad una risorsa web attraverso una richiesta al suo \glossaryItem{sp}. La richiesta viene reindirizzata verso il server che si occupa della federazione del \glossaryItem{sp} per autenticare l'utente.
\item Questo server invia un \textit{form} contenente una \textit{SAMLRequest} per l'\glossaryItem{autenticazione} (\textbf{AuthN Request}) al \textit{browser} dell'utente.
\item Se l'utente non ha già effettuato il \textit{login}, l'\glossaryItem{idp} gli richiede le credenziali.
\item (\textbf{Opzionale}) Alcuni attributi aggiuntivi possono essere aggiunti alla \textit{SAMLResponse}. Questi attributi vengono ricavati dal \textit{database} delle \glossaryItem{identita} e sono determinati sulla base dei requisiti dell'applicazione.
\item Il server di federazione dell'\glossaryItem{idp} ritorna un \textit{form} contenente l'asserzione \glossaryItem{saml}, eventualmente con gli attributi aggiuntivi, al \textit{browser} dell'utente. Automaticamente il \textit{form} viene inoltrato al server di federazione del \glossaryItem{sp}.
\item Dopo aver validato l'asserzione e la firma dell'\glossaryItem{idp}, il \glossaryItem{sp} stabilisce una sessione con l'utente e il \textit{browser} viene reindirizzato alla risorsa richiesta inizialmente.
\end{enumerate}
In questo caso, dunque, l'utente prima richiede la risorsa e successivamente esegue il \textit{login}, se non era già stato fatto in precedenza.

\paragraph{Modalità utilizzata} \mbox{} \\
Lo scenario più comune di utilizzo di \glossaryItem{saml} per un'applicazione web è quello \glossaryItem{sp} Initiated: l'utente può salvare un segnalibro, o seguire un \textit{link}, per arrivare all'applicazione. Il \glossaryItem{sp} reindirizza, se necessario, l'utente verso l'\glossaryItem{idp} per permettergli di autenticarsi. Quest'ultimo crea ad-hoc un'asserzione e la rimanda al \glossaryItem{sp}, che decide se concedere l'accesso alla risorsa richiesta.

Al contrario, nella modalità \glossaryItem{idp} Initiated, l'utente è già autenticato presso un \glossaryItem{idp} e da esso accede alle risorse che ha a disposizione. 

Catalogue Manager rientra perfettamente in quest'ultima categoria: è, infatti, un'applicazione presente direttamente in Monokee e fortemente legata ad esso (tanto che, in questa prima versione, ne condivide il \textit{database}). Catalogue Manager si fida di Monokee e del suo sistema di \glossaryItem{autenticazione}: Monokee è l'\glossaryItem{idp} e Catalogue Manager è il \glossaryItem{sp}. 

Per questo motivo, nella progettazione dell'applicazione si è scelto di adottare la modalità di \glossaryItem{sso} \glossaryItem{idp} Initiated. 

Questa modalità, per quanto più adatta alle esigenze del progetto, richiede che l'\glossaryItem{idp} sia configurato in modo tale da reindirizzare l'utente verso l'applicazione. Fortunatamente, Monokee è stato progettato per supportare entrambe le modalità. Di conseguenza non si è resa necessaria nessuna modifica o configurazione aggiuntiva. 

\section{Principali componenti} 
Di seguito verranno analizzati i principali componenti dell'architettura dell'applicazione, in particolare modelli (\ref{modelli}) e moduli (\ref{moduli}). Per aiutare la spiegazione, ogni componente viene presentato con un diagramma delle classi che ne mostra attributi e metodi (se presenti). La trattazione non è completa: le \textit{routes}, infatti, non vengono discusse in questa sede, ma lo saranno più avanti. La scelta è motivata dal fatto che di seguito vengono descritte solo le componenti rilevanti ai fine dell'architettura e quelle che meglio mostrano l'applicazione di una progettazione orientata agli oggetti. 
 
\subsection{Modelli} \label{modelli}
\paragraph{Catalogue} \mbox{} \\
In Figura~\ref{fig:Catalogue} è mostrato il modello \texttt{Catalogue}, che rappresenta il catalogo delle applicazioni. Ogni \textit{document} della \textit{collection} descrive un'applicazione da catalogo, non importa se di quello di Monokee o di quello di un dominio.
\begin{figure}[h]
  \begin{center}
    \includegraphics[scale=0.6]{Classi/Catalogue}
  \caption[Modello Catalogue]{Modello Catalogue}
  \label{fig:Catalogue}
  \end{center} 
\end{figure}
\subparagraph{Attributi}
\begin{itemize}
\item \texttt{name}: nome dell'applicazione;
\item \texttt{description}: descrizione dell'applicazione;
\item \texttt{url}: \glossaryItem{url} dell'applicazione;
\item \texttt{private\_application}: \texttt{true} se e solo se l'applicazione è privata, ovvero se appartiene al catalogo di un dominio; \texttt{false} altrimenti;
\item \texttt{image}: \textit{path} dell'immagine dell'applicazione, salvata su un server di Monokee;
\item \texttt{categories}: \textit{array} di oggetti \texttt{Category} rappresentante le categorie dell'applicazione. \texttt{Category} (Figura~\ref{fig:Category}) è caratterizzato da due stringhe:
	\begin{itemize}
	\item \texttt{category}: il nome della categoria;
	\item \texttt{sovra\_category}: il nome della ''sovra categoria''.
	\end{itemize}
	\begin{figure}[h]
	  \begin{center}
	    \includegraphics[scale=0.7]{Classi/Category}
	  \caption[Attributi di Category]{Attributi di Category}
	  \label{fig:Category}
	  \end{center} 
	\end{figure}
La consistenza dell'associazione categoria/sovra categoria è assicurata durante la validazione dei dati effettuata da mongoose.js. Tutte le categorie accettabili sono state definite in un \textit{file}, \texttt{categories.json}, che viene letto per assicurare che vengano inseriti solo dati corretti;
\item \texttt{auth\_type}: identifica il tipo di \glossaryItem{autenticazione} per l'applicazione;
\item \texttt{application\_form}: riferimento al \textit{document} contenente i dati per l'\glossaryItem{autenticazione} \textit{form-based}. Se l'applicazione ha un tipo di \glossaryItem{autenticazione} diverso (\glossaryItem{saml} o terzo tipo) è \texttt{null};
\item \texttt{application\_saml}: riferimento al \textit{document} contenente i dati per l'\glossaryItem{autenticazione} \glossaryItem{saml}. Se l'applicazione ha un tipo di \glossaryItem{autenticazione} diverso (\textit{form-based} o terzo tipo) è \texttt{null};
\item \texttt{application\_third\_type}: riferimento al \textit{document} contenente i dati per l'\glossaryItem{autenticazione} del terzo tipo. Se l'applicazione ha un tipo di \glossaryItem{autenticazione} diverso (\glossaryItem{saml} o  \textit{form-based}) è \texttt{null};
\item \texttt{removed}: \textit{flag} per la \glossaryItem{softdeletion}. \texttt{true} se l'applicazione deve essere rimossa, \texttt{false} altrimenti;
\item \texttt{group}: riferimento al \textit{document} contenente i dettagli del gruppo. È \texttt{null} se l'applicazione non appartiene ad un gruppo;
\item \texttt{published}: \texttt{true} se e solo se l'applicazione è pubblicata, \texttt{false} altrimenti;
\item \texttt{work\_in\_progress}: \texttt{true} se e solo se l'applicazione è in manutenzione; \texttt{false} altrimenti;
\item \texttt{available}: \texttt{true} se l'applicazione può essere aggiunta agli \textit{application brokers}, \texttt{false} altrimenti. Il valore di questo \textit{flag} è modificato durante la rimozione dell'applicazione. Se quest'ultima ha associazioni con qualche utente, infatti, non può essere eliminata definitivamente perché non funzionerebbe più il \glossaryItem{sso} per gli utenti che la vedono nel loro \textit{application broker}. Per questo motivo l'applicazione resta presente nel \textit{database}, ma non è più aggiungibile ad alcun \textit{application broker}. È compito di Monokee segnalare la rimozione dell'applicazione originale agli utenti che ne fanno uso.
\end{itemize}

\paragraph{CatalogueDomain} \mbox{} \\
In Figura~\ref{fig:CatalogueDomain} è mostrato il modello \texttt{CatalogueDomain}, che rappresenta il catalogo privato di un dominio. Ogni \textit{document} della \textit{collection} rappresentata da questo modello contiene un \textit{array} di riferimenti ad applicazioni (\textit{documents} del modello \texttt{Catalogue}): l'\textit{array} è il catalogo del dominio.
\begin{figure}[h]
  \begin{center}
    \includegraphics[scale=0.6]{Classi/CatalogueDomain}
  \caption[Modello CatalogueDomain]{Modello CatalogueDomain}
  \label{fig:CatalogueDomain}
  \end{center} 
\end{figure}
\subparagraph{Attributi}
\begin{itemize}
\item \texttt{catalogue\_applications}: \textit{array} contenente i riferimento alle applicazioni del catalogo del dominio;
\item \texttt{removed}: \textit{flag} per la \glossaryItem{softdeletion}. \texttt{true} se il catalogo di dominio deve essere rimosso, \texttt{false} altrimenti.
\end{itemize}

\paragraph{CatalogueForm} \mbox{} \\
In Figura~\ref{fig:CatalogueForm} è mostrato il modello \texttt{CatalogueForm}, che rappresenta i dati necessari per il \glossaryItem{sso} \textit{form-based}. 

Si nota che gli attributi coprono i principali \textit{browser} utilizzati. Sebbene la distinzione per \textit{browser} non sia una buona tecnica per discriminare le azioni da eseguire, questa si è rivelata necessaria a causa delle differenze, anche sostanziali, che essi presentano quando si cerca di effettuare richieste \glossaryItem{ajax}. A seconda della tipologia di \glossaryItem{sso} (tramite \textit{form fulfillment} o \glossaryItem{ajax}), ciascun attributo contiene i dati necessari a compilare il \textit{form} di accesso o per eseguire la richiesta POST.
\begin{figure}[hbpc]
  \begin{center}
    \includegraphics[scale=0.6]{Classi/CatalogueForm}
  \caption[Modello CatalogueForm]{Modello CatalogueForm}
  \label{fig:CatalogueForm}
  \end{center} 
\end{figure}
\subparagraph{Attributi}
\begin{itemize}
\item \texttt{method\_sso\_chrome}: dati per Google Chrome;
\item \texttt{method\_sso\_ie}: dati per Microsoft Internet Explorer;
\item \texttt{method\_sso\_safari}: dati per Safari;
\item \texttt{method\_sso\_edge}: dati per Microsoft Edge;
\item \texttt{method\_sso\_firefox}: dati per Mozilla Firefox;
\item \texttt{removed}: \textit{flag} per la \glossaryItem{softdeletion}. \texttt{true} se il \textit{document} deve essere rimosso, \texttt{false} altrimenti.
\end{itemize}

\paragraph{CatalogueGroup} \mbox{} \\
In Figura~\ref{fig:CatalogueGroup} è mostrato il modello \texttt{CatalogueGroup}, che rappresenta un gruppo di applicazioni del catalogo (di Monokee o di dominio).
\begin{figure}[hbpc]
  \begin{center}
    \includegraphics[scale=0.6]{Classi/CatalogueGroup}
  \caption[Modello CatalogueGroup]{Modello CatalogueGroup}
  \label{fig:CatalogueGroup}
  \end{center} 
\end{figure}
\subparagraph{Attributi}
\begin{itemize}
\item \texttt{name}: nome del gruppo;
\item \texttt{private\_group}: \texttt{true} se e solo se il gruppo è privato, ovvero se è specifico di un dominio; \texttt{false} altrimenti;
\item \texttt{removed}: \textit{flag} per la \glossaryItem{softdeletion}. \texttt{true} se il gruppo deve essere rimosso, \texttt{false} altrimenti;
\item \texttt{domain}: riferimento al dominio di appartenenza del gruppo. È \texttt{null} se il gruppo è pubblico;
\item \texttt{description}: descrizione del gruppo;
\item \texttt{image}: \textit{path} dell'immagine del gruppo, salvata su un server di Monokee.
\end{itemize}

\paragraph{CatalogueLog} \mbox{} \\
In Figura~\ref{fig:CatalogueLog} è mostrato il modello \texttt{CatalogueLog}, che rappresenta un \textit{log} salvato su \textit{database}. Questo modello è utilizzato sia per i \textit{log} delle operazioni eseguite con successo sia per quelle che hanno generato un errore: la distinzione si basa unicamente su un codice definito in fase di progettazione. 
\begin{figure}[h]
  \begin{center}
    \includegraphics[scale=0.6]{Classi/CatalogueLog}
  \caption[Modello CatalogueLog]{Modello CatalogueLog}
  \label{fig:CatalogueLog}
  \end{center} 
\end{figure}
\subparagraph{Attributi}
\begin{itemize}
\item \texttt{infos}: informazioni sull'operazione eseguita. \texttt{Log} è stato progettato per essere il più generico possibile e per poter contenere, di conseguenza, informazioni molto eterogenee tra loro. In Figura~\ref{fig:Log} sono mostrati i suoi attributi. Ogni istanza di \texttt{Log} è caratterizzata principalmente da una coppia chiave/valore, utilizzata per poter recuperare le informazioni durante la visualizzazione dei \textit{log}.
	\begin{figure}[hbpc]
	\begin{center}
  		\includegraphics[scale=0.6]{Classi/Log}
 		\caption[Attributi di Log]{Attributi di Log}
 		\label{fig:Log}
 	\end{center} 
	\end{figure}
	\begin{itemize}
		\item \texttt{key}: chiave, utilizzata per recuperare l'informazione richiesta;
		\item \texttt{value}: valore dell'informazione;
		\item \texttt{element\_id}: ID dell'elemento. È utilizzato per risalire all'elemento (applicazione, gruppo, dominio, eccetera) ''bersaglio'' dell'operazione eseguita e per recuperare, se necessario, informazioni aggiuntive. L'ID è inoltre usato per identificare univocamente l'elemento, soprattutto nel caso di \textit{log} di errore.
	\end{itemize}
\item \texttt{code}: codice del \textit{log};
\item \texttt{user}: indirizzo email dell'utente che ha eseguito l'operazione;
\item \texttt{created\_at}: \textit{timestamp} di esecuzione dell'operazione;
\end{itemize}
La struttura del modello \texttt{CatalogueLog}, e in particolare di \texttt{Log}, rende difficile e possibilmente confusionario l'utilizzo delle informazioni: non tutti i \textit{log} hanno le stesse informazioni, quindi sarebbero necessari molti controlli per capire quali attributi sono presenti di volta in volta. Per questo, prima di essere trasmesse al client, esse sono ''riformattate'' utilizzando una struttura più facilmente utilizzabile. Tale struttura differisce anche di molto in base al codice del \textit{log}: se il client accetta la struttura specifica non deve eseguire nessun controllo aggiuntivo: nel \glossaryItem{json} ritornato sono presenti tutti e soli gli attributi necessari.

\paragraph{CatalogueSAML} \mbox{} \\
In Figura~\ref{fig:CatalogueSAML} è mostrato il modello \texttt{CatalogueSAML}, che rappresenta le istruzioni per la configurazione del \glossaryItem{sso} con un'applicazione di tipo \glossaryItem{saml}. Dato che queste applicazioni devono essere configurate a mano dagli utilizzatori, il catalogo offre solo una serie di azioni da compiere per configurarle al meglio, eventualmente accompagnate da un'immagine.
\begin{figure}[hbpc]
	\begin{center}
  		\includegraphics[scale=0.6]{Classi/CatalogueSAML}
 		\caption[Modello CatalogueSAML]{Modello CatalogueSAML}
 		\label{fig:CatalogueSAML}
 	\end{center} 
\end{figure}
\subparagraph{Attributi}
\begin{itemize}
\item \texttt{instructions}: \textit{array} contenente le istruzioni per la configurazione, viste come usa serie di passi accompagnati da un'immagine. Ogni passo è caratterizzato da nome e descrizione (Figura~\ref{fig:Instruction});
\begin{figure}[hbpc]
	\begin{center}
  		\includegraphics[scale=0.7]{Classi/Instruction}
 		\caption[Attributi di Instruction e Step]{Attributi di Instruction e Step}
 		\label{fig:Instruction}
 	\end{center} 
\end{figure}
\item \texttt{removed}: \textit{flag} per la \glossaryItem{softdeletion}. \texttt{true} se il \textit{document} deve essere rimosso, \texttt{false} altrimenti.
\end{itemize}

\paragraph{CatalogueThirdType} \mbox{} \\
In Figura~\ref{fig:CatalogueThirdType} è mostrato il modello \texttt{CatalogueThirdType}, che rappresenta le informazioni necessarie al \glossaryItem{sso} con applicazioni di terze parti. Un'\glossaryItem{autenticazione} di questo tipo è caratterizzata da una richiesta POST al servizio di accesso dell'applicazione stessa. Di conseguenza vengono memorizzate tutte le informazioni necessarie a popolare la richiesta.

\begin{figure}[hbpc]
	\begin{center}
  		\includegraphics[scale=0.6]{Classi/CatalogueThirdType}
 		\caption[Modello CatalogueThirdType]{Modello CatalogueThirdType}
 		\label{fig:CatalogueThirdType}
 	\end{center} 
\end{figure}
\subparagraph{Attributi}
\begin{itemize}
\item \texttt{post\_url}: \glossaryItem{url} verso il quale effettuare la richiesta POST;
\item \texttt{properties}: dati necessari a popolare la richiesta. In Figura~\ref{fig:Property} viene mostrata in dettaglio la struttura di queste informazioni:
	\begin{figure}[hbpc]
		\begin{center}
	  		\includegraphics[scale=0.7]{Classi/Property}
	 		\caption[Attributi di Property]{Attributi di Property}
	 		\label{fig:Property}
	 	\end{center} 
	\end{figure}
	\begin{itemize}
	\item \texttt{property}: nome della proprietà da mostrare all'utente nel \textit{form} di inserimento dei dati;
	\item \texttt{type}: valore dell'attributo \texttt{type} del \textit{tag input} corrispondente alla \textit{property} nel \textit{form} di inserimento dei dati;
	\item \texttt{post\_property}: nome della \textit{property} da utilizzare nella richiesta;
	\item \texttt{hidden}: \textit{true} se e solo se il corrispondente \textit{tag input} deve essere marcato come nascosto;
	\item \texttt{value}: valore della \textit{property}.
	\end{itemize}
\item \texttt{headers}: \textit{headers} da utilizzare nella richiesta. Come mostrato in Figura~\ref{fig:Header}, un \textit{header} è caratterizzato da un nome (\texttt{name}) e dal valore (\texttt{value});
	\begin{figure}[h]
		\begin{center}
	  		\includegraphics[scale=0.7]{Classi/Header}
	 		\caption[Attributi di Header]{Attributi di Header}
	 		\label{fig:Header}
	 	\end{center} 
	\end{figure}
\item \texttt{removed}: \textit{flag} per la \glossaryItem{softdeletion}. \texttt{true} se il \textit{document} deve essere rimosso, \texttt{false} altrimenti.
\end{itemize}

\paragraph{Domain} \mbox{} \\
Il modello Domain viene utilizzato principalmente dal modulo \texttt{DomainCatalogue} per recuperare l'ID del catalogo di dominio associato. In Figura~\ref{fig:Domain} vengono riportati solo gli attributi utili all'applicazione Catalogue Manager e vengono tralasciati quelli utilizzati solamente da Monokee.

\begin{figure}[h]
	\begin{center}
  		\includegraphics[scale=0.7]{Classi/Domain}
 		\caption[Modello Domain]{Modello Domain}
 		\label{fig:Domain}
 	\end{center} 
\end{figure}

\subparagraph{Attributi}
\begin{itemize}
\item \texttt{name}: nome del dominio. Viene utilizzato per ricercare un dominio a partire dall'applicazione Catalogue Manager;
\item \texttt{type}: tipo del dominio. Come già detto, un dominio può essere \textbf{personale} (\textit{personal}) o \textbf{aziendale} (\textit{company}). I domini \textit{personal} non possono avere un catalogo associato, mentre quelli \textit{company} si. La consistenza dell'attributo \texttt{type} è controllata durante la validazione effettuata da mongoose.js;
\item \texttt{catalogue}: riferimento al \textit{document} di \texttt{CatalogueDomain} contenente il catalogo del dominio.
\end{itemize}

%\paragraph{User e UserApplication} \mbox{} \\
%I modelli User e UserApplication sono utilizzati dal servizio \textbf{/acs} per verificare l'associazione tra utente e applicazione Catalogue Manager. Se questa associazione non è presente l'utente non è autorizzato ad accedere, e viene riportato (e salvato) l'errore. 
%
%Non vengono mostrati i diagrammi delle classi di questi modelli, in quanto non considerati utili ai fini del documento.

  
\subsection{Moduli} \label{moduli}
\begin{center}
\textit{Di seguito verranno trattati i tre moduli di appoggio principali e maggiormente degni di nota dal punto di vista progettuale, ovvero quelli per la gestione delle immagini, degli errori e per il salvataggio dei log su \textit{database}.}
\end{center}
\paragraph{Gestione delle immagini} \mbox{} \\
In Figura~\ref{fig:ImageHandler} è mostrata la gerarchia di moduli utilizzata per gestire le immagini salvate. In particolare, \texttt{ImageHandler} contiene i due metodi principali: \texttt{save} e \texttt{remove}. Di fatto è un'istanza del \textit{design pattern} \textbf{Template Method}: \texttt{save} e \texttt{remove} chiamano dei metodi implementati dalle sottoclassi che ritornano i percorsi della cartella in cui ci sarà (o c'è) l’immagine. Questi metodi sono \texttt{make\_directories} per \texttt{save} e \texttt{get\_base\_path} per \texttt{remove}. Nelle sottoclassi vengono anche salvati, come campi statici, i percorsi delle cartelle delle immagini. 
\begin{figure}[h]
  \begin{center}
    \includegraphics[scale=0.4]{Classi/ImageHandler}
  \caption[Gerarchia per la gestione delle immagini]{Gerarchia per la gestione delle immagini}
  \label{fig:ImageHandler}
  \end{center} 
\end{figure}

\subparagraph{Attributi} \mbox{} \\
Gli attributi principali sono \texttt{id} e \texttt{image}: il primo contiene l'ID dell'elemento (applicazione, gruppo o istruzioni \glossaryItem{saml}), mentre il secondo rappresenta l'immagine codificata con \glossaryItem{base64}. Oltre a questi, ogni sottoclasse contiene dei campi statici per memorizzare i \textit{path} delle cartelle delle immagini. In particolare:
\begin{itemize}
\item \texttt{APPS\_DIR, CATALOGUE\_DIR, GROUP\_DIR e SAML\_DIR} contengono i \textit{path} delle cartelle delle immagini di, rispettivamente, applicazioni (APPS e CATALOGUE), gruppi e istruzioni per applicazioni \glossaryItem{saml};
\item \texttt{DEFAULT\_IMAGE}: \textit{path} dell'immagine di \textit{default}. Per le applicazioni l'immagine è obbligatoria, quindi non è prevista nessuna immagine di \textit{default}.
\end{itemize}

\subparagraph{Metodi} \mbox{} \\
Come già detto, i due metodi principali sono \texttt{save} e \texttt{remove}. Il primo salva l'immagine, utilizzando come nome il valore dell'attributo \texttt{id}; il secondo rimuove l'immagine con nome uguale al valore dell'attributo \texttt{id}. Entrambi, però, sono metodi \textit{template}, ovvero si appoggiano a metodi implementati nelle sottoclassi. In questo modo è possibile implementare le parti invarianti dei due algoritmi una volta sola, evitando la duplicazione del codice. Sono le sottoclassi a fornire il comportamento concreto implementando i metodi lasciati astratti: \texttt{ImageHandler} definisce solo lo scheletro e l'ordine delle operazioni, senza preoccuparsi di come saranno implementate.

\texttt{make\_directories} crea le \textit{directories} che conterranno l'immagine, se non sono già presenti. \texttt{get\_base\_path}, invece, ritorna il percorso generale della \textit{directory} in cui sono salvate le immagini. 

\texttt{\_init} ha la funzione di costruttore, e viene richiamato automaticamente alla creazione di un oggetto tramite \texttt{new}.

\texttt{get\_default\_image}, infine, è un metodo statico che ritorna l'immagine di \textit{default}.

\newpage
\paragraph{ErrorHandler} \mbox{} \\
\begin{figure}[hbpc]
  \begin{center}
    \includegraphics[scale=0.5]{Classi/ErrorHandler}
  \caption[Modulo per la gestione degli errori]{Modulo per la gestione degli errori}
  \label{fig:ErrorHandler}
  \end{center} 
\end{figure}
In Figura~\ref{fig:ErrorHandler} è mostrata il modulo utilizzato per gestire gli errori, \texttt{ErrorHandler}. Ogni risposta di errore passa da qui, indipendentemente dal tipo di errore (non autorizzato, \textit{bad request}, errore del server o interno). In caso di errore, il back end di Catalogue Manager invia al client un \glossaryItem{json} che segue la seguente struttura (\lstlistingname~\ref{jsonErrore}):
\begin{lstlisting}[
		caption={Struttura del JSON di errore},
		label=jsonErrore,
		language=json,
		firstnumber=1
	]
{
	status: false,
	message: "Messaggio di errore",
	error_code: code
}
\end{lstlisting}
Dove \texttt{code} è, ovviamente, il codice dell'errore registrato. Se si tratta di un errore del server, la proprietà \texttt{error\_code} ha valore \texttt{undefined} e non compare nel \glossaryItem{json} ritornato. 
\subparagraph{Attributi}
\begin{itemize}
\item \texttt{res}: oggetto \texttt{Resource} di Express.js. Viene utilizzato per inviare la risposta al client;
\item \texttt{user}: indirizzo email dell'utente. Viene utilizzato per salvare il \textit{log} dell'errore;
\item \texttt{errors}: oggetto contenente tutti i codici degli errori utilizzati da Catalogue Manager e il messaggio di errore associato. Un esempio è mostrato nel \lstlistingname~\ref{errors}:
\begin{lstlisting}[
		caption={Esempio di oggetto contenente i messaggi di errore},
		label=errors,
		language=json,
		firstnumber=1
	]
{
	3000: "Domain catalogue not found",
	3001: "auth_type value is not supported.",
	// ...
	3042: "Private application with public group."
}
\end{lstlisting}
\end{itemize}
\subparagraph{Metodi}
\begin{itemize}
\item \texttt{server\_error}: viene chiamato nel caso in cui ci sia un errore interno del server (esempio quelli generati da mongoose.js). Il parametro \texttt{err} contiene l'errore. Per errori normali (ovvero non di validazione, ma sollevati durante l'esecuzione di un \textit{middleware}) viene semplicemente inviata una risposa di errore e salvato il \textit{log}. Altrimenti (errore di validazione) viene fatto il \textit{parsing} per recuperare le informazioni errate da salvare nel \textit{log} per funzioni di \textit{debug}. Al fine di collegare il \textit{log} ad un'entità di Catalogue Manager (applicazione, gruppo, eccetera), il parametro \texttt{main\_entity} può contenere le informazioni necessarie per memorizzare tale collegamento;
\item \texttt{internal\_error}: invia una risposta di errore con codice \texttt{code} e messaggio \texttt{errors[code]}, senza salvare nessun \textit{log};
\item \texttt{bad\_request}: invia una risposta con \textit{status} \glossaryItem{http} 400. Nella risposta viene anche inserito l’\textit{array} \texttt{missing\_fields}, se presente, per segnalare se alcuni parametri obbligatori non sono stati ricevuti dal servizio invocato. Il messaggio di errore è contenuto in \texttt{msg};
\item \texttt{unauthorized}: invia una risposta con \textit{status} \glossaryItem{http} 401;
\item \texttt{handle\_and\_save\_log}: chiama \texttt{internal\_error} per gestire l'errore e poi salva il \textit{log}. Oltre alle informazioni sulle principali entità coinvolte nell'errore (applicazione, gruppo, ecc) presenti nell’\textit{array} \texttt{refs}, possono essere incluse delle informazioni aggiuntive non collegabili ad un modello di mongoose.js che vanno inserite nell’\textit{array} \texttt{extra\_infos}. In Figura~\ref{fig:handle_and_save_log} sono mostrate le proprietà contenute nei due parametri. \texttt{refs} viene passato così com'è alla funzione di salvataggio: \texttt{model} rappresenta il nome del modello dell'entità coinvolta, \texttt{dispname} il valore dell'attributo \texttt{name} dell'entità e \texttt{element\_id} il suo ID. In alternativa a \texttt{dispname} è possibile utilizzare \texttt{name}: in questo caso esso contiene il nome dell'attributo dell'entità da mostrare in fase di visualizzazione del \textit{log}. La funzione di salvataggio effettuerà un'interrogazione al \textit{database} per ricavare il valore dell'attributo rappresentato da \texttt{name} nel modello \texttt{model}. \texttt{extra\_infos} contiene invece, come già detto, informazioni aggiuntive non collegabili a nessun modello. La sua struttura è molto semplice e simile a quella di Log (\texttt{element\_id} viene lasciato ad \texttt{undefined}).
\begin{figure}[hbpc]
  \begin{center}
    \includegraphics[scale=0.6]{Classi/handle_and_save_log}
  \caption[Proprietà dei parametri per il salvataggio dei log]{Proprietà dei parametri per il salvataggio dei log}
  \label{fig:handle_and_save_log}
  \end{center} 
\end{figure}
\end{itemize}

\newpage
\paragraph{Salvataggio dei log su database} \mbox{} \\
\begin{figure}[hbpc]
  \begin{center}
    \includegraphics[scale=0.5]{Classi/DBLogger}
  \caption[Modulo per il salvataggio dei log su database]{Modulo per il salvataggio dei log su database}
  \label{fig:DBLogger}
  \end{center} 
\end{figure}
In Figura~\ref{fig:DBLogger} è mostrato il modulo \texttt{DBLogger}, utilizzato per il salvataggio dei \textit{log} su \textit{database}. È l'unico modulo di Catalogue Manager che scrive sulla \textit{collection} rappresentata dal modello \texttt{CatalogueLog} e, grazie alla funzione \texttt{extract\_infos}, rende facilmente utilizzabili le informazioni salvate su tale \textit{collection}. Memorizza inoltre tutti i codici di \textit{log} presenti nell'applicazione.

\subparagraph{Attributi} \mbox{} \\
L'unico attributo di \texttt{DBLogger} è \texttt{codes}, che contiene tutti i codici dei log corrispondenti a operazioni riuscite (quindi non quelli per gli errori, memorizzati in \texttt{ErrorHandler}). Questi codici sono accedibili tramite proprietà di questo attributo, che ha la seguente struttura (\lstlistingname~\ref{codiciLog}):
\begin{lstlisting}[
		caption={Esempio di oggetto contenente i codici dei log},
		label=codiciLog,
		language=json,
		firstnumber=1
	]
{
	APP_CREATED: 1,
	PRIVATE_APP_CREATED: 2,
	APP_REMOVED: 3,
	PRIVATE_APP_REMOVED: 4,
	// ...
	ACCESS_ALLOWED: 29,
	ACCESS_DENIED: 30
}
\end{lstlisting}

\subparagraph{Metodi} 
\begin{itemize}
\item \texttt{save}: salva i \textit{log} su \textit{database}. \texttt{user} rappresenta l'indirizzo email dell'utente e \texttt{code} il codice del \textit{log}. Quest'ultimo viene validato attraverso la funzione \texttt{check\_code}, che ritorna \texttt{true} se e solo se il codice è contenuto in \texttt{codes} o se è un codice di errore. Nell'\textit{array} \texttt{data}, invece, sono contenute le informazioni da salvare. La struttura di questo parametro è conforme a quella di \texttt{ModelLogInfo}, in modo da mantenere la flessibilità data dalla doppia proprietà per il nome da mostrare. Come già detto, infatti, \texttt{model} rappresenta il nome del modello dell'entità coinvolta nell'operazione, \texttt{dispname} il valore dell'attributo \texttt{name} dell'entità e \texttt{element\_id} il suo ID. In alternativa a \texttt{dispname} è possibile utilizzare \texttt{name}: in questo caso esso contiene il nome dell'attributo dell'entità da mostrare in fase di visualizzazione del \textit{log}. La funzione di salvataggio effettuerà un'interrogazione al \textit{database} per ricavare il valore dell'attributo rappresentato da \texttt{name} nel modello \texttt{model}. Ad esempio, dato il modello \textit{M}, se \texttt{name} contiene la stringa ''description'', verrà memorizzato nell'attributo \texttt{value} di \texttt{CatalogueLog} il valore \texttt{description} del \textit{document} del modello \textit{M} identificato univocamente da \texttt{element\_id}. In Tabella~\ref{tab:esempioLog} è mostrato un esempio con dei valori. 
\begin{center}
  \bgroup
  
  \begin{longtable}{ | m{4cm} | m{5cm} |}
    \hline
    \cellcolor[gray]{0.9}\textbf{ID} & \cellcolor[gray]{0.9}\textbf{\textit{description}} \\ \hline
    5770fadf7a1725abe77e0382 & Lorem ipsum dolor sit amet \\ \hline
    575ffd1c8259f70d4b775646 & Mauris suscipit semper dui \\ \hline
    \caption[Esempio di salvataggio di un log]{Esempio di salvataggio di un log}
    \label{tab:esempioLog} 
    \end{longtable}
  \egroup
\end{center} 
Supponiamo 
\newline \texttt{data.element\_id} == 5770fadf7a1725abe77e0382 e 
\newline \texttt{data.name} == \textit{description}

In tal caso verrà salvato come \texttt{CatalogueLog.value} la stringa ''Lorem ipsum dolor sit amet''. 

Al contrario, se 
\newline \texttt{data.element\_id} == 5770fadf7a1725abe77e0382 e 
\newline \texttt{data.dispname} == \textit{Mario Rossi} 

verrà salvato come \texttt{CatalogueLog.value} la stringa ''Mario Rossi''.
\item \texttt{extract\_infos}: questa funzione serve per rendere più facilmente utilizzabili le informazioni contenute il \texttt{CatalogueLog}. In particolare, dato un \textit{log} con la struttura mostrata nel \lstlistingname~\ref{rawlog} restituisce lo stesso \textit{log}, ma con la struttura mostrata nel \lstlistingname~\ref{formattedlog}:
\begin{lstlisting}[
		caption={Document di CatalogueLog},
		label=rawlog,
		language=json,
		firstnumber=1
	]
{
	_id: ObjectId,
	code: Number,
	user: String,
	created_at: Date,
	infos: Array[{
		key: String,
		value: String,
		element_id: ObjectId
	}]
}
\end{lstlisting}

\begin{lstlisting}[
		caption={Log riformattato},
		label=formattedlog,
		language=json,
		firstnumber=1
	]
{
	date: Date,
	code: Number,
	infos: Array[{
		application: String,
		group: String,
		domain: String,
		browser: String
		user: String
	}],
	errored: Array[{
		// struttura dipendente dal codice di errore
	}]
}
\end{lstlisting}
Le proprietà di \texttt{infos} sono potenzialmente vuote. Ad esempio, se il \textit{log} riguarda la creazione di un'applicazione pubblica, \texttt{group, domain} e \texttt{browser} saranno stringhe vuote. Le proprietà di \texttt{errored}, invece, dipendono dal codice di errore specifico e, in generale, riportano i valori dei dati errati.
\end{itemize}

\section{Design Pattern}
La progettazione ad oggetti presuppone l'applicazione di alcuni \textit{patterns} noti e molto utilizzati per risolvere problemi comuni. Di seguito verranno esposti quelli utilizzati durante la definizione dell'architettura di Catalogue Manager.
\subsection{Module Pattern}
I moduli sono parte integrante di qualsiasi applicazione di grandi dimensioni, e tipicamente aiutano ad organizzare il codice. In JavaScript ci sono diverse opzioni per implementare un modulo; tra le più conosciute ci sono la \textbf{Object literal notation} e il \textbf{Module pattern}.

\paragraph{Object Literal} \mbox{} \\
Nella \textit{object literal notation} un oggetto viene descritto come un insieme di coppie nome/valore separate da virgole (',') e contenute tra parentesi graffe ('\{\}'). Il \lstlistingname~\ref{objectLiteral} mostra un esempio:
\begin{lstlisting}[
		caption={Oggetto JavaScript in notazione classica},
		label=objectLiteral,
		language=JavaScript,
		firstnumber=1
	]
var contatore = {
	k: 0,
	incrementa_e_stampa: function() {
		this.k++;;
		console.log(this.k);
	}
};

contatore.incrementa_e_stampa(); // stampa "1"
contatore.i = 50; // aggiunto i con valore 50
contatore.k = 20; // k ora vale 20
contatore.incrementa_e_stampa(); // stampa "21"
\end{lstlisting}
Questa notazione non richiede l'utilizzo dell'operatore \texttt{new}. Dall'esterno, tuttavia, chiunque può aggiungere proprietà all'oggetto \texttt{contatore}. L'istruzione \texttt{contatore.i = 50;} ne è un esempio. Oltre a questo, chiunque può modificare \texttt{contatore.k} senza utilizzare il metodo dedicato: non verrà quindi stampato a video il nuovo valore e l'utilizzatore non noterà, inizialmente, nessun risultato.

Questa soluzione manca completamente di \textbf{incapsulazione}: tutti possono vedere, modificare o aggiungere proprietà all'oggetto.

\paragraph{Una soluzione migliore: il Module Pattern} \mbox{} \\
In JavaScript, il \textbf{Module Pattern} è utilizzato per \textit{emulare} il concetto di classe, in modo da avere attributi e metodi pubblici e privati. È quindi possibile decidere quali parti del modulo esporre e quali no, semplicemente ritornando un oggetto contenente le proprietà pubbliche del modulo. 

È importante notare che in JavaScript non è presente il concetto di ''\textit{privacy}'', in quanto non esistono i modificatori di accesso presenti negli altri linguaggi. Le variabili non possono essere dichiarate \textit{pubbliche} o \textit{private}, ed è necessario utilizzare l'ambito di visibilità delle funzioni per simulare questo concetto. Con il Module \textit{pattern} le variabili e le funzioni dichiarate dentro il modulo sono private; al contrario, quelle contenute nell'oggetto ritornato sono pubbliche. 

Il \lstlistingname~\ref{modulePattern} mostra un esempio:
\begin{lstlisting}[
		caption={Module pattern},
		label=modulePattern,
		language=JavaScript,
		firstnumber=1
	]
var contatore = (function() {
	var k = 0; // attributo privato
	
	var incrementa_e_stampa = function() {
		k++;
		stampa();
	};
	
	var decrementa_e_stampa = function() {
		k--;
		stampa();
	};
	
	var get_contatore = function() {
		return k;
	};
	
	// metodo privato
	var stampa = function() {
		console.log(k);
	};
	
	// pubbliche
	return {
		incrementa_e_stampa: incrementa_e_stampa,
		decrementa_e_stampa: decrementa_e_stampa,
		get_contatore: get_contatore
	};
})();

contatore.incrementa_e_stampa(); // stampa "1"
contatore.decrementa_e_stampa(); // stampa "0"
contatore.k = 20;
contatore.get_contatore(); // ritorna "0"
contatore.k; // "20"
\end{lstlisting}

\newpage
\subparagraph{Vantaggi}
\begin{itemize}
\item Aggiunge il concetto di incapsulazione a JavaScript.
\item Se vengono aggiunti dei metodi esternamente alla definizione del modulo, questi non possono accedere alle variabili private.
\end{itemize}

\subparagraph{Svantaggi}
\begin{itemize}
\item Complica l'utilizzo dell'ereditarietà.
\item Rispetto alla notazione ad oggetti classica complica il \textit{testing} perché alcuni membri sono inaccessibili.
\end{itemize}

\subparagraph{Contestualizzazione} \mbox{} \\
In Catalogue Manager tutti i moduli sono realizzati con il Module pattern. Questo ha consentito di progettarli in modo molto più \textit{object-oriented} e di ottenere l'incapsulazione che sarebbe altrimenti assente.

\subsection{Template Method} 
In Figura~\ref{fig:templateMethod} è riportata la struttura del \textit{design pattern} \textbf{Template Method}.
\begin{figure}[h]
  \begin{center}
    \includegraphics[scale=0.6]{designPattern/TemplateMethod}
  \caption[Design pattern Template Method]{Design pattern Template Method}
  \label{fig:templateMethod}
  \end{center} 
\end{figure}

Questo \textit{pattern} consente di definire la struttura di un algoritmo, lasciando alle sottoclassi il compito di implementare alcuni passi secondo le loro necessità. In questo modo si può ridefinire e personalizzare parte del comportamento nelle varie sottoclassi, senza dover riscrivere più volte il codice in comune. Inoltre evita la duplicazione di codice nelle sottoclassi e aderisce all'\textbf{Hollywood Principle}\footnote{''\textbf{Don't call us, we'll call you}''. Proviene dalla filosofia di Hollywood, secondo lo quale sono le case produttrici a chiamare gli attori se hanno bisogno di loro. Contestualizzando, le componenti di alto livello (superclassi) decidono quando e come utilizzare le componenti di basso livello (sottoclassi)}. 

\paragraph{Vantaggi}
\begin{itemize}
\item Nessuna duplicazione di codice.
\item Il riutilizzo di codice avviene per ereditarietà e non per composizione; solo alcuni metodi devono subire l'\textit{override}.
\item Le sottoclassi decidono come implementare i passi dell'algoritmo, migliorando la flessibilità.
\end{itemize}

\paragraph{Svantaggi}
\begin{itemize}
\item Aumenta la difficoltà di \textit{debugging}.
\item Rende più difficile comprendere il flusso di esecuzione.
\end{itemize}

\paragraph{Contestualizzazione} \mbox{} \\
Il \textit{design pattern} Template Method è utilizzato per la gerarchia di classi che gestiscono le immagini salvate su \textit{file system}. \texttt{ImageHandler} espone due metodi, \texttt{save} e \texttt{remove}, che chiamano dei metodi astratti che devono essere implementati dalle sottoclassi. 

\subsection{Middleware}
Questo \textit{pattern} consente di definire uno o più intermediari tra i vari componenti \textit{software} dell'applicazione in modo da semplificare la loro connessione e collaborazione. In generale è molto utile nello sviluppo e nella gestione di di sistemi distribuiti complessi, contesto nel quale Catalogue Manager si colloca perfettamente. 

Viene utilizzato da Express.js per fornire una libreria di funzioni comuni: definisce una serie di livelli (o funzioni) per gestire le varie richieste dell'applicazione. Tutti i componenti del \textit{pattern} \textbf{Middleware} sono collegati l'uno con l'altro e ricevono a turno una richiesta in ingresso finché uno di questi non decide di partire con l'elaborazione: l'\textit{output} di un \textit{middleware} diventa l'\textit{input} per il successivo. Per questo è anche conosciuto con il nome di \textbf{Pipeline}.

Anche mongoose.js utilizza questo concetto: come già spiegato è possibile definire delle funzioni da eseguire prima o dopo un'operazione specificata. 

%In Figura~\ref{fig:middlewareStruttura} è mostrata una possibile struttura del \textit{pattern} Middleware. Si nota che il \textit{Client} contiene un \textit{array} di \textit{middlewares}: questo \textit{array} rappresenta la catena. Per aggiungerne uno è sufficiente utilizzare la funzione \textit{use}, passando come parametro un'istanza di una classe che implementa l'interfaccia \textit{Middleware}. Quest'ultima interfaccia serve solamente a fornire un ''tipo'' utilizzabile da \textit{Client}. In JavaScript il discorso dei tipi è quasi inesistente, e il fatto che anche le funzioni siano trattate come oggetti consente di utilizzare queste ultime come \textit{middlewares}.
%
%\begin{figure}[hbpc]
%  \begin{center}
%    \includegraphics[scale=0.6]{designPattern/MiddlewareStruttura}
%  \caption[Design pattern Middleware]{Design pattern Middleware}
%  \label{fig:middlewareStruttura}
%  \end{center} 
%\end{figure}

Middleware è fortemente legato al \textbf{Chain of Responsibility}, descritto più avanti.

\subparagraph{Vantaggi}
\begin{itemize}
\item Riutilizzo di codice comune a più parti dell'applicazione.
\item Aderenza al \textbf{Single Responsibility Principle}.\footnote{Ogni elemento di un \textit{software} deve avere una sola responsabilità, e tale responsabilità deve essere interamente incapsulata dall'elemento stesso. Tutti i servizi offerti dall'elemento dovrebbero essere strettamente allineati a tale responsabilità.}
\end{itemize}

\subparagraph{Svantaggi}
\begin{itemize}
\item Se mal utilizzato rende difficile la comprensione del flusso di controllo.
\end{itemize}

\subparagraph{Contestualizzazione}\mbox{} \\
In Catalogue Manager il \textit{pattern} Middleware viene utilizzato ampiamente sia nel contesto di Express.js sia in quello di mongoose.js.

Per Express.js serve per registrare funzioni di validazione del \textit{token}, le \textit{routes}, il gestore dell'errore 404 e così via.

Per mongoose.js, invece, serve, ad esempio, a validare i dati inviati (\glossaryItem{url} e categorie in particolare), a forzare il mantenimento di vincoli personalizzati (come quello che impone che applicazioni pubbliche non abbiano lo stesso nome), ad escludere dalle ricerche su \textit{database} determinati valori (ad esempio le applicazioni con il \textit{flag} \texttt{removed == true} o con \texttt{available == false}), eccetera. 

%\begin{figure}[hbpc]
%  \begin{center}
%    \includegraphics[scale=0.6]{designPattern/MiddlewareCatMgr}
%  \caption[Design pattern Middleware in Catalogue Manager]{Design pattern Middleware in Catalogue Manager}
%  \label{fig:middlewareCatMgr}
%  \end{center} 
%\end{figure}
\subsection{Chain of Responsibility}
Il \textit{pattern} \textbf{Chain of Responsibility} permette ad un oggetto di inviare un evento senza sapere chi lo riceverà e chi lo gestirà. Questo passaggio rende i due (o più) oggetti parte di una catena: ciascun oggetto nella catena può gestire l'evento, passarlo al successivo o entrambi. 

Lo scopo è quello di evitare un collegamento statico tra chi emette l'evento e chi lo gestisce: formando una catena la richiesta viene passata da un oggetto all'altro, senza sapere chi e quando la gestirà. Ogni nodo della catena decide se esaudire la richiesta o no, delegando, in quest'ultimo caso, l'onere al nodo successivo. La catena viene attraversata finché un nodo non può eseguire l'ordine del mittente. In questo modo si evita l’\textbf{accoppiamento} fra il mittente di una richiesta e il destinatario.

Un \textit{pattern} di questo tipo si lega facilmente e fortemente al Middleware, e trova in lui una naturale istanziazione: Express.js, infatti, lo utilizza per la gestione dei \textit{middlewares} (così come mongoose.js) e del \textit{routing}. In Figura~\ref{fig:cor} è mostrata la struttura.

\begin{figure}[h]
  \begin{center}
    \includegraphics[scale=0.5]{designPattern/ChainOfResponsibility}
  \caption[Design pattern ChainOfResponsibility]{Design pattern ChainOfResponsibility}
  \label{fig:cor}
  \end{center} 
\end{figure}

Nel gergo di Express.js i \textit{middleware} corrispondono agli oggetti \texttt{ConcreteHandler} del \textit{design pattern}. Sebbene il comportamento e lo scopo sia quasi identico, l'implementazione di Express.js presenta alcune differenze:
\begin{itemize}
\item i \textit{middlewares} di Express.js possono essere classi con un metodo \texttt{handle} o semplici funzioni, in pieno accordo con lo stile funzionale utilizzato dalla maggioranza delle librerie e delle applicazioni scritte in Node.js. In Catalogue Manager è stata utilizzata principalmente la seconda versione;
\item il \textit{design pattern} prevede che l'oggetto \texttt{Handler} abbia un riferimento \texttt{successor} all’\texttt{Handler} successivo. Express.js, invece, passa al metodo di esecuzione del \textit{middleware} una \texttt{callback}. Il \textit{middleware}, eseguendo la \texttt{callback}, passa nuovamente il controllo all'oggetto del server di Express.js il quale passerà il controllo al successivo \textit{middleware};
\item Express.js divide i \textit{middlewares} in due gruppi:  standard e di gestione degli errori. Si distinguono per il numero di parametri che possono gestire (tre e quattro, rispettivamente). Ogni \textit{middleware} può decidere a quale dei due passare il controllo semplicemente variando il numero di parametri (nel secondo caso va specificato l'errore da gestire). Questa funzionalità serve per permettere la gestione di errori senza utilizzare i costrutti \texttt{try catch}, tipici dei linguaggi imperativi, ma inefficaci quando si utilizza lo stile di programmazione asincrono;
\item ogni \textit{middleware} di Express.js deve essere invocato con i seguenti parametri: l'eventuale errore, l'oggetto della richiesta (\texttt{Express.Request}), l'oggetto della risposta (\texttt{Express.Response}), la \texttt{callback} da utilizzare per passare il controllo al successivo \textit{middleware}. L'ordine è importante.
\end{itemize}

\subparagraph{Vantaggi} \mbox{} \\
\begin{itemize}
\item Riduzione dell'accoppiamento.
\item Maggiore flessibilità nell'assegnazione delle responsabilità.
\end{itemize}

\subparagraph{Svantaggi} \mbox{} \\
\begin{itemize}
\item La gestione di una richiesta non è garantita, sia per la possibile mancanza di un \textit{middleware} con quella specifica responsabilità sia per una configurazione sbagliata della catena.
\end{itemize}

\subparagraph{Contestualizzazione}\mbox{} \\
Come già detto per il \textit{pattern} Middleware, il Chain of Responsibility viene utilizzato largamente da Express.js e mongoose.js.
\chapter{Implementazione} \label{implementazione}
\section{Autenticazione a Catalogue Manager}
\subsection{Implementazione dei JWT}
L'autenticazione a Catalogue Manager avviene tramite \glossaryItem{saml} nella modalità \textbf{\glossaryItem{idp} Initiated}. L'accesso deve essere eseguito attraverso Monokee e, in seguito ad esso, viene generato un \textit{token} \glossaryItem{jwt} trasmesso al front end di Catalogue Manager, che, successivamente, invierà questo \textit{token} al back end per ogni richiesta effettuata. Quest'ultimo verificherà la firma del \textit{token} stesso per controllare che le informazioni non siano state alterate e che l'utente sia chi dice di essere. 

Non essendo previsto un sistema di ruoli, l'unico controllo viene effettuato al momento della generazione del \textit{token} e riguarda l'associazione tra utente e applicazione: in questo modo può accedere a Catalogue Manager solo chi ha questa applicazione nel proprio \textit{application broker}. 

Il periodo di validità del \textit{token} è di nove ore, dopodiché è necessario autenticarsi nuovamente. È stato scelto questo periodo di tempo in modo da rendere valido il \textit{token} per una giornata di lavoro e per imporre una nuova \glossaryItem{autenticazione} ogni giorno.

Il \lstlistingname~\ref{header} mostra un esempio di \textit{header} di un \textit{token} di Catalogue Manager.
\begin{lstlisting}[
		caption={Header di un JWT di Catalogue Manager},
		label=header,
		language=json,
		firstnumber=1
	]
{
  typ: "JWT",
  alg: "HS256"
}
\end{lstlisting}
Come descritto in precedenza, questa parte del \textit{token} contiene le informazioni su tipo e algoritmo di firma. La scelta per la modalità di firma è caduta su \textbf{HS256} (\glossaryItem{hmac} con \textit{\glossaryItem{sha}-256}) e non su \textit{RS256} (\glossaryItem{rsa}) per un motivo sostanzialmente legato alle \textit{performance}: la firma e la verifica, nel primo caso, sono molto più veloci rispetto al secondo. Inoltre, la dimensione del \textit{token} è molto minore. 

La differenza sostanziale, comunque, risiede nella modalità di firma. Con \glossaryItem{hmac} chi pensa all'\glossaryItem{autenticazione} ha la chiave (il ''\textbf{secret}''); fornisce la chiave e il messaggio all'algoritmo scelto, che produce la versione firmata. Dopodiché invia il messaggio originale e quello firmato al verificatore, che, disponendo della stessa chiave, ricalcola la firma, controllando se quanto ottenuto è uguale a quanto ricevuto. Con \glossaryItem{rsa}, invece, esistono due chiavi diverse (pubblica e privata). Il messaggio viene firmato con quella privata e inviato al verificatore, che, attraverso l'algoritmo di verifica (ora diverso da quello di firma) controlla se il messaggio originale è uguale a quello ottenuto utilizzando la chiave pubblica. 

È chiaro che nel primo caso entrambe le parti condividono la stessa chiave e il verificatore può, se vuole, generare messaggi che vengono validati senza problemi. Nel secondo caso non accade, perché la chiave pubblica funziona solo per verificare i messaggi e non per firmarli.

Nel caso specifico di Catalogue Manager il compito di creare i \textit{token} e quello di verificarli ricadono su due componenti della stessa applicazione, quindi una può fidarsi, senza problemi, dell'altra (la verifica vera e propria viene effettuata utilizzando un noto modulo di Node.js, \textit{expressjwt}). Tolto questo problema, \glossaryItem{hmac} funziona molto più velocemente, e pertanto è stato preferito a \glossaryItem{rsa}.

Il \lstlistingname~\ref{payload} mostra un esempio di \textit{payload} di un \textit{token} di Catalogue Manager.
\begin{lstlisting}[
		caption={Payload di un JWT di Catalogue Manager},
		label=payload,
		language=json,
		firstnumber=1
	]
{
  id: "5783969ebf947349bb34f7b7",
  email: "matteo.dipirro@email.com",
  iat: 1468413576,
  exp: 1470213576
}
\end{lstlisting}
Come si nota le informazioni fornite sono molto basilari e, come già detto, nessuna di esse è riservata. Oltre ai vincoli sulla validità vengono inviati solo l'ID dell'applicazione Catalogue Manager e l'indirizzo email dell'utente, che sarà utilizzato per salvare i \textit{log} delle operazioni svolte durante la sessione. È ovviamente compito del client inviare il \textit{token} ad ogni richiesta, tramite \textit{header} \glossaryItem{http}. L'ID di Catalogue Manager serve per verificare la presenza dell'associazione tra l'applicazione e l'utente.

\subsection{Modalità SAML IdP Initiated}
In Figura~\ref{fig:idpinitiatedcatmgr} è mostrato il processo di \glossaryItem{autenticazione} \glossaryItem{idp} Initiated utilizzato. L'utente deve innanzitutto autenticarsi a Monokee e avere in uno dei suoi \textit{application brokers} Catalogue Manager. Il compito di accertare l'identità è quindi di Monokee.

\newpage

\begin{figure}[hbpc]
  \begin{center}
    \includegraphics[scale=0.2]{IdPInitiatedCATMGR}
  \caption[Single Sign On IdP Initiated in Catalogue Manager]{Single Sign On IdP Initiated in Catalogue Manager}
  \label{fig:idpinitiatedcatmgr}
  \end{center} 
\end{figure}

Di seguito saranno descritti i passi compiuti, automaticamente, durante l'accesso.
\begin{enumerate}
\item L'utente clicca sull'applicazione Catalogue Manager. Questo causa l'invocazione del un servizio di Monokee che si occupa del \glossaryItem{sso} \glossaryItem{idp} Initiated.
\item Il back end di Monokee effettua una chiamata all'\glossaryItem{idp}, parametrizzata con il \textit{token} dell'utente, l'ID dell'applicazione (Catalogue Manager in questo caso) e l'ID del dominio di appartenenza dell'utente.
\item L'\glossaryItem{idp} può comunicare ulteriormente con il back end di Monokee per ricevere degli attributi aggiuntivi dell'utente. Nel caso di Catalogue Manager, viene prelevato l'indirizzo email.
\item L'\glossaryItem{idp} fornisce la \textit{SAMLResponse} al back end di Monokee.
\item Il back end di Monokee invia la \textit{SAMLResponse} tramite \textit{form} al \textit{browser} dell'utente. Questo, automaticamente, la inoltra ad un servizio di Catalogue Manager (\textbf{/acs}), l'\textit{Assertion Consumer Service} di \glossaryItem{saml}. Il compito di questo servizio è di inoltrare la \textit{SAMLResponse} al \glossaryItem{sp}, aspettare la risposta, validarla e generare il \textit{token} corrispondente.
\item Il back end di Catalogue Manager chiama il \glossaryItem{sp} inviando la \textit{SAMLResponse} ricevuta in precedenza.
\item Il \glossaryItem{sp} risponde al back end di Catalogue Manager. 
\item Il back end di Catalogue Manager chiama il front end in base alla risposta del \glossaryItem{sp}:
	\begin{itemize}
	\item \textbf{Errore nella risposta del \glossaryItem{sp}}: viene chiamato il front end di Catalogue Manager e segnalato l'errore.
	\item \textbf{Successo}: arriva l'indirizzo email dell'utente. Viene controllato che l'utente sia collegato all'applicazione:
	\begin{itemize}
	\item se no, l'errore viene notificato al front end di Catalogue Manager;
	\item se sì, viene generato il \textit{token} dell'utente.
	\end{itemize}
	\end{itemize}
\end{enumerate}
%\section{Prodotto realizzato}
%Parallelamente al back end è stato sviluppato, da una dipendente di \textit{Athesys S.r.l}, il front end di Catalogue Manager. Questo ha permesso di integrare dopo poco tempo l'applicazione con Monokee e di utilizzarla a supporto delle attività di sviluppo del prodotto padre. Come verrà descritto in seguito, inoltre, è stato possibile eseguire dei test di utilizzo dei servizi sviluppati e di ricevere dei \textit{feedbacks} sul livello di esperienza utente. Di seguito verranno descritte le funzionalità sviluppate.

\section{Elenco dei servizi implementati} \label{servizi}
%Di seguito viene fornito un elenco dei servizi implementati .

%\subsection{Moduli}
%In Tabella~\ref{tab:moduliREST} vengono elencati i moduli di appoggio per i servizi \glossaryItem{rest} implementati. Questi moduli permettono di raggruppare le operazioni comuni alle \textit{routes}, evitando la duplicazione di codice e rendendo, di conseguenza, il prodotto più manutenibile. Ogni modulo è fortemente coeso, e dipende in misura quasi completamente nulla dagli altri moduli. L'unica eccezione è rappresentata dalla dipendenza nei confronti del modulo di \textit{logging}.
%\begin{center}
%  \bgroup
%  
%  \begin{longtable}{ | m{4.8cm} | p{7cm} |}
%    \hline
%    \cellcolor[gray]{0.9} \textbf{Nome} & \cellcolor[gray]{0.9} \textbf{Descrizione} \\ \hline
%    ApplicationImageHandler.js & Utilizzato per semplificare la gestione delle immagini delle applicazioni. \\ \hline
%    ErrorHandler.js & Utilizzato per la gestione degli errori. \\ \hline
%    GroupImageHandler.js & Utilizzato per semplificare la gestione delle immagini dei gruppi. \\ \hline
%    ImageHandler.js & Modulo base per la gestione delle immagini. \\ \hline
%    SAMLInstructionsImageHandler.js & Utilizzato per semplificare la gestione delle immagini delle istruzioni per la configurazione di applicazioni \glossaryItem{saml}. \\ \hline
%    check\_required\_fields.js & Utilizzato per controllare la correttezza dei parametri obbligatori dei servizi \glossaryItem{rest}. \\ \hline
%    clear\_db.js & Utilizzato per eliminare i residui della \glossaryItem{softdeletion} dal \textit{database} di Monokee. \\ \hline
%    db\_logger.js & Utilizzato per salvare i log delle operazioni o degli errori. \\ \hline
%    domain\_catalogue.js & Utilizzato per popolare il catalogo di uno specifico dominio. \\ \hline
%    group\_applications\_handler.js& Utilizzato per raggruppare le operazioni sui gruppi di applicazioni. \\ \hline
%    logger.js & Utilizzato per mostrare nella \textit{console} i log prodotti dai servizi \glossaryItem{rest} e per salvarli su \textit{file}. \\ \hline
%    remove\_app.js & Utilizzato per eliminare un'applicazione dal \textit{database} di Monokee. \\ \hline
%    reset\_first\_sign\_in.js & Utilizzato per reimpostare al valore di \textit{default} dei \textit{flag} utilizzati da Monokee per capire quando un utente deve inserire le credenziali. \\ \hline
%    \caption[Moduli del back end di Catalogue Manager]{Moduli del back end di Catalogue Manager}
%    \label{tab:moduliREST} 
%    \end{longtable}
%  \egroup
%\end{center} 

%\subsection{Servizi} \label{servizi}
In Tabella~\ref{tab:serviziREST} vengono elencati i servizi \glossaryItem{rest} implementati ed esposti dal back end di Catalogue Manager. I nomi sono stati assegnati seguendo le convenzioni utilizzate dal \textit{team} di sviluppo.

Tutti i servizi che contengono \texttt{auth/} nel loro \glossaryItem{url} richiedono l'invio di un \textit{token} valido per essere eseguiti.
\begin{center}
  \bgroup
  
  \begin{longtable}{ | m{6.5cm} | p{6.5cm} |}
    \hline
    \cellcolor[gray]{0.9} \textbf{Nome} & \cellcolor[gray]{0.9} \textbf{Descrizione} \\ \hline
    \texttt{/acs} & Data la \textit{SAMLResponse}, verifica che l'utente abbia accesso all'applicazione Catalogue Manager e genera il \textit{token} corrispondente. All'interno di esso viene memorizzato l'indirizzo email dell'utente e il suo identificativo all'interno del \textit{database} di Monokee. \\ \hline
\texttt{/auth/add\_applications\_in\_group} & Aggiunge un insieme di applicazioni ad un gruppo. Le applicazioni e il gruppo vengono identificate tramite i loro ID, e vengono effettuati tutti i controlli necessari per non creare inconsistenze. In particolare:
\begin{itemize}
\item la visibilità dell'applicazione e del gruppo deve essere la stessa;
\item se il gruppo è privato (specifico di un dominio, l'applicazione deve appartenere al catalogo di quel dominio);
\item l'applicazione non deve essere già inserita in un altro gruppo.
\end{itemize}
\\ \hline
\texttt{/auth/create\_application} & Crea un'applicazione pubblica o privata e la aggiunge al catalogo di Monokee, nel primo caso, o al catalogo di un dominio, nel secondo. \\ \hline
\texttt{/auth/create\_domain\_catalogue} & Crea un catalogo di dominio per il dominio selezionato. Se il dominio ha già un catalogo viene sollevato un errore. \\ \hline
\texttt{/auth/create\_group} & Crea un gruppo (pubblico o privato). Durante la creazione di un gruppo è possibile decidere quali applicazioni faranno parte di quel gruppo. Se vengono forniti degli ID di applicazioni vengono fatti gli stessi controlli descritti per \texttt{add\_applications\_in\_group}. \\ \hline
\texttt{/auth/get\_access\_by\_intervals} & Ritorna il numero di accessi effettuati nelle ultime 24 ore e nell'ultima settimana. \\ \hline
%get\_access\_logs & Ritorna i log riguardanti gli accessi permessi o negati in un intervallo temporale deciso dall'utente. \\ \hline
\texttt{/auth/get\_activities\_from\_to} & Ritorna un qualsiasi tipo di \textit{log} (riguardanti le attività su gruppi, applicazioni, accessi o domini) in un intervallo temporale deciso dall'utente. \\ \hline
\texttt{/auth/get\_addable\_applications} & Ritorna, dato l'ID di un gruppo, le applicazioni che possono essere aggiunte a tale gruppo. \\ \hline
\texttt{/auth/get\_application\_details} & Ritorna, dato l'ID di un'applicazione, i dettagli di tale applicazione. I dettagli includono nome, descrizione, \glossaryItem{url}, informazioni sul tipo di \glossaryItem{autenticazione}, eccetera. \\ \hline
%get\_application\_logs & Ritorna i log riguardanti le attività su applicazioni in un intervallo temporale deciso dall'utente. \\ \hline
\texttt{/auth/get\_applications\_by\_dates} & Ritorna il numero di applicazioni, pubbliche e private, aggiunte e rimosse in un intervallo temporale deciso dall'utente. Le informazioni vengono separate per settimana. \\ \hline
\texttt{/auth/get\_applications\_by\_group} & Dato l'ID di un gruppo, ritorna le applicazioni contenute in quel gruppo. \\ \hline
\texttt{/auth/get\_applications\_by\_intervals} & Ritorna le applicazioni aggiunte e rimosse nelle ultime 24 ore, nell'ultima settimana, mese e anno. \\ \hline
\texttt{/auth/get\_applications\_list} & Ritorna la lista di tutte le applicazioni pubbliche. \\ \hline
\texttt{/auth/get\_categories} & Ritorna la lista delle categorie, complete delle sotto categorie. \\ \hline
\texttt{/auth/get\_domain\_by\_name} & Ritorna una lista di domini con nome simile a quello inviato dall'utente. \\ \hline
\texttt{/auth/get\_domain\_catalogue} & Dato un dominio, ritorna le applicazioni del catalogo di quel dominio. \\ \hline
%get\_domain\_logs & Ritorna i log riguardanti le attività su domini in un intervallo temporale deciso dall'utente. \\ \hline
\texttt{/auth/get\_domains\_with\_catalogue} & Ritorna i domini che hanno un catalogo di applicazioni associato. \\ \hline
%get\_error\_logs & Ritorna i log riguardanti le operazioni che hanno generato degli errori in un intervallo temporale deciso dall'utente. \\ \hline
\texttt{/auth/get\_group\_details} & Ritorna i dettagli di un gruppo di applicazioni, come il nome, la descrizione e l'immagine. \\ \hline
\texttt{/auth/get\_groups} & Ritorna i gruppi pubblici di applicazioni presenti in Catalogue Manager. \\ \hline
%get\_groups\_logs & Ritorna i log riguardanti le attività su gruppi in un intervallo temporale deciso dall'utente. \\ \hline
\texttt{/auth/get\_properties} & Ritorna le \texttt{properties} utilizzabili per l'accesso in \glossaryItem{sso} di tipo \textit{form-based}. \\ \hline
\texttt{/auth/get\_sovra\_category\_stats} & Ritorna il numero di applicazioni, pubbliche e private, presenti in ciascuna sotto categoria. \\ \hline
\texttt{/auth/get\_sso\_actions} & Ritorna una lista delle \texttt{actions} da eseguire per l'accesso in \glossaryItem{sso} di tipo \textit{form-based}. \\ \hline
\texttt{/auth/get\_stats} & Ritorna il numero totale di applicazioni e gruppi, pubblici e privati. Vengono inoltre ritornate il numero di applicazioni presenti in ciascuna categoria. \\ \hline
\texttt{/auth/get\_user\_info} & Ritorna le informazioni dell'utente che ha effettuato il \textit{login}. \\ \hline
\texttt{/auth/remove\_application} & Rimuove definitivamente un'applicazione dato il suo ID. \\ \hline
\texttt{/auth/remove\_applications\_from\_group} & Rimuove un insieme di applicazioni da un gruppo. \\ \hline
\texttt{/auth/remove\_domain\_catalogue} & Dato l'ID di un dominio, rimuove il catalogo di quel dominio. Le applicazioni contenute in quel catalogo vengono eliminate definitivamente. \\ \hline
\texttt{/auth/remove\_group} & Dato l'ID di un gruppo, lo rimuove. La rimozione del gruppo causa solo la cancellazione del collegamento con le applicazioni che vi erano inserite. Queste applicazioni restano nel catalogo di Monokee, o in quello di un dominio. \\ \hline
\texttt{/auth/set\_learning\_catalogue} & Consente di utilizzare il \textit{plug in} di Monokee per il \textit{learning}. Il \textit{learning} serve per il \glossaryItem{sso} tramite \textit{form fulfillment}. \\ \hline
\texttt{/auth/set\_learning\_mobile} & Consente inserire i dati per il \textit{form fulfillment} per \textit{browsers} \textit{mobile}. \\ \hline
\texttt{/auth/set\_maintenance} & Consente di aggiornare il \textit{flag} di manutenzione di un'applicazione. \\ \hline
\texttt{/auth/update\_3rd\_type\_data} & Consente di aggiornare i dati di \glossaryItem{autenticazione} per un'applicazione con accesso del terzo tipo. L'\glossaryItem{autenticazione} per queste applicazioni viene effettuata tramite una richiesta POST \glossaryItem{ajax}. È quindi possibile modificare tutti gli elementi di questa richiesta, come le \texttt{properties} utilizzate e gli \texttt{headers} della richiesta. \\ \hline
\texttt{/auth/update\_formSSO\_data} & Consente di aggiornare i dati di \glossaryItem{autenticazione} per un'applicazione di tipo \textit{form-based}. Oltre all'utilizzo del \textit{learning}, quindi, è possibile impostare questi dati anche manualmente. Questa possibilità si rivela molto utile se l'accesso viene effettuato tramite richieste POST \glossaryItem{ajax} e non tramite \textit{form-fulfillment}. \\ \hline
\texttt{/auth/update\_general\_data} & Consente di aggiornare i dati generali di un'applicazione, come il nome, la descrizione, l'\glossaryItem{url} o l'immagine. \\ \hline
\texttt{/auth/update\_group} & Consente di aggiornare i dati generali di un gruppo, ma non di aggiungere o rimuovere applicazioni da esso. \\ \hline
\texttt{/auth/update\_saml\_instructions} & Consente di aggiornare le istruzioni per la configurazione dell'accesso tramite \glossaryItem{saml}. Essendo l'applicazione nel catalogo generale e indipendente dagli utilizzatori, è permesso solo configurare le istruzioni (viste come una serie di azioni da compiere) per configurare le informazioni richieste per l'utilizzo di \glossaryItem{saml}. È compito di chi aggiunge l'applicazione desiderata al proprio \textit{application broker} inserire queste informazioni. \\ \hline
    \caption[Servizi REST esposti dal back end di Catalogue Manager]{Servizi REST esposti dal back end di Catalogue Manager}
    \label{tab:serviziREST} 
    \end{longtable}
  \egroup
\end{center} 
\chapter{Verifica e validazione} \label{vev}

\section{Verifica}
Per \textbf{verifica} si intende il processo che ha il compito di accertare che l'esecuzione delle altre attività non abbia introdotto errori. Risponde quindi alla domanda ''\textbf{Did I build the system right?}''.

Per assicurare un'alta qualità, il \textit{software} è stato analizzato con due diverse tecniche:
\begin{itemize}
\item \textbf{analisi statica}, non prevede l'esecuzione del prodotto;
\item \textbf{analisi dinamica}, prevede l'esecuzione del prodotto.
\end{itemize}

Si è cercato di raggiungere la correttezza del prodotto \textbf{by construction}: la definizione di regole e strategie ha portato alla stesura di codice facile da verificare fin dall'inizio. L'alternativa è la cosiddetta correttezza \textbf{by correction}, che prevede di correggere al volo senza però fare nulla per evitare a priori gli errori più comuni.

\subsection{Analisi statica}
L'analisi statica si basa sulla verifica del codice senza eseguire lo stesso: attraverso strumenti esterni è possibile analizzare quanto scritto per rilevare le principali fonti di \glossaryItemPl{bug}. 

Uno dei principali strumenti in questo campo è \hyperref[jshint]{JSHint}. Grazie alla semplicità e velocità di configurazione, JSHint si integra con moltissimi \textit{editors} di testo e analizza il codice ''al volo'', durante la scrittura. Dopo averlo configurato con le regole di codifica più adatte, JSHint analizzerà il codice alla ricerca di violazioni delle regole precedentemente impostate. Molto spesso, infatti, un utilizzo errato del linguaggio di programmazione scelto inserisce nel prodotto dei \glossaryItemPl{bug} difficili da trovare. L'esempio più classico sono le variabili definite e non inizializzate, ma ce ne sono molti altri: variabili utilizzate e mai dichiarate, conversioni di tipo automatiche, variabili solo scritte e mai lette, eccetera. Un linguaggio estremamente permissivo come JavaScript, inoltre, rende molto difficile l'individuazione di errori di questo tipo, che diventano visibili solo a \textit{run time}: come sarà spiegato successivamente, risalire alla causa degli errori rilevati a \textit{run time} è spesso molto complesso.

JSHint, invece, permette di rilevare alcuni problemi senza eseguire il codice, velocizzando e facilitando il percorso verso la correttezza. La configurazione di regole di codifica ben precise, inoltre, aiuta a scrivere codice facilmente comprensibile, leggibile e, di conseguenza, manutenibile. 

\subsection{Analisi dinamica}
Secondo quanto affermato da Edsger Wybe Dijkstra nel trattato \textbf{Notes On Structured Programming} (1970), il \textit{testing}\footnote{Con \textit{testing} si intendono le tecniche di analisi dinamica.} di un \textit{software} può solo dimostrare la presenza di \glossaryItemPl{bug}, ma mai la loro assenza:

\begin{displayquote}
\centering
\textit{Program testing can be used to show the presence of bugs, but never to show their absence!}

\textit{Edsger Wybe Dijkstra}
\end{displayquote}

È dunque necessario stabilire il giusto numero di \textit{test} per poter trovare il maggior numero di \glossaryItem{bug}, evitando però di cadere nella ridondanza. La progettazione e la scrittura dei \textit{test} ha un costo che deve essere analizzato e non sottovalutato. 

Quando si parla di \textit{testing} del \textit{software} è necessario distinguere tra \glossaryItem{fault} e \glossaryItem{failure}. L'esecuzione dei \textit{test} rileva i \glossaryItemPl{failure}, ma è compito di chi ha scritto il codice risalire ai \glossaryItemPl{fault}: questo compito è particolarmente difficile, visto che non esiste un modo per risalire in modo univoco al \glossaryItem{fault} presente.

Il bersaglio di un \textit{test} può variare: un singolo modulo, un gruppo di moduli (collegati tra loro per uso, struttura, scopo o comportamento) o l'intero sistema. In base a questa distinzione possono essere identificati tre livelli di \textit{testing}: 
\begin{itemize}
\item \textbf{unità}: verificano il funzionamento in isolamento di un elemento \textit{software} atomico (\textbf{unità}\footnote{Con \textbf{unità} si intende la più piccola quantità di \textit{software} che conviene provare da sola. Nel presente documento una unità è rappresentata da un modulo JavaScript o da una \textit{route} di Express.js.});
\item \textbf{integrazione}: verificano l'interazione tra i diversi componenti. Tipicamente si utilizza una strategia incrementale che mira a mettere insieme solo parti già provate, soprattutto per \textit{software} di grandi dimensioni. La strategia opposta è detta \textit{big bang testing} e consiste nel mettere insieme tutte le componenti un'unica vola e provarle tutte insieme. In questo modo, però, risalire ai \glossaryItemPl{fault} può essere molto complicato;
\item \textbf{sistema}: verificano il comportamento dell'intero sistema. A questo punto i \textit{test} di unità e integrazione hanno rilevato la maggior parte dei difetti del prodotto: i \textit{test} di sistema mirano quindi a provare le caratteristiche non funzionali del \textit{software}, come sicurezza o velocità.
\end{itemize}
Esiste un altro tipo di \textit{test}, di \textbf{regressione}, che viene eseguito in seguito alla modifica di una parte \textit{P} del sistema \textit{S} per verificare che la modifica di \textit{P} non abbia introdotto errori né in \textit{P} né nelle altre parti di \textit{S} che hanno relazione con \textit{P}.

Catalogue Manager è stato verificato utilizzando la strategia incrementale descritta in precedenza. Inizialmente sono stati provati i moduli comuni utilizzati dalle \textit{routes} e le \textit{routes} che non utilizzano nessun modulo. Successivamente, man mano che il loro funzionamento è stato verificato, le componenti sono state combinate fino a verificare l'intero sistema nel complesso. 

Per essere efficaci i \textit{test} devono essere \textbf{ripetibili}: le condizioni (stato iniziale del sistema, \textit{input} e \textit{output} attesi) di svolgimento devono quindi essere sempre uguali, in modo da poter valutare e confrontare i risultati ottenuti. Affinché la valutazione dei risultati sia obiettiva, inoltre, l'esecuzione dei \textit{test} deve essere automatizzata. Per questo motivo, per la maggior parte della verifica del prodotto sviluppato è stato utilizzato il \glossaryItem{framework} \hyperref[mocha]{Mocha.js}, con la libreria di asserzioni offerta da \hyperref[express]{Express.js}. Per prove veloci è stato anche utilizzato lo strumento \hyperref[dhc]{DHC}: ogni \textit{test} fallito (che ha rivelato \glossaryItem{failure}) è successivamente stato aggiunto ai \textit{test} automatizzati.

Lo sviluppo parallelo del front end, infine, ha portato alla luce diversi \glossaryItemPl{failure} e ha contribuito alla definizione dei casi di \textit{test}.

\section{Validazione}
Per \textbf{validazione} si intende il processo che conferma, attraverso misurazioni e prove oggettive, che le specifiche del \textit{software} siano conformi ai bisogni dell'utente. Risponde quindi alla domanda ''\textbf{Did I build the right system?}''.

I \textbf{test di accettazione} sono utilizzati per fornire queste prove oggettive, e controllano che i requisiti fissati dall'utente e identificati nell'iniziale processo di analisi siano stati rispettati. 

Catalogue Manager rispetta tutti i requisiti definiti inizialmente, a ha soddisfatto tutte le aspettative.
\chapter{Valutazione retrospettiva} \label{conclusioni}

\section{Obiettivi fissati}
Gli obiettivi fissati erano molto ambiziosi: è stato inizialmente richiesto un elevato numero di funzionalità, che è cresciuto ulteriormente durante lo svolgimento dell'attività di stage. Monokee è un prodotto in continua evoluzione: di conseguenza i requisiti sono soggetti a cambiamenti frequenti e, talvolta, inaspettati. 

Le principali modifiche apportate, in corso d'opera, ai requisiti sono le seguenti:
\begin{itemize}
\item \textbf{tipologie di applicazioni permesse}: come descritto in \ref{catmgr} sono presenti tre tipi di autenticazione diversa: \textit{form based}, tramite \glossaryItem{saml} o di terzo tipo. Inizialmente, tuttavia, solamente le prime due erano state richieste. L'introduzione del nuovo tipo ha portato all'aggiunta di altri servizi \glossaryItem{rest}, di un nuovo modello di mongoose.js e la modifica di alcuni servizi e moduli già scritti;
\item \textbf{servizio di logging e statistiche}: sebbene non fossero funzionalità definite inizialmente, sono diventate importanti grazie al parallelo sviluppo di Monokee, e sono pertanto state inserite come requisiti desiderabili;
\item \textbf{sviluppo di un'applicazione mobile}: ha richiesto la definizione di un ulteriore servizio \glossaryItem{rest} (\texttt{set\_learning\_mobile}) e la modifica di uno dei modelli utilizzati da Catalogue Manager (\texttt{CatalogueForm}). 
\end{itemize}
Oltre a questo, lo sviluppo parallelo del front end ha, in qualche caso, portato alla luce degli aspetti tralasciati durante l'esposizione del progetto e durante le riunioni di analisi del prodotto e dei requisiti.

\section{Obiettivi raggiunti}
Grazie alla collaborazione del personale di \textit{Athesys S.r.l.} e del tutor aziendale Roberto Griggio, il periodo di apprendimento e di analisi si è concluso velocemente e in anticipo rispetto a quanto pianificato. Monokee è un prodotto molto complesso, e capirne a pieno la struttura si è rivelato, inizialmente, molto complicato: al di là delle funzionalità offerte, il sistema di \glossaryItem{idm} ha richiesto del tempo per essere compreso a fondo. Inoltre, l'accesso a Catalogue Manager è di tipo federato con \glossaryItem{saml}, e riuscire a capire il funzionamento di questo standard si è rivelato difficile, ma fondamentale.

L'architettura riflette quella classica utilizzata per applicazioni web basate sul \glossaryItem{framework} Express.js ed ha potuto beneficiare della struttura esistente di Monokee. Se da un lato questo è un aspetto positivo, perché molte scelte si sono rivelate già prese, adottate e provate, dall'altro è stata necessaria una particolare attenzione all'integrazione con gli elementi esistenti, in modo da preservare la scalabilità e la manutenibilità del prodotto. 

Nonostante i cambiamenti nei requisiti e le nuove funzionalità richieste, tutti i requisiti, sia obbligatori che desiderabili, sono stati soddisfatti. 

\section{Conclusioni}
Al momento della scelta dello stage, mi sono trovato davanti una lista di aziende proposte dall'Università di Padova, operanti in diversi settori. Dopo la giornata di presentazione di STAGE-IT, ho avuto modo di approfondire la conoscenza di alcune di esse tramite colloqui individuali e, in seguito, ho scelto \textit{Athesys S.r.l.} per i contenuti dei progetti proposti ed il buon clima aziendale che si è percepito già in fase di contatto. L'ambito dell'\glossaryItem{IAM} era a me sconosciuto, e l'impatto iniziale è stato difficile, ma grazie al clima di collaborazione e all'atteggiamento socievole del \textit{team} è stato facile capire come funzionavano le cose.

Giunti al termine dell'attività risulta comunque doveroso valutare in modo critico quanto prodotto. Il sistema, nell'arco di tutto il suo sviluppo, è stato continuamente oggetto di confronto e discussione da parte del \textit{team} di sviluppo dell'intera applicazione Monokee; questo approccio, se da una parte ha portato alcune volte a modifiche successive (talvolta onerose) di requisiti e componenti, ha indubbiamente permesso al prodotto di crescere come visione collettiva di un insieme di persone anziché di un unico responsabile, guadagnandone in termini di funzionalità individuate e completezza di visione del problema.

Ritengo quindi l'esperienza decisamente positiva. Tutti gli obiettivi fissati sono stati portati a termine ed è nata un'applicazione funzionante ed estensibile che diventerà parte di un progetto interessante e dal grande futuro. L'intera attività di stage si è svolta con il \textit{team} di sviluppo di Monokee, che mi ha aiutato ad apprendere nozioni nuove ed affascinanti. Questa esperienza mi ha fatto capire che da soli possiamo fare molto poco e che grandi risultati possono derivare solo dalla collaborazione di persone motivate e sempre disponibili ad aiutarsi in caso di bisogno.

Una delle cose che mi ha colpito maggiormente durante tutta il periodo di stage è la serietà con cui è stata presa in considerazione ogni mia singola opinione; ognuno ha sempre potuto esprimere le proprie idee, i propri dubbi e le proprie perplessità su quanto stava sviluppando così come sulle altre parti dell'applicativo. Questo atteggiamento mi ha fatto superare la mia iniziale timidezza e mi ha spinto ad esprimere le mie opinioni ed i miei dubbi ogniqualvolta ne sentissi l'utilità, nella speranza di provare a contribuire con qualche piccola idea. 

L'atteggiamento socievole del \textit{team}, inoltre, mi ha permesso di integrarmi all'interno del gruppo con gran semplicità, favorendo la collaborazione con gli altri componenti e risultando parte attiva nella buona riuscita dell'attività.

Ho infine avuto la possibilità di applicare le nozioni apprese durante le molte ore di studio e di lezione di questi tre anni università, ma anche di studiare in modo autonomo nuovi concetti e tecnologie. Si sono rivelate molto utili tutte le conoscenze acquisite durante l'insegnamento di Ingegneria del Software, dall'esistenza di database \glossaryItem{nosql}, all'utilizzo di Node.js e JavaScript per lo sviluppo del back end di applicazioni web, dalle tecniche di analisi dei requisiti alle modalità di stesura della documentazione di un'architettura e del codice. 

Ho capito ancora meglio come l'obiettivo del Corso di Laurea in Informatica sia quello di fornire un ''modo di pensare'' che permetta allo studente (o laureato) di far fronte a qualsiasi situazione, di affrontare autonomamente le difficoltà e di saper scegliere sempre la strada giusta. Molti studenti, tuttavia, durante il loro percorso di studi, valutano la loro preparazione informatica in relazione alle tecnologie che hanno appreso fino a quel momento, ricercando e richiedendo le tecnologie più nuove e criticando i corsi che propongono linguaggi per loro obsoleti. Questa visione, a mio giudizio, è eccessivamente miope: l'informatica è un'area dinamica e in continua evoluzione. Le cose progrediscono e cambiano molto velocemente, e quello che oggi è attuale già domani potrebbe essere obsoleto. Al contrario, le capacità analitiche e di ragionamento sono utili in qualsiasi situazione e permettono di adattarsi al meglio alle difficoltà.

È quindi soprattutto merito di questa visione se mi sono travato ad apprendere velocemente nuove tecnologie e nozioni in ambiti a me sconosciuti e se non ho riscontrato significativi problemi nel corso dell'attività di stage. La capacità di gestire un progetto complesso, di affrontare difficoltà e scadenze e di ragionare a fondo su ogni aspetto sono state provvidenziali e utilissime.

\begin{displayquote}
\centering
\textit{It is not the task of the University to offer what society asks for, but to give what society needs.}

\textit{Edsger Wybe Dijkstra}
\end{displayquote}

%**************************************************************
% Materiale finale
%**************************************************************
\backmatter
\printglossaries
\bibliography{bibliografia} % database di biblatex 


\defbibheading{bibliography}
{
    \phantomsection 
    \addcontentsline{toc}{chapter}{\bibname}
    \chapter*{\bibname\markboth{\bibname}{\bibname}}
}

\setlength\bibitemsep{1.5\itemsep} % spazio tra entry

\DeclareBibliographyCategory{opere}
\DeclareBibliographyCategory{web}

\addtocategory{web}{site:gartner}
\addtocategory{web}{site:scrum}
\addtocategory{web}{site:agilemanifesto}
\addtocategory{web}{site:scrumroles}
\addtocategory{web}{site:spidgov}
\addtocategory{web}{site:spidatt}
\addtocategory{web}{site:nodejava}
\addtocategory{web}{site:paypalNode}
\addtocategory{web}{site:mongodb}
\addtocategory{web}{site:expressjs}
\addtocategory{web}{site:git}
\addtocategory{web}{site:bitbucket}
\addtocategory{web}{site:guelphsso}
\addtocategory{web}{site:jwtintro}
\addtocategory{web}{site:signjwt}
\addtocategory{web}{site:SPvsIdPinitiated}
\addtocategory{web}{site:expressMiddleware}

\addtocategory{opere}{reg:spid}
\addtocategory{opere}{gilchrist05}
\addtocategory{opere}{doglio05}
\addtocategory{opere}{grt:G00269748}
\addtocategory{opere}{grt:G00292924}
\addtocategory{opere}{grt:G00279633}
\addtocategory{opere}{rountree12}
\addtocategory{opere}{std:saml}
\addtocategory{opere}{grt:G00261583}
\addtocategory{opere}{grt:G00296572}
\addtocategory{opere}{swebok}
\addtocategory{opere}{rfc:7519}
\addtocategory{opere}{designPattern}
\addtocategory{opere}{javascriptPatterns}
\addtocategory{opere}{abacNIST}
\addtocategory{opere}{grt:G00260705}

\defbibheading{opere}{\section*{Riferimenti bibliografici}}
\defbibheading{web}{\section*{Siti Web}}
\end{document}