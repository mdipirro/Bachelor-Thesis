\chapter{Introduzione}
\textit{In questo primo capitolo verrà esposta l'organizzazione dell'elaborato e verranno spiegate le convenzioni tipografiche adottate.}

\section{Organizzazione dell'elaborato}
\hyperref[azienda]{Il secondo capitolo} esporrà il contesto aziendale nel quale si è svolta l'attività di stage. \\ \\
\hyperref[identita]{Il terzo capitolo} illustrerà la realtà attuale dei sistemi di \glossaryItem{iam}, le possibili evoluzioni future e gli ambiti di utilizzo. \\ \\
\hyperref[progetto]{Il quarto capitolo} esporrà il prodotto per il quale si è svolta l'attività di stage, Monokee. Verrà inoltre esposto il ruolo del progetto svolto. \\ \\
\hyperref[tecnologie]{Il quinto capitolo} esporrà le tecnologie, gli strumenti e i linguaggi utilizzati nel corso dell'attività di stage. \\ \\
\hyperref[adr]{Il sesto capitolo} presenterà le attività di analisi dei requisiti e l'individuazione delle principali funzionalità da implementare. \\ \\
\hyperref[progettazione]{Il settimo capitolo} presenterà ad alto livello la progettazione effettuata e le principali scelte progettuali implementate. \\ \\
\hyperref[implementazione]{L'ottavo capitolo} illustrerà i risultati dell'attività di implementazione di quanto progettato. \\ \\
\hyperref[vev]{Il nono capitolo} esporrà le attività di verifica e validazione. \\ \\
\hyperref[conclusioni]{Il decimo capitolo} riassumerà alcune considerazioni finali, relativamente al prodotto implementato, all'attività di stage nel suo complesso ed all'intero percorso di studi dello studente.

\section{Convenzioni adottate}
Nella stesura del presente documento sono state adottate le seguenti convenzioni tipografiche:
\begin{itemize}
\item ogni occorrenza di termini tecnici, ambigui o di acronimi verrà marcata in corsivo e con una G a pedice. Questo indica che quel termine è presente nel glossario in fondo al documento;
\item la prima occorrenza di un acronimo viene riportata con la dicitura estesa, in corsivo, e le lettere che compongono l'acronimo vengono riportate in grassetto. Ogni occorrenza successiva presenterà solo la forma ridotta;
\item le parole chiave presenti in ciascun paragrafo saranno marcate in grassetto, in modo da consentirne una facile individuazione;
\item ogni termine corrispondente a nomi di file, componenti dell'architettura o codice verrà marcato utilizzando un \textit{font} non proporzionale (a larghezza fissa);
\item i termini in lingua inglese non presenti nel glossario, e non marcati in grassetto, sono evidenziati semplicemente in corsivo, senza la G a pedice.
\end{itemize}
Inoltre, per ciascun diagramma \glossaryItem{uml} è utilizzato lo standard 2.0. 