Il back end si basa su uno stile \glossaryItem{rest}-\textit{like}, ovvero con le seguenti caratteristiche:
\begin{itemize}
\item stato dell'applicazione e funzionalità divisi in risorse web;
\item ogni risorsa è unica e indirizzabile attraverso un \glossaryItem{uri};
\item tutte le risorse sono condivise come interfaccia uniforme per il trasferimento di stato tra client e risorse. Questo trasferimento consiste in:
\begin{itemize}
\item un insieme vincolato di operazioni ben definite;
\item un insieme vincolato di contenuti, opzionalmente supportato da codice a richiesta;
\item un protocollo:
\begin{itemize}
\item client-server;
\item privo di stato;
\item memorizzabile in cache;
\item a livelli.
\end{itemize}
\end{itemize}
\end{itemize}

\glossaryItem{rest} utilizza il concetto di risorsa (aggregato di dati con un nome e una rappresentazione interna), sulla quale è possibile invocare operazioni \glossaryItem{crud} secondo la corrispondenza indicata in Tabella~\ref{tab:RESTCRUD}.
\begin{center}
  \bgroup
  
  \begin{longtable}{ | m{2cm} | m{2cm} | p{7cm} |}
    \hline
    \cellcolor[gray]{0.9}\textbf{Metodo HTTP} & \cellcolor[gray]{0.9}\textbf{Operazione CRUD} & \cellcolor[gray]{0.9}\textbf{Descrizione} \\ \hline
    GET & Read & Ricava e ritorna informazioni su una risorsa \\ \hline
    PUT & Update & Aggiorna una risorsa \\ \hline
    POST & Create & Crea una risorsa \\ \hline
    DELETE & Delete & Cancella una risorsa \\ \hline
    \caption[Corrispondenza tra CRUD e HTTP]{Corrispondenza tra CRUD e HTTP}
    \label{tab:RESTCRUD} 
    \end{longtable}
  \egroup
\end{center} 
Per la rappresentazione dei dati si è scelto di utilizzare \glossaryItem{json} perché si integra molto bene con le tecnologie utilizzate e con il linguaggio JavaScript. Questo non è vero per \glossaryItem{xml} o \glossaryItem{csv}, che richiederebbero librerie specifiche. Inoltre \glossaryItem{json} è molto meno verboso e molto più flessibile di \glossaryItem{xml}, e si adatta molto bene al dominio dell'applicazione.

Uno stile architetturale di questo tipo permette l'indipendenza completa tra back end e front end, permettendo così espansioni su altre piattaforme senza dover modificare il back end dell'applicazione.