\chapter{Progettazione} \label{progettazione}

\section{Interfaccia REST-like}
Il back end si basa su uno stile \glossaryItem{rest}-\textit{like}, ovvero con le seguenti caratteristiche:
\begin{itemize}
\item stato dell'applicazione e funzionalità divisi in risorse web;
\item ogni risorsa è unica e indirizzabile attraverso un \glossaryItem{uri};
\item tutte le risorse sono condivise come interfaccia uniforme per il trasferimento di stato tra client e risorse. Questo trasferimento consiste in:
\begin{itemize}
\item un insieme vincolato di operazioni ben definite;
\item un insieme vincolato di contenuti, opzionalmente supportato da codice a richiesta;
\item un protocollo:
\begin{itemize}
\item client-server;
\item privo di stato;
\item memorizzabile in cache;
\item a livelli.
\end{itemize}
\end{itemize}
\end{itemize}

\glossaryItem{rest} utilizza il concetto di risorsa (aggregato di dati con un nome e una rappresentazione interna), sulla quale è possibile invocare operazioni \glossaryItem{crud} secondo la corrispondenza indicata in Tabella~\ref{tab:RESTCRUD}.
\begin{center}
  \bgroup
  
  \begin{longtable}{ | m{2cm} | m{2cm} | p{7cm} |}
    \hline
    \cellcolor[gray]{0.9}\textbf{Metodo HTTP} & \cellcolor[gray]{0.9}\textbf{Operazione CRUD} & \cellcolor[gray]{0.9}\textbf{Descrizione} \\ \hline
    GET & Read & Ricava e ritorna informazioni su una risorsa \\ \hline
    PUT & Update & Aggiorna una risorsa \\ \hline
    POST & Create & Crea una risorsa \\ \hline
    DELETE & Delete & Cancella una risorsa \\ \hline
    \caption[Corrispondenza tra CRUD e HTTP]{Corrispondenza tra CRUD e HTTP}
    \label{tab:RESTCRUD} 
    \end{longtable}
  \egroup
\end{center} 
Per la rappresentazione dei dati si è scelto di utilizzare \glossaryItem{json} perché si integra molto bene con le tecnologie utilizzate e con il linguaggio JavaScript. Questo non è vero per \glossaryItem{xml} o \glossaryItem{csv}, che richiederebbero librerie specifiche. Inoltre \glossaryItem{json} è molto meno verboso e molto più flessibile di \glossaryItem{xml}, e si adatta molto bene al dominio dell'applicazione.

Uno stile architetturale di questo tipo permette l'indipendenza completa tra back end e front end, permettendo così espansioni su altre piattaforme senza dover modificare il back end dell'applicazione.

\section{Architettura}
In Figura~\ref{fig:architetturaGenerale} è rappresentata l'architettura di Catalogue Manager. Il diagramma dei \textit{package} rappresenta i componenti ad un livello di dettaglio molto basso, ma sufficiente a capire le relazioni principali. Come si nota, Catalogue Manager utilizza i modelli di mongoose.js (\textit{package} \texttt{Monokee.models}) di Monokee. Questa scelta è stata imposta dall'architettura esistente: alcuni servizi di Monokee utilizzavano, e utilizzano, alcuni modelli riguardanti il catalogo. Uno spostamento completo avrebbe causato numerosi problemi e cambiamenti. È stato pertanto deciso di importare solo e soltanto i modelli necessari allo svolgimento delle operazioni di Catalogue Manager. Successivamente verranno descritti i modelli importati.

\begin{figure}[hbpc]
  \begin{center}
    \includegraphics[scale=0.4]{Classi/architettura}
  \caption[Architettura generale]{Architettura generale}
  \label{fig:architetturaGenerale}
  \end{center} 
\end{figure}

\subsection{Moduli esterni}
Catalogue Manager fa uso di numerosi moduli esterni: nel diagramma in Figura~\ref{fig:architetturaGenerale} sono stati riportati solamente quelli principali.

\paragraph{mongoose.js} \mbox{} \\
\textbf{mongoose.js} è uno strumento di modellazione ad oggetti per MongoDB progettato per lavorare in ambiente asincrono (e quindi ottimo per Node.js) che offre grande supporto per le interrogazioni al \textit{database}. La modellazione ad oggetti consente di progettare con precisione le collezioni attraverso la definizione degli attributi, dei loro tipi e delle loro relazioni: in questo modo è possibile definire uno schema sul quale basarsi. 

La definizione dei tipi, in particolare, consente di controllare la consistenza dei dati inseriti. Grazie alla definizione di \textit{middlewares}\footnote{I \textit{middlewares} sono delle funzioni alle quali è passato il controllo durante l'esecuzione di funzioni asincrone. Sono specifici a livello di schema.}, inoltre, è possibile eseguire delle operazioni prima, o dopo, le interrogazioni al \textit{database}, evitando la replicazione di codice e aderendo al principio \glossaryItem{dry}. Esistono due tipi di \textit{middleware}: \textbf{document}, che agiscono a livello dell'intero \textit{document} MongoDB e che possono essere definiti su operazioni come salvataggio, rimozione o validazione, e \textbf{query}, che agiscono durante le interrogazioni al \textit{database}, in particolare per il \texttt{find} (ricerca), l'\texttt{update} (aggiornamento) e il \texttt{count} (conteggio). La possibilità di eseguire codice personale prima dello svolgimento di queste operazioni consente, ad esempio, di effettuare controlli specifici sui dati e di modificarli se necessario. È uno strumento molto potente del quale si è fatto grande uso.

Un \textit{middleware} meno conosciuto, ma importantissimo per la gestione degli errori in Catalogue Manager, è quello che viene eseguito dopo la sollevazione di un errore (\texttt{post error}). Durante la sua esecuzione si dispone dell'errore sollevato, ed è possibile aggiungerci delle informazioni sfruttando il permissivo paradigma ad oggetti di JavaScript. Questi dati aggiuntivi consentono di salvare dei \textit{log} precisi ed accurati.
\subparagraph{Dipendenze}
Come si nota dal diagramma, mongoose.js è utilizzato dal \textit{package} \texttt{models} di Monokee.

\paragraph{async.js} \mbox{} \\
\textbf{async.js} è un modulo di utilità che fornisce potenti funzioni per lavorare in modo asincrono con JavaScript. È stato progettato per un uso con Node.js, ma è utilizzabile anche direttamente nel \textit{browser}. Tra le circa 70 funzioni presenti si trovano molte utili per operare con gli \textit{array} (come \texttt{map}, \texttt{reduce}, \texttt{filter}, eccetera) e altre che implementano \textit{pattern} comuni per un flusso di controllo asincrono. Tutte queste seguono le convenzioni di Node.js e prevedono una sola funzione di \texttt{callback} (con due parametri: l'errore sollevato, se presente, e i risultati dell'intera esecuzione o fino al verificarsi dell'errore) che va chiamata una sola volta.
\subparagraph{Dipendenze}
async.js è utilizzato da tutto Catalogue Manager, sia dal \textit{package} \texttt{routes} sia da \texttt{modules}.

\paragraph{body-parser} \mbox{} \\
\textbf{body-parser} effettua il \textit{parsing} del \textit{body} delle richieste alle \glossaryItem{api} e lo rende disponibile nella proprietà \texttt{body} dell'oggetto \texttt{req} (\texttt{Request})di Express.js. Mette a disposizione numerose funzioni che definiscono come deve essere effettuato il \textit{parsing}: Catalogue Manager utilizza la funzione \texttt{json} in modo da analizzare solamente \textit{body} di tipo \glossaryItem{json}.
\subparagraph{Dipendenze}
body-parser è utilizzato dallo \textit{script} JavaScript utilizzato per avviare l'applicazione (\texttt{app.js}).

\paragraph{Express.js} \mbox{} \\
Come abbondantemente descritto in \ref{express}, \textbf{Express.js} è utilizzato per definire gli \textit{endpoints} del back end di Catalogue Manager. 

\subparagraph{Dipendenze}
Ogni elemento del \textit{package} \texttt{routes} dipende da Express.js, oltre allo script principale per l'avvio dell'applicazione, che lo utilizza per definire gli \glossaryItem{uri} dei servizi esposti.

\paragraph{expressjwt e jsonwebtoken} \mbox{} \\
Questi due moduli permettono di utilizzare i \glossaryItem{jwt}. \textbf{jsonwebtoken} è un'implementazione che rispetta il documento RFC7519 (\cite{rfc:7519}) ed è utilizzato per generare e firmare \textit{token} \glossaryItem{jwt}. \textbf{expressjwt}, invece, è utilizzato per validare i \glossaryItem{jwt} e per inserire il contenuto del \textit{token} nella proprietà \textbf{user} dell'oggetto \texttt{req} di Express.js. Quest'ultimo modulo consente di autenticare richieste \glossaryItem{http} utilizzando \textit{token} \glossaryItem{jwt} in applicazioni Node.js. 

\subparagraph{Dipendenze}
jsonwebtoken è utilizzato da un unico servizio, \texttt{/acs}, ovvero quello che, dopo aver ricevuto la \textit{SAMLResponse} dall'\glossaryItem{idp}, genera il \textit{token} e lo invia al front end di Catalogue Manager.

expressjwt, invece, è utilizzato nello \textit{script} per l'avvio dell'applicazione e ''protegge'' i servizi per i quali è richiesta l'\glossaryItem{autenticazione}. 

\subsection{Models di Monokee}
Il \textit{package} \texttt{models} di Monokee contiene i modelli di mongoose.js utilizzati da Catalogue Manager, che verranno descritti in dettaglio successivamente (vedi \ref{modelli}). In Figura~\ref{fig:models} sono riportati quelli di maggior interesse.

\begin{figure}[hbpc]
  \begin{center}
    \includegraphics[scale=0.4]{Classi/models}
  \caption[Package models]{Package models}
  \label{fig:models}
  \end{center} 
\end{figure}

Come si può notare dai nomi, la quasi totalità di questi riguarda esclusivamente il catalogo. Il modello principale è \texttt{Catalogue}, che specifica lo schema delle applicazioni da catalogo, non importa se quello di Monokee o uno di dominio. Le applicazioni appartenenti ad uno stesso dominio sono raggruppate nei \textit{document} di \texttt{CatalogueDomain}: ad ogni \textit{document} corrisponde un catalogo di dominio.

\texttt{CatalogueForm}, \texttt{CatalogueSAML} e \texttt{CatalogueThirdType} definiscono le informazioni specifiche per i tre tipi di accesso diversi.

\texttt{CatalogueGroup} quelle dei gruppi di applicazioni del catalogo, mentre \texttt{CatalogueLog} specifica lo schema dei \textit{log}.

\texttt{Domain}, invece, è utilizzato solamente per mantenere la consistenza dei dati e per recuperare il giusto catalogo da \texttt{CatalogueDomain}: definisce le informazioni dei domini di Monokee.

\subsection{Modules di Catalogue Manager}
Il \textit{package} \texttt{modules} di Catalogue Manager (Figura~\ref{fig:modules}) contiene i moduli di utilità usati dalle \textit{routes}, che verranno descritti in dettaglio successivamente (vedi \ref{moduli}).  Questi moduli permettono di raggruppare le operazioni comuni, evitando la duplicazione di codice e rendendo, di conseguenza, il prodotto più manutenibile. Ogni modulo è fortemente coeso, e dipende in misura quasi completamente nulla dagli altri moduli. L'unica eccezione è rappresentata dalla dipendenza nei confronti dei due moduli di \textit{logging}.

\begin{figure}[hbpc]
  \begin{center}
    \includegraphics[scale=0.4]{Classi/modules}
  \caption[Package modules]{Package modules}
  \label{fig:modules}
  \end{center} 
\end{figure}

\texttt{Logger} e \texttt{DBLogger} si occupano di salvare i \textit{log} rispettivamente su \textit{file} e nel \textit{database} (attraverso il modello \texttt{CatalogueLog}).

La gerarchia di \texttt{ImageHandler} implementa il \textit{design pattern} \textbf{Template Method} per il salvataggio e la rimozione delle immagini dei gruppi (\texttt{GroupImageHandler}), delle applicazioni (\texttt{ApplicationImageHandler}) e delle istruzioni per la configurazione di applicazioni di tipo \glossaryItem{saml} (\texttt{SAMLInstructionsImageHandler}). 

\texttt{ClearDB} è utilizzato per eliminare i residui della \glossaryItem{softdeletion} dal \textit{database} di Monokee. Per consentire il \glossaryItem{rollback} dei dati, infatti, i \textit{document} non sono direttamente eliminati dal \textit{database}, ma viene impostato a \textit{true} un \textit{flag} (\texttt{removed}). Alla fine del processo di cancellazione, se non sono stati rilevati errori, \texttt{ClearDB} si occupa di eliminare tutto ciò che ha \texttt{removed} a \texttt{true}. 

\texttt{CheckRequiredFields} controlla semplicemente che tutti i campi obbligatori per il servizio che lo invoca siano presenti nella proprietà \texttt{body} dell'oggetto \texttt{req} di Express.js.

\texttt{DomainCatalogue} popola il catalogo di un dominio specifico. 

\texttt{ErrorHandler} è utilizzato per gestire gli errori riscontrati: oltre a salvare il \textit{log} invia risposte diverse al client in base a quanto è successo. È inoltre in grado di intercettare gli errori sollevati direttamente da mongoose.js (ad esempio, una validazione fallita) e di effettuare un \textit{parsing} per presentare al client un \glossaryItem{json} conforme alla definizione definita durante la progettazione.

\texttt{GroupApplicationsHandler} raggruppa le operazioni che coinvolgono gruppi di applicazioni, come l'aggiunta e la rimozione di applicazioni e il controllo sull'associazione gruppo/applicazione.

\texttt{RemoveApp} si occupa di rimuovere (tramite \glossaryItem{softdeletion}) un'applicazione dal \textit{database} di Monokee, ed è usato anche come \glossaryItem{rollback} se si verificano errori durante una creazione.

\texttt{ResetFirstSignIn}, infine, reimposta i \textit{flag} di \textit{first sign in} in seguito alla modifica delle informazioni sull'\glossaryItem{autenticazione} \textit{form-based}. 

\subsection{Routes di Catalogue Manager}
Il \textit{package} \texttt{routes} (Figura~\ref{fig:routes}) di Catalogue Manager contiene gli \textit{endpoints} esposti dal server. In generale, ogni \textit{route} corrisponde a (e soddisfa un) requisito funzionale di alto livello. I servizi \glossaryItem{rest} definiti sono circa 40, e un diagramma delle classi che li mostri tutti risulterebbe illeggibile e inutile. Una descrizione di ciascuna \textit{route} viene fornita in \ref{servizi}.

\begin{figure}[hbpc]
  \begin{center}
    \includegraphics[scale=0.4]{Classi/routes}
  \caption[Package routes]{Package routes}
  \label{fig:routes}
  \end{center} 
\end{figure}

\section{Modalità di autenticazione}
Come già visto, l'\glossaryItem{autenticazione} a Catalogue Manager deve avvenire tramite \glossaryItem{saml}. Tuttavia non esiste un modo univoco di effettuare questa \glossaryItem{autenticazione}, in quanto \glossaryItem{saml} prevede due diverse modalità di accesso, ognuna corrispondente ad uno specifico caso d'uso: in \ref{ssoSAML} verranno analizzate entrambe. 

Risolto il problema dell'accesso, è importante stabilire come verrà mantenuta la sessione con l'utente autenticato, ovvero se questa sarà memorizzata sul server o meno. La scelta è stata quella di lasciare al client l'onere di far sapere se l'\glossaryItem{autenticazione} è avvenuta o no, attraverso l'uso dei \glossaryItem{jwt}, descritti in \ref{descrJWT}.

\subsection{JSON Web Token} \label{descrJWT}
\paragraph{Descrizione} \mbox{} \\
\glossaryItem{jwt} (logo in Figura~\ref{fig:jwt}) è uno standard \textit{open} (\cite{rfc:7519}) che definisce un modo \textbf{compatto} e \textbf{self-contained} per trasmettere informazioni in modo sicuro sotto forma di oggetti \glossaryItem{json}. Le informazioni possono essere verificate dato che sono firmate: la firma può avvenire attraverso una stringa (il cosiddetto \textbf{secret}), con l'algoritmo \glossaryItem{hmac}, o usando una coppia di chiavi pubbliche e private, grazie a \glossaryItem{rsa}.

\begin{figure}[h]
  \begin{center}
    \includegraphics[scale=0.5]{jwt}
  \caption[Logo di JWT]{Logo di JWT\protect\footnotemark}
  \label{fig:jwt}
  \end{center} 
\end{figure}
\footnotetext{Immagine tratta da \cite{site:jwtintro}}

\glossaryItem{jwt} è \textbf{compatto} perché grazie alla sua dimensione ridotta può essere inviato attraverso un \glossaryItem{url}, come parametro di una richiesta POST o dentro un \textit{header} \glossaryItem{http}. Inoltre, occupando poco spazio, può essere trasmesso velocemente. 

\glossaryItem{jwt} è \textbf{self-contained} perché il \textbf{payload} contiene tutte le informazioni riguardo l'utente, evitando di dover interrogare il \textit{database} più di una volta.

I due scenari di utilizzo più comuni sono i seguenti:
\begin{itemize}
\item \textbf{\glossaryItem{autenticazione}}: è il caso d'uso più comune. Dopo la prima \glossaryItem{autenticazione} ogni successiva richiesta conterrà il \textit{token} e permetterà all'utente di accedere a tutto ciò che il suo ruolo gli consente (servizi, risorse, eccetera). \glossaryItem{jwt} è molto usato con il \glossaryItem{sso} grazie alla sua interoperabilità e alla facilità di utilizzo;
\item \textbf{scambio di informazioni}: \glossaryItem{jwt} è un ottimo modo per trasmettere informazioni in modo sicuro tra parti diverse, grazie alla possibilità di firmarli. Visto che la firma è calcolata sul contenuto, si può anche controllare che le informazioni non siano state alterate.
\end{itemize}

\paragraph{Struttura} \mbox{} \\
In Figura~\ref{fig:token} è mostrato un esempio di \textit{token} utilizzato in Catalogue Manager. Come si può notare è composto da tre parti separate da '.':
\begin{itemize}
\item \textbf{Header};
\item \textbf{Payload};
\item \textbf{Signature}.
\end{itemize}

\begin{figure}[hbpc]
  \begin{center}
    \includegraphics[scale=0.4]{token}
  \caption[Esempio di token di Catalogue Manager]{Esempio di token di Catalogue Manager}
  \label{fig:token}
  \end{center} 
\end{figure}

L'\textbf{header} tipicamente consiste di due parti: il tipo del \textit{token} (\texttt{typ}, che deve essere \textbf{JWT}) e l'algoritmo di firma utilizzato (\texttt{alg}), che può essere \glossaryItem{hmac} con \textit{\glossaryItem{sha}-256} o \glossaryItem{rsa}. Nel \lstlistingname~\ref{headerEsempio} è mostrato un esempio di \textit{header}.
\begin{lstlisting}[
		caption={Esempio di header JWT},
		label=headerEsempio,
		language=json,
		firstnumber=1
	]
{
  typ: "JWT",
  alg: "RS256" // RSA
}
\end{lstlisting}
Il tutto è poi codificato con \glossaryItem{base64} e forma la prima parte del \textit{token}.

La seconda parte è il \textbf{payload} e contiene le ''affermazioni'' (o \textbf{claims}), ovvero delle asserzioni di sicurezza riguardo un'entità (tipicamente l'utente) e dati aggiuntivi. Ci sono tre tipi di affermazioni:
\begin{itemize}
\item \textbf{riservate}: insieme di asserzioni predefinite, non obbligatorie, ma raccomandate per fornire un \textit{token} utile. In particolare sono:
	\begin{itemize}
	\item \texttt{iss}: indica l'\texttt{issuer} (emittente) del \textit{token}, e il suo utilizzo generalmente è fortemente dipendente dall'applicazione;
	\item \texttt{sub}: indice il \texttt{subject} (soggetto) del \textit{token} e deve essere unico globalmente o per singolo \texttt{issuer}. Come per il campo \texttt{iss}, anche il suo utilizzo è  fortemente dipendente dall'applicazione;
	\item \texttt{aud}: indica l'\texttt{audience} (pubblico) del \textit{token}. Chiunque voglia utilizzare il \textit{token} deve identificarsi in uno dei casi descritti in questo campo. Se questo non accade, il \glossaryItem{jwt} deve essere scartato. Solitamente è un \textit{array} di stringhe che identificano i diversi casi d'uso;
	\item \texttt{exp}: indica l'\texttt{expiration time} (tempo di scadenza) oltre il quale il \textit{token} non deve essere accettato;
	\item \texttt{nbf}: indica il \texttt{not before time} (non prima di) prima del quale il \textit{token} non deve essere accettato;
	\item \texttt{iat}: indica il momento in cui il \textit{token} è stato emesso (\texttt{issued at}) e può essere usato per determinare l'età del \glossaryItem{jwt};
	\item \texttt{jti}: fornisce un identificatore univoco (\glossaryItem{jwt} ID) per il \textit{token}. L'ID deve essere assegnato in modo tale da garantire l'unicità; se l'applicazione usa più di un \texttt{issuer}, le collisioni devono essere previste ed evitate. È una stringa \textit{case sensitive}, e in generale è utilizzata per evitare la replicazione dei \textit{token}.
	\end{itemize}
I nomi sono tutti di tre caratteri per enfatizzare la compattezza di \glossaryItem{jwt};
\item \textbf{pubbliche}: possono essere definite dagli utilizzatori dei \glossaryItem{jwt}, ma devono essere registrate presso il \textbf{\glossaryItem{JWT} \glossaryItem{iana} Registry} o definite come \glossaryItem{uri}, in modo da evitare collisioni;
\item \textbf{private}: personalizzate ed utilizzate per scambiare informazioni tra due parti in accordo sulle modalità di utilizzo.
\end{itemize}
Un esempio di \textit{payload} è riportato nel \lstlistingname~\ref{payloadEsempio}.
\begin{lstlisting}[
		caption={Esempio di payload JWT},
		label=payloadEsempio,
		language=json,
		firstnumber=1
	]
{
  sub: "1234567890",
  name: "John Doe",
  admin: true
}
\end{lstlisting}
Il tutto è poi codificato con \glossaryItem{base64} e forma la seconda parte del \textit{token}.

Per creare la terza parte tutto quello che serve è l'\textit{header} codificato, il \textit{payload} codificato, una chiave (il \textbf{secret}) e l'algoritmo specificato nell'\textit{header}. Tutto questo va firmato. Nel \lstlistingname~\ref{signatureEsempio} è mostrato un esempio utilizzando \glossaryItem{hmac} con \glossaryItem{sha}-256.
\begin{lstlisting}[
		caption={Esempio di signature JWT},
		label=signatureEsempio,
		language=signature,
		firstnumber=1
	]
HMACSHA256(
  base64UrlEncode(header) + "." +
  base64UrlEncode(payload),
  secret)
\end{lstlisting}
La firma è utilizzata per verificare che chi invia il \glossaryItem{jwt} sia chi dice di essere e per assicurare che il messaggio non venga modificato.

L'\textit{output} finale è un insieme di tre stringhe codificate con \glossaryItem{base64} separate da punti ('.') che può essere facilmente trasmessa con il protocollo \glossaryItem{http}. La stringa ottenuta è inoltre molto più compatta di altre alternative, come le asserzioni \glossaryItem{saml}. Un esempio è stato mostrato in Figura~\ref{fig:token}.

È importante notare come le informazioni siano solo codificate con \glossaryItem{base64} e non criptate: questo le rende visibili da chiunque intercetti il \textit{token}. Vanno quindi inseriti solamente dati pubblici e assolutamente non sensibili.

\paragraph{Utilizzo} \mbox{} \\
Nel caso d'uso dell'\glossaryItem{autenticazione}, quando un utente effettua il \textit{login} utilizzando le proprie credenziali viene generato, e ritornato, un \glossaryItem{jwt} che deve essere salvato localmente. Questa modalità differisce da quella ''classica'', che prevedeva di creare una sessione lato server e di ritornare un \glossaryItem{cookie}.

Ogniqualvolta l'utente vuole accedere ad un servizio, o risorsa, protetto, il \textit{token} deve essere inviato al server. L'invio avviene tipicamente nell'\textit{header} \textbf{Authorization} usando lo schema \texttt{Bearer}. Il contenuto dell'\textit{header} sarà dunque quello mostrato nel \lstlistingname~\ref{bearer}
\begin{lstlisting}[
		caption={Esempio di header HTTP per l'invio di un JWT},
		label=bearer,
		language=bearer,
		firstnumber=1
	]
Authorization: Bearer <token>
\end{lstlisting}
L'\glossaryItem{autenticazione} effettuata in questo modo è \textit{stateless} perché lo stato dell'utente non è mai salvato nella memoria del server. I servizi protetti del server controlleranno se il \glossaryItem{jwt} inviato è valido e se permette di accedere alla risorsa richiesta. Dato che i \textit{token} \glossaryItem{jwt} sono \textit{self-contained}, tutte le informazioni necessarie sono già presenti, e non è necessario interrogare, nuovamente, il \textit{database}. Il diagramma in Figura~\ref{fig:jwtdiagram} mostra l'intero processo.
\begin{figure}[h]
  \begin{center}
    \includegraphics[scale=0.2]{jwt-diagram}
  \caption[Autenticazione con JWT]{Autenticazione con JWT\protect\footnotemark}
  \label{fig:jwtdiagram}
  \end{center} 
\end{figure}
\footnotetext{Immagine tratta da \cite{site:jwtintro}}

\paragraph{Vantaggi} \mbox{} \\
\begin{itemize}
\item \textbf{Compattezza}: le dimensioni ridotte dei \glossaryItem{jwt} li rendono molto versatili.
\item \textbf{Completezza}: il \textit{payload} può contenere tutti i dati necessari a definire l'identità dell'utente, evitando numerose e ricorrenti interrogazioni al \textit{database}.
\item \textbf{Sicurezza}: i \textit{token} vengono firmati in modo da poter verificare che nessuno modifichi i dati che contengono.
\item \textbf{Supporto a \glossaryItem{cors}}: essendo il processo \textit{stateless} non importa quale dominio fornisce le \glossaryItem{api}. Le informazioni necessarie sono tutte nel \textit{token} e nessun \glossaryItem{cookie} viene memorizzato lato server, quindi l'\glossaryItem{autenticazione} è assolutamente indipendente dal dominio.
\end{itemize}

\paragraph{Svantaggi} \mbox{} \\
\begin{itemize}
\item \textbf{\glossaryItem{xss}}: nonostante risultino molto più sicuri dei \textit{cookie} per certi tipi di minacce, anche i \glossaryItem{jwt} sono vulnerabili ad attacchi di tipo \glossaryItem{xss}; la
grande differenza, però, sta nel fatto che mentre i \textit{cookie} hanno la possibilità di rendersi invisibili a codice JavaScript, i \textit{token} (salvati nello \textit{storage} del \textit{browser}) non possono impedire a codice malevolo di accedervi. Una buona strategia per mitigare i rischi di un attacco \glossaryItem{xss} è quello di impostare un valore basso per la scadenza del \textit{token}, prediligendo un rinnovo più frequente dello stesso e limitando
temporalmente la possibilità che un \textit{token} rubato possa essere utilizzato per accedere a risorse protette.
\end{itemize}

\paragraph{Motivazioni della scelta} \mbox{} \\
Si è scelto l'utilizzo dei \textit{token} \glossaryItem{jwt} per la natura di Catalogue Manager: essendo un sistema \glossaryItem{saas} ogni azione effettuata dall'utente, attraverso il front end, genera una richiesta ad un apposito servizio fornito dal back end. La corretta identificazione dell'utente, quindi, risulta fondamentale, non tanto, attualmente, per una questione di \glossaryItem{autorizzazione} (che comunque è prevista nelle versioni successive dell'applicazione), ma per una questione di \textit{logging} delle operazioni svolte. I \glossaryItem{jwt} grazie alla capacità di includere al loro interno informazioni sull'utente in modo sicuro, firmate in modo tale che qualsiasi tentativo di corruzione del \textit{token} venga rilevato durante la verifica di integrità, sono lo strumento ideale a tale scopo. Si adattano, inoltre, molto bene a diversi tipi di dispositivi, consentendo agli utenti di poter utilizzare l'applicativo anche da ambienti \textit{mobile}. La facilità con cui i \glossaryItem{jwt} supportano comunicazioni di tipo \glossaryItem{cors} garantisce infine scalabilità all'applicazione.

\subsection{SSO con SAML} \label{ssoSAML}
In Figura~\ref{fig:spidpinit} sono mostrate le due modalità di accesso in \glossaryItem{sso} con \glossaryItem{saml}: \glossaryItem{idp} e \glossaryItem{sp} Initiated. 

\begin{figure}[h]
\centering
\mbox{
	\begin{subfigure}[b]{\textwidth}
    \centering
    \includegraphics[scale=0.6,clip=false]{idp-init-sso-post}
    \caption{SSO IdP Initiated}
    \label{fig:idpinit}
    \end{subfigure}
}

\mbox{
	\begin{subfigure}[b]{\textwidth}
    \centering
    \includegraphics[scale=0.6,clip=false]{sp-init-sso-post-post}
    \caption{SSO SP Initiated}
    \label{fig:spinit}
    \end{subfigure}
}
\caption[Confronto tra SSO SP e IdP Initiated]{Confronto tra SSO SP e IdP Initiated\protect\footnotemark}\label{fig:spidpinit}
\end{figure}

La differenza principale tra le due modalità è rappresentata dall'azione iniziale dell'utente. Con \textbf{\glossaryItem{idp} Initiated} (Figura~\ref{fig:idpinit}) l'utente richiede immediatamente di effettuare il \textit{login} e, successivamente, di accedere alla risorsa. Al contrario, con \textbf{\glossaryItem{sp} Initiated} (Figura~\ref{fig:spinit}) l'utente cerca fin da subito di accedere alla risorsa. È compito del \glossaryItem{sp} verificare che l'utente abbia effettuato il \textit{login} e, in caso contrario, di reindirizzarlo alla pagina di \textit{login} dell'\glossaryItem{idp}.

Di seguito viene descritta in dettaglio la sequenza di azioni di entrambe le modalità.

\paragraph{IdP Initiated}
\begin{enumerate}
\item L'utente effettua il \textit{login} presso l'\glossaryItem{idp}.
\item L'utente richiede l'accesso ad una risorsa protetta del \glossaryItem{sp}.
\item (\textbf{Opzionale}) Alcuni attributi aggiuntivi possono essere aggiunti alla \textit{SAMLResponse}. Questi attributi vengono ricavati dal \textit{database} delle \glossaryItem{identita} e sono determinati sulla base dei requisiti dell'applicazione.
\item L'\glossaryItem{idp} ritorna un \textit{form} al \textit{browser} dell'utente con l'asserzione \glossaryItem{saml} ed, eventualmente, gli attributi aggiuntivi. Il \textit{browser} invia come POST il \textit{form} al \glossaryItem{sp}.
\item Dopo aver validato l'asserzione e la firma dell'\glossaryItem{idp}, il \glossaryItem{sp} stabilisce una sessione con l'utente e il \textit{browser} viene reindirizzato alla risorsa richiesta.
\end{enumerate}
\footnotetext{Immagini tratte da \cite{site:SPvsIdPinitiated}}
Come si può notare l'utente effettua inizialmente il \textit{login} e, successivamente, cerca di accedere alla risorsa. Il primo componente che entra in gioco è l'\glossaryItem{idp}.

\paragraph{SP Initiated}
\begin{enumerate}
\item L'utente cerca di accedere ad una risorsa web attraverso una richiesta al suo \glossaryItem{sp}. La richiesta viene reindirizzata verso il server che si occupa della federazione del \glossaryItem{sp} per autenticare l'utente.
\item Questo server invia un \textit{form} contenente una \textit{SAMLRequest} per l'\glossaryItem{autenticazione} (\textbf{AuthN Request}) al \textit{browser} dell'utente.
\item Se l'utente non ha già effettuato il \textit{login}, l'\glossaryItem{idp} gli richiede le credenziali.
\item (\textbf{Opzionale}) Alcuni attributi aggiuntivi possono essere aggiunti alla \textit{SAMLResponse}. Questi attributi vengono ricavati dal \textit{database} delle \glossaryItem{identita} e sono determinati sulla base dei requisiti dell'applicazione.
\item Il server di federazione dell'\glossaryItem{idp} ritorna un \textit{form} contenente l'asserzione \glossaryItem{saml}, eventualmente con gli attributi aggiuntivi, al \textit{browser} dell'utente. Automaticamente il \textit{form} viene inoltrato al server di federazione del \glossaryItem{sp}.
\item Dopo aver validato l'asserzione e la firma dell'\glossaryItem{idp}, il \glossaryItem{sp} stabilisce una sessione con l'utente e il \textit{browser} viene reindirizzato alla risorsa richiesta inizialmente.
\end{enumerate}
In questo caso, dunque, l'utente prima richiede la risorsa e successivamente esegue il \textit{login}, se non era già stato fatto in precedenza.

\paragraph{Modalità utilizzata} \mbox{} \\
Lo scenario più comune di utilizzo di \glossaryItem{saml} per un'applicazione web è quello \glossaryItem{sp} Initiated: l'utente può salvare un segnalibro, o seguire un \textit{link}, per arrivare all'applicazione. Il \glossaryItem{sp} reindirizza, se necessario, l'utente verso l'\glossaryItem{idp} per permettergli di autenticarsi. Quest'ultimo crea ad-hoc un'asserzione e la rimanda al \glossaryItem{sp}, che decide se concedere l'accesso alla risorsa richiesta.

Al contrario, nella modalità \glossaryItem{idp} Initiated, l'utente è già autenticato presso un \glossaryItem{idp} e da esso accede alle risorse che ha a disposizione. 

Catalogue Manager rientra perfettamente in quest'ultima categoria: è, infatti, un'applicazione presente direttamente in Monokee e fortemente legata ad esso (tanto che, in questa prima versione, ne condivide il \textit{database}). Catalogue Manager si fida di Monokee e del suo sistema di \glossaryItem{autenticazione}: Monokee è l'\glossaryItem{idp} e Catalogue Manager è il \glossaryItem{sp}. 

Per questo motivo, nella progettazione dell'applicazione si è scelto di adottare la modalità di \glossaryItem{sso} \glossaryItem{idp} Initiated. 

Questa modalità, per quanto più adatta alle esigenze del progetto, richiede che l'\glossaryItem{idp} sia configurato in modo tale da reindirizzare l'utente verso l'applicazione. Fortunatamente, Monokee è stato progettato per supportare entrambe le modalità. Di conseguenza non si è resa necessaria nessuna modifica o configurazione aggiuntiva. 

\section{Principali componenti} 
Di seguito verranno analizzati i principali componenti dell'architettura dell'applicazione, in particolare modelli (\ref{modelli}) e moduli (\ref{moduli}). Per aiutare la spiegazione, ogni componente viene presentato con un diagramma delle classi che ne mostra attributi e metodi (se presenti). La trattazione non è completa: le \textit{routes}, infatti, non vengono discusse in questa sede, ma lo saranno più avanti. La scelta è motivata dal fatto che di seguito vengono descritte solo le componenti rilevanti ai fine dell'architettura e quelle che meglio mostrano l'applicazione di una progettazione orientata agli oggetti. 
 
\subsection{Modelli} \label{modelli}
\paragraph{Catalogue} \mbox{} \\
In Figura~\ref{fig:Catalogue} è mostrato il modello \texttt{Catalogue}, che rappresenta il catalogo delle applicazioni. Ogni \textit{document} della \textit{collection} descrive un'applicazione da catalogo, non importa se di quello di Monokee o di quello di un dominio.
\begin{figure}[h]
  \begin{center}
    \includegraphics[scale=0.6]{Classi/Catalogue}
  \caption[Modello Catalogue]{Modello Catalogue}
  \label{fig:Catalogue}
  \end{center} 
\end{figure}
\subparagraph{Attributi}
\begin{itemize}
\item \texttt{name}: nome dell'applicazione;
\item \texttt{description}: descrizione dell'applicazione;
\item \texttt{url}: \glossaryItem{url} dell'applicazione;
\item \texttt{private\_application}: \texttt{true} se e solo se l'applicazione è privata, ovvero se appartiene al catalogo di un dominio; \texttt{false} altrimenti;
\item \texttt{image}: \textit{path} dell'immagine dell'applicazione, salvata su un server di Monokee;
\item \texttt{categories}: \textit{array} di oggetti \texttt{Category} rappresentante le categorie dell'applicazione. \texttt{Category} (Figura~\ref{fig:Category}) è caratterizzato da due stringhe:
	\begin{itemize}
	\item \texttt{category}: il nome della categoria;
	\item \texttt{sovra\_category}: il nome della ''sovra categoria''.
	\end{itemize}
	\begin{figure}[h]
	  \begin{center}
	    \includegraphics[scale=0.7]{Classi/Category}
	  \caption[Attributi di Category]{Attributi di Category}
	  \label{fig:Category}
	  \end{center} 
	\end{figure}
La consistenza dell'associazione categoria/sovra categoria è assicurata durante la validazione dei dati effettuata da mongoose.js. Tutte le categorie accettabili sono state definite in un \textit{file}, \texttt{categories.json}, che viene letto per assicurare che vengano inseriti solo dati corretti;
\item \texttt{auth\_type}: identifica il tipo di \glossaryItem{autenticazione} per l'applicazione;
\item \texttt{application\_form}: riferimento al \textit{document} contenente i dati per l'\glossaryItem{autenticazione} \textit{form-based}. Se l'applicazione ha un tipo di \glossaryItem{autenticazione} diverso (\glossaryItem{saml} o terzo tipo) è \texttt{null};
\item \texttt{application\_saml}: riferimento al \textit{document} contenente i dati per l'\glossaryItem{autenticazione} \glossaryItem{saml}. Se l'applicazione ha un tipo di \glossaryItem{autenticazione} diverso (\textit{form-based} o terzo tipo) è \texttt{null};
\item \texttt{application\_third\_type}: riferimento al \textit{document} contenente i dati per l'\glossaryItem{autenticazione} del terzo tipo. Se l'applicazione ha un tipo di \glossaryItem{autenticazione} diverso (\glossaryItem{saml} o  \textit{form-based}) è \texttt{null};
\item \texttt{removed}: \textit{flag} per la \glossaryItem{softdeletion}. \texttt{true} se l'applicazione deve essere rimossa, \texttt{false} altrimenti;
\item \texttt{group}: riferimento al \textit{document} contenente i dettagli del gruppo. È \texttt{null} se l'applicazione non appartiene ad un gruppo;
\item \texttt{published}: \texttt{true} se e solo se l'applicazione è pubblicata, \texttt{false} altrimenti;
\item \texttt{work\_in\_progress}: \texttt{true} se e solo se l'applicazione è in manutenzione; \texttt{false} altrimenti;
\item \texttt{available}: \texttt{true} se l'applicazione può essere aggiunta agli \textit{application brokers}, \texttt{false} altrimenti. Il valore di questo \textit{flag} è modificato durante la rimozione dell'applicazione. Se quest'ultima ha associazioni con qualche utente, infatti, non può essere eliminata definitivamente perché non funzionerebbe più il \glossaryItem{sso} per gli utenti che la vedono nel loro \textit{application broker}. Per questo motivo l'applicazione resta presente nel \textit{database}, ma non è più aggiungibile ad alcun \textit{application broker}. È compito di Monokee segnalare la rimozione dell'applicazione originale agli utenti che ne fanno uso.
\end{itemize}

\paragraph{CatalogueDomain} \mbox{} \\
In Figura~\ref{fig:CatalogueDomain} è mostrato il modello \texttt{CatalogueDomain}, che rappresenta il catalogo privato di un dominio. Ogni \textit{document} della \textit{collection} rappresentata da questo modello contiene un \textit{array} di riferimenti ad applicazioni (\textit{documents} del modello \texttt{Catalogue}): l'\textit{array} è il catalogo del dominio.
\begin{figure}[h]
  \begin{center}
    \includegraphics[scale=0.6]{Classi/CatalogueDomain}
  \caption[Modello CatalogueDomain]{Modello CatalogueDomain}
  \label{fig:CatalogueDomain}
  \end{center} 
\end{figure}
\subparagraph{Attributi}
\begin{itemize}
\item \texttt{catalogue\_applications}: \textit{array} contenente i riferimento alle applicazioni del catalogo del dominio;
\item \texttt{removed}: \textit{flag} per la \glossaryItem{softdeletion}. \texttt{true} se il catalogo di dominio deve essere rimosso, \texttt{false} altrimenti.
\end{itemize}

\paragraph{CatalogueForm} \mbox{} \\
In Figura~\ref{fig:CatalogueForm} è mostrato il modello \texttt{CatalogueForm}, che rappresenta i dati necessari per il \glossaryItem{sso} \textit{form-based}. 

Si nota che gli attributi coprono i principali \textit{browser} utilizzati. Sebbene la distinzione per \textit{browser} non sia una buona tecnica per discriminare le azioni da eseguire, questa si è rivelata necessaria a causa delle differenze, anche sostanziali, che essi presentano quando si cerca di effettuare richieste \glossaryItem{ajax}. A seconda della tipologia di \glossaryItem{sso} (tramite \textit{form fulfillment} o \glossaryItem{ajax}), ciascun attributo contiene i dati necessari a compilare il \textit{form} di accesso o per eseguire la richiesta POST.
\begin{figure}[hbpc]
  \begin{center}
    \includegraphics[scale=0.6]{Classi/CatalogueForm}
  \caption[Modello CatalogueForm]{Modello CatalogueForm}
  \label{fig:CatalogueForm}
  \end{center} 
\end{figure}
\subparagraph{Attributi}
\begin{itemize}
\item \texttt{method\_sso\_chrome}: dati per Google Chrome;
\item \texttt{method\_sso\_ie}: dati per Microsoft Internet Explorer;
\item \texttt{method\_sso\_safari}: dati per Safari;
\item \texttt{method\_sso\_edge}: dati per Microsoft Edge;
\item \texttt{method\_sso\_firefox}: dati per Mozilla Firefox;
\item \texttt{removed}: \textit{flag} per la \glossaryItem{softdeletion}. \texttt{true} se il \textit{document} deve essere rimosso, \texttt{false} altrimenti.
\end{itemize}

\paragraph{CatalogueGroup} \mbox{} \\
In Figura~\ref{fig:CatalogueGroup} è mostrato il modello \texttt{CatalogueGroup}, che rappresenta un gruppo di applicazioni del catalogo (di Monokee o di dominio).
\begin{figure}[hbpc]
  \begin{center}
    \includegraphics[scale=0.6]{Classi/CatalogueGroup}
  \caption[Modello CatalogueGroup]{Modello CatalogueGroup}
  \label{fig:CatalogueGroup}
  \end{center} 
\end{figure}
\subparagraph{Attributi}
\begin{itemize}
\item \texttt{name}: nome del gruppo;
\item \texttt{private\_group}: \texttt{true} se e solo se il gruppo è privato, ovvero se è specifico di un dominio; \texttt{false} altrimenti;
\item \texttt{removed}: \textit{flag} per la \glossaryItem{softdeletion}. \texttt{true} se il gruppo deve essere rimosso, \texttt{false} altrimenti;
\item \texttt{domain}: riferimento al dominio di appartenenza del gruppo. È \texttt{null} se il gruppo è pubblico;
\item \texttt{description}: descrizione del gruppo;
\item \texttt{image}: \textit{path} dell'immagine del gruppo, salvata su un server di Monokee.
\end{itemize}

\paragraph{CatalogueLog} \mbox{} \\
In Figura~\ref{fig:CatalogueLog} è mostrato il modello \texttt{CatalogueLog}, che rappresenta un \textit{log} salvato su \textit{database}. Questo modello è utilizzato sia per i \textit{log} delle operazioni eseguite con successo sia per quelle che hanno generato un errore: la distinzione si basa unicamente su un codice definito in fase di progettazione. 
\begin{figure}[h]
  \begin{center}
    \includegraphics[scale=0.6]{Classi/CatalogueLog}
  \caption[Modello CatalogueLog]{Modello CatalogueLog}
  \label{fig:CatalogueLog}
  \end{center} 
\end{figure}
\subparagraph{Attributi}
\begin{itemize}
\item \texttt{infos}: informazioni sull'operazione eseguita. \texttt{Log} è stato progettato per essere il più generico possibile e per poter contenere, di conseguenza, informazioni molto eterogenee tra loro. In Figura~\ref{fig:Log} sono mostrati i suoi attributi. Ogni istanza di \texttt{Log} è caratterizzata principalmente da una coppia chiave/valore, utilizzata per poter recuperare le informazioni durante la visualizzazione dei \textit{log}.
	\begin{figure}[hbpc]
	\begin{center}
  		\includegraphics[scale=0.6]{Classi/Log}
 		\caption[Attributi di Log]{Attributi di Log}
 		\label{fig:Log}
 	\end{center} 
	\end{figure}
	\begin{itemize}
		\item \texttt{key}: chiave, utilizzata per recuperare l'informazione richiesta;
		\item \texttt{value}: valore dell'informazione;
		\item \texttt{element\_id}: ID dell'elemento. È utilizzato per risalire all'elemento (applicazione, gruppo, dominio, eccetera) ''bersaglio'' dell'operazione eseguita e per recuperare, se necessario, informazioni aggiuntive. L'ID è inoltre usato per identificare univocamente l'elemento, soprattutto nel caso di \textit{log} di errore.
	\end{itemize}
\item \texttt{code}: codice del \textit{log};
\item \texttt{user}: indirizzo email dell'utente che ha eseguito l'operazione;
\item \texttt{created\_at}: \textit{timestamp} di esecuzione dell'operazione;
\end{itemize}
La struttura del modello \texttt{CatalogueLog}, e in particolare di \texttt{Log}, rende difficile e possibilmente confusionario l'utilizzo delle informazioni: non tutti i \textit{log} hanno le stesse informazioni, quindi sarebbero necessari molti controlli per capire quali attributi sono presenti di volta in volta. Per questo, prima di essere trasmesse al client, esse sono ''riformattate'' utilizzando una struttura più facilmente utilizzabile. Tale struttura differisce anche di molto in base al codice del \textit{log}: se il client accetta la struttura specifica non deve eseguire nessun controllo aggiuntivo: nel \glossaryItem{json} ritornato sono presenti tutti e soli gli attributi necessari.

\paragraph{CatalogueSAML} \mbox{} \\
In Figura~\ref{fig:CatalogueSAML} è mostrato il modello \texttt{CatalogueSAML}, che rappresenta le istruzioni per la configurazione del \glossaryItem{sso} con un'applicazione di tipo \glossaryItem{saml}. Dato che queste applicazioni devono essere configurate a mano dagli utilizzatori, il catalogo offre solo una serie di azioni da compiere per configurarle al meglio, eventualmente accompagnate da un'immagine.
\begin{figure}[hbpc]
	\begin{center}
  		\includegraphics[scale=0.6]{Classi/CatalogueSAML}
 		\caption[Modello CatalogueSAML]{Modello CatalogueSAML}
 		\label{fig:CatalogueSAML}
 	\end{center} 
\end{figure}
\subparagraph{Attributi}
\begin{itemize}
\item \texttt{instructions}: \textit{array} contenente le istruzioni per la configurazione, viste come usa serie di passi accompagnati da un'immagine. Ogni passo è caratterizzato da nome e descrizione (Figura~\ref{fig:Instruction});
\begin{figure}[hbpc]
	\begin{center}
  		\includegraphics[scale=0.7]{Classi/Instruction}
 		\caption[Attributi di Instruction e Step]{Attributi di Instruction e Step}
 		\label{fig:Instruction}
 	\end{center} 
\end{figure}
\item \texttt{removed}: \textit{flag} per la \glossaryItem{softdeletion}. \texttt{true} se il \textit{document} deve essere rimosso, \texttt{false} altrimenti.
\end{itemize}

\paragraph{CatalogueThirdType} \mbox{} \\
In Figura~\ref{fig:CatalogueThirdType} è mostrato il modello \texttt{CatalogueThirdType}, che rappresenta le informazioni necessarie al \glossaryItem{sso} con applicazioni di terze parti. Un'\glossaryItem{autenticazione} di questo tipo è caratterizzata da una richiesta POST al servizio di accesso dell'applicazione stessa. Di conseguenza vengono memorizzate tutte le informazioni necessarie a popolare la richiesta.

\begin{figure}[hbpc]
	\begin{center}
  		\includegraphics[scale=0.6]{Classi/CatalogueThirdType}
 		\caption[Modello CatalogueThirdType]{Modello CatalogueThirdType}
 		\label{fig:CatalogueThirdType}
 	\end{center} 
\end{figure}
\subparagraph{Attributi}
\begin{itemize}
\item \texttt{post\_url}: \glossaryItem{url} verso il quale effettuare la richiesta POST;
\item \texttt{properties}: dati necessari a popolare la richiesta. In Figura~\ref{fig:Property} viene mostrata in dettaglio la struttura di queste informazioni:
	\begin{figure}[hbpc]
		\begin{center}
	  		\includegraphics[scale=0.7]{Classi/Property}
	 		\caption[Attributi di Property]{Attributi di Property}
	 		\label{fig:Property}
	 	\end{center} 
	\end{figure}
	\begin{itemize}
	\item \texttt{property}: nome della proprietà da mostrare all'utente nel \textit{form} di inserimento dei dati;
	\item \texttt{type}: valore dell'attributo \texttt{type} del \textit{tag input} corrispondente alla \textit{property} nel \textit{form} di inserimento dei dati;
	\item \texttt{post\_property}: nome della \textit{property} da utilizzare nella richiesta;
	\item \texttt{hidden}: \textit{true} se e solo se il corrispondente \textit{tag input} deve essere marcato come nascosto;
	\item \texttt{value}: valore della \textit{property}.
	\end{itemize}
\item \texttt{headers}: \textit{headers} da utilizzare nella richiesta. Come mostrato in Figura~\ref{fig:Header}, un \textit{header} è caratterizzato da un nome (\texttt{name}) e dal valore (\texttt{value});
	\begin{figure}[h]
		\begin{center}
	  		\includegraphics[scale=0.7]{Classi/Header}
	 		\caption[Attributi di Header]{Attributi di Header}
	 		\label{fig:Header}
	 	\end{center} 
	\end{figure}
\item \texttt{removed}: \textit{flag} per la \glossaryItem{softdeletion}. \texttt{true} se il \textit{document} deve essere rimosso, \texttt{false} altrimenti.
\end{itemize}

\paragraph{Domain} \mbox{} \\
Il modello Domain viene utilizzato principalmente dal modulo \texttt{DomainCatalogue} per recuperare l'ID del catalogo di dominio associato. In Figura~\ref{fig:Domain} vengono riportati solo gli attributi utili all'applicazione Catalogue Manager e vengono tralasciati quelli utilizzati solamente da Monokee.

\begin{figure}[h]
	\begin{center}
  		\includegraphics[scale=0.7]{Classi/Domain}
 		\caption[Modello Domain]{Modello Domain}
 		\label{fig:Domain}
 	\end{center} 
\end{figure}

\subparagraph{Attributi}
\begin{itemize}
\item \texttt{name}: nome del dominio. Viene utilizzato per ricercare un dominio a partire dall'applicazione Catalogue Manager;
\item \texttt{type}: tipo del dominio. Come già detto, un dominio può essere \textbf{personale} (\textit{personal}) o \textbf{aziendale} (\textit{company}). I domini \textit{personal} non possono avere un catalogo associato, mentre quelli \textit{company} si. La consistenza dell'attributo \texttt{type} è controllata durante la validazione effettuata da mongoose.js;
\item \texttt{catalogue}: riferimento al \textit{document} di \texttt{CatalogueDomain} contenente il catalogo del dominio.
\end{itemize}

%\paragraph{User e UserApplication} \mbox{} \\
%I modelli User e UserApplication sono utilizzati dal servizio \textbf{/acs} per verificare l'associazione tra utente e applicazione Catalogue Manager. Se questa associazione non è presente l'utente non è autorizzato ad accedere, e viene riportato (e salvato) l'errore. 
%
%Non vengono mostrati i diagrammi delle classi di questi modelli, in quanto non considerati utili ai fini del documento.

  
\subsection{Moduli} \label{moduli}
\begin{center}
\textit{Di seguito verranno trattati i tre moduli di appoggio principali e maggiormente degni di nota dal punto di vista progettuale, ovvero quelli per la gestione delle immagini, degli errori e per il salvataggio dei log su \textit{database}.}
\end{center}
\paragraph{Gestione delle immagini} \mbox{} \\
In Figura~\ref{fig:ImageHandler} è mostrata la gerarchia di moduli utilizzata per gestire le immagini salvate. In particolare, \texttt{ImageHandler} contiene i due metodi principali: \texttt{save} e \texttt{remove}. Di fatto è un'istanza del \textit{design pattern} \textbf{Template Method}: \texttt{save} e \texttt{remove} chiamano dei metodi implementati dalle sottoclassi che ritornano i percorsi della cartella in cui ci sarà (o c'è) l’immagine. Questi metodi sono \texttt{make\_directories} per \texttt{save} e \texttt{get\_base\_path} per \texttt{remove}. Nelle sottoclassi vengono anche salvati, come campi statici, i percorsi delle cartelle delle immagini. 
\begin{figure}[h]
  \begin{center}
    \includegraphics[scale=0.4]{Classi/ImageHandler}
  \caption[Gerarchia per la gestione delle immagini]{Gerarchia per la gestione delle immagini}
  \label{fig:ImageHandler}
  \end{center} 
\end{figure}

\subparagraph{Attributi} \mbox{} \\
Gli attributi principali sono \texttt{id} e \texttt{image}: il primo contiene l'ID dell'elemento (applicazione, gruppo o istruzioni \glossaryItem{saml}), mentre il secondo rappresenta l'immagine codificata con \glossaryItem{base64}. Oltre a questi, ogni sottoclasse contiene dei campi statici per memorizzare i \textit{path} delle cartelle delle immagini. In particolare:
\begin{itemize}
\item \texttt{APPS\_DIR, CATALOGUE\_DIR, GROUP\_DIR e SAML\_DIR} contengono i \textit{path} delle cartelle delle immagini di, rispettivamente, applicazioni (APPS e CATALOGUE), gruppi e istruzioni per applicazioni \glossaryItem{saml};
\item \texttt{DEFAULT\_IMAGE}: \textit{path} dell'immagine di \textit{default}. Per le applicazioni l'immagine è obbligatoria, quindi non è prevista nessuna immagine di \textit{default}.
\end{itemize}

\subparagraph{Metodi} \mbox{} \\
Come già detto, i due metodi principali sono \texttt{save} e \texttt{remove}. Il primo salva l'immagine, utilizzando come nome il valore dell'attributo \texttt{id}; il secondo rimuove l'immagine con nome uguale al valore dell'attributo \texttt{id}. Entrambi, però, sono metodi \textit{template}, ovvero si appoggiano a metodi implementati nelle sottoclassi. In questo modo è possibile implementare le parti invarianti dei due algoritmi una volta sola, evitando la duplicazione del codice. Sono le sottoclassi a fornire il comportamento concreto implementando i metodi lasciati astratti: \texttt{ImageHandler} definisce solo lo scheletro e l'ordine delle operazioni, senza preoccuparsi di come saranno implementate.

\texttt{make\_directories} crea le \textit{directories} che conterranno l'immagine, se non sono già presenti. \texttt{get\_base\_path}, invece, ritorna il percorso generale della \textit{directory} in cui sono salvate le immagini. 

\texttt{\_init} ha la funzione di costruttore, e viene richiamato automaticamente alla creazione di un oggetto tramite \texttt{new}.

\texttt{get\_default\_image}, infine, è un metodo statico che ritorna l'immagine di \textit{default}.

\newpage
\paragraph{ErrorHandler} \mbox{} \\
\begin{figure}[hbpc]
  \begin{center}
    \includegraphics[scale=0.5]{Classi/ErrorHandler}
  \caption[Modulo per la gestione degli errori]{Modulo per la gestione degli errori}
  \label{fig:ErrorHandler}
  \end{center} 
\end{figure}
In Figura~\ref{fig:ErrorHandler} è mostrata il modulo utilizzato per gestire gli errori, \texttt{ErrorHandler}. Ogni risposta di errore passa da qui, indipendentemente dal tipo di errore (non autorizzato, \textit{bad request}, errore del server o interno). In caso di errore, il back end di Catalogue Manager invia al client un \glossaryItem{json} che segue la seguente struttura (\lstlistingname~\ref{jsonErrore}):
\begin{lstlisting}[
		caption={Struttura del JSON di errore},
		label=jsonErrore,
		language=json,
		firstnumber=1
	]
{
	status: false,
	message: "Messaggio di errore",
	error_code: code
}
\end{lstlisting}
Dove \texttt{code} è, ovviamente, il codice dell'errore registrato. Se si tratta di un errore del server, la proprietà \texttt{error\_code} ha valore \texttt{undefined} e non compare nel \glossaryItem{json} ritornato. 
\subparagraph{Attributi}
\begin{itemize}
\item \texttt{res}: oggetto \texttt{Resource} di Express.js. Viene utilizzato per inviare la risposta al client;
\item \texttt{user}: indirizzo email dell'utente. Viene utilizzato per salvare il \textit{log} dell'errore;
\item \texttt{errors}: oggetto contenente tutti i codici degli errori utilizzati da Catalogue Manager e il messaggio di errore associato. Un esempio è mostrato nel \lstlistingname~\ref{errors}:
\begin{lstlisting}[
		caption={Esempio di oggetto contenente i messaggi di errore},
		label=errors,
		language=json,
		firstnumber=1
	]
{
	3000: "Domain catalogue not found",
	3001: "auth_type value is not supported.",
	// ...
	3042: "Private application with public group."
}
\end{lstlisting}
\end{itemize}
\subparagraph{Metodi}
\begin{itemize}
\item \texttt{server\_error}: viene chiamato nel caso in cui ci sia un errore interno del server (esempio quelli generati da mongoose.js). Il parametro \texttt{err} contiene l'errore. Per errori normali (ovvero non di validazione, ma sollevati durante l'esecuzione di un \textit{middleware}) viene semplicemente inviata una risposa di errore e salvato il \textit{log}. Altrimenti (errore di validazione) viene fatto il \textit{parsing} per recuperare le informazioni errate da salvare nel \textit{log} per funzioni di \textit{debug}. Al fine di collegare il \textit{log} ad un'entità di Catalogue Manager (applicazione, gruppo, eccetera), il parametro \texttt{main\_entity} può contenere le informazioni necessarie per memorizzare tale collegamento;
\item \texttt{internal\_error}: invia una risposta di errore con codice \texttt{code} e messaggio \texttt{errors[code]}, senza salvare nessun \textit{log};
\item \texttt{bad\_request}: invia una risposta con \textit{status} \glossaryItem{http} 400. Nella risposta viene anche inserito l’\textit{array} \texttt{missing\_fields}, se presente, per segnalare se alcuni parametri obbligatori non sono stati ricevuti dal servizio invocato. Il messaggio di errore è contenuto in \texttt{msg};
\item \texttt{unauthorized}: invia una risposta con \textit{status} \glossaryItem{http} 401;
\item \texttt{handle\_and\_save\_log}: chiama \texttt{internal\_error} per gestire l'errore e poi salva il \textit{log}. Oltre alle informazioni sulle principali entità coinvolte nell'errore (applicazione, gruppo, ecc) presenti nell’\textit{array} \texttt{refs}, possono essere incluse delle informazioni aggiuntive non collegabili ad un modello di mongoose.js che vanno inserite nell’\textit{array} \texttt{extra\_infos}. In Figura~\ref{fig:handle_and_save_log} sono mostrate le proprietà contenute nei due parametri. \texttt{refs} viene passato così com'è alla funzione di salvataggio: \texttt{model} rappresenta il nome del modello dell'entità coinvolta, \texttt{dispname} il valore dell'attributo \texttt{name} dell'entità e \texttt{element\_id} il suo ID. In alternativa a \texttt{dispname} è possibile utilizzare \texttt{name}: in questo caso esso contiene il nome dell'attributo dell'entità da mostrare in fase di visualizzazione del \textit{log}. La funzione di salvataggio effettuerà un'interrogazione al \textit{database} per ricavare il valore dell'attributo rappresentato da \texttt{name} nel modello \texttt{model}. \texttt{extra\_infos} contiene invece, come già detto, informazioni aggiuntive non collegabili a nessun modello. La sua struttura è molto semplice e simile a quella di Log (\texttt{element\_id} viene lasciato ad \texttt{undefined}).
\begin{figure}[hbpc]
  \begin{center}
    \includegraphics[scale=0.6]{Classi/handle_and_save_log}
  \caption[Proprietà dei parametri per il salvataggio dei log]{Proprietà dei parametri per il salvataggio dei log}
  \label{fig:handle_and_save_log}
  \end{center} 
\end{figure}
\end{itemize}

\newpage
\paragraph{Salvataggio dei log su database} \mbox{} \\
\begin{figure}[hbpc]
  \begin{center}
    \includegraphics[scale=0.5]{Classi/DBLogger}
  \caption[Modulo per il salvataggio dei log su database]{Modulo per il salvataggio dei log su database}
  \label{fig:DBLogger}
  \end{center} 
\end{figure}
In Figura~\ref{fig:DBLogger} è mostrato il modulo \texttt{DBLogger}, utilizzato per il salvataggio dei \textit{log} su \textit{database}. È l'unico modulo di Catalogue Manager che scrive sulla \textit{collection} rappresentata dal modello \texttt{CatalogueLog} e, grazie alla funzione \texttt{extract\_infos}, rende facilmente utilizzabili le informazioni salvate su tale \textit{collection}. Memorizza inoltre tutti i codici di \textit{log} presenti nell'applicazione.

\subparagraph{Attributi} \mbox{} \\
L'unico attributo di \texttt{DBLogger} è \texttt{codes}, che contiene tutti i codici dei log corrispondenti a operazioni riuscite (quindi non quelli per gli errori, memorizzati in \texttt{ErrorHandler}). Questi codici sono accedibili tramite proprietà di questo attributo, che ha la seguente struttura (\lstlistingname~\ref{codiciLog}):
\begin{lstlisting}[
		caption={Esempio di oggetto contenente i codici dei log},
		label=codiciLog,
		language=json,
		firstnumber=1
	]
{
	APP_CREATED: 1,
	PRIVATE_APP_CREATED: 2,
	APP_REMOVED: 3,
	PRIVATE_APP_REMOVED: 4,
	// ...
	ACCESS_ALLOWED: 29,
	ACCESS_DENIED: 30
}
\end{lstlisting}

\subparagraph{Metodi} 
\begin{itemize}
\item \texttt{save}: salva i \textit{log} su \textit{database}. \texttt{user} rappresenta l'indirizzo email dell'utente e \texttt{code} il codice del \textit{log}. Quest'ultimo viene validato attraverso la funzione \texttt{check\_code}, che ritorna \texttt{true} se e solo se il codice è contenuto in \texttt{codes} o se è un codice di errore. Nell'\textit{array} \texttt{data}, invece, sono contenute le informazioni da salvare. La struttura di questo parametro è conforme a quella di \texttt{ModelLogInfo}, in modo da mantenere la flessibilità data dalla doppia proprietà per il nome da mostrare. Come già detto, infatti, \texttt{model} rappresenta il nome del modello dell'entità coinvolta nell'operazione, \texttt{dispname} il valore dell'attributo \texttt{name} dell'entità e \texttt{element\_id} il suo ID. In alternativa a \texttt{dispname} è possibile utilizzare \texttt{name}: in questo caso esso contiene il nome dell'attributo dell'entità da mostrare in fase di visualizzazione del \textit{log}. La funzione di salvataggio effettuerà un'interrogazione al \textit{database} per ricavare il valore dell'attributo rappresentato da \texttt{name} nel modello \texttt{model}. Ad esempio, dato il modello \textit{M}, se \texttt{name} contiene la stringa ''description'', verrà memorizzato nell'attributo \texttt{value} di \texttt{CatalogueLog} il valore \texttt{description} del \textit{document} del modello \textit{M} identificato univocamente da \texttt{element\_id}. In Tabella~\ref{tab:esempioLog} è mostrato un esempio con dei valori. 
\begin{center}
  \bgroup
  
  \begin{longtable}{ | m{4cm} | m{5cm} |}
    \hline
    \cellcolor[gray]{0.9}\textbf{ID} & \cellcolor[gray]{0.9}\textbf{\textit{description}} \\ \hline
    5770fadf7a1725abe77e0382 & Lorem ipsum dolor sit amet \\ \hline
    575ffd1c8259f70d4b775646 & Mauris suscipit semper dui \\ \hline
    \caption[Esempio di salvataggio di un log]{Esempio di salvataggio di un log}
    \label{tab:esempioLog} 
    \end{longtable}
  \egroup
\end{center} 
Supponiamo 
\newline \texttt{data.element\_id} == 5770fadf7a1725abe77e0382 e 
\newline \texttt{data.name} == \textit{description}

In tal caso verrà salvato come \texttt{CatalogueLog.value} la stringa ''Lorem ipsum dolor sit amet''. 

Al contrario, se 
\newline \texttt{data.element\_id} == 5770fadf7a1725abe77e0382 e 
\newline \texttt{data.dispname} == \textit{Mario Rossi} 

verrà salvato come \texttt{CatalogueLog.value} la stringa ''Mario Rossi''.
\item \texttt{extract\_infos}: questa funzione serve per rendere più facilmente utilizzabili le informazioni contenute il \texttt{CatalogueLog}. In particolare, dato un \textit{log} con la struttura mostrata nel \lstlistingname~\ref{rawlog} restituisce lo stesso \textit{log}, ma con la struttura mostrata nel \lstlistingname~\ref{formattedlog}:
\begin{lstlisting}[
		caption={Document di CatalogueLog},
		label=rawlog,
		language=json,
		firstnumber=1
	]
{
	_id: ObjectId,
	code: Number,
	user: String,
	created_at: Date,
	infos: Array[{
		key: String,
		value: String,
		element_id: ObjectId
	}]
}
\end{lstlisting}

\begin{lstlisting}[
		caption={Log riformattato},
		label=formattedlog,
		language=json,
		firstnumber=1
	]
{
	date: Date,
	code: Number,
	infos: Array[{
		application: String,
		group: String,
		domain: String,
		browser: String
		user: String
	}],
	errored: Array[{
		// struttura dipendente dal codice di errore
	}]
}
\end{lstlisting}
Le proprietà di \texttt{infos} sono potenzialmente vuote. Ad esempio, se il \textit{log} riguarda la creazione di un'applicazione pubblica, \texttt{group, domain} e \texttt{browser} saranno stringhe vuote. Le proprietà di \texttt{errored}, invece, dipendono dal codice di errore specifico e, in generale, riportano i valori dei dati errati.
\end{itemize}

\section{Design Pattern}
La progettazione ad oggetti presuppone l'applicazione di alcuni \textit{patterns} noti e molto utilizzati per risolvere problemi comuni. Di seguito verranno esposti quelli utilizzati durante la definizione dell'architettura di Catalogue Manager.
\subsection{Module Pattern}
I moduli sono parte integrante di qualsiasi applicazione di grandi dimensioni, e tipicamente aiutano ad organizzare il codice. In JavaScript ci sono diverse opzioni per implementare un modulo; tra le più conosciute ci sono la \textbf{Object literal notation} e il \textbf{Module pattern}.

\paragraph{Object Literal} \mbox{} \\
Nella \textit{object literal notation} un oggetto viene descritto come un insieme di coppie nome/valore separate da virgole (',') e contenute tra parentesi graffe ('\{\}'). Il \lstlistingname~\ref{objectLiteral} mostra un esempio:
\begin{lstlisting}[
		caption={Oggetto JavaScript in notazione classica},
		label=objectLiteral,
		language=JavaScript,
		firstnumber=1
	]
var contatore = {
	k: 0,
	incrementa_e_stampa: function() {
		this.k++;;
		console.log(this.k);
	}
};

contatore.incrementa_e_stampa(); // stampa "1"
contatore.i = 50; // aggiunto i con valore 50
contatore.k = 20; // k ora vale 20
contatore.incrementa_e_stampa(); // stampa "21"
\end{lstlisting}
Questa notazione non richiede l'utilizzo dell'operatore \texttt{new}. Dall'esterno, tuttavia, chiunque può aggiungere proprietà all'oggetto \texttt{contatore}. L'istruzione \texttt{contatore.i = 50;} ne è un esempio. Oltre a questo, chiunque può modificare \texttt{contatore.k} senza utilizzare il metodo dedicato: non verrà quindi stampato a video il nuovo valore e l'utilizzatore non noterà, inizialmente, nessun risultato.

Questa soluzione manca completamente di \textbf{incapsulazione}: tutti possono vedere, modificare o aggiungere proprietà all'oggetto.

\paragraph{Una soluzione migliore: il Module Pattern} \mbox{} \\
In JavaScript, il \textbf{Module Pattern} è utilizzato per \textit{emulare} il concetto di classe, in modo da avere attributi e metodi pubblici e privati. È quindi possibile decidere quali parti del modulo esporre e quali no, semplicemente ritornando un oggetto contenente le proprietà pubbliche del modulo. 

È importante notare che in JavaScript non è presente il concetto di ''\textit{privacy}'', in quanto non esistono i modificatori di accesso presenti negli altri linguaggi. Le variabili non possono essere dichiarate \textit{pubbliche} o \textit{private}, ed è necessario utilizzare l'ambito di visibilità delle funzioni per simulare questo concetto. Con il Module \textit{pattern} le variabili e le funzioni dichiarate dentro il modulo sono private; al contrario, quelle contenute nell'oggetto ritornato sono pubbliche. 

Il \lstlistingname~\ref{modulePattern} mostra un esempio:
\begin{lstlisting}[
		caption={Module pattern},
		label=modulePattern,
		language=JavaScript,
		firstnumber=1
	]
var contatore = (function() {
	var k = 0; // attributo privato
	
	var incrementa_e_stampa = function() {
		k++;
		stampa();
	};
	
	var decrementa_e_stampa = function() {
		k--;
		stampa();
	};
	
	var get_contatore = function() {
		return k;
	};
	
	// metodo privato
	var stampa = function() {
		console.log(k);
	};
	
	// pubbliche
	return {
		incrementa_e_stampa: incrementa_e_stampa,
		decrementa_e_stampa: decrementa_e_stampa,
		get_contatore: get_contatore
	};
})();

contatore.incrementa_e_stampa(); // stampa "1"
contatore.decrementa_e_stampa(); // stampa "0"
contatore.k = 20;
contatore.get_contatore(); // ritorna "0"
contatore.k; // "20"
\end{lstlisting}

\newpage
\subparagraph{Vantaggi}
\begin{itemize}
\item Aggiunge il concetto di incapsulazione a JavaScript.
\item Se vengono aggiunti dei metodi esternamente alla definizione del modulo, questi non possono accedere alle variabili private.
\end{itemize}

\subparagraph{Svantaggi}
\begin{itemize}
\item Complica l'utilizzo dell'ereditarietà.
\item Rispetto alla notazione ad oggetti classica complica il \textit{testing} perché alcuni membri sono inaccessibili.
\end{itemize}

\subparagraph{Contestualizzazione} \mbox{} \\
In Catalogue Manager tutti i moduli sono realizzati con il Module pattern. Questo ha consentito di progettarli in modo molto più \textit{object-oriented} e di ottenere l'incapsulazione che sarebbe altrimenti assente.

\subsection{Template Method} 
In Figura~\ref{fig:templateMethod} è riportata la struttura del \textit{design pattern} \textbf{Template Method}.
\begin{figure}[h]
  \begin{center}
    \includegraphics[scale=0.6]{designPattern/TemplateMethod}
  \caption[Design pattern Template Method]{Design pattern Template Method}
  \label{fig:templateMethod}
  \end{center} 
\end{figure}

Questo \textit{pattern} consente di definire la struttura di un algoritmo, lasciando alle sottoclassi il compito di implementare alcuni passi secondo le loro necessità. In questo modo si può ridefinire e personalizzare parte del comportamento nelle varie sottoclassi, senza dover riscrivere più volte il codice in comune. Inoltre evita la duplicazione di codice nelle sottoclassi e aderisce all'\textbf{Hollywood Principle}\footnote{''\textbf{Don't call us, we'll call you}''. Proviene dalla filosofia di Hollywood, secondo lo quale sono le case produttrici a chiamare gli attori se hanno bisogno di loro. Contestualizzando, le componenti di alto livello (superclassi) decidono quando e come utilizzare le componenti di basso livello (sottoclassi)}. 

\paragraph{Vantaggi}
\begin{itemize}
\item Nessuna duplicazione di codice.
\item Il riutilizzo di codice avviene per ereditarietà e non per composizione; solo alcuni metodi devono subire l'\textit{override}.
\item Le sottoclassi decidono come implementare i passi dell'algoritmo, migliorando la flessibilità.
\end{itemize}

\paragraph{Svantaggi}
\begin{itemize}
\item Aumenta la difficoltà di \textit{debugging}.
\item Rende più difficile comprendere il flusso di esecuzione.
\end{itemize}

\paragraph{Contestualizzazione} \mbox{} \\
Il \textit{design pattern} Template Method è utilizzato per la gerarchia di classi che gestiscono le immagini salvate su \textit{file system}. \texttt{ImageHandler} espone due metodi, \texttt{save} e \texttt{remove}, che chiamano dei metodi astratti che devono essere implementati dalle sottoclassi. 

\subsection{Middleware}
Questo \textit{pattern} consente di definire uno o più intermediari tra i vari componenti \textit{software} dell'applicazione in modo da semplificare la loro connessione e collaborazione. In generale è molto utile nello sviluppo e nella gestione di di sistemi distribuiti complessi, contesto nel quale Catalogue Manager si colloca perfettamente. 

Viene utilizzato da Express.js per fornire una libreria di funzioni comuni: definisce una serie di livelli (o funzioni) per gestire le varie richieste dell'applicazione. Tutti i componenti del \textit{pattern} \textbf{Middleware} sono collegati l'uno con l'altro e ricevono a turno una richiesta in ingresso finché uno di questi non decide di partire con l'elaborazione: l'\textit{output} di un \textit{middleware} diventa l'\textit{input} per il successivo. Per questo è anche conosciuto con il nome di \textbf{Pipeline}.

Anche mongoose.js utilizza questo concetto: come già spiegato è possibile definire delle funzioni da eseguire prima o dopo un'operazione specificata. 

%In Figura~\ref{fig:middlewareStruttura} è mostrata una possibile struttura del \textit{pattern} Middleware. Si nota che il \textit{Client} contiene un \textit{array} di \textit{middlewares}: questo \textit{array} rappresenta la catena. Per aggiungerne uno è sufficiente utilizzare la funzione \textit{use}, passando come parametro un'istanza di una classe che implementa l'interfaccia \textit{Middleware}. Quest'ultima interfaccia serve solamente a fornire un ''tipo'' utilizzabile da \textit{Client}. In JavaScript il discorso dei tipi è quasi inesistente, e il fatto che anche le funzioni siano trattate come oggetti consente di utilizzare queste ultime come \textit{middlewares}.
%
%\begin{figure}[hbpc]
%  \begin{center}
%    \includegraphics[scale=0.6]{designPattern/MiddlewareStruttura}
%  \caption[Design pattern Middleware]{Design pattern Middleware}
%  \label{fig:middlewareStruttura}
%  \end{center} 
%\end{figure}

Middleware è fortemente legato al \textbf{Chain of Responsibility}, descritto più avanti.

\subparagraph{Vantaggi}
\begin{itemize}
\item Riutilizzo di codice comune a più parti dell'applicazione.
\item Aderenza al \textbf{Single Responsibility Principle}.\footnote{Ogni elemento di un \textit{software} deve avere una sola responsabilità, e tale responsabilità deve essere interamente incapsulata dall'elemento stesso. Tutti i servizi offerti dall'elemento dovrebbero essere strettamente allineati a tale responsabilità.}
\end{itemize}

\subparagraph{Svantaggi}
\begin{itemize}
\item Se mal utilizzato rende difficile la comprensione del flusso di controllo.
\end{itemize}

\subparagraph{Contestualizzazione}\mbox{} \\
In Catalogue Manager il \textit{pattern} Middleware viene utilizzato ampiamente sia nel contesto di Express.js sia in quello di mongoose.js.

Per Express.js serve per registrare funzioni di validazione del \textit{token}, le \textit{routes}, il gestore dell'errore 404 e così via.

Per mongoose.js, invece, serve, ad esempio, a validare i dati inviati (\glossaryItem{url} e categorie in particolare), a forzare il mantenimento di vincoli personalizzati (come quello che impone che applicazioni pubbliche non abbiano lo stesso nome), ad escludere dalle ricerche su \textit{database} determinati valori (ad esempio le applicazioni con il \textit{flag} \texttt{removed == true} o con \texttt{available == false}), eccetera. 

%\begin{figure}[hbpc]
%  \begin{center}
%    \includegraphics[scale=0.6]{designPattern/MiddlewareCatMgr}
%  \caption[Design pattern Middleware in Catalogue Manager]{Design pattern Middleware in Catalogue Manager}
%  \label{fig:middlewareCatMgr}
%  \end{center} 
%\end{figure}
\subsection{Chain of Responsibility}
Il \textit{pattern} \textbf{Chain of Responsibility} permette ad un oggetto di inviare un evento senza sapere chi lo riceverà e chi lo gestirà. Questo passaggio rende i due (o più) oggetti parte di una catena: ciascun oggetto nella catena può gestire l'evento, passarlo al successivo o entrambi. 

Lo scopo è quello di evitare un collegamento statico tra chi emette l'evento e chi lo gestisce: formando una catena la richiesta viene passata da un oggetto all'altro, senza sapere chi e quando la gestirà. Ogni nodo della catena decide se esaudire la richiesta o no, delegando, in quest'ultimo caso, l'onere al nodo successivo. La catena viene attraversata finché un nodo non può eseguire l'ordine del mittente. In questo modo si evita l’\textbf{accoppiamento} fra il mittente di una richiesta e il destinatario.

Un \textit{pattern} di questo tipo si lega facilmente e fortemente al Middleware, e trova in lui una naturale istanziazione: Express.js, infatti, lo utilizza per la gestione dei \textit{middlewares} (così come mongoose.js) e del \textit{routing}. In Figura~\ref{fig:cor} è mostrata la struttura.

\begin{figure}[h]
  \begin{center}
    \includegraphics[scale=0.5]{designPattern/ChainOfResponsibility}
  \caption[Design pattern ChainOfResponsibility]{Design pattern ChainOfResponsibility}
  \label{fig:cor}
  \end{center} 
\end{figure}

Nel gergo di Express.js i \textit{middleware} corrispondono agli oggetti \texttt{ConcreteHandler} del \textit{design pattern}. Sebbene il comportamento e lo scopo sia quasi identico, l'implementazione di Express.js presenta alcune differenze:
\begin{itemize}
\item i \textit{middlewares} di Express.js possono essere classi con un metodo \texttt{handle} o semplici funzioni, in pieno accordo con lo stile funzionale utilizzato dalla maggioranza delle librerie e delle applicazioni scritte in Node.js. In Catalogue Manager è stata utilizzata principalmente la seconda versione;
\item il \textit{design pattern} prevede che l'oggetto \texttt{Handler} abbia un riferimento \texttt{successor} all’\texttt{Handler} successivo. Express.js, invece, passa al metodo di esecuzione del \textit{middleware} una \texttt{callback}. Il \textit{middleware}, eseguendo la \texttt{callback}, passa nuovamente il controllo all'oggetto del server di Express.js il quale passerà il controllo al successivo \textit{middleware};
\item Express.js divide i \textit{middlewares} in due gruppi:  standard e di gestione degli errori. Si distinguono per il numero di parametri che possono gestire (tre e quattro, rispettivamente). Ogni \textit{middleware} può decidere a quale dei due passare il controllo semplicemente variando il numero di parametri (nel secondo caso va specificato l'errore da gestire). Questa funzionalità serve per permettere la gestione di errori senza utilizzare i costrutti \texttt{try catch}, tipici dei linguaggi imperativi, ma inefficaci quando si utilizza lo stile di programmazione asincrono;
\item ogni \textit{middleware} di Express.js deve essere invocato con i seguenti parametri: l'eventuale errore, l'oggetto della richiesta (\texttt{Express.Request}), l'oggetto della risposta (\texttt{Express.Response}), la \texttt{callback} da utilizzare per passare il controllo al successivo \textit{middleware}. L'ordine è importante.
\end{itemize}

\subparagraph{Vantaggi} \mbox{} \\
\begin{itemize}
\item Riduzione dell'accoppiamento.
\item Maggiore flessibilità nell'assegnazione delle responsabilità.
\end{itemize}

\subparagraph{Svantaggi} \mbox{} \\
\begin{itemize}
\item La gestione di una richiesta non è garantita, sia per la possibile mancanza di un \textit{middleware} con quella specifica responsabilità sia per una configurazione sbagliata della catena.
\end{itemize}

\subparagraph{Contestualizzazione}\mbox{} \\
Come già detto per il \textit{pattern} Middleware, il Chain of Responsibility viene utilizzato largamente da Express.js e mongoose.js.