\chapter{Contesto aziendale} \label{azienda}
\section{Profilo dell'azienda}
L'attività di stage descritta nel presente documento è stata svolta presso l'azienda \textbf{iVoxIT S.r.l} (logo in Figura~\ref{fig:ivoxit}) di Padova. 

\begin{figure}[h]
\begin{center}
\includegraphics[scale=0.3]{iVoxIT}
\caption{Logo di iVoxIT}
\label{fig:ivoxit}
\end{center}
\end{figure}
\textit{iVoxIT S.r.l.} fa parte del gruppo \textbf{Athesys S.r.l.}, fondato nel 2010 dall'unione di professionisti dell'\glossaryItem{it} con l'obiettivo di fornire consulenza ad alto livello tecnologico e progettuale. Tra le altre cose, \textit{Athesys S.r.l} fornisce supporto nell'istanziazione del processo di \glossaryItem{dlm}, con particolare attenzione alla sicurezza nella conservazione e nell'esposizione dei dati gestiti.

L'azienda opera in tutto il territorio nazionale, prevalentemente nel Nord Italia, e vanta esperienze a livello europeo in paesi quali Olanda, Regno Unito e Svizzera.

Grazie all'adozione delle \glossaryItemPl{bestpractice} definite dalle linee guida \glossaryItem{itil} e alla certificazione \textit{ISO 9001} il gruppo è in grado di assicurare un'alta qualità professionale.

\subsection{I servizi}
I principali ambiti di consulenza del personale di \textit{Athesys S.r.l} sono i seguenti:
\begin{itemize}
\item \textbf{Database Management}: nelle aziende è sempre più necessario memorizzare grosse quantità di dati in modo da renderli facilmente reperibili. I \glossaryItem{dba} di \textit{Athesys S.r.l} guidano il cliente nella scelta della tecnologia più appropriata, occupandosi anche dell'integrazione con il sistema informativo preesistente. In particolare:
	\begin{itemize}
	\item identificazione e progettazione dell'infrastruttura più adatta in base alle esigenze di affidabilità e scalabilità;
	\item progettazione, realizzazione e manutenzione di basi di dati efficienti;
	\item interazione e migrazione fra diverse piattaforme;
	\item protezione dei dati nei confronti di accessi o manipolazioni non autorizzate;
	\item \textit{backup} e ripristino di dati.
	\end{itemize}
\item \textbf{Identity and Access Management}: come sarà maggiormente descritto in seguito, gli strumenti di \glossaryItem{iam} consentono alle aziende di effettuare il controllo dell'\glossaryItem{identita} e degli accessi ai propri servizi attraverso sistemi di \glossaryItem{sso}. Fra i servizi offerti ci sono:
	\begin{itemize}
	\item studio del sistema informativo aziendale;
	\item individuazione del sistema \glossaryItem{iam} più adatto;
	\item implementazione di motori di \glossaryItem{workflow} coinvolti nei processi di \glossaryItem{provisioning};
	\item implementazione di moduli di \glossaryItem{autenticazione} e \glossaryItem{autorizzazione}.
	\end{itemize}
\item \textbf{System Integration}: integrazione dei sistemi informativi e delle infrastrutture aziendali in modo da garantire \glossaryItem{efficienza}, \glossaryItem{scalabilita} e \glossaryItem{affidabilita}.
\item \textbf{Business Intelligence}: un'azienda deve essere in grado di reperire, elaborare e analizzare i dati. Questi dati, però, provengono da sorgenti sempre più eterogenee. \textit{Athesys S.r.l} si occupa di raccogliere questi dati e di renderli disponibili sotto forma di \textit{report}.
\end{itemize}

\section{Metodo di lavoro}
\textit{Athesys S.r.l}, e di conseguenza \textit{iVoxIT S.r.l}, utilizza un ciclo di sviluppo \glossaryItem{agile} nel quale il \textit{team} è gestito secondo una metodologia \textit{Scrum}. \textit{Scrum} è una metodologia di sviluppo semplice da comprendere, ma difficile da padroneggiare: non è un processo o una tecnica per costruire i prodotti, al contrario è un \glossaryItem{framework} del quale si possono impiegare i vari processi e tecniche. In \textit{Scrum} sono presenti uno o più \textit{team}; i cui membri hanno ruoli, regole e capacità diverse, e tutti competono al successo finale.

\subsection{Scrum}
 Gli aspetti fondamentali di Scrum sono:
\begin{itemize}
\item \textbf{Trasparenza}: gli aspetti significativi del processo devono essere visibili ai responsabili;
\item \textbf{Ispezione}: gli utenti devono ispezionare ciò che è stato fatto, in modo da controllare se le cose stanno andando nel verso giusto. Le ispezioni non dovrebbero essere tanto frequenti da bloccare il flusso di lavoro, e sono utili quando condotte da ispettori qualificati.
\item \textbf{Adattamento}: se un'ispezione determina che uno o più aspetti deviano dai limiti accettabili, e che, di conseguenza, il prodotto finale non sarà accettabile, il processo o il materiale ispezionato deve essere corretto. La correzione deve essere fatta il prima possibile, per evitare ulteriori deviazioni.
\end{itemize}
In \textit{Scrum} esistono solamente tre ruoli principali (mostrati in Figura~\ref{fig:scrumroles}):
\begin{itemize}
\item \textbf{Product Owner}, mantiene la visione complessiva del prodotto;
\item \textbf{Scrum Master}, aiuta il \textit{team} a rimanere concentrato e focalizzato sull'obiettivo finale;
\item \textbf{Development Team}, realizza il prodotto.
\end{itemize}
I membri del \textit{team} si auto organizzano il lavoro al posto di ricevere indicazioni esterne, e dispongono di competenze che non necessitano dipendenze da altri membri dello stesso \textit{team}. In questo modo si ottimizzano flessibilità, creatività e produttività.
\begin{figure}[h]
\begin{center}
\includegraphics[scale=0.2]{ScrumRoles}
\caption[Ruoli del framework Scrum]{Ruoli del framework Scrum\protect\footnotemark}
\label{fig:scrumroles}
\end{center}
\end{figure}
\footnotetext{Fonte: \url{https://www.scrumalliance.org/employer-resources/scrum-roles-demystified}}
\subsection{Ciclo Scrum}

\textit{Scrum} è un metodo iterativo che divide il progetto in blocchi rapidi di lavoro chiamati \textbf{Sprint}, della durata massima di quattro settimane. Alla fine di ogni \textit{Sprint} si ottiene un incremento del prodotto.

In Figura~\ref{fig:scrum} è mostrato un tipico ciclo \textit{Scrum}. Il \textit{Product Owner} crea una ''lista di desideri'' con delle priorità associate, chiamata \textbf{Product Backlog}. Questa lista contiene tutto ciò di cui potrebbe aver bisogno il prodotto, ed è l'unica fonte di requisiti. L'unico responsabile della \textit{Product Backlog} è il \textit{Product Owner}, che ne mantiene il contenuto e l'ordinamento.

Ad ogni \textit{Sprint} il \textit{Development Team} prende in considerazione un pezzo della lista e decide come implementare le funzionalità richieste (\textbf{Sprint Planning}). La durata massima dello \textit{Sprint Planning} è otto ore al giorno per un mese. È compito dello \textit{Scrum Master} assicurarsi del rispetto dei tempi: quotidianamente effettua una breve riunione, della durata di circa quindici minuti, per valutare i progressi fatti (\textbf{Daily Scrum}). Queste consentono al \textit{team} di pianificare le attività per le successive 24 ore, valutando il lavoro svolto dall'ultimo \textit{Daily Scrum}. Lo \textit{Scrum Master} si assicura che il \textit{Daily Scrum} venga svolto, ma di fatto questo viene condotto dal \textit{Development Team}. Così facendo si promuove la comunicazione.
\begin{figure}[h]
\begin{center}
\includegraphics[scale=0.5]{Scrum}
\caption[Framework \textit{Scrum}]{Framework Scrum\protect\footnotemark}
\label{fig:scrum}
\end{center}
\end{figure}
\footnotetext{Fonte: \url{https://www.scrumalliance.org/why-scrum}}

Lo \textit{\textit{Scrum}} termina con una \textbf{Sprint Review} e con una \textbf{Sprint Retrospective}. La prima ispeziona i risultati ottenuti e li corregge se necessario. È una riunione informale, nella quale lo \textit{Scrum Master} collabora con gli \glossaryItemPl{stakeholder} per valutare ciò che è stato fatto. Il \textit{Product Owner} espone cosa è stato fatto e cosa no, analizza le priorità della \textit{Product Backlog} e decide, insieme agli altri, cosa sarà fatto nel prossimo \textit{Sprint}. La \textit{Sprint Retrospective}, invece, analizza i miglioramenti che possono essere fatti al \textit{team}, e si svolge dopo la \textit{Sprint Review}. Vengono identificati gli effetti dello \textit{Sprint} appena concluso sulle persone e sugli strumenti, e, se necessario, vengono pianificati dei miglioramenti. È compito dello \textit{Scrum Master} incoraggiare il \textit{team} a migliorarsi.

Alla fine di ogni \textit{Sprint} il prodotto è potenzialmente consegnabile al cliente.