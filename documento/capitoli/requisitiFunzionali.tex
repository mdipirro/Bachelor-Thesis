R1O 1 & Un utente deve poter visualizzare la lista delle applicazioni del catalogo di Monokee. \\ \hline 
R1O 1.1 & Un utente deve poter visualizzare i dettagli di un'applicazione presente nel catalogo di Monokee. \\ \hline 
R1O 1.2 & Un utente deve poter visualizzare il nome dell'applicazione del catalogo di Monokee. \\ \hline 
R1O 1.3 & Un utente deve poter visualizzare la descrizione dell'applicazione del catalogo di Monokee. \\ \hline 
R1O 1.4 & Un utente deve poter visualizzare l'immagine di un'applicazione del catalogo di Monokee. \\ \hline 
R1O 1.5 & Un utente deve poter visualizzare la lista delle categorie associate ad un'applicazione del catalogo di Monokee. \\ \hline 
R1O 1.6 & Un utente deve poter visualizzare il tipo di \glossaryItem{autenticazione} di un'applicazione del catalogo di Monokee. \\ \hline 
R1O 1.7 & Un utente deve poter visualizzare ogni dettaglio di \glossaryItem{autenticazione} per un'applicazione del catalogo di Monokee. \\ \hline 
R1O 1.8 & Un utente deve poter visualizzare la visibilità (pubblica o privata) dell'applicazione. \\ \hline 
R1O 1.9 & Un utente deve poter visualizzare il nome del gruppo nel quale è inserita l'applicazione. \\ \hline 
R1O 1.10 & Un utente deve poter visualizzare se l'applicazione è in manutenzione o meno. \\ \hline 
R1O 1.10 & Un utente deve poter visualizzare se l'applicazione è stata pubblicata o meno. \\ \hline 
R1O 2 & Un utente deve poter inserire una nuova applicazione. \\ \hline 
R1O 2.1 & Un utente deve poter inserire il nome della nuova applicazione. \\ \hline 
R1O 2.2 & Un utente deve poter inserire la descrizione della nuova applicazione. \\ \hline 
R1O 2.3 & Un utente deve poter inserire l'\glossaryItem{url} della nuova applicazione. \\ \hline 
R1O 2.4 & Un utente deve poter inserire l'immagine della nuova applicazione. \\ \hline 
R1O 2.5 & Un utente deve poter selezionare le categorie di appartenenza della nuova applicazione. \\ \hline 
R1O 2.6 & Un utente deve poter selezionare il tipo di \glossaryItem{autenticazione} della nuova applicazione. \\ \hline 
R1O 2.7 & Un utente deve poter inserire inserire i dati di \glossaryItem{autenticazione} della nuova applicazione. \\ \hline 
R1O 2.8 & Un utente deve poter inserire il nome del gruppo nel quale sarà inserita la nuova applicazione. \\ \hline 
R1O 3 & Un utente deve poter modificare un'applicazione. \\ \hline 
R1O 3.1 & Un utente deve poter modificare il nome dell'applicazione. \\ \hline 
R1O 3.2 & Un utente deve poter modificare la descrizione dell'applicazione.\\ \hline 
R1O 3.3 & Un utente deve poter modificare l'\glossaryItem{url} dell'applicazione.\\ \hline 
R1O 3.4 & Un utente deve poter modificare l'immagine dell'applicazione. \\ \hline 
R1O 3.5 & Un utente deve poter selezionare le categorie di appartenenza dell'applicazione. \\ \hline 
R1O 3.6 & Un utente deve poter modificare modificare i dati di autenticazione dell'applicazione. \\ \hline 
R1O 3.7 & Un utente deve poter modificare il nome del gruppo nel quale sarà inserita l'applicazione. \\ \hline 
R1O 3.8 & Un utente deve poter mettere un'applicazione in manutenzione. \\ \hline 
R1O 3.9 & Un utente deve poter pubblicare un'applicazione. \\ \hline 
R1O 4 & Un utente deve poter rimuovere un'applicazione. \\ \hline 
R1O 5 & Un utente deve poter gestire i cataloghi di dominio di Monokee. \\ \hline 
R1O 5.1 & Un utente deve poter aggiungere un catalogo di dominio. \\ \hline 
R1O 5.2 & Un utente deve visualizzare un messaggio di errore se il dominio aziendale è già collegato ad un catalogo di dominio. \\ \hline 
R1O 5.3 & Un utente deve poter visualizzare i cataloghi di dominio esistenti. \\ \hline 
R1O 5.4 & Un utente deve poter gestire le applicazioni presenti in un catalogo di dominio. \\ \hline 
R1O 5.4.1 & Un utente deve poter visualizzare le applicazioni nel catalogo. \\ \hline 
R1O 5.4.2 & Un utente deve poter aggiungere un'applicazione al catalogo. \\ \hline 
R1O 5.4.3 & Un utente deve poter rimuovere un'applicazione dal catalogo. \\ \hline 
R1O 5.5 & L'utente deve poter rimuovere un catalogo di dominio esistente. \\ \hline 
R1O 5.6 & La rimozione di un catalogo di dominio deve comportare la rimozione di tute le applicazioni collegate a quel catalogo. \\ \hline 
R1O 6 & Un utente deve poter cercare un'applicazione nel catalogo. \\ \hline 
R1O 10 & Un utente deve poter gestire i gruppi di applicazioni. \\ \hline 
R1O 10.1 & Un utente deve poter aggiungere un gruppo. \\ \hline 
R1O 10.1.1 & Un utente deve poter inserire il nome del gruppo. \\ \hline 
R1O 10.1.2 & Un utente deve poter inserire la descrizione del gruppo. \\ \hline 
R1O 10.1.3 & Un utente deve poter inserire l'immagine del gruppo. \\ \hline 
R1O 10.2 & Un utente deve poter visualizzare i gruppi di applicazioni esistenti. \\ \hline 
R1O 10.3 & Un utente deve poter gestire un gruppo di applicazioni esistente. \\ \hline 
R1O 10.3.1 & Un utente deve poter aggiungere un'applicazione al gruppo.  \\ \hline 
R1O 10.3.2 & Un utente deve poter rimuovere un'applicazione dal gruppo. \\ \hline 
R1O 10.3.3 & Un utente deve visualizzare un messaggio di errore se l'applicazione selezionata è già presente nel gruppo. \\ \hline 
R1O 10.4 & Un utente deve poter modificare un gruppo. \\ \hline 
R1O 10.4.1 & Un utente deve poter inserire il nuovo nome del gruppo. \\ \hline 
R1O 10.4.2 & Un utente deve poter inserire la nuova descrizione del gruppo. \\ \hline 
R1O 10.4.3 & Un utente deve poter inserire la nuova immagine del gruppo. \\ \hline 
R1O 10.5 & Un utente deve poter rimuovere un gruppo.\\ \hline 
R1O 13 & Un utente deve visualizzare un messaggio di errore se l'applicazione che si sta cercando di inserire è già presente. \\ \hline
R1D 14 & Un utente deve poter visualizzare le statistiche dell'applicazione Catalogue Manager. \\ \hline 
R1D 14.1 & Un utente deve poter visualizzare il numero di applicazioni aggiunte e rimosse in intervalli di tempo definiti a priori: ultime 24 ore, ultima settimana, ultimo mese e ultimo anno. \\ \hline 
R1D 14.2 & Un utente deve poter visualizzare il numero di accessi in intervalli di tempo definiti a priori: ultime 24 ore e ultima settimana. \\ \hline 
R1D 14.3 & Un utente deve poter visualizzare il numero di utenti attivi. \\ \hline 
R1D 14.4 & Un utente deve poter visualizzare il numero di applicazioni e gruppi pubblici e privati. \\ \hline 
R1D 14.5 & Un utente deve poter visualizzare il numero di applicazioni, pubbliche e private, appartenenti ad ogni categoria (intesa come ''sotto categoria''). \\ \hline 
R1D 14.6 & Un utente deve poter visualizzare il numero di applicazioni, pubbliche e private, appartenenti ad una specifica ''sovra categoria''. \\ \hline 
R1D 15 & Un utente deve poter visualizzare i \textit{log} dell'applicazione Catalogue Manager. \\ \hline 
R1D 16 & Un utente deve poter visualizzare i \textit{log} delle operazioni eseguite con successo. \\ \hline 
R1D 17 & Un utente deve poter visualizzare i \textit{log} delle operazioni che hanno generato errori. \\ \hline 
R1O 18 & Un utente deve poter effettuare una ricerca tra i domini aziendali di Monokee. \\ \hline 
R1O 19 & Un utente deve visualizzare un messaggio di errore se il dominio ha già un catalogo associato. \\ \hline