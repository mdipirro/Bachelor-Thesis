\chapter{Implementazione} \label{implementazione}
\section{Autenticazione a Catalogue Manager}
\subsection{Implementazione dei JWT}
L'autenticazione a Catalogue Manager avviene tramite \glossaryItem{saml} nella modalità \textbf{\glossaryItem{idp} Initiated}. L'accesso deve essere eseguito attraverso Monokee e, in seguito ad esso, viene generato un \textit{token} \glossaryItem{jwt} trasmesso al front end di Catalogue Manager, che, successivamente, invierà questo \textit{token} al back end per ogni richiesta effettuata. Quest'ultimo verificherà la firma del \textit{token} stesso per controllare che le informazioni non siano state alterate e che l'utente sia chi dice di essere. 

Non essendo previsto un sistema di ruoli, l'unico controllo viene effettuato al momento della generazione del \textit{token} e riguarda l'associazione tra utente e applicazione: in questo modo può accedere a Catalogue Manager solo chi ha questa applicazione nel proprio \textit{application broker}. 

Il periodo di validità del \textit{token} è di nove ore, dopodiché è necessario autenticarsi nuovamente. È stato scelto questo periodo di tempo in modo da rendere valido il \textit{token} per una giornata di lavoro e per imporre una nuova \glossaryItem{autenticazione} ogni giorno.

Il \lstlistingname~\ref{header} mostra un esempio di \textit{header} di un \textit{token} di Catalogue Manager.
\begin{lstlisting}[
		caption={Header di un JWT di Catalogue Manager},
		label=header,
		language=json,
		firstnumber=1
	]
{
  typ: "JWT",
  alg: "HS256"
}
\end{lstlisting}
Come descritto in precedenza, questa parte del \textit{token} contiene le informazioni su tipo e algoritmo di firma. La scelta per la modalità di firma è caduta su \textbf{HS256} (\glossaryItem{hmac} con \textit{\glossaryItem{sha}-256}) e non su \textit{RS256} (\glossaryItem{rsa}) per un motivo sostanzialmente legato alle \textit{performance}: la firma e la verifica, nel primo caso, sono molto più veloci rispetto al secondo. Inoltre, la dimensione del \textit{token} è molto minore. 

La differenza sostanziale, comunque, risiede nella modalità di firma. Con \glossaryItem{hmac} chi pensa all'\glossaryItem{autenticazione} ha la chiave (il ''\textbf{secret}''); fornisce la chiave e il messaggio all'algoritmo scelto, che produce la versione firmata. Dopodiché invia il messaggio originale e quello firmato al verificatore, che, disponendo della stessa chiave, ricalcola la firma, controllando se quanto ottenuto è uguale a quanto ricevuto. Con \glossaryItem{rsa}, invece, esistono due chiavi diverse (pubblica e privata). Il messaggio viene firmato con quella privata e inviato al verificatore, che, attraverso l'algoritmo di verifica (ora diverso da quello di firma) controlla se il messaggio originale è uguale a quello ottenuto utilizzando la chiave pubblica. 

È chiaro che nel primo caso entrambe le parti condividono la stessa chiave e il verificatore può, se vuole, generare messaggi che vengono validati senza problemi. Nel secondo caso non accade, perché la chiave pubblica funziona solo per verificare i messaggi e non per firmarli.

Nel caso specifico di Catalogue Manager il compito di creare i \textit{token} e quello di verificarli ricadono su due componenti della stessa applicazione, quindi una può fidarsi, senza problemi, dell'altra (la verifica vera e propria viene effettuata utilizzando un noto modulo di Node.js, \textit{expressjwt}). Tolto questo problema, \glossaryItem{hmac} funziona molto più velocemente, e pertanto è stato preferito a \glossaryItem{rsa}.

Il \lstlistingname~\ref{payload} mostra un esempio di \textit{payload} di un \textit{token} di Catalogue Manager.
\begin{lstlisting}[
		caption={Payload di un JWT di Catalogue Manager},
		label=payload,
		language=json,
		firstnumber=1
	]
{
  id: "5783969ebf947349bb34f7b7",
  email: "matteo.dipirro@email.com",
  iat: 1468413576,
  exp: 1470213576
}
\end{lstlisting}
Come si nota le informazioni fornite sono molto basilari e, come già detto, nessuna di esse è riservata. Oltre ai vincoli sulla validità vengono inviati solo l'ID dell'applicazione Catalogue Manager e l'indirizzo email dell'utente, che sarà utilizzato per salvare i \textit{log} delle operazioni svolte durante la sessione. È ovviamente compito del client inviare il \textit{token} ad ogni richiesta, tramite \textit{header} \glossaryItem{http}. L'ID di Catalogue Manager serve per verificare la presenza dell'associazione tra l'applicazione e l'utente.

\subsection{Modalità SAML IdP Initiated}
In Figura~\ref{fig:idpinitiatedcatmgr} è mostrato il processo di \glossaryItem{autenticazione} \glossaryItem{idp} Initiated utilizzato. L'utente deve innanzitutto autenticarsi a Monokee e avere in uno dei suoi \textit{application brokers} Catalogue Manager. Il compito di accertare l'identità è quindi di Monokee.

\newpage

\begin{figure}[hbpc]
  \begin{center}
    \includegraphics[scale=0.2]{IdPInitiatedCATMGR}
  \caption[Single Sign On IdP Initiated in Catalogue Manager]{Single Sign On IdP Initiated in Catalogue Manager}
  \label{fig:idpinitiatedcatmgr}
  \end{center} 
\end{figure}

Di seguito saranno descritti i passi compiuti, automaticamente, durante l'accesso.
\begin{enumerate}
\item L'utente clicca sull'applicazione Catalogue Manager. Questo causa l'invocazione del un servizio di Monokee che si occupa del \glossaryItem{sso} \glossaryItem{idp} Initiated.
\item Il back end di Monokee effettua una chiamata all'\glossaryItem{idp}, parametrizzata con il \textit{token} dell'utente, l'ID dell'applicazione (Catalogue Manager in questo caso) e l'ID del dominio di appartenenza dell'utente.
\item L'\glossaryItem{idp} può comunicare ulteriormente con il back end di Monokee per ricevere degli attributi aggiuntivi dell'utente. Nel caso di Catalogue Manager, viene prelevato l'indirizzo email.
\item L'\glossaryItem{idp} fornisce la \textit{SAMLResponse} al back end di Monokee.
\item Il back end di Monokee invia la \textit{SAMLResponse} tramite \textit{form} al \textit{browser} dell'utente. Questo, automaticamente, la inoltra ad un servizio di Catalogue Manager (\textbf{/acs}), l'\textit{Assertion Consumer Service} di \glossaryItem{saml}. Il compito di questo servizio è di inoltrare la \textit{SAMLResponse} al \glossaryItem{sp}, aspettare la risposta, validarla e generare il \textit{token} corrispondente.
\item Il back end di Catalogue Manager chiama il \glossaryItem{sp} inviando la \textit{SAMLResponse} ricevuta in precedenza.
\item Il \glossaryItem{sp} risponde al back end di Catalogue Manager. 
\item Il back end di Catalogue Manager chiama il front end in base alla risposta del \glossaryItem{sp}:
	\begin{itemize}
	\item \textbf{Errore nella risposta del \glossaryItem{sp}}: viene chiamato il front end di Catalogue Manager e segnalato l'errore.
	\item \textbf{Successo}: arriva l'indirizzo email dell'utente. Viene controllato che l'utente sia collegato all'applicazione:
	\begin{itemize}
	\item se no, l'errore viene notificato al front end di Catalogue Manager;
	\item se sì, viene generato il \textit{token} dell'utente.
	\end{itemize}
	\end{itemize}
\end{enumerate}
%\section{Prodotto realizzato}
%Parallelamente al back end è stato sviluppato, da una dipendente di \textit{Athesys S.r.l}, il front end di Catalogue Manager. Questo ha permesso di integrare dopo poco tempo l'applicazione con Monokee e di utilizzarla a supporto delle attività di sviluppo del prodotto padre. Come verrà descritto in seguito, inoltre, è stato possibile eseguire dei test di utilizzo dei servizi sviluppati e di ricevere dei \textit{feedbacks} sul livello di esperienza utente. Di seguito verranno descritte le funzionalità sviluppate.

\section{Elenco dei servizi implementati} \label{servizi}
%Di seguito viene fornito un elenco dei servizi implementati .

%\subsection{Moduli}
%In Tabella~\ref{tab:moduliREST} vengono elencati i moduli di appoggio per i servizi \glossaryItem{rest} implementati. Questi moduli permettono di raggruppare le operazioni comuni alle \textit{routes}, evitando la duplicazione di codice e rendendo, di conseguenza, il prodotto più manutenibile. Ogni modulo è fortemente coeso, e dipende in misura quasi completamente nulla dagli altri moduli. L'unica eccezione è rappresentata dalla dipendenza nei confronti del modulo di \textit{logging}.
%\begin{center}
%  \bgroup
%  
%  \begin{longtable}{ | m{4.8cm} | p{7cm} |}
%    \hline
%    \cellcolor[gray]{0.9} \textbf{Nome} & \cellcolor[gray]{0.9} \textbf{Descrizione} \\ \hline
%    ApplicationImageHandler.js & Utilizzato per semplificare la gestione delle immagini delle applicazioni. \\ \hline
%    ErrorHandler.js & Utilizzato per la gestione degli errori. \\ \hline
%    GroupImageHandler.js & Utilizzato per semplificare la gestione delle immagini dei gruppi. \\ \hline
%    ImageHandler.js & Modulo base per la gestione delle immagini. \\ \hline
%    SAMLInstructionsImageHandler.js & Utilizzato per semplificare la gestione delle immagini delle istruzioni per la configurazione di applicazioni \glossaryItem{saml}. \\ \hline
%    check\_required\_fields.js & Utilizzato per controllare la correttezza dei parametri obbligatori dei servizi \glossaryItem{rest}. \\ \hline
%    clear\_db.js & Utilizzato per eliminare i residui della \glossaryItem{softdeletion} dal \textit{database} di Monokee. \\ \hline
%    db\_logger.js & Utilizzato per salvare i log delle operazioni o degli errori. \\ \hline
%    domain\_catalogue.js & Utilizzato per popolare il catalogo di uno specifico dominio. \\ \hline
%    group\_applications\_handler.js& Utilizzato per raggruppare le operazioni sui gruppi di applicazioni. \\ \hline
%    logger.js & Utilizzato per mostrare nella \textit{console} i log prodotti dai servizi \glossaryItem{rest} e per salvarli su \textit{file}. \\ \hline
%    remove\_app.js & Utilizzato per eliminare un'applicazione dal \textit{database} di Monokee. \\ \hline
%    reset\_first\_sign\_in.js & Utilizzato per reimpostare al valore di \textit{default} dei \textit{flag} utilizzati da Monokee per capire quando un utente deve inserire le credenziali. \\ \hline
%    \caption[Moduli del back end di Catalogue Manager]{Moduli del back end di Catalogue Manager}
%    \label{tab:moduliREST} 
%    \end{longtable}
%  \egroup
%\end{center} 

%\subsection{Servizi} \label{servizi}
In Tabella~\ref{tab:serviziREST} vengono elencati i servizi \glossaryItem{rest} implementati ed esposti dal back end di Catalogue Manager. I nomi sono stati assegnati seguendo le convenzioni utilizzate dal \textit{team} di sviluppo.

Tutti i servizi che contengono \texttt{auth/} nel loro \glossaryItem{url} richiedono l'invio di un \textit{token} valido per essere eseguiti.
\begin{center}
  \bgroup
  
  \begin{longtable}{ | m{6.5cm} | p{6.5cm} |}
    \hline
    \cellcolor[gray]{0.9} \textbf{Nome} & \cellcolor[gray]{0.9} \textbf{Descrizione} \\ \hline
    \texttt{/acs} & Data la \textit{SAMLResponse}, verifica che l'utente abbia accesso all'applicazione Catalogue Manager e genera il \textit{token} corrispondente. All'interno di esso viene memorizzato l'indirizzo email dell'utente e il suo identificativo all'interno del \textit{database} di Monokee. \\ \hline
\texttt{/auth/add\_applications\_in\_group} & Aggiunge un insieme di applicazioni ad un gruppo. Le applicazioni e il gruppo vengono identificate tramite i loro ID, e vengono effettuati tutti i controlli necessari per non creare inconsistenze. In particolare:
\begin{itemize}
\item la visibilità dell'applicazione e del gruppo deve essere la stessa;
\item se il gruppo è privato (specifico di un dominio, l'applicazione deve appartenere al catalogo di quel dominio);
\item l'applicazione non deve essere già inserita in un altro gruppo.
\end{itemize}
\\ \hline
\texttt{/auth/create\_application} & Crea un'applicazione pubblica o privata e la aggiunge al catalogo di Monokee, nel primo caso, o al catalogo di un dominio, nel secondo. \\ \hline
\texttt{/auth/create\_domain\_catalogue} & Crea un catalogo di dominio per il dominio selezionato. Se il dominio ha già un catalogo viene sollevato un errore. \\ \hline
\texttt{/auth/create\_group} & Crea un gruppo (pubblico o privato). Durante la creazione di un gruppo è possibile decidere quali applicazioni faranno parte di quel gruppo. Se vengono forniti degli ID di applicazioni vengono fatti gli stessi controlli descritti per \texttt{add\_applications\_in\_group}. \\ \hline
\texttt{/auth/get\_access\_by\_intervals} & Ritorna il numero di accessi effettuati nelle ultime 24 ore e nell'ultima settimana. \\ \hline
%get\_access\_logs & Ritorna i log riguardanti gli accessi permessi o negati in un intervallo temporale deciso dall'utente. \\ \hline
\texttt{/auth/get\_activities\_from\_to} & Ritorna un qualsiasi tipo di \textit{log} (riguardanti le attività su gruppi, applicazioni, accessi o domini) in un intervallo temporale deciso dall'utente. \\ \hline
\texttt{/auth/get\_addable\_applications} & Ritorna, dato l'ID di un gruppo, le applicazioni che possono essere aggiunte a tale gruppo. \\ \hline
\texttt{/auth/get\_application\_details} & Ritorna, dato l'ID di un'applicazione, i dettagli di tale applicazione. I dettagli includono nome, descrizione, \glossaryItem{url}, informazioni sul tipo di \glossaryItem{autenticazione}, eccetera. \\ \hline
%get\_application\_logs & Ritorna i log riguardanti le attività su applicazioni in un intervallo temporale deciso dall'utente. \\ \hline
\texttt{/auth/get\_applications\_by\_dates} & Ritorna il numero di applicazioni, pubbliche e private, aggiunte e rimosse in un intervallo temporale deciso dall'utente. Le informazioni vengono separate per settimana. \\ \hline
\texttt{/auth/get\_applications\_by\_group} & Dato l'ID di un gruppo, ritorna le applicazioni contenute in quel gruppo. \\ \hline
\texttt{/auth/get\_applications\_by\_intervals} & Ritorna le applicazioni aggiunte e rimosse nelle ultime 24 ore, nell'ultima settimana, mese e anno. \\ \hline
\texttt{/auth/get\_applications\_list} & Ritorna la lista di tutte le applicazioni pubbliche. \\ \hline
\texttt{/auth/get\_categories} & Ritorna la lista delle categorie, complete delle sotto categorie. \\ \hline
\texttt{/auth/get\_domain\_by\_name} & Ritorna una lista di domini con nome simile a quello inviato dall'utente. \\ \hline
\texttt{/auth/get\_domain\_catalogue} & Dato un dominio, ritorna le applicazioni del catalogo di quel dominio. \\ \hline
%get\_domain\_logs & Ritorna i log riguardanti le attività su domini in un intervallo temporale deciso dall'utente. \\ \hline
\texttt{/auth/get\_domains\_with\_catalogue} & Ritorna i domini che hanno un catalogo di applicazioni associato. \\ \hline
%get\_error\_logs & Ritorna i log riguardanti le operazioni che hanno generato degli errori in un intervallo temporale deciso dall'utente. \\ \hline
\texttt{/auth/get\_group\_details} & Ritorna i dettagli di un gruppo di applicazioni, come il nome, la descrizione e l'immagine. \\ \hline
\texttt{/auth/get\_groups} & Ritorna i gruppi pubblici di applicazioni presenti in Catalogue Manager. \\ \hline
%get\_groups\_logs & Ritorna i log riguardanti le attività su gruppi in un intervallo temporale deciso dall'utente. \\ \hline
\texttt{/auth/get\_properties} & Ritorna le \texttt{properties} utilizzabili per l'accesso in \glossaryItem{sso} di tipo \textit{form-based}. \\ \hline
\texttt{/auth/get\_sovra\_category\_stats} & Ritorna il numero di applicazioni, pubbliche e private, presenti in ciascuna sotto categoria. \\ \hline
\texttt{/auth/get\_sso\_actions} & Ritorna una lista delle \texttt{actions} da eseguire per l'accesso in \glossaryItem{sso} di tipo \textit{form-based}. \\ \hline
\texttt{/auth/get\_stats} & Ritorna il numero totale di applicazioni e gruppi, pubblici e privati. Vengono inoltre ritornate il numero di applicazioni presenti in ciascuna categoria. \\ \hline
\texttt{/auth/get\_user\_info} & Ritorna le informazioni dell'utente che ha effettuato il \textit{login}. \\ \hline
\texttt{/auth/remove\_application} & Rimuove definitivamente un'applicazione dato il suo ID. \\ \hline
\texttt{/auth/remove\_applications\_from\_group} & Rimuove un insieme di applicazioni da un gruppo. \\ \hline
\texttt{/auth/remove\_domain\_catalogue} & Dato l'ID di un dominio, rimuove il catalogo di quel dominio. Le applicazioni contenute in quel catalogo vengono eliminate definitivamente. \\ \hline
\texttt{/auth/remove\_group} & Dato l'ID di un gruppo, lo rimuove. La rimozione del gruppo causa solo la cancellazione del collegamento con le applicazioni che vi erano inserite. Queste applicazioni restano nel catalogo di Monokee, o in quello di un dominio. \\ \hline
\texttt{/auth/set\_learning\_catalogue} & Consente di utilizzare il \textit{plug in} di Monokee per il \textit{learning}. Il \textit{learning} serve per il \glossaryItem{sso} tramite \textit{form fulfillment}. \\ \hline
\texttt{/auth/set\_learning\_mobile} & Consente inserire i dati per il \textit{form fulfillment} per \textit{browsers} \textit{mobile}. \\ \hline
\texttt{/auth/set\_maintenance} & Consente di aggiornare il \textit{flag} di manutenzione di un'applicazione. \\ \hline
\texttt{/auth/update\_3rd\_type\_data} & Consente di aggiornare i dati di \glossaryItem{autenticazione} per un'applicazione con accesso del terzo tipo. L'\glossaryItem{autenticazione} per queste applicazioni viene effettuata tramite una richiesta POST \glossaryItem{ajax}. È quindi possibile modificare tutti gli elementi di questa richiesta, come le \texttt{properties} utilizzate e gli \texttt{headers} della richiesta. \\ \hline
\texttt{/auth/update\_formSSO\_data} & Consente di aggiornare i dati di \glossaryItem{autenticazione} per un'applicazione di tipo \textit{form-based}. Oltre all'utilizzo del \textit{learning}, quindi, è possibile impostare questi dati anche manualmente. Questa possibilità si rivela molto utile se l'accesso viene effettuato tramite richieste POST \glossaryItem{ajax} e non tramite \textit{form-fulfillment}. \\ \hline
\texttt{/auth/update\_general\_data} & Consente di aggiornare i dati generali di un'applicazione, come il nome, la descrizione, l'\glossaryItem{url} o l'immagine. \\ \hline
\texttt{/auth/update\_group} & Consente di aggiornare i dati generali di un gruppo, ma non di aggiungere o rimuovere applicazioni da esso. \\ \hline
\texttt{/auth/update\_saml\_instructions} & Consente di aggiornare le istruzioni per la configurazione dell'accesso tramite \glossaryItem{saml}. Essendo l'applicazione nel catalogo generale e indipendente dagli utilizzatori, è permesso solo configurare le istruzioni (viste come una serie di azioni da compiere) per configurare le informazioni richieste per l'utilizzo di \glossaryItem{saml}. È compito di chi aggiunge l'applicazione desiderata al proprio \textit{application broker} inserire queste informazioni. \\ \hline
    \caption[Servizi REST esposti dal back end di Catalogue Manager]{Servizi REST esposti dal back end di Catalogue Manager}
    \label{tab:serviziREST} 
    \end{longtable}
  \egroup
\end{center} 