\chapter{Valutazione retrospettiva} \label{conclusioni}

\section{Obiettivi fissati}
Gli obiettivi fissati erano molto ambiziosi: è stato inizialmente richiesto un elevato numero di funzionalità, che è cresciuto ulteriormente durante lo svolgimento dell'attività di stage. Monokee è un prodotto in continua evoluzione: di conseguenza i requisiti sono soggetti a cambiamenti frequenti e, talvolta, inaspettati. 

Le principali modifiche apportate, in corso d'opera, ai requisiti sono le seguenti:
\begin{itemize}
\item \textbf{tipologie di applicazioni permesse}: come descritto in \ref{catmgr} sono presenti tre tipi di autenticazione diversa: \textit{form based}, tramite \glossaryItem{saml} o di terzo tipo. Inizialmente, tuttavia, solamente le prime due erano state richieste. L'introduzione del nuovo tipo ha portato all'aggiunta di altri servizi \glossaryItem{rest}, di un nuovo modello di mongoose.js e la modifica di alcuni servizi e moduli già scritti;
\item \textbf{servizio di logging e statistiche}: sebbene non fossero funzionalità definite inizialmente, sono diventate importanti grazie al parallelo sviluppo di Monokee, e sono pertanto state inserite come requisiti desiderabili;
\item \textbf{sviluppo di un'applicazione mobile}: ha richiesto la definizione di un ulteriore servizio \glossaryItem{rest} (\texttt{set\_learning\_mobile}) e la modifica di uno dei modelli utilizzati da Catalogue Manager (\texttt{CatalogueForm}). 
\end{itemize}
Oltre a questo, lo sviluppo parallelo del front end ha, in qualche caso, portato alla luce degli aspetti tralasciati durante l'esposizione del progetto e durante le riunioni di analisi del prodotto e dei requisiti.

\section{Obiettivi raggiunti}
Grazie alla collaborazione del personale di \textit{Athesys S.r.l.} e del tutor aziendale Roberto Griggio, il periodo di apprendimento e di analisi si è concluso velocemente e in anticipo rispetto a quanto pianificato. Monokee è un prodotto molto complesso, e capirne a pieno la struttura si è rivelato, inizialmente, molto complicato: al di là delle funzionalità offerte, il sistema di \glossaryItem{idm} ha richiesto del tempo per essere compreso a fondo. Inoltre, l'accesso a Catalogue Manager è di tipo federato con \glossaryItem{saml}, e riuscire a capire il funzionamento di questo standard si è rivelato difficile, ma fondamentale.

L'architettura riflette quella classica utilizzata per applicazioni web basate sul \glossaryItem{framework} Express.js ed ha potuto beneficiare della struttura esistente di Monokee. Se da un lato questo è un aspetto positivo, perché molte scelte si sono rivelate già prese, adottate e provate, dall'altro è stata necessaria una particolare attenzione all'integrazione con gli elementi esistenti, in modo da preservare la scalabilità e la manutenibilità del prodotto. 

Nonostante i cambiamenti nei requisiti e le nuove funzionalità richieste, tutti i requisiti, sia obbligatori che desiderabili, sono stati soddisfatti. 

\section{Conclusioni}
Al momento della scelta dello stage, mi sono trovato davanti una lista di aziende proposte dall'Università di Padova, operanti in diversi settori. Dopo la giornata di presentazione di STAGE-IT, ho avuto modo di approfondire la conoscenza di alcune di esse tramite colloqui individuali e, in seguito, ho scelto \textit{Athesys S.r.l.} per i contenuti dei progetti proposti ed il buon clima aziendale che si è percepito già in fase di contatto. L'ambito dell'\glossaryItem{IAM} era a me sconosciuto, e l'impatto iniziale è stato difficile, ma grazie al clima di collaborazione e all'atteggiamento socievole del \textit{team} è stato facile capire come funzionavano le cose.

Giunti al termine dell'attività risulta comunque doveroso valutare in modo critico quanto prodotto. Il sistema, nell'arco di tutto il suo sviluppo, è stato continuamente oggetto di confronto e discussione da parte del \textit{team} di sviluppo dell'intera applicazione Monokee; questo approccio, se da una parte ha portato alcune volte a modifiche successive (talvolta onerose) di requisiti e componenti, ha indubbiamente permesso al prodotto di crescere come visione collettiva di un insieme di persone anziché di un unico responsabile, guadagnandone in termini di funzionalità individuate e completezza di visione del problema.

Ritengo quindi l'esperienza decisamente positiva. Tutti gli obiettivi fissati sono stati portati a termine ed è nata un'applicazione funzionante ed estensibile che diventerà parte di un progetto interessante e dal grande futuro. L'intera attività di stage si è svolta con il \textit{team} di sviluppo di Monokee, che mi ha aiutato ad apprendere nozioni nuove ed affascinanti. Questa esperienza mi ha fatto capire che da soli possiamo fare molto poco e che grandi risultati possono derivare solo dalla collaborazione di persone motivate e sempre disponibili ad aiutarsi in caso di bisogno.

Una delle cose che mi ha colpito maggiormente durante tutta il periodo di stage è la serietà con cui è stata presa in considerazione ogni mia singola opinione; ognuno ha sempre potuto esprimere le proprie idee, i propri dubbi e le proprie perplessità su quanto stava sviluppando così come sulle altre parti dell'applicativo. Questo atteggiamento mi ha fatto superare la mia iniziale timidezza e mi ha spinto ad esprimere le mie opinioni ed i miei dubbi ogniqualvolta ne sentissi l'utilità, nella speranza di provare a contribuire con qualche piccola idea. 

L'atteggiamento socievole del \textit{team}, inoltre, mi ha permesso di integrarmi all'interno del gruppo con gran semplicità, favorendo la collaborazione con gli altri componenti e risultando parte attiva nella buona riuscita dell'attività.

Ho infine avuto la possibilità di applicare le nozioni apprese durante le molte ore di studio e di lezione di questi tre anni università, ma anche di studiare in modo autonomo nuovi concetti e tecnologie. Si sono rivelate molto utili tutte le conoscenze acquisite durante l'insegnamento di Ingegneria del Software, dall'esistenza di database \glossaryItem{nosql}, all'utilizzo di Node.js e JavaScript per lo sviluppo del back end di applicazioni web, dalle tecniche di analisi dei requisiti alle modalità di stesura della documentazione di un'architettura e del codice. 

Ho capito ancora meglio come l'obiettivo del Corso di Laurea in Informatica sia quello di fornire un ''modo di pensare'' che permetta allo studente (o laureato) di far fronte a qualsiasi situazione, di affrontare autonomamente le difficoltà e di saper scegliere sempre la strada giusta. Molti studenti, tuttavia, durante il loro percorso di studi, valutano la loro preparazione informatica in relazione alle tecnologie che hanno appreso fino a quel momento, ricercando e richiedendo le tecnologie più nuove e criticando i corsi che propongono linguaggi per loro obsoleti. Questa visione, a mio giudizio, è eccessivamente miope: l'informatica è un'area dinamica e in continua evoluzione. Le cose progrediscono e cambiano molto velocemente, e quello che oggi è attuale già domani potrebbe essere obsoleto. Al contrario, le capacità analitiche e di ragionamento sono utili in qualsiasi situazione e permettono di adattarsi al meglio alle difficoltà.

È quindi soprattutto merito di questa visione se mi sono travato ad apprendere velocemente nuove tecnologie e nozioni in ambiti a me sconosciuti e se non ho riscontrato significativi problemi nel corso dell'attività di stage. La capacità di gestire un progetto complesso, di affrontare difficoltà e scadenze e di ragionare a fondo su ogni aspetto sono state provvidenziali e utilissime.

\begin{displayquote}
\centering
\textit{It is not the task of the University to offer what society asks for, but to give what society needs.}

\textit{Edsger Wybe Dijkstra}
\end{displayquote}