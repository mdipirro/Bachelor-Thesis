\chapter{Analisi dei requisiti}\label{adr}
\section{Attori coinvolti}
L'attore principale dell'applicazione Catalogue Manager è l'Utente Autenticato. Una gerarchia di utenti, ognuno associato a permessi e funzionalità crescenti, è prevista per le successive versioni del gestore del catalogo applicativo di Monokee, ma non è stata inserita tra i requisiti del progetto per volontà del tutor aziendale. 

L'autenticazione è di tipo federato e fa uso di \glossaryItem{saml}: non è pertanto presente una fase di registrazione. L'applicazione ha il ruolo di \glossaryItem{sp} e utilizza Monokee come \glossaryItem{idp}. Un utente non autenticato (ma autenticato presso Monokee) dipendentemente dal suo ruolo può:
\begin{itemize}
\item aggiungere manualmente l'applicazione ad un dominio;
\item accedere all'applicazione.
\end{itemize}
Tutte le funzionalità di Catalogue Manager sono accessibili solamente dopo aver effettuato l'accesso e, come già detto, non esiste nessuna distinzione tra utenti: questo significa che un amministratore di dominio e un semplice utente di Monokee in Catalogue Manager hanno gli stessi poteri.

\section{Casi d'uso ad alto livello}
Considerando l'elevato numero di funzionalità previste in Catalogue Manager di seguito vengono riportati solo i casi d'uso di alto livello, in modo da poter dare una visione d'insieme più precisa del prodotto sviluppato.

Ad ogni caso d'uso è associato un identificatore univoco così formato:

\begin{center}
\textbf{UCTX.Y.Z}
\end{center}
dove:
\begin{itemize}
\item \textbf{T} corrisponde al tipo di caso d'uso:
	\begin{itemize}
	\item \textbf{U} per l'Utente Non Autenticato;
	\item \textbf{A} per l'Utente Autenticato;
	\end{itemize}
\item \textbf{X} corrisponde all'identificatore del caso d'uso ''padre'';
\item \textbf{Y} corrisponde all'identificatore del caso d'uso ''figlio'' di primo livello;
\item \textbf{Z} corrisponde all'identificatore del caso d'uso ''figlio'' di secondo livello;
\end{itemize}
Man mano che si scende nella gerarchia (da padre a figlio di primo e poi secondo livello) si aumenta il dettaglio del caso d'uso.

\subsection{Operazioni permesse ad un Utente Non Autenticato}
\begin{figure}[hbpc]
  \begin{center}
    \includegraphics[width=12cm]{UC/UtenteNonAutenticato}
  \caption[Operazioni Generali per l'Utente Non Autenticato]{Operazioni di alto livello permesse ad un Utente Non Autenticato}
  \end{center} 
\end{figure}

\begin{center}
  \bgroup
  \def\arraystretch{1.8}     
  \begin{longtable}{  p{3.5cm} | p{8cm} } 
    \hline
    \multicolumn{2}{ | c | }{ \cellcolor[gray]{0.9} \textbf{Operazioni generali per l'Utente Non Autenticato}} \\
    \textbf{Attori Primari} & Utente Non Autenticato \\ 
    \textbf{Scopo e Descrizione} & L'Utente Non Autenticato può aggiungere, tramite le funzionalità di Monokee, l'applicazione Catalogue Manager ad un \textit{application broker} di un dominio. Dopo averla aggiunta può decidere in qualsiasi momento di cambiarne gli attributi di configurazione attraverso la pagina di modifica di Monokee. Infine può accedere all'applicazione per gestire il catalogo applicativo. Si ricorda che, sebbene venga definito come Utente Non Autenticato, l'utente in questione deve essere autenticato ed autorizzato tramite Monokee. \\ 
    \textbf{Precondizioni}  & L'Utente Non Autenticato ha effettuato l'accesso a Monokee. \\
    \textbf{Postcondizioni} & Monokee ha preso in carico, ed eseguito, l'operazione voluta dall'Utente Non Autenticato.  \\ 
    \textbf{Flusso Principale} & 
    1. L'Utente Non Autenticato aggiunge l'applicazione Catalogue Manager ad un \textit{application broker} di un dominio di Monokee. (UCU1) \newline
    2. L'Utente Non Autenticato modifica gli attributi di Catalogue Manager. (UCU2) \newline
    3. L'Utente Non Autenticato accede a Catalogue Manager. (UCU3) \\
    \textbf{Estensioni} & L'Utente Non Autenticato visualizza un messaggio di errore perché non è autorizzato ad accedere a Catalogue Manager. (UCU4)
  \end{longtable}
  \egroup
\end{center}

\subsubsection{UCU1 - Aggiunta di Catalogue Manager ad un Application Broker}
\begin{center}
  \bgroup
  \def\arraystretch{1.8}     
  \begin{longtable}{  p{3.5cm} | p{8cm} } 
    \multicolumn{2}{ | c | }{ \cellcolor[gray]{0.9} \textbf{UCU1 - Aggiunta di Catalogue Manager ad un Application Broker}} \\
    \hline
    
    \textbf{Attori Primari} & Utente Non Autenticato \\ 
    \textbf{Scopo e Descrizione} & L'Utente Non Autenticato può aggiungere l'applicazione Catalogue Manager ad un \textit{application broker} di un dominio di Monokee. Dato che l'\glossaryItem{autenticazione} avviene tramite \glossaryItem{saml}, per aggiungere l'applicazione è necessario inserire tutti i parametri richiesti da questo standard. \\ 
    
    \textbf{Precondizioni}  & L'Utente Non Autenticato ha effettuato l'accesso a Monokee e si trova nella pagina dell'aggiunta di una nuova applicazione \glossaryItem{saml}. \\ 
    
    \textbf{Postcondizioni} & Monokee ha aggiunto all'\textit{application broker} del dominio selezionato dall'Utente Non Autenticato l'applicazione Catalogue Manager. \\ 
    \textbf{Flusso Principale} & 
    1. L'Utente Non Autenticato inserisce l'\glossaryItem{url} dell'\textit{assertion consumer service}. \newline
    2. L'Utente Non Autenticato inserisce l'\glossaryItem{uri} del \glossaryItem{sp}. \newline
    3. L'Utente Non Autenticato inserisce il certificato del \glossaryItem{sp}. \newline
    4. L'Utente Non Autenticato inserisce l'\glossaryItem{url} della pagina successiva al \textit{login}. \newline
    5. L'Utente Non Autenticato inserisce le regole dell'asserzione \glossaryItem{saml}. \newline
    6. L'Utente Non Autenticato inserisce l'algoritmo di firma. \newline
    7. L'Utente Non Autenticato inserisce l'\glossaryItem{uri} della pagina di \textit{log out}. \newline
    8. L'Utente Non Autenticato inserisce l'\glossaryItem{uri} della pagina di risposta al \textit{log out}. 
  \end{longtable}
  \egroup
\end{center}

\subsubsection{UCU2 - Modifica della configurazione specifica di un dominio di Catalogue Manager}
\begin{center}
  \bgroup
  \def\arraystretch{1.8}     
  \begin{longtable}{  p{3.5cm} | p{8cm} } 
    \multicolumn{2}{ | c | }{ \cellcolor[gray]{0.9} \textbf{UCU2 - Modifica della configurazione specifica di un dominio di Catalogue Manager}} \\
    \hline
    
    \textbf{Attori Primari} & Utente Non Autenticato \\ 
    \textbf{Scopo e Descrizione} & L'Utente Non Autenticato può modificare gli attributi di configurazione di Catalogue Manager per un dominio tramite la pagina di modifica di un'applicazione prevista da Monokee. \\ 
    
    \textbf{Precondizioni}  & L'Utente Non Autenticato ha effettuato l'accesso a Monokee e si trova nella pagina di modifica di un'applicazione esistente. \\ 
    
    \textbf{Postcondizioni} & Monokee ha modificato l'applicazione Catalogue Manager per il dominio selezionato dall'Utente Non Autenticato. \\ 
    \textbf{Flusso Principale} & 
    1. L'Utente Non Autenticato modifica l'\glossaryItem{url} dell'\textit{assertion consumer service}. \newline
    2. L'Utente Non Autenticato modifica l'\glossaryItem{uri} del \glossaryItem{sp}. \newline
    3. L'Utente Non Autenticato modifica il certificato del \glossaryItem{sp}. \newline
    4. L'Utente Non Autenticato modifica l'\glossaryItem{url} della pagina successiva al \textit{login}. \newline
    5. L'Utente Non Autenticato modifica le regole dell'asserzione \glossaryItem{saml}. \newline
    6. L'Utente Non Autenticato modifica l'algoritmo di firma. \newline
    7. L'Utente Non Autenticato modifica l'\glossaryItem{uri} della pagina di \textit{log out}. \newline
    8. L'Utente Non Autenticato modifica l'\glossaryItem{uri} della pagina di risposta al \textit{log out}. 
  \end{longtable}
  \egroup
\end{center}

\subsubsection{UCU3 - Accesso a Catalogue Manager}
\begin{center}
  \bgroup
  \def\arraystretch{1.8}     
  \begin{longtable}{  p{3.5cm} | p{8cm} } 
    \multicolumn{2}{ | c | }{ \cellcolor[gray]{0.9} \textbf{UCU3 - Accesso a Catalogue Manager}} \\
    \hline
    
    \textbf{Attori Primari} & Utente Non Autenticato \\ 
    \textbf{Scopo e Descrizione} & L'Utente Non Autenticato può accedere a Catalogue Manager tramite l'\textit{application broker} di un dominio di Monokee. \\ 
    
    \textbf{Precondizioni}  & L'Utente Non Autenticato ha effettuato l'accesso a Monokee e si trova nell'\textit{application broker} di un dominio. \\ 
    
    \textbf{Postcondizioni} & L'Utente Non Autenticato ha effettuato l'accesso a Catalogue Manager e si trova nella pagina principale della nuova applicazione. \\ 
    \textbf{Flusso Principale} & 
    1. L'Utente Non Autenticato richiede (tramite un \textit{click}) l'accesso a Catalogue Manager. \\
    \textbf{Estensioni} & L'Utente Non Autenticato visualizza un messaggio di errore perché l'\glossaryItem{idp} di Monokee non gli ha consentito l'accesso a Catalogue Manager.
  \end{longtable}
  \egroup
\end{center}

\subsubsection{UCU4 - Visualizzazione messaggio di errore per utente non autorizzato}
\begin{center}
  \bgroup
  \def\arraystretch{1.8}     
  \begin{longtable}{  p{3.5cm} | p{8cm} } 
    \multicolumn{2}{ | c | }{ \cellcolor[gray]{0.9} \textbf{UCU4 - Visualizzazione messaggio di errore per utente non autorizzato}} \\
    \hline
    
    \textbf{Attori Primari} & Utente Non Autenticato \\ 
    \textbf{Scopo e Descrizione} & L'Utente Non Autenticato visualizza un messaggio di errore in seguito all'accesso negato, da parte dell'\glossaryItem{idp} di Monokee, all'applicazione Catalogue Manager. \\ 
    
    \textbf{Precondizioni}  & L'Utente Non Autenticato ha effettuato l'accesso a Monokee e si trova nell'\textit{application broker} di un dominio. ha inoltre richiesto l'accesso all'applicazione Catalogue Manager, ma gli è stata negata dall'\glossaryItem{idp} di Monokee. \\ 
    
    \textbf{Postcondizioni} & L'Utente Non Autenticato ha visualizzato il messaggio di errore per autenticazione non riuscita. \\ 
    \textbf{Flusso Principale} & 
    1. L'Utente Non Autenticato visualizza il messaggio di errore conseguente all'accesso negato a Catalogue Manager.
  \end{longtable}
  \egroup
\end{center} % casi d'uso per utente non autenticato
\newpage
\subsection{Operazioni generali per l'Utente Autenticato}
\begin{figure}[hbpc]
  \begin{center}
    \includegraphics[width=12cm]{UC/UtenteAutenticato}
  \caption[Operazioni Generali per l'Utente Autenticato]{Operazioni di alto livello permesse ad un Utente Autenticato}
  \end{center} 
\end{figure}
\newpage
\begin{center}
  \bgroup
  \def\arraystretch{1.8}     
  \begin{longtable}{  p{3.5cm} | p{8cm} } 
    \hline
    \multicolumn{2}{ | c | }{ \cellcolor[gray]{0.9} \textbf{Operazioni generali per l'Utente Autenticato}} \\
    \textbf{Attori Primari} & Utente Autenticato \\ 
    \textbf{Scopo e Descrizione} & All'Utente Autenticato viene mostrata una \textit{dashboard} a partire dalla quale può eseguire varie operazioni. Queste operazioni permettono di gestire l'intero catalogo applicativo, i raggruppamenti di applicazioni e i cataloghi specifici per i domini aziendali. È possibile anche cercare le applicazioni sulla base di vari filtri, visualizzare statistiche e \textit{log}. \\ 
    \textbf{Precondizioni}  & L'Utente Autenticato ha effettuato l'accesso a Catalogue Manager a partire da Monokee attraverso \glossaryItem{saml}. \\
    \textbf{Postcondizioni} & Catalogue Manager ha preso in carico, ed eseguito, l'operazione voluta dall'Utente Autenticato.  \\ 
    \textbf{Flusso Principale} & 
    1. L'Utente Autenticato visualizza la lista di applicazioni presenti nel catalogo. (UCA1) \newline
    2. L'Utente Autenticato aggiunge un'applicazione al catalogo. (UCA2) \newline
    3. L'Utente Autenticato modifica i dati di un'applicazione del catalogo. (UCA3) \newline 
    4. L'Utente Autenticato rimuove un'applicazione. (UCA4) \newline
    5. L'Utente Autenticato gestisce i cataloghi specifici dei domini aziendali. (UCA5) \newline
    5. L'Utente Autenticato ricerca un'applicazione. (UCA6) \newline
    6. L'Utente Autenticato gestisce i raggruppamenti di applicazioni. (UCA10) \newline
    7. L'Utente Autenticato visualizza le statistiche sul numero di applicazioni e gruppi (UCA14) \newline
    8. L'Utente Autenticato visualizza i \textit{log} sulle operazioni effettuate da lui e dagli altri utenti dell'applicazione. (UCA15) \newline
    I \textit{log} riguardano sia le operazioni eseguite con successo (UCA16) sia quelle che hanno generato errori (UCA17).\\
    \textbf{Estensioni} & L'Utente Autenticato cerca di inserire un'applicazione pubblica che è già presente nel catalogo. (UCA13)
  \end{longtable}
  \egroup
\end{center}

\subsubsection{UCA1 - Visualizzazione Applicazioni}
\begin{center}
  \bgroup
  \def\arraystretch{1.8}     
  \begin{longtable}{  p{3.5cm} | p{8cm} } 
    \multicolumn{2}{ | c | }{ \cellcolor[gray]{0.9} \textbf{UCA1 - Visualizzazione Applicazioni}} \\
    \hline
    
    \textbf{Attori Primari} & Utente Autenticato \\ 
    \textbf{Scopo e Descrizione} & L'Utente Autenticato può visualizzare le applicazioni presenti nel catalogo di Monokee. \\ 
    
    \textbf{Precondizioni}  & Catalogue Manager presenta all'Utente Autenticato una pagina contenente l'elenco delle applicazioni presenti in Monokee. \\ 
    
    \textbf{Postcondizioni} & L'Utente Autenticato ha visualizzato le applicazioni presenti in Monokee. \\ 
    \textbf{Flusso Principale} & 
    1. L'Utente Autenticato visualizza i dettagli di un'applicazione presente in Monokee.   
  \end{longtable}
  \egroup
\end{center}

\subsubsection{UCA2 - Aggiunta Applicazione Pubblica}
\begin{center}
  \bgroup
  \def\arraystretch{1.8}     
  \begin{longtable}{  p{3.5cm} | p{8cm} } 
    \multicolumn{2}{ | c | }{ \cellcolor[gray]{0.9} \textbf{UCA2 - Aggiunta Applicazione Pubblica}} \\
    \hline
    
    \textbf{Attori Primari} & Utente Autenticato \\ 
    \textbf{Scopo e Descrizione} & L'Utente Autenticato può aggiungere una nuova applicazione al catalogo di Monokee. \\ 
    
    \textbf{Precondizioni}  & Catalogue Manager presenta all'Utente Autenticato una pagina contenente un \textit{form} per l'aggiunta di un'applicazione al catalogo di Monokee. \\ 
    
    \textbf{Postcondizioni} & L'Utente Autenticato ha aggiunto un'applicazione al catalogo di Monokee. \\ 
    \textbf{Flusso Principale} & 
    1. L'Utente Autenticato inserisce il nome dell'applicazione. \newline
    2. L'Utente Autenticato inserisce la descrizione dell'applicazione. \newline
    3. L'Utente Autenticato inserisce l'\glossaryItem{url} dell'applicazione. \newline
    4. L'Utente Autenticato inserisce l'immagine dell'applicazione. \newline
    5. L'Utente Autenticato seleziona le categorie di appartenenza dell'applicazione. \newline
    6. L'Utente Autenticato seleziona la tipologia di \glossaryItem{autenticazione}. \newline
    7. L'Utente Autenticato seleziona il nome del gruppo di applicazioni nel quale inserire l'applicazione. \\
    \textbf{Estensioni} & 
        1. L'Utente Autenticato visualizza un messaggio di errore come conseguenza al tentativo di inserimento dello stesso nome di un'altra applicazione pubblica di Monokee.
  \end{longtable}
  \egroup
\end{center}

\subsubsection{UCA3 - Modifica Applicazione Pubblica}
\begin{center}
  \bgroup
  \def\arraystretch{1.8}     
  \begin{longtable}{  p{3.5cm} | p{8cm} } 
    \multicolumn{2}{ | c | }{ \cellcolor[gray]{0.9} \textbf{UCA3 - Modifica Applicazione Pubblica}} \\
    \hline
    
    \textbf{Attori Primari} & Utente Autenticato \\ 
    \textbf{Scopo e Descrizione} & L'Utente Autenticato può modificare un'applicazione esistente nel catalogo di Monokee. \\ 
    
    \textbf{Precondizioni}  & Catalogue Manager presenta all'Utente Autenticato una pagina contenente un \textit{form} per la modifica di un'applicazione nel catalogo di Monokee. \\ 
    
    \textbf{Postcondizioni} & L'Utente Autenticato ha modificato un'applicazione nel catalogo Monokee. \\ 
    \textbf{Flusso Principale} & 
    1. L'Utente Autenticato modifica il nome dell'applicazione. \newline
    2. L'Utente Autenticato modifica la descrizione dell'applicazione. \newline
    3. L'Utente Autenticato modifica l'\glossaryItem{url} dell'applicazione. \newline
    4. L'Utente Autenticato modifica l'immagine dell'applicazione. \newline
    5. L'Utente Autenticato seleziona le categorie di appartenenza dell'applicazione. \newline
    6. L'Utente Autenticato modifica i dati necessari all'\glossaryItem{autenticazione}. \newline
    	Tali dati possono riguardare l'accesso \textit{form-based}, tramite \glossaryItem{saml} o di terzo tipo. \newline
    7. L'Utente Autenticato seleziona il nome del gruppo di applicazioni nel quale modificare l'applicazione. \newline
    8. L'Utente Autenticato decide se pubblicare l'applicazione. \newline
    9. L'Utente Autenticato decide se mettere l'applicazione in manutenzione.\\
    \textbf{Estensioni} & 
    1. L'Utente Autenticato visualizza un messaggio di errore come conseguenza al tentativo di inserimento dello stesso nome di un'altra applicazione pubblica di Monokee.
  \end{longtable}
  \egroup
\end{center}

\subsubsection{UCA4 - Rimozione Applicazione Pubblica}
\begin{center}
  \bgroup
  \def\arraystretch{1.8}     
  \begin{longtable}{  p{3.5cm} | p{8cm} } 
    \multicolumn{2}{ | c | }{ \cellcolor[gray]{0.9} \textbf{UC4 - Rimozione Applicazione Pubblica}} \\
    \hline
    \textbf{Attori Primari} & Utente Autenticato \\ 
    \textbf{Scopo e Descrizione} & L'Utente Autenticato può rimuovere un'applicazione pubblica dal catalogo di Monokee. \\ 
    
    \textbf{Precondizioni}  & L'applicazione selezionata è presente nel catalogo di Monokee e l'Utente Autenticato ha selezionato il comando di rimozione su di essa. \\ 
    
    \textbf{Postcondizioni} & L'applicazione selezionata è stata rimossa dal catalogo di Monokee. \\ 
    \textbf{Flusso Principale} &
    1. L'Utente Autenticato conferma l'operazione di rimozione. \\
    \textbf{Inclusioni} & Richiesta di conferma rimozione.
  \end{longtable}
  \egroup
\end{center}

\subsubsection{UCA5 - Gestione Cataloghi di Dominio}
\begin{center}
  \bgroup
  \def\arraystretch{1.8}     
  \begin{longtable}{  p{3.5cm} | p{8cm} } 
    \multicolumn{2}{ | c | }{ \cellcolor[gray]{0.9} \textbf{UCA5 - Gestione Cataloghi di Dominio}} \\
    \hline
    
    \textbf{Attori Primari} & Utente Autenticato \\ 
    \textbf{Scopo e Descrizione} & L'Utente Autenticato può gestire i cataloghi associati a domini aziendali. \\ 
    
    \textbf{Precondizioni}  & Catalogue Manager presenta all'Utente Autenticato la pagina per la gestione dei cataloghi associati a domini aziendali. \\ 
    
    \textbf{Postcondizioni} & Catalogue Manager ha preso in carico, ed eseguito, le operazioni richieste dall'Utente Autenticato. \\ 
    \textbf{Flusso Principale} & 
    1. L'Utente Autenticato può aggiungere un nuovo catalogo di dominio ad un dominio senza catalogo. \newline 
    2. L'Utente Autenticato può visualizzare i cataloghi di dominio esistenti. \newline 
    3. L'Utente Autenticato può aggiungere un'applicazione ad un catalogo di dominio esistente. \newline
    4. L'Utente Autenticato può rimuovere un'applicazione da un catalogo di dominio esistente. \newline
    5. L'Utente Autenticato può rimuovere un catalogo di dominio esistente. \newline 
    6. L'Utente Autenticato può cercare un catalogo di dominio tra quelli esistenti. \newline
    	La ricerca può avvenire in base al nome del dominio. \\
    \textbf{Estensioni} &
    1. L'Utente Autenticato visualizza un messaggio di errore dovuto all'esistenza di un altro catalogo per il dominio selezionato.
  \end{longtable}
  \egroup
\end{center}

\subsubsection{UCA6 - Ricerca Applicazione}
\begin{center}
  \bgroup
  \def\arraystretch{1.8}     
  \begin{longtable}{  p{3.5cm} | p{8cm} } 
    \multicolumn{2}{ | c | }{ \cellcolor[gray]{0.9} \textbf{UCA6 - Ricerca Applicazione}} \\
    \hline
    
    \textbf{Attori Primari} & Utente Autenticato \\ 
    \textbf{Scopo e Descrizione} & L'Utente Autenticato può cercare un'applicazione del catalogo di Monokee. La ricerca può avvenire per nome o per categoria. \\ 
    
    \textbf{Precondizioni}  & Catalogue Manager ha mostrato all'utente la pagina di ricerca. L'Utente Autenticato può effettuare ricerche di vario tipo. \\ 
    
    \textbf{Postcondizioni} & Catalogue Manager ha mostrato i risultati della ricerca. \\ 
    \textbf{Flusso Principale} & 
    1. L'Utente Autenticato seleziona le modalità di ricerca e inserisce le informazioni richieste.
  \end{longtable}
  \egroup
\end{center}

\subsubsection{UCA10 - Gestione Gruppi di Applicazioni}
\begin{center}
  \bgroup
  \def\arraystretch{1.8}     
  \begin{longtable}{  p{3.5cm} | p{8cm} } 
    \multicolumn{2}{ | c | }{ \cellcolor[gray]{0.9} \textbf{UCA10 - Gestione Gruppi di Applicazioni}} \\
    \hline
    \textbf{Attori Primari} & Utente Autenticato \\ 
    \textbf{Scopo e Descrizione} & L'Utente Autenticato può gestire i gruppi di applicazioni del catalogo di Monokee. \\ 
    
    \textbf{Precondizioni}  & Catalogue Manager mostra all'Utente Autenticato la pagina di gestione dei gruppi di applicazioni. \\ 
    
    \textbf{Postcondizioni} & Catalogue Manager ha preso in carico le richieste dell'Utente Autenticato e le ha eseguite. \\ 
    \textbf{Flusso Principale} &
    1. L'Utente Autenticato può aggiungere un gruppo di applicazioni. \newline
    2. L'Utente Autenticato può visualizzare i gruppi di applicazioni presenti. \newline
    3. L'Utente Autenticato può gestire un gruppo di applicazioni esistente. \newline
    	In particolare, L'Utente Autenticato può aggiungere o rimuovere un'applicazione dal gruppo selezionato. \newline
    4. L'Utente Autenticato può modificare gli attributi di un gruppo di applicazioni esistente. \newline
    5. L'Utente Autenticato può rimuovere un gruppo di applicazioni esistente. 
  \end{longtable}
  \egroup
\end{center}

\subsubsection{UCA13 - Visualizzazione di un messaggio di errore per applicazione pubblica già presente}
\begin{center}
  \bgroup
  \def\arraystretch{1.8}     
  \begin{longtable}{  p{3.5cm} | p{8cm} } 
    \multicolumn{2}{ | c | }{ \cellcolor[gray]{0.9} \textbf{UCA13 - Visualizzazione di un messaggio di errore per applicazione pubblica già presente}} \\
    \hline
    
    \textbf{Attori Primari} & Utente Autenticato \\ 
    \textbf{Scopo e Descrizione} & L'Utente Autenticato ha cercato di inserire un nome corrispondente ad un'applicazione pubblica già presente nel catalogo di Monokee. Catalogue Manager presenta all'Utente Autenticato un messaggio di errore. \\ 
    
    \textbf{Precondizioni}  & L'Utente Autenticato ha inserito (nella pagina di aggiunta o di modifica di un'applicazione) un nome già utilizzato per un'applicazione pubblica. \\ 
    
    \textbf{Postcondizioni} & L'Utente Autenticato ha visualizzato il messaggio di errore presentato da Catalogue Manager. \\
    \textbf{Flusso Principale} & 
    1. L'Utente Autenticato visualizza il messaggio di errore.
  \end{longtable}
  \egroup
\end{center}

\subsubsection{UCA14 - Visualizzazione Statistiche}
\begin{center}
  \bgroup
  \def\arraystretch{1.8}     
  \begin{longtable}{  p{3.5cm} | p{8cm} } 
    \multicolumn{2}{ | c | }{ \cellcolor[gray]{0.9} \textbf{UCA14 - Visualizzazione Statistiche}} \\
    \hline
    
    \textbf{Attori Primari} & Utente Autenticato \\ 
    \textbf{Scopo e Descrizione} & L'Utente Autenticato può visualizzare delle statistiche riguardanti l'applicazione Catalogue Manager. In particolare, le statistiche riguardano:
    	\begin{enumerate}
 	    \item numero di applicazioni aggiunte e rimosse in intervalli di tempo definiti a priori: ultime 24 ore, ultima settimana, ultimo mese e ultimo anno;
 	    \item numero di accessi in intervalli di tempo definiti a priori: ultime 24 ore e ultima settimana;
 	    \item numero di utenti attivi;
 	    \item numero di applicazioni e gruppi pubblici e privati;
 	    \item numero di applicazioni, pubbliche e private, appartenenti ad ogni categoria (intesa come ''sotto categoria'');
 	    \item numero di applicazioni, pubbliche e private, appartenenti ad una specifica ''sovra categoria''.
 	  	\end{enumerate} \\ 
    
    \textbf{Precondizioni}  & L'Utente Autenticato ha richiesto la visualizzazione delle statistiche dell'applicazione Catalogue Manager. \\ 
    
    \textbf{Postcondizioni} & L'Utente Autenticato ha visualizzato le statistiche dell'applicazione Catalogue Manager. \\
    \textbf{Flusso Principale} & 
    1. L'Utente Autenticato visualizza il numero di applicazioni aggiunte e rimosse in intervalli di tempo definiti a priori: ultime 24 ore, ultima settimana, ultimo mese e ultimo anno. \newline
    2. L'Utente Autenticato visualizza il numero di accessi in intervalli di tempo definiti a priori: ultime 24 ore e ultima settimana. \newline
    3. L'Utente Autenticato visualizza il numero di utenti attivi. \newline
    4. L'Utente Autenticato visualizza il numero di applicazioni e gruppi pubblici e privati. \newline
    5. L'Utente Autenticato visualizza il numero di applicazioni, pubbliche e private, appartenenti ad ogni categoria (intesa come ''sotto categoria''). \newline
    6. L'Utente Autenticato visualizza il numero di applicazioni, pubbliche e private, appartenenti ad una specifica ''sovra categoria''.
  \end{longtable}
  \egroup
\end{center}

\subsubsection{UCA15 - Visualizzazione Log}
\begin{center}
  \bgroup
  \def\arraystretch{1.8}     
  \begin{longtable}{  p{3.5cm} | p{8cm} } 
    \multicolumn{2}{ | c | }{ \cellcolor[gray]{0.9} \textbf{UCA15 - Visualizzazione Log}} \\
    \hline
    
    \textbf{Attori Primari} & Utente Autenticato \\ 
    \textbf{Scopo e Descrizione} & L'Utente Autenticato può visualizzare i \textit{log} delle operazioni eseguite nell'applicazione Catalogue Manager. In particolare i \textit{log} possono riguardare le operazioni eseguite con successo (UCA16) e quelle che hanno generato un errore (UCA17). I \textit{log} salvati possono essere filtrati per intervallo di date, per \textit{keywords} e per tipologia. \\
    
    \textbf{Precondizioni}  & L'Utente Autenticato ha richiesto la visualizzazione dei \textit{log} dell'applicazione Catalogue Manager. \\ 
    
    \textbf{Postcondizioni} & L'Utente Autenticato ha visualizzato i \textit{log} dell'applicazione Catalogue Manager. \\
    \textbf{Flusso Principale} & 
    1. L'Utente Autenticato visualizza i \textit{log} dell'applicazione. 
  \end{longtable}
  \egroup
\end{center}

\subsubsection{UCA16 - Visualizzazione Log sulle operazioni che hanno avuto successo}
\begin{center}
  \bgroup
  \def\arraystretch{1.8}     
  \begin{longtable}{  p{3.5cm} | p{8cm} } 
    \multicolumn{2}{ | c | }{ \cellcolor[gray]{0.9} \textbf{UCA16 - Visualizzazione Log sulle operazioni che hanno avuto successo}} \\
    \hline
    
    \textbf{Attori Primari} & Utente Autenticato \\ 
    \textbf{Scopo e Descrizione} & L'Utente Autenticato può visualizzare i \textit{log} delle operazioni eseguite con successo nell'applicazione Catalogue Manager. I \textit{log} salvati possono essere filtrati per intervallo di date, per \textit{keywords} e per tipologia. \\
    
    \textbf{Precondizioni}  & L'Utente Autenticato ha richiesto la visualizzazione dei \textit{log} dell'applicazione Catalogue Manager. Ha successivamente selezionato la visualizzazione dei \textit{log} delle operazioni eseguite con successo. \\ 
    
    \textbf{Postcondizioni} & L'Utente Autenticato ha visualizzato i \textit{log} delle operazioni eseguite con successo dell'applicazione Catalogue Manager. \\
    \textbf{Flusso Principale} & 
    1. L'Utente Autenticato visualizza i \textit{log} delle operazioni eseguite con successo.
  \end{longtable}
  \egroup
\end{center}

\subsubsection{UCA17 - Visualizzazione Log delle operazioni che hanno generato errori}
\begin{center}
  \bgroup
  \def\arraystretch{1.8}     
  \begin{longtable}{  p{3.5cm} | p{8cm} } 
    \multicolumn{2}{ | c | }{ \cellcolor[gray]{0.9} \textbf{UCA17 - Visualizzazione Log delle operazioni che hanno generato errori}} \\
    \hline
    
    \textbf{Attori Primari} & Utente Autenticato \\ 
    \textbf{Scopo e Descrizione} & L'Utente Autenticato può visualizzare i \textit{log} delle operazioni che hanno generato errori nell'applicazione Catalogue Manager. I \textit{log} salvati possono essere filtrati per intervallo di date, per \textit{keywords} e per tipologia. \\
    
    \textbf{Precondizioni}  & L'Utente Autenticato ha richiesto la visualizzazione dei \textit{log} dell'applicazione Catalogue Manager. Ha successivamente selezionato la visualizzazione dei \textit{log} delle operazioni che hanno generato errori. \\ 
    
    \textbf{Postcondizioni} & L'Utente Autenticato ha visualizzato i \textit{log} delle operazioni che hanno generato errori dell'applicazione Catalogue Manager. \\
    \textbf{Flusso Principale} & 
    1. L'Utente Autenticato visualizza i \textit{log} delle operazioni che hanno generato errore. 
  \end{longtable}
  \egroup
\end{center}

\subsubsection{UCA18 - Ricerca Domini Aziendali di Monokee}
\begin{center}
  \bgroup
  \def\arraystretch{1.8}     
  \begin{longtable}{  p{3.5cm} | p{8cm} } 
    \multicolumn{2}{ | c | }{ \cellcolor[gray]{0.9} \textbf{UCA18 - Ricerca Domini Aziendali di Monokee}} \\
    \hline
    \textbf{Attori Primari} & Utente Autenticato \\ 
    \textbf{Scopo e Descrizione} & L'Utente Autenticato può effettuare una ricerca tra i domini aziendali di Monokee. La ricerca avviene per nome. \\ 
    
    \textbf{Precondizioni}  & Catalogue Manager mostra all'Utente Autenticato la pagina di gestione dei cataloghi di dominio di Monokee. \\ 
    
    \textbf{Postcondizioni} & Catalogue Manager ha preso cercato tra i domini aziendali di Monokee e ha mostrato all'Utente Autenticato i risultati. \\ 
    \textbf{Flusso Principale} &
    1. L'Utente Autenticato cerca un dominio aziendale di Monokee. 
  \end{longtable}
  \egroup
\end{center}

\subsubsection{UCA19 - Visualizzazione di un messaggio di errore per catalogo già presente}
\begin{center}
  \bgroup
  \def\arraystretch{1.8}     
  \begin{longtable}{  p{3.5cm} | p{8cm} } 
    \multicolumn{2}{ | c | }{ \cellcolor[gray]{0.9} \textbf{UCA19 - Visualizzazione di un messaggio di errore per catalogo già presente}} \\
    \hline
    
    \textbf{Attori Primari} & Utente Autenticato \\ 
    \textbf{Scopo e Descrizione} & L'Utente Autenticato ha cercato di aggiungere un catalogo di dominio ad un dominio che ha già un catalogo. Catalogue Manager presenta all'Utente Autenticato un messaggio di errore. \\ 
    
    \textbf{Precondizioni}  & L'Utente Autenticato ha cercato di aggiungere un catalogo di dominio ad un dominio che ha già un catalogo. \\ 
    
    \textbf{Postcondizioni} & L'Utente Autenticato ha visualizzato il messaggio di errore presentato da Catalogue Manager. \\
    \textbf{Flusso Principale} & 
    1. L'Utente Autenticato visualizza il messaggio di errore.
  \end{longtable}
  \egroup
\end{center} % casi d'uso per utente autenticato


\section{Requisiti}
I requisiti funzionali e di vincolo individuati sono riportati nelle seguenti tabelle. Viene inoltre indicato se si tratta di un requisito fondamentale, desiderabile o facoltativo e una sua descrizione.

Ogni requisito è identificato da un codice, che segue il seguente formalismo:
\begin{center}
		\textbf{RXY Gerarchia}
\end{center}

Dove:
\begin{itemize}
 \item \textbf{X} corrisponde alla tipologia del requisito e può assumere i seguenti valori:
		\begin{itemize}
		 \item[] \textbf{1} = Funzionale;
		 \item[] \textbf{2} = Vincolo.
		\end{itemize}

 \item \textbf{Y} corrisponde alla priorità del requisito e può assumere i seguenti valori:
		\begin{itemize}
		 \item[] \textbf{O} = Obbligatorio;
		 \item[] \textbf{D} = Desiderabile;
		 \item[] \textbf{F} = Facoltativo o Opzionale.
		\end{itemize}

 \item \textbf{Gerarchia} identifica la relazione gerarchica che c'è tra i requisiti di uno stesso tipo. Vi è dunque una struttura gerarchica per ogni tipologia di requisito.
\end{itemize}

\subsection{Principali requisiti funzionali}
Nella Tabella~\ref{tab:reqfunzionali} vengono elencati i principali requisiti funzionali dell'applicazione Catalogue Manager.
\begin{center}
  \bgroup
  \def\arraystretch{1.8}
  \begin{longtable}{ | l | p{8.4cm} |}
    \hline
    \cellcolor[gray]{0.9} \textbf{Requisito} & \cellcolor[gray]{0.9} \textbf{Descrizione} \\ \hline
    R1O 1 & Un utente deve poter visualizzare la lista delle applicazioni del catalogo di Monokee. \\ \hline 
R1O 1.1 & Un utente deve poter visualizzare i dettagli di un'applicazione presente nel catalogo di Monokee. \\ \hline 
R1O 1.2 & Un utente deve poter visualizzare il nome dell'applicazione del catalogo di Monokee. \\ \hline 
R1O 1.3 & Un utente deve poter visualizzare la descrizione dell'applicazione del catalogo di Monokee. \\ \hline 
R1O 1.4 & Un utente deve poter visualizzare l'immagine di un'applicazione del catalogo di Monokee. \\ \hline 
R1O 1.5 & Un utente deve poter visualizzare la lista delle categorie associate ad un'applicazione del catalogo di Monokee. \\ \hline 
R1O 1.6 & Un utente deve poter visualizzare il tipo di \glossaryItem{autenticazione} di un'applicazione del catalogo di Monokee. \\ \hline 
R1O 1.7 & Un utente deve poter visualizzare ogni dettaglio di \glossaryItem{autenticazione} per un'applicazione del catalogo di Monokee. \\ \hline 
R1O 1.8 & Un utente deve poter visualizzare la visibilità (pubblica o privata) dell'applicazione. \\ \hline 
R1O 1.9 & Un utente deve poter visualizzare il nome del gruppo nel quale è inserita l'applicazione. \\ \hline 
R1O 1.10 & Un utente deve poter visualizzare se l'applicazione è in manutenzione o meno. \\ \hline 
R1O 1.10 & Un utente deve poter visualizzare se l'applicazione è stata pubblicata o meno. \\ \hline 
R1O 2 & Un utente deve poter inserire una nuova applicazione. \\ \hline 
R1O 2.1 & Un utente deve poter inserire il nome della nuova applicazione. \\ \hline 
R1O 2.2 & Un utente deve poter inserire la descrizione della nuova applicazione. \\ \hline 
R1O 2.3 & Un utente deve poter inserire l'\glossaryItem{url} della nuova applicazione. \\ \hline 
R1O 2.4 & Un utente deve poter inserire l'immagine della nuova applicazione. \\ \hline 
R1O 2.5 & Un utente deve poter selezionare le categorie di appartenenza della nuova applicazione. \\ \hline 
R1O 2.6 & Un utente deve poter selezionare il tipo di \glossaryItem{autenticazione} della nuova applicazione. \\ \hline 
R1O 2.7 & Un utente deve poter inserire inserire i dati di \glossaryItem{autenticazione} della nuova applicazione. \\ \hline 
R1O 2.8 & Un utente deve poter inserire il nome del gruppo nel quale sarà inserita la nuova applicazione. \\ \hline 
R1O 3 & Un utente deve poter modificare un'applicazione. \\ \hline 
R1O 3.1 & Un utente deve poter modificare il nome dell'applicazione. \\ \hline 
R1O 3.2 & Un utente deve poter modificare la descrizione dell'applicazione.\\ \hline 
R1O 3.3 & Un utente deve poter modificare l'\glossaryItem{url} dell'applicazione.\\ \hline 
R1O 3.4 & Un utente deve poter modificare l'immagine dell'applicazione. \\ \hline 
R1O 3.5 & Un utente deve poter selezionare le categorie di appartenenza dell'applicazione. \\ \hline 
R1O 3.6 & Un utente deve poter modificare modificare i dati di autenticazione dell'applicazione. \\ \hline 
R1O 3.7 & Un utente deve poter modificare il nome del gruppo nel quale sarà inserita l'applicazione. \\ \hline 
R1O 3.8 & Un utente deve poter mettere un'applicazione in manutenzione. \\ \hline 
R1O 3.9 & Un utente deve poter pubblicare un'applicazione. \\ \hline 
R1O 4 & Un utente deve poter rimuovere un'applicazione. \\ \hline 
R1O 5 & Un utente deve poter gestire i cataloghi di dominio di Monokee. \\ \hline 
R1O 5.1 & Un utente deve poter aggiungere un catalogo di dominio. \\ \hline 
R1O 5.2 & Un utente deve visualizzare un messaggio di errore se il dominio aziendale è già collegato ad un catalogo di dominio. \\ \hline 
R1O 5.3 & Un utente deve poter visualizzare i cataloghi di dominio esistenti. \\ \hline 
R1O 5.4 & Un utente deve poter gestire le applicazioni presenti in un catalogo di dominio. \\ \hline 
R1O 5.4.1 & Un utente deve poter visualizzare le applicazioni nel catalogo. \\ \hline 
R1O 5.4.2 & Un utente deve poter aggiungere un'applicazione al catalogo. \\ \hline 
R1O 5.4.3 & Un utente deve poter rimuovere un'applicazione dal catalogo. \\ \hline 
R1O 5.5 & L'utente deve poter rimuovere un catalogo di dominio esistente. \\ \hline 
R1O 5.6 & La rimozione di un catalogo di dominio deve comportare la rimozione di tute le applicazioni collegate a quel catalogo. \\ \hline 
R1O 6 & Un utente deve poter cercare un'applicazione nel catalogo. \\ \hline 
R1O 10 & Un utente deve poter gestire i gruppi di applicazioni. \\ \hline 
R1O 10.1 & Un utente deve poter aggiungere un gruppo. \\ \hline 
R1O 10.1.1 & Un utente deve poter inserire il nome del gruppo. \\ \hline 
R1O 10.1.2 & Un utente deve poter inserire la descrizione del gruppo. \\ \hline 
R1O 10.1.3 & Un utente deve poter inserire l'immagine del gruppo. \\ \hline 
R1O 10.2 & Un utente deve poter visualizzare i gruppi di applicazioni esistenti. \\ \hline 
R1O 10.3 & Un utente deve poter gestire un gruppo di applicazioni esistente. \\ \hline 
R1O 10.3.1 & Un utente deve poter aggiungere un'applicazione al gruppo.  \\ \hline 
R1O 10.3.2 & Un utente deve poter rimuovere un'applicazione dal gruppo. \\ \hline 
R1O 10.3.3 & Un utente deve visualizzare un messaggio di errore se l'applicazione selezionata è già presente nel gruppo. \\ \hline 
R1O 10.4 & Un utente deve poter modificare un gruppo. \\ \hline 
R1O 10.4.1 & Un utente deve poter inserire il nuovo nome del gruppo. \\ \hline 
R1O 10.4.2 & Un utente deve poter inserire la nuova descrizione del gruppo. \\ \hline 
R1O 10.4.3 & Un utente deve poter inserire la nuova immagine del gruppo. \\ \hline 
R1O 10.5 & Un utente deve poter rimuovere un gruppo.\\ \hline 
R1O 13 & Un utente deve visualizzare un messaggio di errore se l'applicazione che si sta cercando di inserire è già presente. \\ \hline
R1D 14 & Un utente deve poter visualizzare le statistiche dell'applicazione Catalogue Manager. \\ \hline 
R1D 14.1 & Un utente deve poter visualizzare il numero di applicazioni aggiunte e rimosse in intervalli di tempo definiti a priori: ultime 24 ore, ultima settimana, ultimo mese e ultimo anno. \\ \hline 
R1D 14.2 & Un utente deve poter visualizzare il numero di accessi in intervalli di tempo definiti a priori: ultime 24 ore e ultima settimana. \\ \hline 
R1D 14.3 & Un utente deve poter visualizzare il numero di utenti attivi. \\ \hline 
R1D 14.4 & Un utente deve poter visualizzare il numero di applicazioni e gruppi pubblici e privati. \\ \hline 
R1D 14.5 & Un utente deve poter visualizzare il numero di applicazioni, pubbliche e private, appartenenti ad ogni categoria (intesa come ''sotto categoria''). \\ \hline 
R1D 14.6 & Un utente deve poter visualizzare il numero di applicazioni, pubbliche e private, appartenenti ad una specifica ''sovra categoria''. \\ \hline 
R1D 15 & Un utente deve poter visualizzare i \textit{log} dell'applicazione Catalogue Manager. \\ \hline 
R1D 16 & Un utente deve poter visualizzare i \textit{log} delle operazioni eseguite con successo. \\ \hline 
R1D 17 & Un utente deve poter visualizzare i \textit{log} delle operazioni che hanno generato errori. \\ \hline 
R1O 18 & Un utente deve poter effettuare una ricerca tra i domini aziendali di Monokee. \\ \hline 
R1O 19 & Un utente deve visualizzare un messaggio di errore se il dominio ha già un catalogo associato. \\ \hline
    \caption[Principali requisiti funzionali]{Principali requisiti funzionali}
    \label{tab:reqfunzionali}
  \end{longtable}
  \egroup
\end{center} 

\subsection{Requisiti di vincolo}
Nella Tabella~\ref{tab:reqvincolo} vengono elencati i requisiti di vincolo dell'applicazione.
\begin{center}
  \bgroup
  \def\arraystretch{1.8}
  \begin{longtable}{ | l | p{8.4cm} |}
    \hline
    \cellcolor[gray]{0.9} \textbf{Requisito} & \cellcolor[gray]{0.9} \textbf{Descrizione} \\ \hline
    R2O 1 & L'applicazione deve utilizzare Node.js e JavaScript. \\ \hline
    R2O 2 & L'applicazione deve memorizzare i dati su MongoDB. \\ \hline
    R2O 3 & L'accesso all'applicazione deve avvenire con \glossaryItem{saml}. \\ \hline
    \caption[Requisiti di vincolo]{Requisiti di vincolo}
    \label{tab:reqvincolo} 
    \end{longtable}
  \egroup
\end{center} 