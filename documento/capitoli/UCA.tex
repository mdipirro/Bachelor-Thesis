\newpage
\subsection{Operazioni generali per l'Utente Autenticato}
\begin{figure}[hbpc]
  \begin{center}
    \includegraphics[width=12cm]{UC/UtenteAutenticato}
  \caption[Operazioni Generali per l'Utente Autenticato]{Operazioni di alto livello permesse ad un Utente Autenticato}
  \end{center} 
\end{figure}
\newpage
\begin{center}
  \bgroup
  \def\arraystretch{1.8}     
  \begin{longtable}{  p{3.5cm} | p{8cm} } 
    \hline
    \multicolumn{2}{ | c | }{ \cellcolor[gray]{0.9} \textbf{Operazioni generali per l'Utente Autenticato}} \\
    \textbf{Attori Primari} & Utente Autenticato \\ 
    \textbf{Scopo e Descrizione} & All'Utente Autenticato viene mostrata una \textit{dashboard} a partire dalla quale può eseguire varie operazioni. Queste operazioni permettono di gestire l'intero catalogo applicativo, i raggruppamenti di applicazioni e i cataloghi specifici per i domini aziendali. È possibile anche cercare le applicazioni sulla base di vari filtri, visualizzare statistiche e \textit{log}. \\ 
    \textbf{Precondizioni}  & L'Utente Autenticato ha effettuato l'accesso a Catalogue Manager a partire da Monokee attraverso \glossaryItem{saml}. \\
    \textbf{Postcondizioni} & Catalogue Manager ha preso in carico, ed eseguito, l'operazione voluta dall'Utente Autenticato.  \\ 
    \textbf{Flusso Principale} & 
    1. L'Utente Autenticato visualizza la lista di applicazioni presenti nel catalogo. (UCA1) \newline
    2. L'Utente Autenticato aggiunge un'applicazione al catalogo. (UCA2) \newline
    3. L'Utente Autenticato modifica i dati di un'applicazione del catalogo. (UCA3) \newline 
    4. L'Utente Autenticato rimuove un'applicazione. (UCA4) \newline
    5. L'Utente Autenticato gestisce i cataloghi specifici dei domini aziendali. (UCA5) \newline
    5. L'Utente Autenticato ricerca un'applicazione. (UCA6) \newline
    6. L'Utente Autenticato gestisce i raggruppamenti di applicazioni. (UCA10) \newline
    7. L'Utente Autenticato visualizza le statistiche sul numero di applicazioni e gruppi (UCA14) \newline
    8. L'Utente Autenticato visualizza i \textit{log} sulle operazioni effettuate da lui e dagli altri utenti dell'applicazione. (UCA15) \newline
    I \textit{log} riguardano sia le operazioni eseguite con successo (UCA16) sia quelle che hanno generato errori (UCA17).\\
    \textbf{Estensioni} & L'Utente Autenticato cerca di inserire un'applicazione pubblica che è già presente nel catalogo. (UCA13)
  \end{longtable}
  \egroup
\end{center}

\subsubsection{UCA1 - Visualizzazione Applicazioni}
\begin{center}
  \bgroup
  \def\arraystretch{1.8}     
  \begin{longtable}{  p{3.5cm} | p{8cm} } 
    \multicolumn{2}{ | c | }{ \cellcolor[gray]{0.9} \textbf{UCA1 - Visualizzazione Applicazioni}} \\
    \hline
    
    \textbf{Attori Primari} & Utente Autenticato \\ 
    \textbf{Scopo e Descrizione} & L'Utente Autenticato può visualizzare le applicazioni presenti nel catalogo di Monokee. \\ 
    
    \textbf{Precondizioni}  & Catalogue Manager presenta all'Utente Autenticato una pagina contenente l'elenco delle applicazioni presenti in Monokee. \\ 
    
    \textbf{Postcondizioni} & L'Utente Autenticato ha visualizzato le applicazioni presenti in Monokee. \\ 
    \textbf{Flusso Principale} & 
    1. L'Utente Autenticato visualizza i dettagli di un'applicazione presente in Monokee.   
  \end{longtable}
  \egroup
\end{center}

\subsubsection{UCA2 - Aggiunta Applicazione Pubblica}
\begin{center}
  \bgroup
  \def\arraystretch{1.8}     
  \begin{longtable}{  p{3.5cm} | p{8cm} } 
    \multicolumn{2}{ | c | }{ \cellcolor[gray]{0.9} \textbf{UCA2 - Aggiunta Applicazione Pubblica}} \\
    \hline
    
    \textbf{Attori Primari} & Utente Autenticato \\ 
    \textbf{Scopo e Descrizione} & L'Utente Autenticato può aggiungere una nuova applicazione al catalogo di Monokee. \\ 
    
    \textbf{Precondizioni}  & Catalogue Manager presenta all'Utente Autenticato una pagina contenente un \textit{form} per l'aggiunta di un'applicazione al catalogo di Monokee. \\ 
    
    \textbf{Postcondizioni} & L'Utente Autenticato ha aggiunto un'applicazione al catalogo di Monokee. \\ 
    \textbf{Flusso Principale} & 
    1. L'Utente Autenticato inserisce il nome dell'applicazione. \newline
    2. L'Utente Autenticato inserisce la descrizione dell'applicazione. \newline
    3. L'Utente Autenticato inserisce l'\glossaryItem{url} dell'applicazione. \newline
    4. L'Utente Autenticato inserisce l'immagine dell'applicazione. \newline
    5. L'Utente Autenticato seleziona le categorie di appartenenza dell'applicazione. \newline
    6. L'Utente Autenticato seleziona la tipologia di \glossaryItem{autenticazione}. \newline
    7. L'Utente Autenticato seleziona il nome del gruppo di applicazioni nel quale inserire l'applicazione. \\
    \textbf{Estensioni} & 
        1. L'Utente Autenticato visualizza un messaggio di errore come conseguenza al tentativo di inserimento dello stesso nome di un'altra applicazione pubblica di Monokee.
  \end{longtable}
  \egroup
\end{center}

\subsubsection{UCA3 - Modifica Applicazione Pubblica}
\begin{center}
  \bgroup
  \def\arraystretch{1.8}     
  \begin{longtable}{  p{3.5cm} | p{8cm} } 
    \multicolumn{2}{ | c | }{ \cellcolor[gray]{0.9} \textbf{UCA3 - Modifica Applicazione Pubblica}} \\
    \hline
    
    \textbf{Attori Primari} & Utente Autenticato \\ 
    \textbf{Scopo e Descrizione} & L'Utente Autenticato può modificare un'applicazione esistente nel catalogo di Monokee. \\ 
    
    \textbf{Precondizioni}  & Catalogue Manager presenta all'Utente Autenticato una pagina contenente un \textit{form} per la modifica di un'applicazione nel catalogo di Monokee. \\ 
    
    \textbf{Postcondizioni} & L'Utente Autenticato ha modificato un'applicazione nel catalogo Monokee. \\ 
    \textbf{Flusso Principale} & 
    1. L'Utente Autenticato modifica il nome dell'applicazione. \newline
    2. L'Utente Autenticato modifica la descrizione dell'applicazione. \newline
    3. L'Utente Autenticato modifica l'\glossaryItem{url} dell'applicazione. \newline
    4. L'Utente Autenticato modifica l'immagine dell'applicazione. \newline
    5. L'Utente Autenticato seleziona le categorie di appartenenza dell'applicazione. \newline
    6. L'Utente Autenticato modifica i dati necessari all'\glossaryItem{autenticazione}. \newline
    	Tali dati possono riguardare l'accesso \textit{form-based}, tramite \glossaryItem{saml} o di terzo tipo. \newline
    7. L'Utente Autenticato seleziona il nome del gruppo di applicazioni nel quale modificare l'applicazione. \newline
    8. L'Utente Autenticato decide se pubblicare l'applicazione. \newline
    9. L'Utente Autenticato decide se mettere l'applicazione in manutenzione.\\
    \textbf{Estensioni} & 
    1. L'Utente Autenticato visualizza un messaggio di errore come conseguenza al tentativo di inserimento dello stesso nome di un'altra applicazione pubblica di Monokee.
  \end{longtable}
  \egroup
\end{center}

\subsubsection{UCA4 - Rimozione Applicazione Pubblica}
\begin{center}
  \bgroup
  \def\arraystretch{1.8}     
  \begin{longtable}{  p{3.5cm} | p{8cm} } 
    \multicolumn{2}{ | c | }{ \cellcolor[gray]{0.9} \textbf{UC4 - Rimozione Applicazione Pubblica}} \\
    \hline
    \textbf{Attori Primari} & Utente Autenticato \\ 
    \textbf{Scopo e Descrizione} & L'Utente Autenticato può rimuovere un'applicazione pubblica dal catalogo di Monokee. \\ 
    
    \textbf{Precondizioni}  & L'applicazione selezionata è presente nel catalogo di Monokee e l'Utente Autenticato ha selezionato il comando di rimozione su di essa. \\ 
    
    \textbf{Postcondizioni} & L'applicazione selezionata è stata rimossa dal catalogo di Monokee. \\ 
    \textbf{Flusso Principale} &
    1. L'Utente Autenticato conferma l'operazione di rimozione. \\
    \textbf{Inclusioni} & Richiesta di conferma rimozione.
  \end{longtable}
  \egroup
\end{center}

\subsubsection{UCA5 - Gestione Cataloghi di Dominio}
\begin{center}
  \bgroup
  \def\arraystretch{1.8}     
  \begin{longtable}{  p{3.5cm} | p{8cm} } 
    \multicolumn{2}{ | c | }{ \cellcolor[gray]{0.9} \textbf{UCA5 - Gestione Cataloghi di Dominio}} \\
    \hline
    
    \textbf{Attori Primari} & Utente Autenticato \\ 
    \textbf{Scopo e Descrizione} & L'Utente Autenticato può gestire i cataloghi associati a domini aziendali. \\ 
    
    \textbf{Precondizioni}  & Catalogue Manager presenta all'Utente Autenticato la pagina per la gestione dei cataloghi associati a domini aziendali. \\ 
    
    \textbf{Postcondizioni} & Catalogue Manager ha preso in carico, ed eseguito, le operazioni richieste dall'Utente Autenticato. \\ 
    \textbf{Flusso Principale} & 
    1. L'Utente Autenticato può aggiungere un nuovo catalogo di dominio ad un dominio senza catalogo. \newline 
    2. L'Utente Autenticato può visualizzare i cataloghi di dominio esistenti. \newline 
    3. L'Utente Autenticato può aggiungere un'applicazione ad un catalogo di dominio esistente. \newline
    4. L'Utente Autenticato può rimuovere un'applicazione da un catalogo di dominio esistente. \newline
    5. L'Utente Autenticato può rimuovere un catalogo di dominio esistente. \newline 
    6. L'Utente Autenticato può cercare un catalogo di dominio tra quelli esistenti. \newline
    	La ricerca può avvenire in base al nome del dominio. \\
    \textbf{Estensioni} &
    1. L'Utente Autenticato visualizza un messaggio di errore dovuto all'esistenza di un altro catalogo per il dominio selezionato.
  \end{longtable}
  \egroup
\end{center}

\subsubsection{UCA6 - Ricerca Applicazione}
\begin{center}
  \bgroup
  \def\arraystretch{1.8}     
  \begin{longtable}{  p{3.5cm} | p{8cm} } 
    \multicolumn{2}{ | c | }{ \cellcolor[gray]{0.9} \textbf{UCA6 - Ricerca Applicazione}} \\
    \hline
    
    \textbf{Attori Primari} & Utente Autenticato \\ 
    \textbf{Scopo e Descrizione} & L'Utente Autenticato può cercare un'applicazione del catalogo di Monokee. La ricerca può avvenire per nome o per categoria. \\ 
    
    \textbf{Precondizioni}  & Catalogue Manager ha mostrato all'utente la pagina di ricerca. L'Utente Autenticato può effettuare ricerche di vario tipo. \\ 
    
    \textbf{Postcondizioni} & Catalogue Manager ha mostrato i risultati della ricerca. \\ 
    \textbf{Flusso Principale} & 
    1. L'Utente Autenticato seleziona le modalità di ricerca e inserisce le informazioni richieste.
  \end{longtable}
  \egroup
\end{center}

\subsubsection{UCA10 - Gestione Gruppi di Applicazioni}
\begin{center}
  \bgroup
  \def\arraystretch{1.8}     
  \begin{longtable}{  p{3.5cm} | p{8cm} } 
    \multicolumn{2}{ | c | }{ \cellcolor[gray]{0.9} \textbf{UCA10 - Gestione Gruppi di Applicazioni}} \\
    \hline
    \textbf{Attori Primari} & Utente Autenticato \\ 
    \textbf{Scopo e Descrizione} & L'Utente Autenticato può gestire i gruppi di applicazioni del catalogo di Monokee. \\ 
    
    \textbf{Precondizioni}  & Catalogue Manager mostra all'Utente Autenticato la pagina di gestione dei gruppi di applicazioni. \\ 
    
    \textbf{Postcondizioni} & Catalogue Manager ha preso in carico le richieste dell'Utente Autenticato e le ha eseguite. \\ 
    \textbf{Flusso Principale} &
    1. L'Utente Autenticato può aggiungere un gruppo di applicazioni. \newline
    2. L'Utente Autenticato può visualizzare i gruppi di applicazioni presenti. \newline
    3. L'Utente Autenticato può gestire un gruppo di applicazioni esistente. \newline
    	In particolare, L'Utente Autenticato può aggiungere o rimuovere un'applicazione dal gruppo selezionato. \newline
    4. L'Utente Autenticato può modificare gli attributi di un gruppo di applicazioni esistente. \newline
    5. L'Utente Autenticato può rimuovere un gruppo di applicazioni esistente. 
  \end{longtable}
  \egroup
\end{center}

\subsubsection{UCA13 - Visualizzazione di un messaggio di errore per applicazione pubblica già presente}
\begin{center}
  \bgroup
  \def\arraystretch{1.8}     
  \begin{longtable}{  p{3.5cm} | p{8cm} } 
    \multicolumn{2}{ | c | }{ \cellcolor[gray]{0.9} \textbf{UCA13 - Visualizzazione di un messaggio di errore per applicazione pubblica già presente}} \\
    \hline
    
    \textbf{Attori Primari} & Utente Autenticato \\ 
    \textbf{Scopo e Descrizione} & L'Utente Autenticato ha cercato di inserire un nome corrispondente ad un'applicazione pubblica già presente nel catalogo di Monokee. Catalogue Manager presenta all'Utente Autenticato un messaggio di errore. \\ 
    
    \textbf{Precondizioni}  & L'Utente Autenticato ha inserito (nella pagina di aggiunta o di modifica di un'applicazione) un nome già utilizzato per un'applicazione pubblica. \\ 
    
    \textbf{Postcondizioni} & L'Utente Autenticato ha visualizzato il messaggio di errore presentato da Catalogue Manager. \\
    \textbf{Flusso Principale} & 
    1. L'Utente Autenticato visualizza il messaggio di errore.
  \end{longtable}
  \egroup
\end{center}

\subsubsection{UCA14 - Visualizzazione Statistiche}
\begin{center}
  \bgroup
  \def\arraystretch{1.8}     
  \begin{longtable}{  p{3.5cm} | p{8cm} } 
    \multicolumn{2}{ | c | }{ \cellcolor[gray]{0.9} \textbf{UCA14 - Visualizzazione Statistiche}} \\
    \hline
    
    \textbf{Attori Primari} & Utente Autenticato \\ 
    \textbf{Scopo e Descrizione} & L'Utente Autenticato può visualizzare delle statistiche riguardanti l'applicazione Catalogue Manager. In particolare, le statistiche riguardano:
    	\begin{enumerate}
 	    \item numero di applicazioni aggiunte e rimosse in intervalli di tempo definiti a priori: ultime 24 ore, ultima settimana, ultimo mese e ultimo anno;
 	    \item numero di accessi in intervalli di tempo definiti a priori: ultime 24 ore e ultima settimana;
 	    \item numero di utenti attivi;
 	    \item numero di applicazioni e gruppi pubblici e privati;
 	    \item numero di applicazioni, pubbliche e private, appartenenti ad ogni categoria (intesa come ''sotto categoria'');
 	    \item numero di applicazioni, pubbliche e private, appartenenti ad una specifica ''sovra categoria''.
 	  	\end{enumerate} \\ 
    
    \textbf{Precondizioni}  & L'Utente Autenticato ha richiesto la visualizzazione delle statistiche dell'applicazione Catalogue Manager. \\ 
    
    \textbf{Postcondizioni} & L'Utente Autenticato ha visualizzato le statistiche dell'applicazione Catalogue Manager. \\
    \textbf{Flusso Principale} & 
    1. L'Utente Autenticato visualizza il numero di applicazioni aggiunte e rimosse in intervalli di tempo definiti a priori: ultime 24 ore, ultima settimana, ultimo mese e ultimo anno. \newline
    2. L'Utente Autenticato visualizza il numero di accessi in intervalli di tempo definiti a priori: ultime 24 ore e ultima settimana. \newline
    3. L'Utente Autenticato visualizza il numero di utenti attivi. \newline
    4. L'Utente Autenticato visualizza il numero di applicazioni e gruppi pubblici e privati. \newline
    5. L'Utente Autenticato visualizza il numero di applicazioni, pubbliche e private, appartenenti ad ogni categoria (intesa come ''sotto categoria''). \newline
    6. L'Utente Autenticato visualizza il numero di applicazioni, pubbliche e private, appartenenti ad una specifica ''sovra categoria''.
  \end{longtable}
  \egroup
\end{center}

\subsubsection{UCA15 - Visualizzazione Log}
\begin{center}
  \bgroup
  \def\arraystretch{1.8}     
  \begin{longtable}{  p{3.5cm} | p{8cm} } 
    \multicolumn{2}{ | c | }{ \cellcolor[gray]{0.9} \textbf{UCA15 - Visualizzazione Log}} \\
    \hline
    
    \textbf{Attori Primari} & Utente Autenticato \\ 
    \textbf{Scopo e Descrizione} & L'Utente Autenticato può visualizzare i \textit{log} delle operazioni eseguite nell'applicazione Catalogue Manager. In particolare i \textit{log} possono riguardare le operazioni eseguite con successo (UCA16) e quelle che hanno generato un errore (UCA17). I \textit{log} salvati possono essere filtrati per intervallo di date, per \textit{keywords} e per tipologia. \\
    
    \textbf{Precondizioni}  & L'Utente Autenticato ha richiesto la visualizzazione dei \textit{log} dell'applicazione Catalogue Manager. \\ 
    
    \textbf{Postcondizioni} & L'Utente Autenticato ha visualizzato i \textit{log} dell'applicazione Catalogue Manager. \\
    \textbf{Flusso Principale} & 
    1. L'Utente Autenticato visualizza i \textit{log} dell'applicazione. 
  \end{longtable}
  \egroup
\end{center}

\subsubsection{UCA16 - Visualizzazione Log sulle operazioni che hanno avuto successo}
\begin{center}
  \bgroup
  \def\arraystretch{1.8}     
  \begin{longtable}{  p{3.5cm} | p{8cm} } 
    \multicolumn{2}{ | c | }{ \cellcolor[gray]{0.9} \textbf{UCA16 - Visualizzazione Log sulle operazioni che hanno avuto successo}} \\
    \hline
    
    \textbf{Attori Primari} & Utente Autenticato \\ 
    \textbf{Scopo e Descrizione} & L'Utente Autenticato può visualizzare i \textit{log} delle operazioni eseguite con successo nell'applicazione Catalogue Manager. I \textit{log} salvati possono essere filtrati per intervallo di date, per \textit{keywords} e per tipologia. \\
    
    \textbf{Precondizioni}  & L'Utente Autenticato ha richiesto la visualizzazione dei \textit{log} dell'applicazione Catalogue Manager. Ha successivamente selezionato la visualizzazione dei \textit{log} delle operazioni eseguite con successo. \\ 
    
    \textbf{Postcondizioni} & L'Utente Autenticato ha visualizzato i \textit{log} delle operazioni eseguite con successo dell'applicazione Catalogue Manager. \\
    \textbf{Flusso Principale} & 
    1. L'Utente Autenticato visualizza i \textit{log} delle operazioni eseguite con successo.
  \end{longtable}
  \egroup
\end{center}

\subsubsection{UCA17 - Visualizzazione Log delle operazioni che hanno generato errori}
\begin{center}
  \bgroup
  \def\arraystretch{1.8}     
  \begin{longtable}{  p{3.5cm} | p{8cm} } 
    \multicolumn{2}{ | c | }{ \cellcolor[gray]{0.9} \textbf{UCA17 - Visualizzazione Log delle operazioni che hanno generato errori}} \\
    \hline
    
    \textbf{Attori Primari} & Utente Autenticato \\ 
    \textbf{Scopo e Descrizione} & L'Utente Autenticato può visualizzare i \textit{log} delle operazioni che hanno generato errori nell'applicazione Catalogue Manager. I \textit{log} salvati possono essere filtrati per intervallo di date, per \textit{keywords} e per tipologia. \\
    
    \textbf{Precondizioni}  & L'Utente Autenticato ha richiesto la visualizzazione dei \textit{log} dell'applicazione Catalogue Manager. Ha successivamente selezionato la visualizzazione dei \textit{log} delle operazioni che hanno generato errori. \\ 
    
    \textbf{Postcondizioni} & L'Utente Autenticato ha visualizzato i \textit{log} delle operazioni che hanno generato errori dell'applicazione Catalogue Manager. \\
    \textbf{Flusso Principale} & 
    1. L'Utente Autenticato visualizza i \textit{log} delle operazioni che hanno generato errore. 
  \end{longtable}
  \egroup
\end{center}

\subsubsection{UCA18 - Ricerca Domini Aziendali di Monokee}
\begin{center}
  \bgroup
  \def\arraystretch{1.8}     
  \begin{longtable}{  p{3.5cm} | p{8cm} } 
    \multicolumn{2}{ | c | }{ \cellcolor[gray]{0.9} \textbf{UCA18 - Ricerca Domini Aziendali di Monokee}} \\
    \hline
    \textbf{Attori Primari} & Utente Autenticato \\ 
    \textbf{Scopo e Descrizione} & L'Utente Autenticato può effettuare una ricerca tra i domini aziendali di Monokee. La ricerca avviene per nome. \\ 
    
    \textbf{Precondizioni}  & Catalogue Manager mostra all'Utente Autenticato la pagina di gestione dei cataloghi di dominio di Monokee. \\ 
    
    \textbf{Postcondizioni} & Catalogue Manager ha preso cercato tra i domini aziendali di Monokee e ha mostrato all'Utente Autenticato i risultati. \\ 
    \textbf{Flusso Principale} &
    1. L'Utente Autenticato cerca un dominio aziendale di Monokee. 
  \end{longtable}
  \egroup
\end{center}

\subsubsection{UCA19 - Visualizzazione di un messaggio di errore per catalogo già presente}
\begin{center}
  \bgroup
  \def\arraystretch{1.8}     
  \begin{longtable}{  p{3.5cm} | p{8cm} } 
    \multicolumn{2}{ | c | }{ \cellcolor[gray]{0.9} \textbf{UCA19 - Visualizzazione di un messaggio di errore per catalogo già presente}} \\
    \hline
    
    \textbf{Attori Primari} & Utente Autenticato \\ 
    \textbf{Scopo e Descrizione} & L'Utente Autenticato ha cercato di aggiungere un catalogo di dominio ad un dominio che ha già un catalogo. Catalogue Manager presenta all'Utente Autenticato un messaggio di errore. \\ 
    
    \textbf{Precondizioni}  & L'Utente Autenticato ha cercato di aggiungere un catalogo di dominio ad un dominio che ha già un catalogo. \\ 
    
    \textbf{Postcondizioni} & L'Utente Autenticato ha visualizzato il messaggio di errore presentato da Catalogue Manager. \\
    \textbf{Flusso Principale} & 
    1. L'Utente Autenticato visualizza il messaggio di errore.
  \end{longtable}
  \egroup
\end{center}